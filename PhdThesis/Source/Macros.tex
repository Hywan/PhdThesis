% Theorem.
%\theoremstyle{definition}
%\newtheorem{definition}{Définition}[chapter]
%\newtheorem{example}{Exemple}[chapter]

\declaretheoremstyle[
    spaceabove=.4cm,
    spacebelow=.4cm,
    bodyfont=\normalfont
]{mytheoremstyle}
\declaretheorem[style=mytheoremstyle, parent=chapter, title=Définition]{definition}
\declaretheorem[style=mytheoremstyle, parent=chapter, title=Exemple]{example}

\makeatletter
\def\ll@definition{%
  \protect\numberline{\thedefinition}\thmt@shortoptarg%
}
\makeatother

\makeatletter
\def\ll@example{%
  \protect\numberline{\theexample}\thmt@shortoptarg%
}
\makeatother

% Style.
\newcommand{\code}[1]{\texttt{#1}}
\newcommand{\mem}[1]{\mathit{#1}}
\newcommand{\m}[1]{\mathcal{#1}}
\newcommand{\fig}[3]{%
    \begin{center}%
    \resizebox{#1}{#2}{%
        \IfSubStr{#3}{.tex}{%
            \begin{tikzpicture}[>=latex,line join=bevel]%
            \pgfsetlinewidth{1bp}%
            \requiredist{Figure/#3}%
            \end{tikzpicture}%
        }{%
            \includegraphics{\distdirectory/Figure/#3}%
        }%
    }%
    \end{center}
}
\newcommand{\circled}[1]{
    \protect\tikz [baseline=(char.base)] {
        \protect\node [shape=circle,draw,inner sep=2pt, minimum height=1.2em] (char) {\scriptsize $#1$};
    }
}

\newenvironment{pre}{
    \begin{alltt}\centering
    \begin{tabular}{l}
}{
    \end{tabular}
    \end{alltt}
}
\newenvironment{bigpre}{
    \footnotesize\begin{pre}
}{
    \end{pre}
}
\newenvironment{bigbigpre}{
    \scriptsize
    \begin{alltt}\centering
    \begin{longtable}{l}
}{
    \end{longtable}
    \end{alltt}
}

% Shortcut.
\def\eg{\textit{e.g. }}
\def\ie{\textit{i.e. }}
\def\sc{\textsc}
\def\via{\textit{via }}
\def\em{\itshape}
\def\strong{\bfseries}

% Praspel.
\def\abehavior{\code{@be\-ha\-vior}\xspace}
\def\adefault{\code{@de\-fault}\xspace}
\def\adescription{\code{@desc\-rip\-tion}\xspace}
\def\aensures{\code{@en\-su\-res}\xspace}
\def\ainvariant{\code{@in\-va\-riant}\xspace}
\def\ais{\code{@is}\xspace}
\def\aolde{\code{\bslash{}old($e$)}\xspace}
\newcommand{\aold}[1]{\code{\bslash{}old($#1$)}\xspace}
\def\apredicate{\code{@pre\-di\-ca\-te}\xspace}
\newcommand{\apred}[1]{\code{\bslash{}pred($#1$)}\xspace}
\def\arequires{\code{@re\-qui\-res}\xspace}
\def\aresult{\code{\bslash{}result}\xspace}
\def\athrowable{\code{@thro\-wa\-ble}\xspace}
\def\false{\texttt{false}}
\def\true{\texttt{true}}

% Math.
\def\B{\mathbb{B}}
\def\N{\mathbb{N}}
\def\R{\mathbb{R}}
\def\Q{\mathbb{Q}}
\def\Z{\mathbb{Z}}
\def\U{\mathcal{U}}
\def\O{\mathcal{O}}
\def\disjunion{\uplus}
\def\Intersection{\bigcap}
\def\Land{\bigwedge}
\def\Lor{\bigvee}
\def\Union{\bigcup}
\def\intersection{\cap}
\def\land{\wedge}
\def\lor{\vee}
\def\mvert{$\;\;|\;\;$}
\def\union{\cup}
\newcommand{\afrac}[2]{\genfrac{}{}{0pt}{0}{#1}{#2}}
\newcommand{\amultiset}[1]{\{\!\!\{#1\}\!\!\}}

% Algo.
\renewcommand{\algorithmicrequire}{\quad\quad\,\,\,\textbf{input\;}}
\renewcommand{\algorithmicensure}{\quad\quad\,\,\,\textbf{output\;}}
\newcommand{\avar}[1]{\mem{#1}}
\newcommand{\atype}[1]{\mathrm{#1}}
\newcommand{\astring}[1]{\mathrm{#1}}
\newcommand{\akw}[1]{\;\mathrm{#1\;}}
\newcommand{\astruct}[1]{\atype{\textbf{struct}~#1}}
\newcommand{\astype}[1]{\langle#1\rangle}
\newcommand{\acall}[1]{\mathrm{#1}}
\newcommand{\asets}[1]{\rightarrow}
\newcommand{\abreak}{\textbf{break}}
\newcommand{\areturn}{\textbf{return~}}

% Language.
\newcommand{\inenglish}[1]{\foreignlanguage{english}{\em #1}}

% Bibliography
\newcommand{\acite}[1]{\citep{#1}}     % normal citation
\newcommand{\acitep}[1]{\citealp{#1}}  % plain citation
\newcommand{\acitei}[1]{\citet{#1}}    % inline citation

% Color
\definecolor{light-gray}{gray}{0.65}
\newcommand{\ingray}[1]{\textcolor{light-gray}{#1}}

% Tikz
\tikzstyle{rectangle}=[
    draw=black,
    thick,
    text centered,
    text width=2cm
]
\tikzstyle{arrow}=[
    ->,
    >=latex,
    shorten >=1pt,
    thick
]
\tikzstyle{myoverlay}=[
    overlay,
    remember picture
]
\tikzstyle{mybrace}=[
    decoration={brace, amplitude=.8em},
    decorate,
    thick
]
\tikzstyle{mybracetext}=[
    right,
    xshift=1.2em
]
\tikzstyle{mywavyarrow}=[
    ->,
    >=latex,
    decorate,
    decoration={snake, amplitude=.4mm, segment length=2mm, post length=1.5mm}
]
\newcommand{\tikzref}[1]{\tikz[myoverlay, anchor=base, baseline] {
    \node (#1) {\phantom{x}};
}}
\newenvironment{tikzbox}[2]{
    \begin{tikzpicture}[remember picture]
    % create a fake node and set a another node to allow to shift the figure.
    \node (0, 0) {};
    \node [#2] (#1) at (5, 0) \bgroup
    \begin{varwidth}{\linewidth}
}{
    \end{varwidth}
    \egroup;
    \end{tikzpicture}
}
\newenvironment{tikzannotation}{
    \begin{tikzpicture}[myoverlay]
}{
    \end{tikzpicture}
}
\newcommand{\tikzdraw}[2]{
    \begin{tikzpicture}[#1]
        #2
    \end{tikzpicture}
}
\newcommand{\drawfig}[4]{
    \begin{center}
    \resizebox{#1}{#2}{
        \begin{tikzpicture}[#3]
            #4
        \end{tikzpicture}
    }
    \end{center}
}

% Others.
\def\aand{\code{and}\xspace}
\def\aor{\code{or}\xspace}
\def\atsign{\code{@}\xspace}
\def\bslash{\symbol{`\\}}
\newcommand{\aquote}[1]{"#1"}
