% !style
%
\usepackage[english,french]{babel}

\usepackage[utf8]{inputenc}
\usepackage[T1]{fontenc}

\usepackage{amsmath}
\usepackage{amssymb}
\usepackage{amsfonts}
\usepackage{amsthm}
\usepackage{tikz}
\usepackage{alltt}
\usepackage{array}
\usepackage{xspace}
\usepackage{latexsym}
\usepackage{tikz}
  \usetikzlibrary{snakes,arrows,shapes,automata}
\usepackage{paralist}
\usepackage{url}
\usepackage[linktoc=all]{hyperref}
  \hypersetup{
      backref       = true,
      pagebackref   = true,
      hyperindex    = true,
      colorlinks    = true,
      breaklinks    = true,
      urlcolor      = black,
      linkcolor     = black,
      citecolor     = black,
      bookmarks     = true,
      bookmarksopen = true
  }
\usepackage{minitoc}
  \renewcommand{\mtctitle}{Table des matières}
\usepackage[round]{natbib}

\everymath{\displaystyle}

% !style
%\declarethesis[Sous-titre]{Titre}{30 juin 2014}{XXX}
%\addauthor[ivan.enderlin@femto-st.fr]{Ivan}{Enderlin}
%
%\addjury{Foo}{Bar}{Rapporteur}{Professeur à l'Université …}
%\addjury{Fabrice}{Bouquet}{Directeur de thèse}{Professeur à l'Université de Franche-Comté}
%\addjury{Frédéric}{Dadeau}{Encadrant de thèse}{Docteur à l'Université de Franche-Comté}
%\addjury{Alain}{Giorgetti}{Encadrant de thèse}{Docteur à l'Université de Franche-Comté}
%
%\thesisabstract[english]{
%    This is an abstract in English.
%}
%\thesiskeywords[english]{Keyword 1, Keyword 2}
%
%\thesisabstract[french]{
%    Ceci est le résumé en français.
%}
%\thesiskeywords[english]{Mot-clé 1, Mot-clé 2}
%
%\Set{speciality}{Informatique}
%\Set{universityname}{Université Sciences et Techniques de Besançon}
