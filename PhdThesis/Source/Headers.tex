\usepackage[english,french]{babel}

\usepackage[utf8]{inputenc}
\usepackage[T1]{fontenc}

\usepackage{amsmath,amssymb,amsfonts,amsthm}
  \allowdisplaybreaks
\usepackage{thmtools}
\usepackage{tikz}
\usepackage{alltt}
\usepackage{longtable}
\usepackage{tabularx}
\usepackage{xspace}
\usepackage{latexsym}
\usepackage{tikz}
  \usetikzlibrary{
      arrows,
      shapes,
      automata,
      decorations.text,
      decorations.pathreplacing,
      decorations.pathmorphing,
      positioning
  }
\usepackage{varwidth}
\usepackage{graphicx}
\usepackage{rotating}
\usepackage{paralist}
\usepackage{algpseudocode}
\usepackage{algorithmicx}
\usepackage{color}
\usepackage{moresize}
\usepackage{url}
\usepackage[linktoc=all]{hyperref}
  \hypersetup{
      backref       = true,
      pagebackref   = true,
      hyperindex    = true,
      colorlinks    = true,
      breaklinks    = true,
      urlcolor      = black,
      linkcolor     = black,
      citecolor     = black,
      bookmarks     = true,
      bookmarksopen = true
  }
\usepackage{tocloft}
\usepackage{minitoc}
  \renewcommand{\mtctitle}{Table des matières}
  \setcounter{minitocdepth}{3}
\usepackage{fancyhdr}
\usepackage[round]{natbib}
\usepackage{xstring}
\usepackage{pdfpages}

\everymath{\displaystyle}


\usepackage{etex}
\usepackage{stmaryrd}

\usepackage[]{pxfonts}
\linespread{1.1}
\usepackage[euler-digits,euler-hat-accent]{eulervm}
\DeclareSymbolFont{symbolsPX}{OMS}{pxsy}{m}{n}
\SetSymbolFont{symbolsPX}{bold}{OMS}{pxsy}{bx}{n}
\DeclareSymbolFont{operatorsPX}{OT1}{pxr}{m}{n}
\SetSymbolFont{operatorsPX}{bold}{OT1}{pxr}{bx}{n}
\DeclareMathSymbol{\forall}{\mathord}{symbolsPX}{56}
\DeclareMathSymbol{\exists}{\mathord}{symbolsPX}{57}
\DeclareMathSymbol{\in}{\mathrel}{symbolsPX}{50}
\DeclareMathSymbol{\circ}{\mathbin}{symbolsPX}{14}
\DeclareMathSymbol{\wedge}{\mathbin}{symbolsPX}{94}
\DeclareMathSymbol{\vee}{\mathbin}{symbolsPX}{95}
\DeclareMathSymbol{\vdash}{\mathrel}{symbolsPX}{96}
\let\land=\wedge
\let\lor=\vee
\DeclareMathSymbol{=}{\mathrel}{operators}{61}
\DeclareMathSymbol{\equiv}{\mathrel}{symbolsPX}{17}
\DeclareMathDelimiter\llbracket{\mathopen}{stmry}{"4A}{stmry}{"71}
\DeclareMathDelimiter\rrbracket{\mathclose}{stmry}{"4B}{stmry}{"79}
\DeclareMathSymbol{\subseteq}{\mathrel}{symbolsPX}{18}
\DeclareMathSymbol{\supseteq}{\mathrel}{symbolsPX}{19}
\DeclareMathSymbol{\neg}{\mathord}{symbolsPX}{58}
  \let\lnot=\neg
%  \let\lbag\@undefined
%  \let\rbag\@undefined
%  \DeclareMathSymbol\lbag\mathbin{stmry}{"2A}
%  \DeclareMathSymbol\rbag\mathbin{stmry}{"2B}
\DeclareMathSymbol{\subset}{\mathrel}{symbolsPX}{26}
\DeclareMathSymbol{\supset}{\mathrel}{symbolsPX}{27}
\DeclareMathSymbol{\top}{\mathord}{symbolsPX}{62}
\DeclareMathSymbol{\bot}{\mathord}{symbolsPX}{63}
%% FOURIER MATHBB
\DeclareFontFamily{U}{futm}{}
\DeclareFontShape{U}{futm}{m}{n}{
  <-> s * [.92] fourier-bb
  }{}
\DeclareSymbolFont{Ufutm}{U}{futm}{m}{n}
\DeclareSymbolFontAlphabet{\mathbb}{Ufutm}

\usepackage[a4paper,
            top=4.4cm,
            bottom=4.4cm,
            outer=4cm,
            inner=3cm,
            heightrounded
          ]{geometry}

\usepackage{xcolor}
\definecolor{base3}{RGB}{253,246,227}
