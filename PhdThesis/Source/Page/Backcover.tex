\newgeometry{top=0cm, right=0cm, bottom=0cm, left=0cm}

\parindent0pt
\null
\vfill

{
\large

\begin{tikzpicture}[overlay,
                    every node/.style={inner sep=0, outer sep=0,
                                       text justified, text width=15cm}]

\node [anchor=south]
      at (.5\paperwidth, .52\paperheight)
      {
      \paragraph{Résumé.}

      Les travaux présentés dans ce mémoire portent sur la validation de
      programmes PHP à travers un nouveau langage de spécification, accompagné
      de ses outils. Ces travaux s'articulent selon trois axes~: langage de
      spécification, génération automatique de données de test et génération
      automatique de tests unitaires. La première contribution est Praspel, un
      nouveau langage de spécification pour PHP, basé sur la programmation par
      contrat. Praspel spécifie les données avec des domaines réalistes~: de
      nouvelles structures permettant de valider et générer des données. À
      partir d'un contrat écrit en Praspel, nous pouvons faire du
      \inenglish{Contract-based Testing}, c'est~à~dire exploiter les contrats
      pour générer automatiquement des tests unitaires. La deuxième contribution
      concerne la génération de données de test. Pour les booléens, les entiers
      et les réels, une génération aléatoire uniforme est employée. Pour les
      tableaux, un solveur de contraintes dédié est employé. Pour les chaînes de
      caractères, un langage de description de grammaires avec un compilateur de
      compilateurs $LL(\star)$ et plusieurs algorithmes de génération de données
      sont employés. Enfin, la génération d'objets est traitée. La troisième
      contribution définit des critères de couverture sur les contrats. Ces
      derniers fournissent des objectifs de tests. Toutes ces contributions ont
      été implémentées et expérimentées dans des outils industriels.

      \paragraph{Mots-clés.} Praspel, \inenglish{Contract-based Testing},
      génération de données, génération automatique de tests unitaires, PHP.
      };

\node [anchor=north]
      at (.5\paperwidth, .48\paperheight)
      {
      \paragraph{\foreignlanguage{english}{Abstract}.}

      \foreignlanguage{english}{The works presented in this memoir are about the
      validation of PHP programs through a new specification language, along
      with its tools. These works follow three axes: specification language,
      automatic test data generation and automatic unit test generation. The
      first contribution is Praspel, a new specification language for PHP, based
      on the Design by Contract. Praspel specifies data with realistic domains:
      new structures allowing to validate and generate data. Based on a
      contract, we are able to perform Contract-based Testing, i.e. using
      contracts to automatically generate unit tests. The second contribution is
      about test data generation. For booleans, integers, reals, a uniform
      random generation is used. For arrays, a dedicated constraint solver is
      used. For strings, a grammar description language along with an
      $LL(\star)$ compiler compilers and several algorithms for data generation
      are used. Finally, the object generation is supported. The third
      contribution defines contracts coverage criteria. These latters provide
      test objectives. All these contributions are implemented and experimented
      in industrial tools.}

      \paragraph{\foreignlanguage{english}{Keywords}.}
      \foreignlanguage{english}{Praspel, Contract-based Testing, data
      generation, automatic unit test generation, PHP.}
      };

\end{tikzpicture}

}

\pagebreak

\restoregeometry
