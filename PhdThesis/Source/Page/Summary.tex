\documentclass[a4paper,11pt]{article}

\usepackage[english,french]{babel}
\usepackage[utf8]{inputenc}
    \newcommand{\inenglish}[1]{\foreignlanguage{english}{\em #1}}
\usepackage[round,colon]{natbib}
    \newcommand{\acite}[1]{\citep{#1}}
    \renewcommand{\bibsection}{}
\usepackage{nopageno}
\usepackage[a4paper,
            top=4.4cm,
            bottom=4.4cm,
            outer=3.5cm,
            inner=3.5cm,
            heightrounded
          ]{geometry}

\begin{document}

\section*{Résumé}

Les travaux présentés dans ce mémoire portent sur la validation de programmes
PHP à travers un nouveau langage de spécification, accompagné de ses outils. Ces
travaux s'articulent selon trois axes majeurs~: langage de spécification,
génération automatique de données de test et génération automatique de tests
unitaires.

La première contribution est Praspel~\acite{EnderlinDGO11}, un nouveau langage
de spécification pour PHP basé sur la programmation par contrat. Praspel
spécifie les données avec des domaines réalistes~: des nouvelles structures
permettant de valider et générer des données. À partir d'un contrat écrit en
Praspel, nous pouvons faire du \inenglish{Contract-based Testing}, c'est~à~dire
que nous exploitons le contrat pour générer automatiquement des tests unitaires.
La précondition est utilisée pour générer des données de test et la
postcondition est utilisée comme oracle. La deuxième contribution concerne la
génération de données de test. Pour les booléens, les entiers et les réels, une
génération aléatoire uniforme est employée~\acite{EnderlinDGO11}. Cette approche
a rapidement montré ses limites pour les tableaux et les chaînes de caractères.
Un solveur de contraintes dédié aux tableaux PHP~\acite{EnderlinGB13} ainsi
qu'un langage de description de grammaires avec un compilateur de compilateurs
$LL(\star)$ et plusieurs algorithmes de génération de
données~\acite{EnderlinDGB12} ont été créés pour répondre à ses limites. Enfin,
la génération d'objet est traitée. La troisième contribution définit des
critères de couverture sur les contrats. Ils servent à fournir des objectifs de
tests. L'ensemble des contributions a été soumis au journal \inenglish{Software
and Systems Modeling}, abrégé SoSyM, pour une édition spéciale sur les méthodes
formelles intégrées.

Toutes ces contributions ont été implémentées et expérimentées dans des outils
industriels.

\paragraph{Mots-clés} Praspel, \inenglish{Contract-based Testing}, géné\-ration
de données, géné\-ration automatique de tests unitaires, PHP.

\bibliographystyle{abbrvnat}
\bibliography{Source/Bibliography}

\end{document}
