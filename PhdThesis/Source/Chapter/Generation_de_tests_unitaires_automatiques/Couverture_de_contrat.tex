\section{Couverture de contrat}
\label{section:test:contract}

Afin de définir des critères de couvertures sur les contrats, nous transformons
ces derniers en graphes. Dans cette partie, nous expliquons comment construire
le graphe d'un contrat Praspel en forme normale (défini dans le
chapitre~\ref{chapter:language} page~\pageref{chapter:language}). Ce graphe
reflète la structure, les déclarations de domaines réalistes et les prédicats
d'un contrat.

Dans toute la suite, nous notons $A$ l'ensemble composé de toutes les
expressions Praspel générées par les entités syntaxiques \grule{expression} et
\grule{exception-identifier} de la grammaire de la
figure~\ref{figure:language:grammar_part2}, et de toutes les expressions de la
forme $\neg t_1 \land \dots \land \neg t_n$ où $t_i$ est généré par l'entité
syntaxique \grule{expression}, pour $1 \leq i \leq n$. Nous considérons deux
familles de graphes~: les {\strong graphes d'expressions} et les {\strong
graphes de contrats}, définis comme suit.

\begin{definition}[Graphe d'expression]

Un {\strong graphe d'expression} est un tuple $(V, D, i, n)$ où $V$ est un
ensemble fini de {\strong sommets}, $D \subseteq V \times A \times V$ est un
ensemble fini d'{\strong arcs} de $V$ dans $V$ annotés par une {\strong
expression} dans $A$, $i \in V$ est le {\strong sommet initial} et $n \in V$ est
le {\strong sommet final}.

\end{definition}

\begin{definition}[Graphe de contrat]

Un {\strong graphe de contrat} est un tuple $(V, D, i, N, E, U)$ où $V$ est un
ensemble fini de {\strong sommets}, $D \subseteq V \times A \times V$ est un
ensemble fini d'{\strong arcs} de $V$ dans $V$ annotés par une expression dans
$A$, $i \in V$ est le {\strong sommet initial}, $N$ est un ensemble de {\strong
sommets finaux normaux}, $E$ est un ensemble de {\strong sommets finaux
exceptionnels} et $U$ est un ensemble de {\strong sommets erreurs}, avec $N
\disjunion E \disjunion U \subseteq V$\footnote{Avec $\disjunion$ représentant
l'union disjointe.}.

\end{definition}

La partie~\ref{subsection:test:expression_graph} transforme n'importe quelle
expression Praspel en un graphe d'expression. Les autres parties définissent la
transformation de tout contrat Praspel en graphe de contrat. Le cas le plus
simple est celui d'un contrat sans comportement, appelé {\strong contrat
atomique}. Il est considéré dans les parties~\ref{subsection:test:atomic_graph}
et \ref{subsection:test:throwable_graph}. Ensuite, les graphes de contrat avec
des comportements sont définis par induction. Le cas simple avec seulement un
comportement est traité dans la partie~\ref{subsection:test:behavior_graph}. Le
pas d'induction considérant un ou plusieurs comportements est traité dans la
partie~\ref{subsection:test:behaviors_graph}. Le cas d'un contrat avec une
clause \adefault (respectivement \ainvariant) est traité dans la
partie~\ref{subsection:test:default_graph} (respectivement
\ref{subsection:test:invariant_graph}). Afin de rester simple, nous allons
seulement considérer des contrats en forme normale.

\subsection{Graphe d'une expression}
\label{subsection:test:expression_graph}

Les expressions Praspel en forme normale sont décrites dans la
figure~\ref{figure:language:grammar_part2}
page~\pageref{figure:language:grammar_part2} par l'entité syntaxique
\grule{expression}. Une expression est une conjonction de déclarations, de
prédicats et de qualifications, respectivement décrit dans la
figure~\ref{figure:language:grammar_part3}
page~\pageref{figure:language:grammar_part3} par les entités syntaxiques
\grule{declaration}, \grule{predicate} et \grule{qualification}. Nous transformons
seulement les deux premières sortes d'expressions car les expressions de sorte
\grule{qualification} ne sont pas pertinentes pour la couverture~: elles
représentent un seul comportement (par exemple \code{a is unique} dans le cas
d'un tableau \code{a}) et sont toujours évaluées.
Les expressions que nous manipulons sont de la forme \code{$d$ and
$p$}, où $d$ (respectivement $p$) est une conjonction d'expressions de sorte
\grule{declaration} (respectivement \grule{predicate}). Son graphe d'expression
est la concaténation par une transition $\varepsilon$ du {\strong graphe des
domaines} obtenu à partir de $d$ et du {\strong graphe des prédicats} obtenu à
partir de $p$. Ces deux constructions sont détaillées ci-dessous.

\subsubsection{Graphe des domaines}

Une conjonction de déclarations a la forme générale \code{$i_1$: $D_1$ and
$\dots$ and $i_m$: $D_m$}, où \code{$D_j = d_{j,1}$ or $\dots$ or $d_{j,k_j}$}
est une disjonction de domaines réalistes, $1 \leq j \leq m$. Nous notons aussi
par $D_j$ la liste $[d_{j,1}, \dots, d_{j,k_j}]$ de domaines réalistes dans la
disjonction. Soit $S = [i_1, \dots, i_m]$ la liste des identifiants déclarés
dans cette expression. La notation $s \in S$ signifie que $s$ est un élément de
la liste $S$. Pour construire le graphe des domaines, nous utilisons la
définition suivante.

\begin{definition}[Graphe des domaines]

Le {\strong graphe des domaines} associé à l'expression \code{$i_1$: $D_1$ and
$\dots$ and $i_m$: $D_m$} est~:
%
$$G = (\{\mathit{dom}\} \union \{v_s \;\vert\; s \in S\}, D, \mathit{dom}, v_{i_m})$$
%
avec $\mathit{dom}$ une étiquette permettant de typer le graphe et~:
%
$$D = \Union_{d \,\in\, D_1} \{(\mathit{dom}, \code{$i_1$: $d$}, v_{i_1})\} \union
      \Union_{2 \,\leq\, j \,\leq\, m}
      \Union_{d \,\in\, D_j} \{(v_{i_{j - 1}}, \code{$i_j$: $d$}, v_{i_{j}})\}$$

\end{definition}

\begin{example}[Expression simple de deux déclarations de domaines réalistes]
\label{example:test:expression_graph1}

Dans l'expression suivante~:
%
$$\code{i: foo() or bar() and j: baz() or qux()}$$
%
nous avons $S = [\code{i}, \code{j}]$, $D_1 = [\code{foo()}, \code{bar()}]$ et
$D_2 = [\code{baz()}, \code{qux()}]$. La première partie de la
figure~\ref{figure:test:expression_graph} présente graphiquement le graphe
associé à l'exemple~\ref{example:test:expression_graph1}.

\end{example}

\begin{figure}

\fig{\textwidth}{!}{Expression_graph.tex}

\caption{\label{figure:test:expression_graph} Graphe d'expressions, composé
d'un graphe des domaines et d'un graphe des prédicats.}

\end{figure}

\subsubsection{Graphe des prédicats}

Un graphe des prédicats est similaire à un graphe des domaines car la forme de
l'expression est très similaire. En effet, trivialement, nous pouvons dire que
l'opérateur \code{and} est remplacé par \code{\&\&} et que \code{or} est
remplacé par \code{||}. Plus formellement, un prédicat Praspel est de la forme
générale \apred{p}, où $p$ est une chaîne de caractères contenant une expression
logique PHP en forme normale conjonctive (\inenglish{Conjunctive Normal Form},
abrégé CNF), c'est~à~dire que $p$ est de la forme \code{$P_1$ and $\dots$ and
$P_m$}, où $P_j = \code{$p_{j,1}$ or $\dots$ or $p_{j,k_j}$}$ est une
disjonction de variables ou de prédicats PHP (fonctions booléennes ou opérations
logiques), $1 \leq j \leq m$. Nous notons aussi par $P_j$ la liste $[p_{j,1},
\dots, p_{j,k_j}]$ de variables ou prédicats PHP déclarés dans la conjonction.
Pour construire le graphe des prédicats, nous utilisons la définition suivante.

\begin{definition}[Graphe des prédicats]

Le {\strong graphe des prédicats} associé à l'expression \apred{'\code{$P_1$ and
$\dots$ and $P_m$}'} est~:
%
$$G = (\{\mathit{pred}\} \union \{v_j \;\vert\; j \in [1; m]\}, D, \mathit{pred}, v_m)$$
%
avec $\mathit{pred}$ une étiquette permettant de typer le graphe et~:
%
$$D = \Union_{p \,\in\, P_1} \{(\mathit{pred}, p, v_1)\} \union
      \Union_{2 \,\leq\, j \,\leq\, m}
      \Union_{p \,\in\, P_j} \{(v_{j - 1}, p, v_j)\}$$

\end{definition}

\begin{example}[Exemple simple d'un prédicat]
\label{example:test:expression_graph2}

Dans l'expression suivante~:
%
$$\apred{\code{'(\$a || \$b) \&\& (\$c || \$d)'}}$$
%
nous avons $m = 2$, $P_1 = [\code{\$a}, \code{\$b}]$ et $P_2 = [\code{\$c},
\code{\$d}]$. La deuxième partie de la
figure~\ref{figure:test:expression_graph} présente graphiquement le graphe
associé à l'exemple~\ref{example:test:expression_graph2}.

\end{example}

\subsection{Graphe d'un contrat atomique}
\label{subsection:test:atomic_graph}

Un {\strong contrat atomique} est un contrat composé uniquement des clauses
\arequires, \aensures et \athrowable. Il ne contient ni clauses \abehavior, ni
\adefault, c'est~à~dire aucun sous-contrat, et n'est par conséquent pas
décomposable en sous-contrats. Un contrat atomique en forme normale adopte
la forme générale suivante~:
%
\begin{pre}
\arequires  \(P\); \\
\aensures   \(Q\); \\
\athrowable \(T\sb{C}\) with \(T\sb{E}\);
\end{pre}
%
où $P$ et $Q$ sont des expressions de sorte \grule{expression} respectivement
associées aux clauses \arequires et \aensures. L'expression de sorte
\grule{exceptional-expression} associée à la clause \athrowable est composée de
deux parties~: $T_C$ représente le nom de la classe de l'exception et
l'identifiant qui porte l'exception, soit une construction de sorte
\grule{exception-identifier}, et $T_E$ représente la postcondition
exceptionnelle, qui est aussi une expression de sorte \grule{expression}. Le
graphe de contrat $G$ associé à ce contrat atomique est~:


$$G = (
  \{v_1, v_2, v_3, v_4, v_5, v_6\},
  D,
  v_1,
  \{v_3\},
  \{v_5\},
  \{v_6\}
)$$
%
où
%
$$D = \{
  (v_1, P, v_2),
  (v_2, Q, v_3),
  (v_2, T_C, v_4),
  (v_4, T_E, v_5),
  (v_2, \neg T_C, v_6)
\}$$

\begin{figure}

\fig{10cm}{!}{Atomic_graph.tex}

\caption{\label{figure:test:atomic_graph} Graphe correspondant à un contrat
atomique.}

\end{figure}

\begin{example}[Graphe d'un contrat atomique]

La figure~\ref{figure:test:atomic_graph} présente graphiquement ce graphe. Dans
cette figure ainsi que dans toutes celles qui vont suivre, les sommets normaux
finaux sont représentés par un double cercle, les sommets finaux
exceptionnels par un double rectangle et les sommets d'erreurs par un cercle
avec des tirets.

\end{example}

\subsection{Contrat atomique avec plusieurs clauses \athrowable}
\label{subsection:test:throwable_graph}

La partie~\ref{subsection:test:atomic_graph} considérait un contrat avec un
seule clause \athrowable. Dans cette partie, nous voyons comment un contrat est
modifié quand nous ajoutons une clause \code{\athrowable $T_C$ with $T_E$} à un
contrat dont le graphe est $G = (V, D, i, N, E, U)$. Nous pouvons montrer par
récurrence sur le nombre de clauses \athrowable que l'ensemble des sommets $V$
contient au moins les six sommets $v_1 = i$, $v_2$, $v_3$, $v_4$, $v_5$ et $v_6$
introduits dans la partie~\ref{subsection:test:atomic_graph}.

Le graphe résultant est $G' = (V \union \{v'\}, D', i, N, E, U)$ avec un nouveau
sommet $v' \notin V$ et un ensemble d'arcs~:
%
\begin{align*}
%
D' =\;
  & D \union \{(v_2, T_C, v'), (v', T_E, v_5)\} \\
  & - \{(v_2, \varphi, v_6) \,\vert\, (v_2, \varphi, v_6) \in D\} \union
      \{(v_2, \varphi \land \neg T_C, v_6) \,\vert\, (v_2, \varphi, v_6) \in D\}
%
\end{align*}
%
avec une expression $\varphi \in A$ quelconque.

\begin{figure}

\fig{11.5cm}{!}{Atomic_graph_two_throwables.tex}

\caption{\label{figure:test:throwable_graph} Graphe correspondant à un contrat
atomique avec deux clauses \athrowable.}

\end{figure}

\begin{example}[Graphe d'un contrat atomique avec deux clauses \athrowable]

La figure~\ref{figure:test:throwable_graph} présente graphiquement ce graphe
quand il y a deux clauses~: \code{\athrowable $T_{C_1}$ with $T_{E_1}$} et
\code{\athrowable $T_{C_2}$ with $T_{E_2}$} (dans la figure, $v'$ est noté
$v_7$).

\end{example}

\subsection{Graphe d'un contrat avec une clause \abehavior}
\label{subsection:test:behavior_graph}

Considérons à présent le cas d'un contrat contenant exactement une clause
\abehavior. Soit $C$ le contrat~:
%
\begin{pre}
\arequires \(P\); \\
\abehavior \(\alpha\) \{ \(B\) \}
\end{pre}
%
où $B$ est le contrat du comportement $\alpha$. Soit $G_\alpha = (V_\alpha,
D_\alpha, i_\alpha, N_\alpha, E_\alpha, U_\alpha)$ le graphe de contrat associé
à $B$. Alors, le graphe de contrat associé au contrat $C$ est~:
%
$$G = (
  \{v\} \union V_\alpha,
  \{(v, P, i_\alpha)\} \union D_\alpha,
  v,
  \{i_\alpha\} \union N_\alpha,
  E_\alpha,
  U_\alpha
)$$
%
avec un nouveau sommet $v \notin V_\alpha$. Ce graphe est composé de tous les
sommets et arcs de $G_\alpha$, complété avec un nouvel arc entre $v$, le nouveau
sommet initial, et $i_\alpha$, le sommet initial de $G_\alpha$. Le sommet
$i_\alpha$ devient également un sommet final normal car la sémantique du contrat
est qu'il n'est pas obligatoire d'emprunter un comportement. Par conséquent, $P$
peut être évalué mais pas nécessairement $P_\alpha$.

\begin{figure}

\fig{13cm}{!}{Graph_one_behavior.tex}

\caption{\label{figure:test:behavior_graph} Graphe correspondant à un contrat
avec une seule clause \abehavior.}

\end{figure}

\begin{example}[Graphe d'un contrat avec une clause \abehavior]

La figure~\ref{figure:test:behavior_graph} présente graphiquement le graphe du
contrat~:
%
\begin{pre}
\arequires \(P\); \\
\abehavior \(\alpha\) \{ \\
    \arequires  \(P\sb{\alpha}\); \\
    \aensures   \(Q\sb{\alpha}\); \\
    \athrowable \(T\sb{C\sb{\alpha}}\) with \(T\sb{E\sb{\alpha}}\); \\
\}
\end{pre}
%
Le comportement est atomique et contient une seule clause \athrowable. Dans la
figure, les sommets sont renumérotés à partir de $v_1$.

\end{example}

\subsection{Contrat avec plusieurs clauses \abehavior}
\label{subsection:test:behaviors_graph}

La partie~\ref{subsection:test:behavior_graph} considérait un contrat avec une
seule clause \abehavior. Dans cette partie, nous voyons comment ajouter une
clause \abehavior~$\beta$ à la liste des comportements dont le graphe de contrat
est $G = (V, D, i, N, E, U)$. Soit $G_\beta = (V_\beta, D_\beta, i_\beta,
N_\beta, E_\beta, U_\beta)$ le graphe de contrat associé au comportement
$\beta$. {\em Modulo} quelques renommages, supposons que $V$ contient un sommet
$v$ tel que $D$ contient un arc $(i, P, v)$. Supposons aussi que tous les
sommets de $G_\beta$ sont distincts de ceux de $G$ ($V \intersection V_\beta =
\emptyset$). Alors, le nouveau graphe de contrat est $G' = (V \union V_\beta -
\{i_\beta\}, D \union D'_\beta, i, N \union N'_\beta, E \union E_\beta, U \union
U_\beta)$, où $D'_\beta$ (respectivement $N'_\beta$) est obtenu de $D_\beta$
(respectivement $N_\beta$) en remplaçant $i_\beta$ par $v$.

\begin{figure}

\fig{13cm}{!}{Graph_two_behaviors.tex}

\caption{\label{figure:test:graph_two_behaviors.tex} Graphe correspondant à
un contrat avec deux clauses \abehavior.}

\end{figure}

\begin{example}[Graphe d'un contrat avec deux clauses \abehavior]

La figure~\ref{figure:test:graph_two_behaviors.tex} présente graphiquement le
graphe du contrat~:
%
\begin{pre}
\arequires \(P\); \\
\abehavior \(\alpha\) \{ \\
    \arequires  \(P\sb{\alpha}\); \\
    \aensures   \(Q\sb{\alpha}\); \\
    \athrowable \(T\sb{C\sb{\alpha}}\) with \(T\sb{E\sb{\alpha}}\); \\
\} \\
\abehavior \(\beta\) \{ \\
    \arequires  \(P\sb{\beta}\); \\
    \aensures   \(Q\sb{\beta}\); \\
    \athrowable \(T\sb{C\sb{\beta}}\) with \(T\sb{E\sb{\beta}}\); \\
\}
\end{pre}
%
Les sous-contrats des deux comportements sont atomiques et ne contiennent chacun
qu'une seule clause \athrowable.

\end{example}

\subsection{Cas de la clause \adefault}
\label{subsection:test:default_graph}

Après un groupe de clauses \abehavior, nous pouvons avoir une clause \adefault.
Rappelons qu'une clause \adefault est strictement équivalente à une clause
\abehavior avec une clause \arequires implicite (voir la
partie~\ref{subsection:language:clauses}
page~\pageref{subsection:language:clauses}). Ainsi, ajouter une clause \adefault
est un cas spécial d'ajout d'une clause \abehavior, considéré dans la
partie~\ref{subsection:test:behaviors_graph}.

\subsection{Cas de la clause \ainvariant}
\label{subsection:test:invariant_graph}

Dans cette dernière partie, nous voyons comment intégrer une clause
\ainvariant~$I$ (située au niveau des attributs de classe) dans le graphe de
contrat d'une méthode $G_M = (V_M, D_M, i_M, N_M, E_M, U_M)$. Nous supposons que
l'ensemble $V_M$ ne contient pas les sommets $v_1$, $v_2$ et $v_3$. Le graphe
résultant est~:
%
$$G = (
  \{v_1, v_2, v_3\} \union V_M,
  D,
  v_1,
  \{v_2\},
  \{v_3\},
  U_M
)$$
%
avec
%
$$D = \{(v_1, I, i_M)\} \union
      \{(n, I, v_2) \,\vert\, n \in N_M\} \union
      \{(e, I, v_3) \,\vert\, e \in E_M\}$$

Le graphe $G$ contient trois sommets supplémentaires et plusieurs nouveaux arcs,
dont la signification est que l'invariant $I$ doit être satisfait avant et après
l'exécution de la méthode, et dans tous les cas, que ce soit une terminaison
normale ou exceptionnelle.

\begin{figure}

\fig{13cm}{!}{Graph_invariant.tex}

\caption{\label{figure:test:invariant_graph} Graphe correspondant à contrat avec
un invariant.}

\end{figure}

\begin{example}[Graphe d'un contrat d'une clause \ainvariant]

La figure~\ref{figure:test:invariant_graph} illustre ce principe d'extension
d'un graphe de contrat avec un invariant. Dans un souci de simplicité, elle
représente seulement un sommet final normal $n_M \in N_M$ et un sommet final
exceptionnel $e_M \in E_M$.

\end{example}
