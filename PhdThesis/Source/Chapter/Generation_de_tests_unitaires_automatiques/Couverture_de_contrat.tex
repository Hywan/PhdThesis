\section{Couverture de contrat}
\label{section:test:contract}

Dans cette section, nous expliquons comment construire le graphe d'un contrat
Praspel en forme normale. Ce graphe reflète les clauses du contrat à travers
les déclarations de domaines réalistes et les prédicats.

Dans la suite, nous notons $A$ l'ensemble composé de toutes les expressions
Praspel générées par les entités syntaxiques \grule{expression} et
\grule{exception-identifier} de la grammaire de la
figure~\ref{figure:language:grammar_part2}, et complété par toutes les
expressions de la forme $\neg t_1 \land \dots \land \neg t_n$ où $t_i$ est
généré par l'entité syntaxique \grule{exception-identifier}, pour $1 \leq i \leq
n$. Nous considérons deux familles de graphes, à savoir {\strong graphes
d'expressions} et {\strong graphes de contrats}, définis comme suit.

\begin{definition}[Graphe d'expression]

Un {\strong graphe d'expression} est un tuple $(V, D, i, n)$ où $V$ est
l'ensemble fini des {\strong sommets}, $D \subseteq V \times A \times V$ est
l'ensemble fini des {\strong arcs} de $V$ dans $V$ annotés par une {\strong
expression} dans $A$, $i \in V$ est le {\strong sommet initial}, $n \in V$ est
le {\strong sommet final normal}.

\end{definition}

\begin{definition}[Graphe de contrat]

Un {\strong graphe de contrat} est un tuple $(V, D, i, N, E, U)$ où $V$ est
l'ensemble fini des {\strong sommets}, $D \subseteq V \times A \times V$ est
l'ensemble fini des {\strong arcs} de $V$ dans $V$ annotés par une expression
dans $A$, $i \in V$ est le {\strong sommet initial}, $N$ est l'ensemble des
{\strong sommets finaux normaux}, $E$ est l'ensemble des {\strong sommets finaux
exceptionnels} et $U$ est l'ensemble des {\strong sommets pièges}/d'{\strong
erreurs}, avec $N \disjunion E \disjunion U \subseteq V$.

\end{definition}

La section~\ref{subsection:test:expression_graph} transforme n'importe quelle
expression Praspel en un graphe d'expression. Les autres sections définissent la
transformation de tout contrat Praspel en graphe de contrat. Le cas le plus
simple est celui d'un contrat sans comportement, appelé {\strong contrat
atomique}. Ils sont considérés dans les
sections~\ref{subsection:test:atomic_graph} et
\ref{subsection:test:throwable_graph}. Ensuite, les graphes de contrat avec des
comportements sont définis par induction. Le cas simple avec seulement un
comportement est traité dans la section~\ref{subsection:test:behavior_graph}. Le
pas d'induction considérant un ou plusieurs comportements est traité dans la
section~\ref{subsection:test:behaviors_graph}. Le cas d'un contrat avec une
clause \adefault (respectivement \ainvariant) est traité dans la
section~\ref{subsection:test:default_graph} (respectivement
\ref{subsection:test:invariant_graph}). Afin de rester simple, nous allons
seulement considérer des contrats en forme normale.

\subsection{Graphe d'une expression}
\label{subsection:test:expression_graph}

Les expressions Praspel sont décrites dans la
figure~\ref{figure:language:grammar_part2} par l'entité syntaxique
\grule{expression}. Une expression est une conjonction de déclarations, de
prédicats et de contraintes, respectivement décritent par les entités
syntaxiques \grule{declaration}, \grule{predicate} et \grule{constraint} dans la
figure~\ref{figure:language:grammar_part3}. Nous transformons seulement les deux
premières sortes d'expressions car les expressions de type \grule{constraint}
ne sont pas pertinentes pour la couverture~: elles représentent un seul
comportement (par exemple, \grule{constraint} et \grule{declaration}
apparaissent toujours ensembles) et sont toujours exécutées.

Une expression sans contrainte en forme normale est de la forme \code{$d$ and
$p$}, où $d$ (respectivement $p$) est une conjonction d'expressions de type
\grule{declaration} (respectivement \grule{predicate}). Son graphe d'expression
est la concaténation par une transition $\varepsilon$ du {\strong graphe des
domaines} obtenu à partir de $d$ et du {\strong graphe des prédicats} obtenu à
partir de $p$. Les deux constructions sont détaillées ci-dessous.

Une conjonction de déclarations a la forme générale \code{$i_1$: $D_1$ and
$\dots$ and $i_m$: $D_m$}, où \code{$D_j = d_{j,1}$ or $\dots$ or $d_{j,k}$} est
une disjonction de domaines réalistes ($1 \leq j \leq m$). Nous notons aussi par
$D_j$ la liste $[d_{j,1}, \dots, d_{j,k}]$ de domaines réalistes dans la
disjonction. Soit $S = [i_1, \dots, i_m]$ la liste des identifiants déclarés
dans cette expression. Soit $\mathrm{last}(S)$ représentant le dernier élément
de la liste $S$. Par exemple, pour l'expression:
%
$$\code{i: foo() or bar() and j: baz() or qux()}$$
%
nous avons $S = [\code{i}, \code{j}]$, $D_1 = [\code{foo()}, \code{bar()}]$ et
$D_2 = [\code{baz()}, \code{qux()}]$.

Le {\strong graphe des domaines} associé à l'expression \code{$i_1$: $D_1$ and
$\dots$ and $i_m$: $D_m$} est $G = (\{\mathit{dom}\} \union \{v_s \vert \in S\},
D, \mathit{dom}, v_{\mathrm{last}(S)})$ où~:
%
$$D = \Union_{d \,\in\, D_1} \{(\mathit{dom}, \code{$i_1$: $d$}, v_{i_1})\} \union
      \Union_{2 \,\leq\, j \,\leq\, m}
      \Union_{d \,\in\, D_j} \{(v_{i_{j - 1}}, \code{$i_j$: $d$}, v_{i_{j}})\}$$
%
Un prédicat est de la forme \apred{p}, où $p$ est une chaîne de caractères
contenant un prédicat PHP en forme normale disjonctive (DNF). Nous lui associons
un {\strong graphe de prédicat}. La définition est similaire à celle des graphes
de domaines, après avoir remplacé \code{\&\&} par \code{and} et \code{||} par
\code{or}, et annoté les disjonctions pour avoir des sommets numérotés. Le
sommet initial n'est plus $\mathit{dom}$ mais $\mathit{pred}$. Par exemple,
\apred{\code{'(\$a || \$b) \&\& (\$c || \$d)'}} devient \code{$(1~:)$ \$a or \$b
and $(2~:)$ \$c and \$d}.

\begin{figure}

\fig{\textwidth}{!}{Expression_graph.tex}

\caption{\label{figure:test:expression_graph} Un graphe d'expressions, composé
d'un graphe de domaine et d'un graphe de prédicat.}

\end{figure}

La figure~\ref{figure:test:expression_graph} présente graphiquement le graphe
associé au dernier exemple.

\subsection{Graphe d'un contrat atomique}
\label{subsection:test:atomic_graph}

Un {\strong contrat atomique} est un contrat composé uniquement des clauses
\arequires, \aensures et \athrowable. Il ne contient ni clauses \abehavior, ni
\adefault, c'est~à~dire aucun sous-contrat, et n'est par conséquent pas
décomposable en sous-contrat. Un contrat atomique en forme normale adopte
la forme générale suivante~:
%
\begin{pre}
\arequires  \(R\); \\
\aensures   \(E\); \\
\athrowable \(T\sb{C}\) with \(T\sb{E}\);
\end{pre}
%
où $R$ et $E$ sont des expressions de type \grule{expressions} respectivement
associées aux clauses \arequires et \aensures. L'expression de type
\grule{exceptional-expression} associée à la clause \athrowable est composée de
deux parties~: $T_C$ représente le nom de la classe de l'exception et
l'identifiant qui porte l'exception, et $T_E$ représente la postcondition
exceptionnelle, qui est aussi une expression de type \grule{expression}. Le
graphe de contrat associé à ce contrat atomique est~:

$$G = (
  \{v_1, v_2, v_3, v_4, v_5, v_6\},
  D,
  v_1,
  \{v_3\},
  \{v_5\},
  \{v_6\}
)$$
%
où
%
$$D = \{
  (v_1, R, v_2),
  (v_2, E, v_3),
  (v_2, T_C, v_4),
  (v_4, T_E, v_5),
  (v_2, \neg T_C, v_6)
\}$$

\begin{figure}

\fig{10cm}{!}{Atomic_graph.tex}

\caption{\label{figure:test:atomic_graph} Graphe correspondant à un contrat
atomique.}

\end{figure}

La figure~\ref{figure:test:atomic_graph} présente graphiquement ce graphe. Dans
cette figure ainsi que dans toutes celles qui vont suivre, le sommet normal
final est représenté par un double cercle, et le sommet final exceptionnel par
un double rectangle.

\subsection{Extension du graphe d'un contrat atomique avec une ou plusieurs
clauses \athrowable}
\label{subsection:test:throwable_graph}

La section~\ref{subsection:test:atomic_graph} considérait un contrat avec un
seule clause \athrowable. Dans cette section, nous voyons comment un contrat est
modifié quand nous ajoutons une ou plusieurs clauses \code{\athrowable $T_C$
with $T_E$} à un contrat dont le graphe est $G = (V, D, i, N, E, U)$. Par
induction, supposons que l'ensemble des sommets $V$ contient au moins les six
sommets $v_1 = i$, $v_2$, $v_3$, $v_4$, $v_5$, $v_6$ introduits dans la
section~\ref{subsection:test:atomic_graph}.

Le graphe résultant est $G' = (V \union \{v'\}, D', i, N, E, U)$ avec le nouveau
sommet $v' \notin V$, et~:
%
\begin{equation*}
\centering
\begin{aligned}
D =\; & D \union \{(v_2, T_C, v'), (v', T_E, v_5)\} \\
    & - \{(v_2, \varphi, v_6) \,\vert\, (v_2, \varphi, v_6) \in D\} \union
      \{(v_2, \varphi \land \neg T_C, v_6) \,\vert\, (v_2, \varphi, v_6) \in D\}
\end{aligned}
\end{equation*}

\begin{figure}

\fig{11cm}{!}{Atomic_graph_two_throwables.tex}

\caption{\label{figure:test:throwable_graph} Graphe correspondant à un contrat
atomique avec deux clauses \athrowable.}

\end{figure}

La figure~\ref{figure:test:throwable_graph} présente graphiquement ce graphe
quand il y a deux clauses \athrowable (dans la figure, $v'$ est noté $v_7$).

\subsection{Graphe d'un contrat avec une clause \abehavior}
\label{subsection:test:behavior_graph}

Maintenant, nous considérons le cas d'un contrat contenant exactement une clause
\abehavior. Soit $C$ le contrat~:
%
\begin{pre}
\arequires \(R\); \\
\abehavior \(\alpha\) \{ \(B\) \}
\end{pre}
%
où $B$ est le contrat du comportement $\alpha$. Soit $G_\alpha = (V_\alpha,
D_\alpha, i_\alpha, N_\alpha, E_\alpha, U_\alpha)$ le graphe de contrat associé
à $B$. Alors, le graphe de contrat associé au contrat $C$ est~:
%
$$G = (
  \{v\} \union V_\alpha,
  \{(v, R, i_\alpha)\} \union D_\alpha,
  v,
  \{i_\alpha\} \union N_\alpha,
  E_\alpha,
  U_\alpha
)$$
%
avec le nouveau sommet $v \notin V_\alpha$. Ce graphe est composé de tous les
sommets et arcs de $G_\alpha$, complété avec le nouvel arc.

\begin{figure}

\fig{\textwidth}{!}{One_behavior_graph.tex}

\caption{\label{figure:test:behavior_graph} Le graphe correspondant à un contrat
avec une seule clause \abehavior, qui contient un (sous-)contrat atomique.}

\end{figure}

La figure~\ref{figure:test:behavior_graph} présente graphiquement ce graphe
quand le comportement est atomique et contient seulement une seule clause
\athrowable.  Dans la figure, les sommets sont renumérotés depuis $v_1$.

\subsection{Extension du graphe d'un contrat avec une ou plusieurs clauses
\abehavior}
\label{subsection:test:behaviors_graph}

La section~\ref{subsection:test:behavior_graph} considérait un contrat avec une
seule clause \abehavior. Dans cette section, nous voyons comment ajouter une
clause \abehavior~$\beta$ à la liste des comportements dont le graphe de contrat
est $(V, D, i, N, E, U)$. Soit $G_\beta = (V_\beta, D_\beta, i_\beta, N_\beta,
E_\beta, U_\beta)$ le graphe de contrat associé au comportement $\beta$. Modulo
quelques renommages, supposons que $V$ contient un sommet $v$ tel que $D$
contient un arc $(i, R, v)$ et que tous les sommets de $G_\beta$ sont distincts
de ceux de $G$ ($V \intersection V_\beta = \emptyset$). Alors, le nouveau graphe
de contrat est $G = (V \union V_\beta - \{i_\beta\}, D \union D'_\beta, i, N
\union N'_\beta, E \union E_\beta, U \union U_\beta)$, où $D'_\beta$
(respectivement $N'_\beta$) est obtenu de $D_\beta$ (respectivement $N_\beta$)
en remplaçant $i_\beta$ par $v$.

\subsection{Cas de la clause \adefault}
\label{subsection:test:default_graph}

Après un groupe de clauses \abehavior, nous pouvons avoir une clause \adefault.
Rappelons-nous qu'une clause \adefault est strictement équivalente à une clause
\abehavior avec une clause \arequires implicite (voir la
section~\ref{subsection:language:clauses}). Ainsi, ajouter une clause \adefault
est un cas spécial d'ajout d'une clause \abehavior, considéré dans la
section~\ref{subsection:test:behaviors_graph}.

\subsection{Cas de la clause \ainvariant}
\label{subsection:test:invariant_graph}

Dans cette dernière section, nous voyons comment intégrer une clause
\ainvariant~$I$ (localisée sur les attributs de classe) dans le graphe de
contrat d'une méthode $G_M = (V_M, D_M, i_M, N_M, E_M, U_M)$. Nous supposons que
l'ensemble $V_M$ ne contient pas les sommets $v_1$, $v_2$ et $v_3$. Le graphe
résultant est~:
%
$$G = (
  \{v_1, v_2, v_3\} \union V_M,
  D,
  v_1,
  \{v_2\},
  \{v_3\},
  U_M
)$$
%
avec
%
$$D = \{(v_1, I, i_M)\} \union
      \{(n, I, v_2) \,\vert\, n \in N_M\} \union
      \{(e, I, v_3) \,\vert\, e \in E_M\}$$

Il contient trois sommets supplémentaires et plusieurs nouveaux arcs, dont la
signification est que l'invariant $I$ doit être satisfait avant et après
l'exécution de la méthode, et dans tous les cas, que ce soit une terminaison
normale ou exceptionnelle.

\begin{figure}

\fig{\textwidth}{!}{Invariant_graph.tex}

\caption{\label{figure:test:invariant_graph} Graphe correspondant à contrat avec
un invariant.}

\end{figure}

La figure~\ref{figure:test:invariant_graph} illustre ce principe d'extension
d'un graphe de contrat avec un invariant. Pour des raisons de simplicité, elle
représente seulement un sommet final normal et un sommet final exceptionnel. \\

Nous remarquons que tous les graphes construits sont acycliques.
