\section{Critères de couverture}
\label{section:test:criteria}

Nous définissons maintenant les critères de couverture par des propriétés de
l'ensemble des chemins du graphe associé à un contrat Praspel annotant une
méthode PHP.

Les exemples suivants sont basés sur le contrat de la
figure~\ref{figure:test:irc} pour la méthode \code{compute} avec deux arguments:
\code{server} qui doit être une classe de type
\code{\bslash{}Irc\bslash{}Server} et \code{buffer} qui doit contenir un message
IRC.
%
\begin{figure}

\begin{bigpre}
/** \\
 * \arequires server: class('\(\bslash\)Irc\(\bslash\)Server'); \\
 * \abehavior message \{ \\
 *     \arequires buffer: /^privmessage .+/ or /^message .+/; \\
 *     \aensures  \aresult: 1..; \\
 * \} \\
 * \abehavior ping \{ \\
 *     \arequires buffer: /^ping\$/ and \\
 *               \(\bslash\)pred('\$server->bufferState   >= 0 or \\
 *                      network\_buffer\_state() >  0'); \\
 *     \aensures  \aresult: 1..; \\
 * \} \\
 * \adefault \{ \\
 *     \athrowable \(\bslash\)Irc\(\bslash\)Exception\(\bslash\)MalformedMessage e \\
 *                    with e->code: 400..491; \\
 * \} \\
 */ \\
public static function compute ( \$server, \$buffer ) \{ … \}
\end{bigpre}

\caption{\label{figure:test:irc} Contrat et méthode manipulant un flux IRC.}

\end{figure}
%
\begin{figure}

\fig{\textwidth}{!}{IRC.tex}

\caption{\label{figure:test:irc_graph} Graphe de contrat associé à la figure
\ref{figure:test:irc}.}

\end{figure}
%
Le graphe de contrat associé est présenté dans la
figure~\ref{figure:test:irc_graph}. Tous les états ont été renommés en $v_i$
avec $1 \leq i \leq 22$, pour éviter toute ambiguïté lorsque nous parlerons des
chemins. Les graphes d'expressions y sont représentés. Par conséquent, sont
représentés tous les sentiers. Les routes sont représentées par des arcs en
pointillés. Certaines étiquettes sont tronquées pour réduire la taille du
graphe. Dans cet exemple, nous avons deux comportements, un pour un message
privé ou public (représenté par une expression régulière à travers le domaine
réaliste \code{regex} sous sa forme simplifiée) et un pour un «~ping~». Dans le
dernier comportement, la précondition a un prédicat pour spécifier que le réseau
est dans un état spécifique. Enfin, le comportement par défaut spécifie qu'une
exception de type \code{\bslash{}Irc\bslash{}Exception\bslash{}MalformedMessage}
doit être levée si le tampon n'est ni un message ni un ping. De plus, le code
d'une exception doit être un entier entre 400 et 491.

\subsection{Critère de Clause}

Le Critère de Clause vise à couvrir la structure d'un contrat. Ainsi, le Critère
de Clause est satisfait par une suite de tests $S$ d'une méthode \code{f} de
graphe de contrat $G(\code{f}) = (V, D, i, N, E, U)$ si tous les états de $V
\bslash U$ sont dans une route de $R_\code{f}(S)$.

\begin{example}[Suite de tests satisfaisant le Critère de Clause]

La syntaxe \code{object($c$)} représente une instance valide d'une classe $c$.
La suite de tests $\{t_1, t_2, t_3\}$ avec~:
%
\begin{itemize}

\item[$t_1.$]
%
\code{compute(object(\bslash{}Irc\bslash{}Server), 'message foobar')}~;

\item[$t_2.$]
%
\code{compute(object(\bslash{}Irc\bslash{}Server), 'ping')}~;

\item[$t_3.$]
%
\code{compute(object(\bslash{}Irc\bslash{}Server), 'xyz')}.

\end{itemize}
%
satisfait le Critère de Couverture de Clause, c'est~à~dire que ces cas de tests
couvrent toutes les clauses avec les chemins respectifs suivants (sur les
routes)~:
%
\begin{itemize}

\item[$R_\code{compute}(\{t_1\}).$]
%
$v_1 \cdot v_2 \cdot v_5 \cdot v_8$~;

\item[$R_\code{compute}(\{t_2\}).$]
%
$v_1 \cdot v_2 \cdot v_{13} \cdot v_{16}$~;

\item[$R_\code{compute}(\{t_3\}).$]
%
$v_1 \cdot v_3 \cdot v_{17} \cdot v_{21}$.

\end{itemize}

\end{example}

\subsection{Critère de Domaine}

Le Critère de Domaine raffine le Critère de Clause en considérant la déclaration
de domaines réalistes à l'intérieur des graphes d'expression. Plus formellement,
le Critère de Domaine est satisfait par une suite de tests $S$ pour une méthode
\code{f} si toutes les transitions de la forme \code{$i$: $d$} dans le graphe de
contrat $G(\code{f})$ (qui sont dans les graphes d'expressions étiquetant les
transitions de $G(\code{f})$ et contenant le nœud $\mathit{dom}$) apparaissent
dans au moins un sentier de $T_\code{f}(S)$.

\begin{example}[Suite de tests satisfaisant le Critère de Domaine]

La suite de tests $\{t_1, t_2, t_4\}$ avec~:
%
\begin{itemize}

\item[$t_4.$]
%
\code{compute(object(\bslash{}Irc\bslash{}Server), 'privmessage foobar')}.

\end{itemize}
%
satisfait le Critère de Couverture de Domaine: les deux déclarations de domaines
réalistes de la variable \code{buffer} sont couverts. Cette suite de test
produit les chemins respectifs suivants (sur les sentiers)~:
%
\begin{itemize}

\item[$T_\code{compute}(\{t_1\}).$]
%
$v_1 \cdot v_2 \cdot v_3 \cdot v_4 \cdot v_5 \cdot v_6 \cdot v_7 \cdot v_8$, \\
avec la transition $(v_3, \code{/\^\empty message .+/}, v_4)$ d'activée~;

\item[$T_\code{compute}(\{t_2\}).$]
%
$v_1 \cdot v_2 \cdot v_9 \cdot v_{10} \cdot v_{11} \cdot v_{12} \cdot v_{13}
\cdot v_{14} \cdot v_{15} \cdot v_{16}$~;

\item[$T_\code{compute}(\{t_4\}).$]
%
$v_1 \cdot v_2 \cdot v_3 \cdot v_4 \cdot v_5 \cdot v_6 \cdot v_7 \cdot v_8$, \\
avec la transition $(v_3, \code{/\^\empty privmessage .+/}, v_4)$ d'activée.

\end{itemize}

\end{example}

\subsection{Critère de Prédicat}

Similairement au Critère de Domaine, le Critère de Prédicat s'applique aux
prédicats dans les graphes d'expression, et vise à tous les couvrir. Plus
formellement, le Critère de Prédicat est satisfait par une suite de tests $S$
pour une méthode \code{f} si toutes les transitions qui ne sont pas de la forme
\code{$i$: $d$} dans le graphe de contrat $G(\code{f})$ (qui sont dans les
graphes d'expressions étiquetant les transitions de $G(\code{f})$ et contenant
le nœud $\mathit{pred}$) apparaissent dans au moins un sentier de
$T_\code{f}(S)$.

\begin{example}[Suite de tests satisfaisant le Critère de Prédicat]

La suite de tests $\{t_5, t_6\}$ avec~:
%
\begin{itemize}

\item[$t_5.$] \code{compute(object(\bslash{}Irc\bslash{}Server), 'ping')} \\
avec \apred{\code{'\$server->bufferState >= 0'}}~;

\item[$t_6.$] \code{compute(object(\bslash{}Irc\bslash{}Server), 'ping')} \\
avec \apred{\code{'network\_buffer\_state() > 0'}}.

\end{itemize}
%
satisfait le Critère de Couverture de Prédicat. Cette suite de test produit les
chemins respectifs suivants (sur les sentiers)~:
%
\begin{itemize}

\item[$T_\code{compute}(\{t_5\}).$]
%
$v_1 \cdot v_2 \cdot v_9 \cdot v_{10} \cdot v_{11} \cdot v_{11} \cdot v_{12}
\cdot v_{13} \cdot v_{14} \cdot v_{15} \cdot v_{16}$ \\
avec la transition $(v_{11}, \code{\$server->bufferState >= 0}, v_{12})$
d'activée~;

\item[$T_\code{compute}(\{t_6\}).$]
%
$v_1 \cdot v_2 \cdot v_9 \cdot v_{10} \cdot v_{11} \cdot v_{11} \cdot v_{12}
\cdot v_{13} \cdot v_{14} \cdot v_{15} \cdot v_{16}$ \\
avec la transition $(v_{11}, \code{network\_buffer\_state() > 0}, v_{12})$
d'activée.

\end{itemize}

\end{example}

\subsection{Combinaisons}
\label{subsection:test:combination}

Avoir de tels critères de couverture n'est pas suffisant pour produire ou
qualifier des cas de test {\em réalistes}. En effet, couvrir toutes les clauses
d'un contrat n'assure pas que tous les domaines réalistes aient été couverts. À
l'inverse, savoir que tous les domaines aient été couverts n'assure pas que
toutes les clauses du contrat aient été exploitées. Ainsi, afin d'avoir des cas
de test plus réalistes, ces critères de couverture de contrat peuvent être
combinés ensemble.  Nous proposons les combinaisons suivantes~:
%
\begin{itemize}

\item le Critère Clause~+~Domaine vise à couvrir la structure et les domaines
réalistes d'un contrat~;

\item le Critère Clause~+~Prédicat vise à couvrir la structure et les prédicats
d'un contrat~;

\item le Critère \inenglish{All}-$G$ vise à satisfaire les critères Clause,
Domaine et Prédicat en même temps.

\end{itemize}
%
Ce {\strong treillis} de critères de couverture de contrats est résumé dans la
figure~\ref{figure:test:lattice}.

\begin{figure}

\fig{11cm}{!}{Criteria.tex}

\caption{\label{figure:test:lattice} Hiérarchie des critères. $A \rightarrow B$
signifie que si le critère $A$ est satisfait, alors $B$ est également
satisfait.}

\end{figure}

\begin{example}[Suite de tests satisfaisant les combinaisons des Critères de
Couverture de Contrat]

La suite de tests $\{t_1, t_2, t_3, t_4\}$ satisfait le Critère de Couverture
Clause~+~Domaine. La suite de tests $\{t_1, t_3, t_5, t_6\}$ satisfait le
Critère de Couverture Clause~+~Prédicat. La suite de tests $\{t_1, t_3, t_4,
t_5, t_6\}$ satisfait le Critère de Couverture \inenglish{All}-$G$

\end{example}
