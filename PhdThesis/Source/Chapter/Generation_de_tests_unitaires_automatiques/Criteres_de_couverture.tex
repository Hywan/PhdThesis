\section{Critères de couverture}
\label{section:test:criteria}

Nous définissons maintenant les critères de couverture pour les contrats par des
propriétés de l'ensemble des chemins du graphe associé à un contrat Praspel
annotant une méthode PHP. Mais tout d'abord, nous rappelons les définitions d'un
chemin, d'un cas de test et d'une suite de tests pour notre contexte.

Tout au long de cette partie, nous considérons une méthode PHP \code{f} et nous
désignons par $G(\code{f})$ ou $G$ son contrat $(V, D, i, N, E, U)$, comme
défini dans la partie~\ref{section:test:contract}.

\subsection{Critère de Clause}

Le Critère de Clause vise à couvrir la structure d'un contrat. Ainsi, le Critère
de Clause est satisfait par une suite de tests $S$ d'une méthode \code{f} de
graphe de contrat $G(\code{f}) = (V, D, i, N, E, U)$ si tous les états de $V
\bslash U$ sont dans une route de $R_\code{f}(S)$.

\begin{example}[Suite de tests satisfaisant le Critère de Clause]

Les cas de test suivants couvrent toutes les clauses. La syntaxe
\code{object($c$)} représente une instance valide d'une classe $c$. La suite de
tests $\{t_1, t_2, t_3\}$ avec~:
%
\begin{itemize}

\item[$(t_1)$] \code{compute(object(\bslash{}Irc\bslash{}Server), 'message foobar')}~;

\item[$(t_2)$] \code{compute(object(\bslash{}Irc\bslash{}Server), 'ping')}~;

\item[$(t_3)$] \code{compute(object(\bslash{}Irc\bslash{}Server), 'xyz')}.

\end{itemize}
%
satisfait le Critère de Couverture de Clause.

\end{example}

\subsection{Critère de Domaine}

Le Critère de Domaine raffine le Critère de Clause en considérant la déclaration
de domaines réalistes à l'intérieur des graphes d'expression. Plus formellement,
le Critère de Domaine est satisfait par une suite de tests $S$ pour une méthode
\code{f} si toutes les transitions de la forme \code{$i$: $d$} dans le graphe de
contrat $G(\code{f})$ (qui sont dans les graphes d'expressions étiquetant les
transitions de $G(\code{f})$ et contenant le nœud $\mathit{dom}$) apparaissent
dans au moins un sentier de $T_\code{f}(S)$.

\begin{example}[Suite de tests satisfaisant le Critère de Domaine]

La suite de tests $\{t_1, t_2, t_4\}$ avec~:
%
\begin{itemize}

\item[$(t_4)$] \code{compute(object(\bslash{}Irc\bslash{}Server), 'privmessage foobar')}.

\end{itemize}
%
satisfait le Critère de Couverture de Domaine: les deux déclarations de domaines
réalistes de la variable \code{buffer} sont couverts.

\end{example}

\subsection{Critère de Prédicat}

Similairement au Critère de Domaine, le Critère de Prédicat s'applique aux
prédicats dans les graphes d'expression, et vise à tous les couvrir. Plus
formellement, le Critère de Prédicat est satisfait par une suite de tests $S$
pour une méthode \code{f} si toutes les transitions qui ne sont pas de la forme
\code{$i$: $d$} dans le graphe de contrat $G(\code{f})$ (qui sont dans les
graphes d'expressions étiquetant les transitions de $G(\code{f})$ et contenant
le nœud $\mathit{pred}$) apparaissent dans au moins un sentier de
$T_\code{f}(S)$.

\begin{example}[Suite de tests satisfaisant le Critère de Prédicat]

La suite de tests $\{t_5, t_6\}$ avec~:
%
\begin{itemize}

\item[$(t_5)$] \code{compute(object(\bslash{}Irc\bslash{}Server), 'ping')} avec \\
\apred{\code{'\$server->bufferState >= 0'}}~;

\item[$(t_6)$] \code{compute(object(\bslash{}Irc\bslash{}Server), 'ping')} avec \\
\apred{\code{'network\_buffer\_state() > 0'}}.

\end{itemize}
%
satisfait le Critère de Couverture de Prédicat.

\end{example}

\subsection{Combinaisons}

Avoir de tels critères de couverture n'est pas suffisant pour produire ou
qualifier des cas de test {\em réalistes}. En effet, couvrir toutes les clauses
d'un contrat n'assure pas que tous les domaines réalistes aient été couverts. À
l'inverse, savoir que tous les domaines ont été couverts n'assure pas que toutes
les clauses du contrat aient été exploitées. Ainsi, afin d'avoir des cas de test
plus réalistes, ces critères de couverture de contrat peuvent être combinés
ensemble.  Nous proposons les combinaisons suivantes~:
%
\begin{itemize}

\item le Critère Clause~+~Domaine vise à couvrir la structure et les domaines
réalistes d'un contrat~;

\item le Critère Clause~+~Prédicat vise à couvrir la structure et les prédicats
d'un contrat~;

\item le Critère \inenglish{All}-$G$ vise à satisfaire les critères Clause,
Domaine et Prédicat en même temps.

\end{itemize}
%
Ce {\strong treillis} de critères de couverture de contrats est résumé dans la
figure~\ref{figure:test:lattice}.

\begin{figure}

\fig{11cm}{!}{Criteria.tex}

\caption{\label{figure:test:lattice} Hiérarchie des critères. $A \rightarrow B$
signifie que si le critère $A$ est satisfait, alors $B$ est également
satisfait.}

\end{figure}

\begin{example}[Suite de tests satisfaisant les combinaisons des Critères de
Couverture de Contrat]

La suite de tests $\{t_1, t_2, t_3, t_4\}$ satisfait le Critère de Couverture
Clause~+~Domaine. La suite de tests $\{t_1, t_3, t_5, t_6\}$ satisfait le
Critère de Couverture Clause~+~Prédicat. La suite de tests $\{t_1, t_3, t_4,
t_5, t_6\}$ satisfait le Critère de Couverture \inenglish{All}-$G$

\end{example}
