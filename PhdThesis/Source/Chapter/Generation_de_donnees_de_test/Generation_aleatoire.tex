\section{Génération pseudo-aléatoire et rejet}
\label{section:data:random}

Commençons par le processus de génération des données dans Praspel, qui se
déroule basiquement en 3~étapes, présents dans la
figure~\ref{figure:data:process} et détaillée ci-après~:

\begin{enumerate}

\item sélection d'une variable~;

\item génération, validation et affectation d'une donnée à cette variable~;

\item sélection de la variable suivante.

\end{enumerate}

La liste des variables $V$ qui nous intéressent sont écrites dans les
préconditions (clause \arequires) ainsi que dans les invariants (clause
\ainvariant). Ces derniers seront traités dans la
partie~\ref{section:data:objects}. Rappelons qu'une précondition est constituée
de déclarations, de prédicats et de contraintes (voir la
figure~\ref{figure:language:grammar_part2}
page~\pageref{figure:language:grammar_part2}). Le processus de génération d'une
donnée est illustré dans la figure~\ref{figure:data:process}. Nous y retrouvons
l'ensemble des variables $V$ avec $v \in V$ une variable. Nous trouvons
également une variable $\tau$ et une constante $\tau_\mathrm{max}$, détaillées
ci-après. Les annotations sur les transitions représentent des gardes pour
activer la transition.
%
\begin{figure}

\fig{!}{9cm}{Generation_process.tex}

\caption{\label{figure:data:process} Processus de génération des données
pseudo-aléatoire.}

\end{figure}

Le premier état de la figure est l'état \code{pick}. Lors d'une génération de
données, nous commençons par traiter les déclarations. Nous choisissons alors
une variable $v$ dans la liste des variables $V$, les unes après les autres
(représenté par l'opération $\mathrm{pop}$ dans la figure) et ce que jusqu'à ce
que $V$ soit non vide, sinon le processus termine. Les variables sont choisies
dans l'ordre décroissant du nombre de contraintes portées~: la variable avec le
plus de contraintes sera la première de la liste.

L'état suivant est \code{sample}. Une variable porte une disjonction de domaines
réalistes, soit $n$ domaine réalistes~:
%
$$\code{$v$: $t_1$(\dots) or \dots or $t_n$(\dots)}$$
%
Nous allons choisir pseudo-aléatoirement le domaine réaliste $t_c$ avec $1 \leq
c \leq n$. Sur ce domaine réaliste $t_c$, nous allons utiliser sa
caractéristique de générabilité, c'est~à~dire appeler sa méthode \code{sample}
pour générer une donnée qui appartient à ce domaine.

Au sein d'un domaine réaliste, quand une donnée est générée, elle est aussitôt
confrontée à sa caractéristique de prédicabilité, c'est~à~dire sa méthode
\code{predicate}, afin de savoir si la donnée est valide ou non. C'est l'état
\code{predicate} de la figure.

Si la précondition a des prédicats (avec la construction \apred{p}), alors ces
derniers seront évalués avec les données fraîchement générées. Si un prédicat a
besoin de plusieurs variables, alors son évaluation sera reportée à plus tard.

Si pendant ces étapes, la propriété de prédicabilité d'un domaine réaliste ou
qu'un prédicat invalide une donnée, il y a un {\strong rejet}. Suite à un rejet,
une donnée est re-générée. Un nombre maximum $\tau_\mathrm{max}$ de
re-générations $\tau$ est fixé afin d'éviter des générations et des rejets en
boucle trop importants et donc trop longs. À chaque génération, $\tau$ est
incrémenté de 1. La valeur de $\tau_\mathrm{max}$ est paramétrable. \\

Mais il n'y a pas que le choix d'un domaine réaliste parmi une disjonction qui
est pseudo-aléatoire. Par défaut, un domaine réaliste génère une donnée de
manière pseudo-aléatoire.  Si nous écrivons \code{boundinteger(7, 42)}, alors un
entier de l'intervalle $[7; 42]$ sera généré pseudo-aléatoirement. Nous avons
précisé dans la partie~\ref{subsection:language:realdom:implementation} qu'une
méthode \code{sample} d'un domaine réaliste reçoit un générateur numérique en
argument. Le seul générateur actuel est pseudo-aléatoire, basé sur l'algorithme
de Mersenne Twister~\acite{MatsumotoN98}. Il permet de générer des entiers et
des réels. Un exemple d'utilisation a été montré dans la
figure~\ref{figure:language:boundinteger}
page~\pageref{figure:language:boundinteger}.

C'est aussi cette méthodologie qu'utilise le domaine réaliste \code{String} pour
générer des chaînes de caractères. Cette approche pseudo-aléatoire pose
rapidement des problèmes. Le nombre de valeurs possibles pour \code{string('a',
'z', 10)}, c'est~à~dire pour une chaîne de 10 caractères entre \code{a} et
\code{z} en minuscules, est de $26^{10}$, soit environ $1.412e^{14}$ valeurs
différentes. La probabilité qu'une de ces valeurs détecte une erreur dans notre
programme est extraimement faible. C'est un problème inhérent à l'approche
pseudo-aléatoire. Cependant, l'ensemble des valeurs générables peut être
considérablement réduit si la donnée est mieux spécifiée~: plus la donnée pourra
être spécifiée avec précision, plus il sera possible de générer des données
pertinentes, c'est~à~dire capables de détecter des erreurs. \\

Dans les parties suivantes, nous allons nous intéresser à mieux spécifier et
mieux générer des tableaux, des chaînes caractères et des objets.
