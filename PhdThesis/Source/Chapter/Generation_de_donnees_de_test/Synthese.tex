\section{Synthèse}
\label{section:data:summary}

Dans ce chapitre, nous avons présenté les différentes techniques de génération
de données utilisées dans Praspel. Nous avons présenté la technique utilisée par
défaut qui est pseudo-aléatoire. Nous avons observé un grand nombre de mauvaises
données générées qui introduisaient du rejet lorsque nous manipulions des
données structurelles ou complexes, comme les tableaux ou les chaînes de
caractères. Nous avons abordé ces problèmes un par un.

Nous avons proposé un solveur de contraintes sur les tableaux. Le langage
Praspel a été étendu pour gérer nativement plusieurs contraintes sur ces
derniers. Nous avons au préalable effectué une étude sur des dizaines de projets
PHP afin de connaître les contraintes les plus populaires sur les tableaux. Ces
travaux ont été publiés dans l'article~\acitei{EnderlinGB13}.

Nous avons également introduit une approche \inenglish{grammar-based testing}
dans Praspel avec les domaines réalistes \code{grammar} et \code{regex}. Ces
derniers permettent de valider et vérifier des données textuelles complexes.
Pour y arriver, nous avons écrit notre propre compilateur de compilateurs
$LL(\star)$ avec son langage de description de grammaire associé, à savoir PP.
Nous avons proposé trois algorithmes de génération de données à partir d'une
grammaire pour différents usages et contextes~: des données variées, aléatoires,
de taille fixe, bornée ou encore libre etc. Ces travaux ont été publiés dans
l'article~\acitei{EnderlinDGB12}. \\

Nous connaissons le language, nous savons comment évaluer le contrat, nous
savons comment générer des données satisfaisants le contrat. Dans le chapitre
suivant, nous allons voir comment générer des tests à partir d'un contrat.
