\section{Génération à partir de solveur pour les tableaux}
\label{section:data:arrays}

\subsection{Étude des contraintes les plus utilisées}

L'idée principale est d'étudier plusieurs projets PHP afin de déterminer quelles
sont les contraintes sur les tableaux les plus utilisées. Pour cela, nous avons
sélectionnés 61~projets PHP sur plusieurs plateformes, comme
Github\footnote{\url{http://github.com/}} ou
Sourceforge\footnote{\url{http://sourceforge.net/}}, à partir de leur
popularité, leur impact sur l'industrie et leur complexité. Tous ces projets
représentent 28~066 fichiers, soit 5~220~547~lignes de code. Dans ces lignes de
code, nous avons compter le nombre d'occurences de chaque fonction de tableaux
disponible dans la bibliothèque standard de PHP. La
Figure~\ref{figure:data:collecting_informations} présente ces résultats~: les
fonctions sont triés par leur nombre d'occurences $n$.
%
\begin{figure}

{
\Huge

\drawfig{\textwidth}{!}{}{

  \tikzset{
     ultra thick/.style={line width=9pt}
  }

  \draw[->] (-.1,   0) -- (46, 0) node[anchor=north] {fonction};
  \draw[->] (  0, -.1) -- (0, 33) node[anchor=east] {$n$};

  \draw (1, 0) node[anchor=north] {\rotatebox{90}{…}};

  \draw[ultra thick] (2, 0) -- (2, 0.245);
  \draw (2, 0) node[anchor=north] {\rotatebox{90}{\code{array\_diff\_key}}};
  \draw[ultra thick] (3, 0) -- (3, 0.2475);
  \draw (3, 0) node[anchor=north] {\rotatebox{90}{\code{array\_fill\_keys}}};
  \draw[ultra thick] (4, 0) -- (4, 0.265);
  \draw (4, 0) node[anchor=north] {\rotatebox{90}{\code{array\_intersect}}};
  \draw[ultra thick] (5, 0) -- (5, 0.2825);
  \draw (5, 0) node[anchor=north] {\rotatebox{90}{\code{array\_change\_key\_case}}};
  \draw[ultra thick] (6, 0) -- (6, 0.295);
  \draw (6, 0) node[anchor=north] {\rotatebox{90}{\code{array\_intersect\_key}}};
  \draw[ultra thick] (7, 0) -- (7, 0.3125);
  \draw (7, 0) node[anchor=north] {\rotatebox{90}{\code{array\_fill}}};
  \draw[ultra thick] (8, 0) -- (8, 0.39);
  \draw (8, 0) node[anchor=north] {\rotatebox{90}{\code{asort}}};
  \draw[ultra thick] (9, 0) -- (9, 0.435);
  \draw (9, 0) node[anchor=north] {\rotatebox{90}{\code{array\_splice}}};
  \draw[ultra thick] (10, 0) -- (10, 0.475);
  \draw (10, 0) node[anchor=north] {\rotatebox{90}{\code{prev}}};
  \draw[ultra thick] (11, 0) -- (11, 0.58);
  \draw (11, 0) node[anchor=north] {\rotatebox{90}{\code{array\_flip}}};
  \draw[ultra thick] (12, 0) -- (12, 0.645);
  \draw (12, 0) node[anchor=north] {\rotatebox{90}{\code{array\_combine}}};
  \draw[ultra thick] (13, 0) -- (13, 0.65);
  \draw (13, 0) node[anchor=north] {\rotatebox{90}{\code{array\_diff}}};
  \draw[ultra thick] (14, 0) -- (14, 0.6525);
  \draw (14, 0) node[anchor=north] {\rotatebox{90}{\code{array\_reverse}}};
  \draw[ultra thick] (15, 0) -- (15, 0.7225);
  \draw (15, 0) node[anchor=north] {\rotatebox{90}{\code{range}}};
  \draw[ultra thick] (16, 0) -- (16, 0.7575);
  \draw (16, 0) node[anchor=north] {\rotatebox{90}{\code{ksort}}};
  \draw[ultra thick] (17, 0) -- (17, 0.7875);
  \draw (17, 0) node[anchor=north] {\rotatebox{90}{\code{array\_filter}}};
  \draw[ultra thick] (18, 0) -- (18, 1.0075);
  \draw (18, 0) node[anchor=north] {\rotatebox{90}{\code{array\_unshift}}};
  \draw[ultra thick] (19, 0) -- (19, 1.0675);
  \draw (19, 0) node[anchor=north] {\rotatebox{90}{\code{array\_search}}};
  \draw[ultra thick] (20, 0) -- (20, 1.285);
  \draw (20, 0) node[anchor=north] {\rotatebox{90}{\code{array\_push}}};
  \draw[ultra thick] (21, 0) -- (21, 1.3425);
  \draw (21, 0) node[anchor=north] {\rotatebox{90}{\code{array\_slice}}};
  \draw[ultra thick] (22, 0) -- (22, 1.5725);
  \draw (22, 0) node[anchor=north] {\rotatebox{90}{\code{array\_unique}}};
  \draw[ultra thick] (23, 0) -- (23, 1.6425);
  \draw (23, 0) node[anchor=north] {\rotatebox{90}{\code{sort}}};
  \draw[ultra thick] (24, 0) -- (24, 1.76);
  \draw (24, 0) node[anchor=north] {\rotatebox{90}{\code{array\_map}}};
  \draw[ultra thick] (25, 0) -- (25, 2.175);
  \draw (25, 0) node[anchor=north] {\rotatebox{90}{\code{array\_values}}};
  \draw[ultra thick] (26, 0) -- (26, 2.2325);
  \draw (26, 0) node[anchor=north] {\rotatebox{90}{\code{sizeof}}};
  \draw[ultra thick] (27, 0) -- (27, 2.3225);
  \draw (27, 0) node[anchor=north] {\rotatebox{90}{\code{compact}}};
  \draw[ultra thick] (28, 0) -- (28, 2.4125);
  \draw (28, 0) node[anchor=north] {\rotatebox{90}{\code{array\_pop}}};
  \draw[ultra thick] (29, 0) -- (29, 2.5225);
  \draw (29, 0) node[anchor=north] {\rotatebox{90}{\code{key}}};
  \draw[ultra thick] (30, 0) -- (30, 2.64);
  \draw (30, 0) node[anchor=north] {\rotatebox{90}{\code{extract}}};
  \draw[ultra thick] (31, 0) -- (31, 2.685);
  \draw (31, 0) node[anchor=north] {\rotatebox{90}{\code{next}}};
  \draw[ultra thick] (32, 0) -- (32, 3.42);
  \draw (32, 0) node[anchor=north] {\rotatebox{90}{\code{array\_shift}}};
  \draw[ultra thick] (33, 0) -- (33, 3.585);
  \draw (33, 0) node[anchor=north] {\rotatebox{90}{\code{reset}}};
  \draw[ultra thick] (34, 0) -- (34, 3.7175);
  \draw (34, 0) node[anchor=north] {\rotatebox{90}{\code{current}}};
  \draw[ultra thick] (35, 0) -- (35, 3.8325);
  \draw (35, 0) node[anchor=north] {\rotatebox{90}{\code{end}}};
  \draw[ultra thick] (36, 0) -- (36, 5.365);
  \draw (36, 0) node[anchor=north] {\rotatebox{90}{\code{array\_keys}}};
  \draw[ultra thick] (37, 0) -- (37, 7.6975);
  \draw (37, 0) node[anchor=north] {\rotatebox{90}{\code{list}}};
  \draw[ultra thick] (38, 0) -- (38, 8.0875);
  \draw (38, 0) node[anchor=north] {\rotatebox{90}{\code{array\_key\_exists}}};
  \draw[ultra thick] (39, 0) -- (39, 8.3475);
  \draw (39, 0) node[anchor=north] {\rotatebox{90}{\code{array\_merge}}};
  \draw[ultra thick] (40, 0) -- (40, 9);
  \draw (40, 0) node[anchor=north] {\rotatebox{90}{\code{each}}};
  \draw[ultra thick] (41, 0) -- (41, 12.025);
  \draw (41, 0) node[anchor=north] {\rotatebox{90}{\code{in\_array}}};
  \draw[ultra thick] (42, 0) -- (42, 19.485);
  \draw (42, 0) node[anchor=north] {\rotatebox{90}{\code{is\_array}}};
  \draw[ultra thick] (43, 0) -- (43, 29.6475);
  \draw (43, 0) node[anchor=north] {\rotatebox{90}{\code{count}}};

  \draw (0, 0) node[anchor=east] {0};
  \draw (0, 2.5) node[anchor=east] {1000};
  \draw (0, 5) node[anchor=east] {2000};
  \draw (0, 7.5) node[anchor=east] {3000};
  \draw (0, 10) node[anchor=east] {4000};
  \draw (0, 12.5) node[anchor=east] {5000};
  \draw (0, 15) node[anchor=east] {6000};
  \draw (0, 17.5) node[anchor=east] {7000};
  \draw (0, 20) node[anchor=east] {8000};
  \draw (0, 22.5) node[anchor=east] {9000};
  \draw (0, 25) node[anchor=east] {10000};
  \draw (0, 27.5) node[anchor=east] {11000};
  \draw (0, 30) node[anchor=east] {12000};

}
}

\caption{\label{figure:data:collecting_informations} Résultats de l'étude de
plusieurs projets PHP pour connaître les contraintes les plus populaires sur les
tableaux.}

\end{figure}
%
Les trois fonctions les plus utilisées sont \code{count()},
\code{array\_key\_exists()} et \code{in\_array()}. La fonction \code{count()}
compte le nombre d'éléments contenus dans un tableau, la fonction
\code{array\_key\_exists()} vérifie si une clé est présente dans un tableau
(indépendamment de la valeur associée, c'est~à~dire que cette fonction retourne
\code{true} même si la valeur est \code{null}), et enfin, la fontion
\code{in\_array()} vérifie si une valeur est présente dans le tableau. Toutes
fonctions travaillent sur un seul tableau à la fois. Cette étude suggère que
nous pouvons considérer ces fonctions booléennes sans effet de bord comme les
{\em conditions} les plus fréquentes sur les tableaux.

\subsection{Conditions sur les tableaux}

Nous étendons la syntaxe d'une déclaration de tableau~:
%
$$\code{a: array($D$, $L$)}$$
%
dans Praspel avec les conditions sur les tableaux suivantes.

Une condition sur une paire est de la forme~:
%
$$\code{a[$K$]: $V$}$$
%
où $K$ et $V$ sont des disjonctions de domaines réalistes. La condition signifie
que les paires contituées de toutes les clés de $K$ et au moins une valeur
dans $V$ sont présentes dans le tableau \code{a}. $K$ accepte seulement les
domaines réalistes qui implémentent les interfaces \code{Constant},
\code{Interval} et \code{Enumerable}. Cette condition est équivalente à utiliser
les fonctions \code{array\_key\_exists()} et \code{in\_array()} combinées.

Si nous voulons exprimer une contrainte seulement sur $K$, nous pouvons utiliser
le symbole \code{\_}. La condition~:
%
$$\code{a[$K$]: \_}$$
%
signifie que toutes les clés de $K$ doivent être présentes dans le tableau
\code{a}. Cette condition est équivalente à utiliser la fonction
\code{array\_key\_exists} avec toutes les clés de $K$ en conjonction.

De même, la condition~:
%
$$\code{a[\_]: $V$}$$
%
signifie que toutes les valeurs de $V$ doivent être présentes dans le tableau
\code{a}. Cette condition est équivalente à utiliser la fonction
\code{in\_array} avec toutes les valeurs de $V$ en conjonction.

Au lieu d'utiliser le symbole \code{:}, nous pouvons utiliser le symbole
\code{!:} pour exprimer la négation. La condition \code{a[$K$]!: $V$} signifie
que toutes les clés de $K$ ont une valeur dans \code{a} et que cette valeur
n'est pas dans $V$. Le fonctionnement est similaire avec le symbole \code{\_}.
Par exemple, la condition \code{a[$K$]!: \_} signifie qu'aucune clé de $K$
n'apparaît dans \code{a}.

Les clés des tableaux sont toujours uniques, but pas les valeurs. Nous pouvons
exprimer la condition d'unicité sur les valeurs en écrivant~:
%
$$\code{a is unique}$$
%
Dans ce cas, nous ne pouvons pas avoir deux fois la même valeur dans le tableau
\code{a}.

\begin{example}[Array conditions]
Pour illustrer toutes les sortes de conditions, nous allons utiliser l'exmple
suivante qui utilise \code{a}:

\begin{pre}
length: 0..5 or 10 \\
a     : array([to string('a', 'e', 1)], length) \\
a[0]  : 'b' or 'd' \\
a is unique
\end{pre}

\end{example}

\subsection{Solveur de contraintes}

\subsubsection{Variables}

\subsubsection{Contraintes de cardinalité}

\subsubsection{Contraintes sur la taille du tableau}

\subsubsection{Contraintes sur les domaines et co-domaines}

\subsubsection{Contraintes sur les paires}

\subsubsection{Contraintes sur les clés et les valeurs}

\subsubsection{Propagation et consistence}

\subsubsection{\inenglish{Labelling}}
