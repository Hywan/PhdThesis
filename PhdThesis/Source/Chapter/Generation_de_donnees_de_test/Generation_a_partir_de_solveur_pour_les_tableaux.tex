\section{Génération à partir de propriétés sur les tableaux}
\label{section:data:arrays}

En PHP, les tableaux sont utilisés pour représenter toutes sortes de
collections. Nous parlons de tableaux associatifs, ou encore
\inenglish{hashmaps}, car ils sont constitués de paires clé-valeur.

Nous allons tout d'abord étudier plusieurs projets PHP pour connaître les
fonctions les plus utilisées sur les tableaux. Ensuite, nous allons étendre
Praspel pour introduire des propriétés sur les tableaux. Enfin, nous verrons
comment ces propriétés sont traduites en contraintes pour notre propre solveur.

\subsection{Étude des propriétés les plus utilisées}

Nous avons sélectionnés 61~projets PHP sur plusieurs plateformes, comme
Github\footnote{Voir \url{http://github.com/}.} ou Sourceforge\footnote{Voir
\url{http://sourceforge.net/}.}, à partir de leur popularité, leur impact sur
l'industrie et leur complexité. Tous ces projets représentent 28~066 fichiers,
soit 5~220~547~lignes de code. Dans ces lignes de code, nous avons compté le
nombre d'occurences de chaque fonction de tableaux disponible dans la
bibliothèque standard de PHP\footnote{Voir \url{http://php.net/ref.array}.}. La
figure~\ref{figure:data:collecting_informations} présente partiellement ces
résultats~: les fonctions sont triées par leur nombre d'occurences $n$~; celles
avec $n < 100$ n'apparaissent pas.
%
\begin{figure}

{
\Huge

\drawfig{!}{11cm}{}{

  \tikzset{
     ultra thick/.style={line width=9pt}
  }

  \draw[->] (-.1,   0) -- (46, 0) node[anchor=north] {fonction};
  \draw[->] (  0, -.1) -- (0, 33) node[anchor=east] {$n$};

  \draw (1, 0) node[anchor=north] {\rotatebox{90}{…}};

  \draw[ultra thick] (2, 0) -- (2, 0.245);
  \draw (2, 0) node[anchor=north] {\rotatebox{90}{\code{array\_diff\_key}}};
  \draw[ultra thick] (3, 0) -- (3, 0.2475);
  \draw (3, 0) node[anchor=north] {\rotatebox{90}{\code{array\_fill\_keys}}};
  \draw[ultra thick] (4, 0) -- (4, 0.265);
  \draw (4, 0) node[anchor=north] {\rotatebox{90}{\code{array\_intersect}}};
  \draw[ultra thick] (5, 0) -- (5, 0.2825);
  \draw (5, 0) node[anchor=north] {\rotatebox{90}{\code{array\_change\_key\_case}}};
  \draw[ultra thick] (6, 0) -- (6, 0.295);
  \draw (6, 0) node[anchor=north] {\rotatebox{90}{\code{array\_intersect\_key}}};
  \draw[ultra thick] (7, 0) -- (7, 0.3125);
  \draw (7, 0) node[anchor=north] {\rotatebox{90}{\code{array\_fill}}};
  \draw[ultra thick] (8, 0) -- (8, 0.39);
  \draw (8, 0) node[anchor=north] {\rotatebox{90}{\code{asort}}};
  \draw[ultra thick] (9, 0) -- (9, 0.435);
  \draw (9, 0) node[anchor=north] {\rotatebox{90}{\code{array\_splice}}};
  \draw[ultra thick] (10, 0) -- (10, 0.475);
  \draw (10, 0) node[anchor=north] {\rotatebox{90}{\code{prev}}};
  \draw[ultra thick] (11, 0) -- (11, 0.58);
  \draw (11, 0) node[anchor=north] {\rotatebox{90}{\code{array\_flip}}};
  \draw[ultra thick] (12, 0) -- (12, 0.645);
  \draw (12, 0) node[anchor=north] {\rotatebox{90}{\code{array\_combine}}};
  \draw[ultra thick] (13, 0) -- (13, 0.65);
  \draw (13, 0) node[anchor=north] {\rotatebox{90}{\code{array\_diff}}};
  \draw[ultra thick] (14, 0) -- (14, 0.6525);
  \draw (14, 0) node[anchor=north] {\rotatebox{90}{\code{array\_reverse}}};
  \draw[ultra thick] (15, 0) -- (15, 0.7225);
  \draw (15, 0) node[anchor=north] {\rotatebox{90}{\code{range}}};
  \draw[ultra thick] (16, 0) -- (16, 0.7575);
  \draw (16, 0) node[anchor=north] {\rotatebox{90}{\code{ksort}}};
  \draw[ultra thick] (17, 0) -- (17, 0.7875);
  \draw (17, 0) node[anchor=north] {\rotatebox{90}{\code{array\_filter}}};
  \draw[ultra thick] (18, 0) -- (18, 1.0075);
  \draw (18, 0) node[anchor=north] {\rotatebox{90}{\code{array\_unshift}}};
  \draw[ultra thick] (19, 0) -- (19, 1.0675);
  \draw (19, 0) node[anchor=north] {\rotatebox{90}{\code{array\_search}}};
  \draw[ultra thick] (20, 0) -- (20, 1.285);
  \draw (20, 0) node[anchor=north] {\rotatebox{90}{\code{array\_push}}};
  \draw[ultra thick] (21, 0) -- (21, 1.3425);
  \draw (21, 0) node[anchor=north] {\rotatebox{90}{\code{array\_slice}}};
  \draw[ultra thick] (22, 0) -- (22, 1.5725);
  \draw (22, 0) node[anchor=north] {\rotatebox{90}{\code{array\_unique}}};
  \draw[ultra thick] (23, 0) -- (23, 1.6425);
  \draw (23, 0) node[anchor=north] {\rotatebox{90}{\code{sort}}};
  \draw[ultra thick] (24, 0) -- (24, 1.76);
  \draw (24, 0) node[anchor=north] {\rotatebox{90}{\code{array\_map}}};
  \draw[ultra thick] (25, 0) -- (25, 2.175);
  \draw (25, 0) node[anchor=north] {\rotatebox{90}{\code{array\_values}}};
  \draw[ultra thick] (26, 0) -- (26, 2.2325);
  \draw (26, 0) node[anchor=north] {\rotatebox{90}{\code{sizeof}}};
  \draw[ultra thick] (27, 0) -- (27, 2.3225);
  \draw (27, 0) node[anchor=north] {\rotatebox{90}{\code{compact}}};
  \draw[ultra thick] (28, 0) -- (28, 2.4125);
  \draw (28, 0) node[anchor=north] {\rotatebox{90}{\code{array\_pop}}};
  \draw[ultra thick] (29, 0) -- (29, 2.5225);
  \draw (29, 0) node[anchor=north] {\rotatebox{90}{\code{key}}};
  \draw[ultra thick] (30, 0) -- (30, 2.64);
  \draw (30, 0) node[anchor=north] {\rotatebox{90}{\code{extract}}};
  \draw[ultra thick] (31, 0) -- (31, 2.685);
  \draw (31, 0) node[anchor=north] {\rotatebox{90}{\code{next}}};
  \draw[ultra thick] (32, 0) -- (32, 3.42);
  \draw (32, 0) node[anchor=north] {\rotatebox{90}{\code{array\_shift}}};
  \draw[ultra thick] (33, 0) -- (33, 3.585);
  \draw (33, 0) node[anchor=north] {\rotatebox{90}{\code{reset}}};
  \draw[ultra thick] (34, 0) -- (34, 3.7175);
  \draw (34, 0) node[anchor=north] {\rotatebox{90}{\code{current}}};
  \draw[ultra thick] (35, 0) -- (35, 3.8325);
  \draw (35, 0) node[anchor=north] {\rotatebox{90}{\code{end}}};
  \draw[ultra thick] (36, 0) -- (36, 5.365);
  \draw (36, 0) node[anchor=north] {\rotatebox{90}{\code{array\_keys}}};
  \draw[ultra thick] (37, 0) -- (37, 7.6975);
  \draw (37, 0) node[anchor=north] {\rotatebox{90}{\code{list}}};
  \draw[ultra thick] (38, 0) -- (38, 8.0875);
  \draw (38, 0) node[anchor=north] {\rotatebox{90}{\code{array\_merge}}};
  \draw[ultra thick] (39, 0) -- (39, 8.3475);
  \draw (39, 0) node[anchor=north] {\rotatebox{90}{\code{is\_array}}};
  \draw[ultra thick] (40, 0) -- (40, 9);
  \draw (40, 0) node[anchor=north] {\rotatebox{90}{\code{each}}};
  \draw[ultra thick] (41, 0) -- (41, 12.025);
  \draw (41, 0) node[anchor=north] {\rotatebox{90}{\code{in\_array}}};
  \draw[ultra thick] (42, 0) -- (42, 19.485);
  \draw (42, 0) node[anchor=north] {\rotatebox{90}{\code{array\_key\_exists}}};
  \draw[ultra thick] (43, 0) -- (43, 29.6475);
  \draw (43, 0) node[anchor=north] {\rotatebox{90}{\code{count}}};

  \draw (0, 0) node[anchor=east] {0};
  \draw (0, 2.5) node[anchor=east] {1000};
  \draw (0, 5) node[anchor=east] {2000};
  \draw (0, 7.5) node[anchor=east] {3000};
  \draw (0, 10) node[anchor=east] {4000};
  \draw (0, 12.5) node[anchor=east] {5000};
  \draw (0, 15) node[anchor=east] {6000};
  \draw (0, 17.5) node[anchor=east] {7000};
  \draw (0, 20) node[anchor=east] {8000};
  \draw (0, 22.5) node[anchor=east] {9000};
  \draw (0, 25) node[anchor=east] {10000};
  \draw (0, 27.5) node[anchor=east] {11000};
  \draw (0, 30) node[anchor=east] {12000};

}
}

\caption{\label{figure:data:collecting_informations} Étude des fonctions sur les
tableaux~: les fonctions en abscisse et leur nombre d'occurences en ordonnée.}

\end{figure}
%
Les trois fonctions les plus utilisées sont \code{count()},
\code{array\_key\_exists()} et \code{in\_array()}. La fonction \code{count()}
compte le nombre d'éléments contenus dans un tableau, la fonction
\code{array\_key\_exists()} vérifie si une clé est présente dans un tableau
(indépendamment de la valeur associée, c'est~à~dire que cette fonction retourne
\code{true} même si la valeur est \code{null}), et enfin, la fontion
\code{in\_array()} vérifie si une valeur est présente dans le tableau. Toutes
ces fonctions travaillent sur un seul tableau à la fois. Cette étude suggère que
nous pouvons considérer ces fonctions booléennes sans effet de bord en tant que
propriétés les plus fréquentes sur les tableaux.

\subsection{Syntaxe des propriétés}

Nous étendons la syntaxe d'une déclaration de tableau~:
%
$$\code{a: array($D$, $L$)}$$
%
dans Praspel avec les propriétés sur les tableaux suivantes. Une propriété sur
une paire est de la forme~:
%
$$\code{a[$K$]: $V$}$$
%
où $K$ et $V$ sont des disjonctions de domaines réalistes. La propriété signifie
que les paires constituées de toutes les valeurs de $K$ avec au moins une valeur
dans $V$ sont présentes dans le tableau \code{a}. $K$ accepte seulement les
domaines réalistes qui implémentent les interfaces \code{Constant},
\code{Interval} et \code{Enumerable}. Cette propriété est équivalente à utiliser
les fonctions \code{array\_key\_exists()} et \code{in\_array()} combinées.

\begin{example}[\code{a[$K$]: $V$} en PHP]

La propriété \code{a[7..8]: true} signifie que \code{true} est associé aux clés
\code{7} et \code{8}, ce qui correspond en PHP au prédicat suivant~:
%
\begin{pre}
   array\_key\_exists(7, \$a) \\
\&\& in\_array(true, \$a) \\
\&\& \$a[7] === true \\
\&\& array\_key\_exists(8, \$a) \\
\&\& in\_array(true, \$a) \\
\&\& \$a[8] === true
\end{pre}
%
Nous vérifions que la clé \code{7} existe dans le tableau \code{\$a}, puis nous
vérifions que la valeur \code{true} existe également, et enfin, nous vérifions
qu'elles forment une paire.

\end{example}

Si nous voulons exprimer une contrainte seulement sur $K$, nous pouvons utiliser
le symbole \code{\_}. La propriété~:
%
$$\code{a[$K$]: \_}$$
%
signifie que toutes les valeurs de $K$ doivent être des clés du tableau
\code{a}. Cette propriété est équivalente à utiliser la fonction
\code{array\_key\_exists} avec toutes les clés de $K$ en conjonction.  De même,
la propriété~:
%
$$\code{a[\_]: $V$}$$
%
signifie que toutes les valeurs de $V$ doivent être des valeurs du tableau
\code{a}. Cette propriété est équivalente à utiliser la fonction
\code{in\_array} avec toutes les valeurs de $V$ en conjonction.

\begin{example}[\code{a[$K$]: \_} et \code{a[\_]: $V$} en PHP]

Les propriétés \code{a[7..8]: \_} et \code{a[\_]: 'foo'} signifient que les clés
\code{7} et \code{8} sont présentes dans le tableau \code{\$a}, ainsi que la
valeur \code{'foo'}~; ce qui correspond en PHP au prédicat suivant~:
%
\begin{pre}
   array\_key\_exists(7, \$a) \\
\&\& array\_key\_exists(8, \$a) \\
\&\& in\_array('foo', \$a)
\end{pre}

\end{example}

Au lieu d'utiliser le symbole \code{:} nous pouvons utiliser le symbole
\code{!:} pour exprimer la négation. La propriété \code{a[$K$]!: $V$} signifie
que toutes les valeurs de $K$ sont des clés de \code{a} et que les valeurs
associées ne sont pas dans $V$. Le fonctionnement est similaire avec le symbole
\code{\_}.  Par exemple, la propriété \code{a[$K$]!: \_} signifie que toutes les
valeurs de $K$ qui sont des clés de \code{a} ont une valeur qui n'est pas dans
\code{\_} (tout), donc qu'aucune valeur de $K$ n'apparaît dans \code{a} en tant
que clé.

Les clés des tableaux sont toujours uniques, mais pas les valeurs. Nous pouvons
exprimer la propriété d'unicité sur les valeurs en écrivant~:
%
$$\code{a is unique}$$
%
Dans ce cas, nous ne pouvons pas avoir deux fois la même valeur dans le tableau
\code{a}.

\begin{example}[Propriétés sur un tableau]
\label{example:data:array_proprietes}
Pour illustrer toutes les sortes de propriétés, nous allons utiliser l'exemple
suivant qui porte sur un tableau \code{a}:

\begin{pre}
length: 0..5 or 10 \\
a     : array([to string('a', 'e', 1)], length) \\
a[0]  : 'b' or 'c' \\
a is unique
\end{pre}

Deux variables sont présentes dans cet exemple~: \code{length} qui représente la
taille du tableau, ici par un intervalle $[0; 5]$ ou 10, et \code{a} qui est un
tableau de caractères de \code{a} à \code{e} et dont la taille est définie par
la variable \code{length}. Il y a deux propriétés~: \code{a[0]: 'b' or 'c'} qui
signifie que les valeurs associées à la clé \code{0} sont le caractère
\code{'b'} ou \code{'c'} et \code{a is unique} qui signifie que toutes les
valeurs du tableau doivent être uniques.

\end{example}

\subsection{Sémantiques ensemblistes des propriétés}

Sur un ensemble de propriétés d'un tableau \code{a}, nous proposons d'invoquer
un solveur de contraintes pour construire un tableau satisfaisant toutes ces
propriétés. Cette partie explique comment les propriétés sur les tableaux sont
transformées en contraintes pour le solveur, ou plus précisément en problème de
satisfaction de contraintes (\inenglish{Constraint Satisfaction Problem}, abrégé
CSP). Un CSP~\acite{Tsang93} est un triplet composé de variables, de valeurs
pour ces variables (aussi appelés domaines) et de contraintes. Les variables
sont déclarées en Praspel, les valeurs sont données par des disjonctions de
domaines réalistes, et les contraintes entre ces variables sont données par des
propriétés sur les tableaux.

Une seule de ces propriétés est supposée être une déclaration de tableau de la
forme \code{a: array($D$, $L$)}, où $D$ est en forme normale, c'est~à~dire que
$D$ est une liste de $p$ constructions \code{from $F_i$ to $T_i$} avec $1 \leq i
\leq p$, et $L$ est une disjonction de domaines réalistes $L_1, \dots, L_m$
héritant du domaine réaliste \code{Integer} et étant positifs ou nuls.

Dans l'exemple~\ref{example:data:array_proprietes}, $p = 1$, $m = 2$, $L_1 =
[\code{0..5}]$ et $L_2 = \{\code{10}\}$. Dans la description du tableau, aucun
domaine n'a été déclaré. Dans ce cas, le domaine réaliste \code{natural(0, 1)}
est utilisé par défaut~: il représente un entier auto-incrémenté commençant à 0
et avec un pas de 1, soit \code{0}, \code{1}, \code{2}, \code{3} etc. Dans cet
exemple, $F_1 = \code{natural(0, 1)}$ et $T_1 = \code{string('a', 'e', 1)}$.

Nous précisons qu'une disjonction de domaines réalistes \code{$D_1$ or $\dots$
or $D_n$} sera souvent identifiée comme l'ensemble $D_1 \union \dots \union
D_n$.

\subsubsection{Variables}

Les variables du CSP sont~:
%
\begin{enumerate}

\item la taille du tableau, notée $S$, qui est un entier positif ou nul~;

\item les ensembles $X$ et $Y$ qui sont respectivement le domaine du tableau
(l'ensemble des clés) et le co-domaine (l'ensemble des valeurs)~;

\item le contenu du tableau, noté $H$, qui est une fonction totale de $X$ vers
$Y$ (totale car les clés sont uniques)~;

\item les domaines réalistes $X_1, \dots, X_p$ (respectivement $Y_1, \dots,
Y_p)$ qui sont des sous-ensembles des domaines réalistes $F_1, \dots, F_p$
(respectivement $T_1, \dots, T_p$), compatibles avec toutes les propriétés sur
le tableau.

\end{enumerate}

Nous sommes essentiellement intéressés par trouver le contenu de $X$ et les
valeurs de la fonction $H$. Les autres variables sont seulement introduites pour
simplifier l'expression des contraintes.

Nous désignons par $H(x) = y$, avec $x \in X$, que la paire $(x, y)$ est dans le
tableau. Nous étendons $H$ aux parties de $X$ par la fonction $\hat{H}$ définie
par $\hat{H}(E) = \{H(x) \;\vert\; x \in E\}$ pour toute partie $E$ de $X$.

\subsubsection{Contraintes de cardinalité}

Nous désignons par $\mathrm{card}(E)$ la cardinalité de l'ensemble fini $E$. Les
con\-train\-tes~:
%
$$\mathrm{card}(X) = S \quad\mathrm{et}\quad S \geq 0$$
%
expriment que la taille du tableau est son nombre de clés et que ce nombre est
positif ou nul. Par défaut, il n'y a pas de contrainte d'unicité sur les
co-domaines, donc nous avons seulement la contrainte~:
%
$$\mathrm{card}(X) \geq \mathrm{card}(Y)$$
%
Cependant, en présence de la propriété \code{a is unique}, cette contrainte
devient~:
%
$$\mathrm{card}(X) = \mathrm{card}(Y)$$

\subsubsection{Contraintes sur la taille du tableau}

Pendant la propagation des contraintes, le solveur peut raffiner les domaines
réalistes $L_1, \dots, L_m$, représentant les valeurs possibles pour la taille
$S$ du tableau. Cette taille appartient à un de ses domaines réalistes, nous
avons donc la contrainte suivante~:
%
$$S \in L_1 \union \dots \union L_m$$

Dans l'exemple~\ref{example:data:array_proprietes}, nous avons $L_1 \subseteq
[\code{0..5}]$ et $L_2 \subseteq \{\code{10}\}$. La taille $S$ est contrainte
par $S \in L_1 \union L_2$.

\subsubsection{Contraintes sur les domaines et co-domaines}

Le domaine $X$ est le co-domaine $Y$ de $H$ sont reliés par la contrainte~:
%
$$Y = \hat{H}(X)$$

Nous attendons du solveur qu'il nous propose une solution pour le domaine $X$
(respectivement le co-domaine $Y$) sous la forme d'une disjonction de domaines
réalistes \code{$X_1$ or $\dots$ or $X_p$} (respectivement \code{$Y_1$ or
$\dots$ or $Y_p$}), compatibles avec toutes les propriétés sur le tableau. Nous
devrions avoir les égalités $X = \Union_{1 \;\leq\; i \;\leq\; p} X_i$ et $Y =
\Union_{1 \;\leq\; i \;\leq\; p} Y_i$, ainsi que les inclusions $X_i \subseteq
F_i$ et $Y_i \subseteq T_i$ pour $1 \leq i \leq p$. La paire $(X_i, Y_i)$
devrait aussi satisfaire la contrainte $\hat{H}(X_i) = Y_i$, c'est~à~dire que
$Y_i$ est le co-domaine de la restriction de $H$ à $X_i$ ($\subseteq X$).

\subsubsection{Contraintes sur les paires}

Pour chaque propriété \code{a[$K$]: $V$}, où $K$ et $V$ sont des disjonctions de
domaines réalistes, nous introduisons les contraintes~:
%
$$K \subseteq X \quad\mathrm{et}\quad \hat{H}(K) \subseteq V$$
%
Une propriété négative sur une paire \code{a[$K$]!: $V$} est traduite en
contraintes~:
%
$$K \subseteq X \quad\mathrm{et}\quad \hat{H}(K) \intersection V = \emptyset$$

Pour la propriété \code{a[0]: 'b' or 'c'} dans
l'exemple~\ref{example:data:array_proprietes}, nous avons $K = \{\code{0}\}$ et
$V = \{\code{'b'}, \code{'c'}\}$. Les contraintes sont $\{\code{0}\} \subseteq
X$ et $\hat{H}(\{\code{0}\}) \subseteq \{\code{'b'}, \code{'c'}\}$.

\subsubsection{Contraintes sur les clés et les valeurs}

La propriété \code{a[$K$]: \_} est traduite en contrainte $K \subseteq X$ et sa
négation \code{a[$K$]!: \_} en contrainte $K \intersection X = \emptyset$. La
propriété \code{a[\_]: $V$} est traduite en contrainte $V \subseteq Y$ et sa
négation \code{a[\_]!: $V$} en contrainte $V \intersection Y = \emptyset$.

\subsection{Solveur de contraintes}

Introduction

\subsubsection{Propagation et consistance}

La propagation des contraintes utilise un algorithme AC3~\acite{Mackworth77},
pour \inenglish{Arc Consistency Algorithm \#3}, implémenté en PHP.

Cet algorithme travaille sur un CSP transformé en graphe orienté où les nœuds
représentent les variables et les arcs représentent les contraintes entre deux
variables. Un arc $(X, Y)$ est considéré comme consistant si et seulement si
pour chaque valeur $x$ de la variable $X$, il existe une valeur $y$ de la
variable $Y$ telle que $(x, y)$ satisfasse la contrainte entre $X$ et $Y$.  Si
ce n'est pas le cas, l'algorithme supprime des valeurs du domaine de $x$ pour
obtenir la consistance de l'arc. Cet algorithme permet aussi de propager les
contraintes d'une variable à l'autre en même temps qu'il vérifie la consistance.
Seules les contraintes unaires et binaires sont gérées par cet algorithme,
c'est~à~dire des contraintes portant sur une et deux variables. La
figure~\ref{figure:data:ac3} présente l'algorithme AC3, tandis que
l'exemple~\ref{example:data:ac3} illustre le problème auquel répond cet
algorithme.

\begin{figure}

\begin{center}
\begin{algorithmic}

\State $\avar{Variables}: \atype{set\astype{Variable}}$
\State $\avar{Domains}: \lambda (\atype{Variable} \rightarrow \atype{set\astype{Domain}})$
\State $\avar{Neighbours}: \lambda (\atype{Variable} \rightarrow \atype{set\astype{Variable}})$

\\\hrulefill

\Function{$\akw{AC3}$}{}

  \Require $\avar{csp}: \atype{CSP}$
  \State $\avar{queue}: \atype{list\astype{(\atype{Variable}, \atype{Variable})}} \gets \akw{all~the~arcs}$

  \While{$\avar{queue} \neq \emptyset$}

      \State $(X, Y) \gets \avar{queue}.\acall{pop}()$

      \If{$\acall{AC3\_Arc\_Reduce}(X, Y)$}

          \ForAll{$Z \in \avar{Neighbours}(X)$}

              \State $\avar{queue}.\acall{push}((Z, X))$

          \EndFor

      \EndIf

  \EndWhile

\EndFunction

\\\hrulefill

\Function{$\akw{AC3\_Arc\_Reduce}$}{}

    \Require $\avar{X}: \atype{Variable}$
    \Require $\avar{Y}: \atype{Variable}$
    \Ensure  $\avar{removed}: \atype{boolean} \gets \akw{false}$

    \ForAll{$x \in \avar{Domains}(X)$}

        \If{$\akw{there~is~no} y \in \avar{Domains}(Y) \akw{such~that} y.\acall{reduce}(y)$}

            \State $\akw{delete} x \akw{from} \avar{Domains}(X)$
            \State $\avar{removed} \gets true$

        \EndIf

    \EndFor

    \State $\areturn \avar{removed}$

\EndFunction

\end{algorithmic}
\end{center}

\caption{\label{figure:data:ac3} Algorithme AC3.}

\end{figure}

\begin{example}[Illustration de l'algorithme AC3]
\label{example:data:ac3}

Supposons que nous avons le CSP suivant~:
%
\begin{itemize}

\item variables~: $X$, $Y$, $Z$~;

\item domaines~: $D(X) = \{4, 5, 6, 7\}$, $D(Y) = \{4, 5, 6, 8, 9\}$ et $D(Z) =
\{3, 5, 6, 7, 9\}$~;

\item contraintes~: $X = Y$ et $Y = Z$.

\end{itemize}

Sans algorithme AC3, intuitivement, nous commençons par vérifier la contrainte
$X = Y$, c'est~à~dire que toutes les valeurs des domaines $D(X)$ et $D(Y)$ qui
ne sont pas égales seront supprimées. Alors nous obtenons $D(X) = \{4, 5, 6\}$,
$D(Y) = \{4, 5, 6\}$ et $D(Z)$ est inchangé. Ensuite, nous vérifions la
contrainte $Y = Z$. Nous obtenons $D(Y) = \{5, 6\}$, $D(Z) = \{5, 6\}$ et $D(X)$
est inchangé.  Mais nous avons un problème~: la contrainte $X = Y$ n'est plus
satisfaite. Nous savons alors qu'il faut re-vérifier la contrainte $X = Y$ pour
obtenir $D(X) = \{5, 6\}$, mais une machine ne le saurait pas nécessairement.

Maintenant, voyons comment opère l'algorithme AC3. Nous avons les mêmes
variables, domaines et contraintes qu'au début de cet exemple. Le graphe du CSP
est le suivant, il représente les liens par les contraintes entre les
variables~:

\fig{6cm}{!}{AC3.tex}

D'après la figure~\ref{figure:data:ac3}, nous avons $\avar{Variables} = \{X, Y,
Z\}$, $\avar{Domains} = \{X \rightarrow \{4, 5, 6, 7\}, Y \rightarrow \{4, 5, 6,
8, 9\}, Z \rightarrow \{3, 5, 6, 7, 9\}\}$ et $\avar{Neighbours} = \{X
\rightarrow \{Y\}, Y \rightarrow \{X, Z\}, Z \rightarrow \{Y\}\}$. Ensuite, dans
l'algorithme $\akw{AC3}$, la variable $\avar{queue}$ contient au départ tous les
arcs du graphe, soit $\avar{queue} = \{(X, Y), (Y, X), (Y, Z), (Z, Y)\}$. Nous
commençons par traiter la première paire de $\avar{queue}$ après l'avoir
supprimée. Nous traitons alors la contrainte impliquant $X$ et $Y$, soit $X = Y$
mais nous n'allons travailler que sûr $X$.  Ainsi $\avar{Domains}(X) = \{4, 5,
6\}$, $\avar{Domains}(Y)$ et $\avar{Domains}(Z)$ restant inchangés. La variable
$X$ n'a pas d'arcs vers d'autres variables, donc nous ne modifions pas
$\avar{queue}$.  Ensuite, la paire suivante est $(Y, X)$, donc nous traitons la
contrainte $X = Y$ mais cette fois-ci, nous modifions $Y$. Ainsi
$\avar{Domains}(Y) = \{4, 5, 6\}$, $\avar{Domains}(X)$ et $\avar{Domains}(Z)$
restant inchangés.  La variable $Y$ a un arc vers $Z$ en plus de vers $X$, donc
nous devrions ajouter la paire $(Z, Y)$ mais elle existe déjà dans
$\avar{Domains}$, nous ne faisons rien.  La paire suivante dans $\avar{queue}$
est $(Y, Z)$, donc nous traitons la contrainte $Y = Z$ en modifiant uniquement
$Y$. Ainsi~: $\avar{Domains}(Y) = \{5, 6\}$, $\avar{Domains}(X)$ et
$\avar{Domains}(Z)$ restant inchangés. La variable $Z$ n'a pas d'arcs vers une
autre variable, donc nous ne modifions pas $\avar{queue}$. La paire suivante est
$(Z, Y)$, donc nous vérifions la contrainte $Y = Z$ en modifiant $Z$. Ainsi (en
rappelant tous les domaines)~: $\avar{Domains}(X) = \{4, 5, 6\}$,
$\avar{Domains}(Y) = \{5, 6\}$ et $\avar{Domains}(Z) = \{5, 6\}$. La variable
$Y$ a un arc vers la variable $X$, donc nous ajoutons la paire $(X, Y)$ dans
$\avar{queue}$.  Comme $\avar{queue}$ était vide, cela devient notre paire
suivante que nous traitons. Nous appliquons alors la contrainte $X = Y$ en
modifiant $X$.  Ainsi, nous obtenons~: $\avar{Domains}(X) = \avar{Domains}(Y) =
\avar{Domains}(Z) = \{5, 6\}$. La variable $X$ n'a pas d'arcs vers d'autres
variables que $Y$, donc nous ne modifions pas $\avar{queue}$. Cette dernière est
vide, donc l'algorithme se termine.

\end{example}

La complexité de cet algorithme est de $\m{O}(mn^3)$ où $m$ est le nombre de
contraintes (d'arcs) et $n$ le nombre de valeurs de la «~plus grande~»
disjonction de domaines réalistes.

Notre implémentation est une variante de l'algorithme AC3 (tout en conservant la
même complexité). Les variables n'ont pas des domaines mais une disjonction de
domaines réalistes. Rappelons que nous manipulons cinq sortes de domaines
réalistes classés par les interfaces \code{Constant}, \code{Interval},
\code{Nonconvex}, \code{Finite} et \code{Enumerable}. Pour chaque sorte de
domaine réaliste, nous avons implémenté une ou des méthodes de raffinement pour
permettre la réduction du domaine réaliste. Par exemple, \code{Nonconvex} attend
une méthode \code{discredit} pour supprimer une valeur du domaine,
\code{Interval} attend les méthodes \code{reduceRightTo} et \code{reduceLeftTo}
respectivement pour réduire la borne droite et gauche etc.

\begin{example}[Propagation et consistance avec un algorithme AC3]

Si nous reprenons notre exemple~\ref{example:data:array_proprietes}, nous avons
les contraintes suivantes~:

\begin{align*}
%
S & \geq 0 \\
%
\mathrm{card}(X) & = S \\
%
\mathrm{card}(X) & = \mathrm{card}(Y) \\
%
S & \in [\code{0..5}] \union \{\code{10}\} \\
%
Y & = \hat{H}(X) \\
%
X & = \ingray{X_1 =\;} \code{natural(0, 1)} \\
%
Y & = \ingray{Y_1 =\;} \code{string('a', 'e', 1)} \\
%
\{\code{0}\} & \subseteq X \\
%
\ingray{\hat{H}(\{\code{0}\})} & \ingray{\;\subseteq \{\code{'b'}, \code{'c'}\}
                                 \quad \text{simplifiée en}} \\
%
H(\code{0}) & = \{\code{'b'}, \code{'c'}\}
%
\end{align*}

Nous appliquons l'algorithme. Nous sélectionnons et retirons la première paire
$(S, X)$ avec la contrainte $\mathrm{card}(X) = S$. Nous révisons les valeurs de
$S$~: la cardinalité de $X = \code{natural(0, 1)}$ est égale à $+\infty$, $S$
reste inchangée. Nous ajoutons la paire $(Y, S)$. Nous sélectionnons et retirons
le deuxième paire $(X, Y)$ lié par la contrainte $\mathrm{card}(X) =
\mathrm{card}(Y)$. Nous n'avons rien à réviser pour l'instant~; en revanche,
nous savons que les deux ensembles auront la même cardinalité. Ensuite, nous
sélectionnons et retirons la paire $(Y, S)$ avec la contrainte $\mathrm{card}(Y)
= S$ (que nous avons déduite par propagation). Nous révisons les valeurs de
$S$~: la variable $Y = \code{string('a', 'e', 1)}$ peut contenir 1 à 5 données
maximum (de \code{a} jusqu'à \code{e}), alors nous enlevons 0 et 10 de $S$. Sa
disjonction est constituée de deux domaines réalistes. Le premier est un
intervalle d'entiers (domaine réaliste \code{Boundinteger}) implémentant entre
autre les interfaces \code{Interval} et \code{Nonconvex}. Nous appelons la
méthode \code{discredit} pour supprimer 0 de l'interval. Le second domaine
réaliste est une constante représentant l'entier 10, nous le supprimons
simplement de la disjonction. Ainsi, nous obtenons $S = [\code{1..5}]$. Nous
ajoutons la paire $(X, Y)$ traitée juste avant. Cette fois-ci, nous pouvons
réviser $Y$ avec la contrainte $\mathrm{card}(X) = \mathrm{card}(Y)$ qui peut
maintenir s'écrire $5 = \mathrm{card}(Y)$. Nous révisons alors \code{natural(0,
1)} pour qu'il ne produise que 5 données au maximum. Ensuite, nous prenons une
autre paire et nous vérifions la consistance de $\{\code{0}\} \subseteq X$, ce
qui est le cas car $\code{0} \in \code{natural(0, 1)}$. Pareil avec
$\{\code{'b'}, \code{'c'}\} \subseteq \code{string('a', 'e', 1)}$. Nous n'avons
plus de paires à vérifier, l'algorithme termine.

\end{example}

La consistance vérifie également qu'il n'y a pas de domaine réaliste vide pour
les quatres variables $S$, $H$, $X$ et $Y$. L'objectif est de détecter les
inconsistances au plus tôt. Toutefois, toutes les inconsistances ne sont pas
détectées, ceci étant un problème NP-difficile.

\subsubsection{\inenglish{Labelling}}

Le \inenglish{labelling} (étiquetage) est le processus qui trouve une valeur
pour chaque variable. Il s'applique si les contraintes sont consistantes. L'idée
est de sélectionner une variable et de lui choisir une valeur, puis de
recommencer avec une autre variable etc. Le choix de la variable à sélectionner
dépend de la dépendance entre les variables (les arcs vus dans la partie
précédente).

À chaque choix de valeur, la consistance entre les contraintes sont
vérifiées. Si une inconsistance est détectée, alors le processus va revenir en
arrière, nous parlons de \inenglish{backtracking}.

À chaque fois que l'algorithme sélectionne une nouvelle variable, sa disjonction
de domaines réalistes est clonée (dupliquée)~; ainsi lors du
\inenglish{backtracking}, nous restaurons la disjonction précédente pour
«~restaurer l'état~». En même temps, la valeur qui a conduite à l'inconsistance
est discréditée afin de ne pas être re-sélectionnée.

Dans notre implémentation, le choix d'une valeur utilise la caractéristique de
générabilité des domaines réalistes, avec un générateur numérique
pseudo-aléatoire. Afin de faire converger le solveur rapidement vers une
solution, nous utilisons une heuristique qui consiste à choisir une valeur pour
la variable $S$ en premier.  Cette approche permet de déplier les
quantificateurs $\forall$ et $\exists$ (puisque $F_i$ et $T_i$ sont énumérables,
nous ne manipulons que des ensembles finis). Ensuite, le solveur essaye de
calculer les ensembles $X_i$ et $Y_i$.

Quand toutes les variables sont étiquetées, c'est~à~dire que chacune a une
valeur valide, le solveur retourne la solution.

\begin{example}[\inenglish{Labelling}]

Une solution pour l'exemple~\ref{example:data:array_proprietes} est~:
%
\begin{pre}
array( \\
    0 => 'c', \\
    1 => 'd', \\
    2 => 'a', \\
    3 => 'e' \\
)
\end{pre}
%
La taille du tableau est de 4, l'index \code{0} a comme valeur \code{'c'} et
toutes les valeurs sont uniques.

\end{example}
