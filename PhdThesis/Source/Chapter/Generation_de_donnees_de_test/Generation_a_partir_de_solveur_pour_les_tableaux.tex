\section{Génération à partir de solveur pour les tableaux}
\label{section:data:arrays}

Les tableaux sont utilisés pour représenter toutes sortes de collections en PHP.
Nous parlons de tableaux associatifs, ou encore \inenglish{hashmaps}, car ils
sont constituées de paires clé-valeur.

Nous allons tout d'abord étudier plusieurs projets PHP pour connaître les
fonctions les plus utilisées sur les tableaux. Ensuite, nous allons étendre
Praspel pour introduire des conditions sur les tableaux. Enfin, nous verrons
comment ces conditions sont traduites en contraintes pour notre propre solveur.

\subsection{Étude des contraintes les plus utilisées}

Nous avons sélectionnés 61~projets PHP sur plusieurs plateformes, comme
Github\footnote{\url{http://github.com/}} ou
Sourceforge\footnote{\url{http://sourceforge.net/}}, à partir de leur
popularité, leur impact sur l'industrie et leur complexité. Tous ces projets
représentent 28~066 fichiers, soit 5~220~547~lignes de code. Dans ces lignes de
code, nous avons compté le nombre d'occurences de chaque fonction de tableaux
disponible dans la bibliothèque standard de PHP. La
figure~\ref{figure:data:collecting_informations} présente partiellement ces
résultats~: les fonctions sont triées par leur nombre d'occurences $n$~; celles
avec $n < 100$ n'apparaissent pas.
%
\begin{figure}

{
\Huge

\drawfig{!}{11cm}{}{

  \tikzset{
     ultra thick/.style={line width=9pt}
  }

  \draw[->] (-.1,   0) -- (46, 0) node[anchor=north] {fonction};
  \draw[->] (  0, -.1) -- (0, 33) node[anchor=east] {$n$};

  \draw (1, 0) node[anchor=north] {\rotatebox{90}{…}};

  \draw[ultra thick] (2, 0) -- (2, 0.245);
  \draw (2, 0) node[anchor=north] {\rotatebox{90}{\code{array\_diff\_key}}};
  \draw[ultra thick] (3, 0) -- (3, 0.2475);
  \draw (3, 0) node[anchor=north] {\rotatebox{90}{\code{array\_fill\_keys}}};
  \draw[ultra thick] (4, 0) -- (4, 0.265);
  \draw (4, 0) node[anchor=north] {\rotatebox{90}{\code{array\_intersect}}};
  \draw[ultra thick] (5, 0) -- (5, 0.2825);
  \draw (5, 0) node[anchor=north] {\rotatebox{90}{\code{array\_change\_key\_case}}};
  \draw[ultra thick] (6, 0) -- (6, 0.295);
  \draw (6, 0) node[anchor=north] {\rotatebox{90}{\code{array\_intersect\_key}}};
  \draw[ultra thick] (7, 0) -- (7, 0.3125);
  \draw (7, 0) node[anchor=north] {\rotatebox{90}{\code{array\_fill}}};
  \draw[ultra thick] (8, 0) -- (8, 0.39);
  \draw (8, 0) node[anchor=north] {\rotatebox{90}{\code{asort}}};
  \draw[ultra thick] (9, 0) -- (9, 0.435);
  \draw (9, 0) node[anchor=north] {\rotatebox{90}{\code{array\_splice}}};
  \draw[ultra thick] (10, 0) -- (10, 0.475);
  \draw (10, 0) node[anchor=north] {\rotatebox{90}{\code{prev}}};
  \draw[ultra thick] (11, 0) -- (11, 0.58);
  \draw (11, 0) node[anchor=north] {\rotatebox{90}{\code{array\_flip}}};
  \draw[ultra thick] (12, 0) -- (12, 0.645);
  \draw (12, 0) node[anchor=north] {\rotatebox{90}{\code{array\_combine}}};
  \draw[ultra thick] (13, 0) -- (13, 0.65);
  \draw (13, 0) node[anchor=north] {\rotatebox{90}{\code{array\_diff}}};
  \draw[ultra thick] (14, 0) -- (14, 0.6525);
  \draw (14, 0) node[anchor=north] {\rotatebox{90}{\code{array\_reverse}}};
  \draw[ultra thick] (15, 0) -- (15, 0.7225);
  \draw (15, 0) node[anchor=north] {\rotatebox{90}{\code{range}}};
  \draw[ultra thick] (16, 0) -- (16, 0.7575);
  \draw (16, 0) node[anchor=north] {\rotatebox{90}{\code{ksort}}};
  \draw[ultra thick] (17, 0) -- (17, 0.7875);
  \draw (17, 0) node[anchor=north] {\rotatebox{90}{\code{array\_filter}}};
  \draw[ultra thick] (18, 0) -- (18, 1.0075);
  \draw (18, 0) node[anchor=north] {\rotatebox{90}{\code{array\_unshift}}};
  \draw[ultra thick] (19, 0) -- (19, 1.0675);
  \draw (19, 0) node[anchor=north] {\rotatebox{90}{\code{array\_search}}};
  \draw[ultra thick] (20, 0) -- (20, 1.285);
  \draw (20, 0) node[anchor=north] {\rotatebox{90}{\code{array\_push}}};
  \draw[ultra thick] (21, 0) -- (21, 1.3425);
  \draw (21, 0) node[anchor=north] {\rotatebox{90}{\code{array\_slice}}};
  \draw[ultra thick] (22, 0) -- (22, 1.5725);
  \draw (22, 0) node[anchor=north] {\rotatebox{90}{\code{array\_unique}}};
  \draw[ultra thick] (23, 0) -- (23, 1.6425);
  \draw (23, 0) node[anchor=north] {\rotatebox{90}{\code{sort}}};
  \draw[ultra thick] (24, 0) -- (24, 1.76);
  \draw (24, 0) node[anchor=north] {\rotatebox{90}{\code{array\_map}}};
  \draw[ultra thick] (25, 0) -- (25, 2.175);
  \draw (25, 0) node[anchor=north] {\rotatebox{90}{\code{array\_values}}};
  \draw[ultra thick] (26, 0) -- (26, 2.2325);
  \draw (26, 0) node[anchor=north] {\rotatebox{90}{\code{sizeof}}};
  \draw[ultra thick] (27, 0) -- (27, 2.3225);
  \draw (27, 0) node[anchor=north] {\rotatebox{90}{\code{compact}}};
  \draw[ultra thick] (28, 0) -- (28, 2.4125);
  \draw (28, 0) node[anchor=north] {\rotatebox{90}{\code{array\_pop}}};
  \draw[ultra thick] (29, 0) -- (29, 2.5225);
  \draw (29, 0) node[anchor=north] {\rotatebox{90}{\code{key}}};
  \draw[ultra thick] (30, 0) -- (30, 2.64);
  \draw (30, 0) node[anchor=north] {\rotatebox{90}{\code{extract}}};
  \draw[ultra thick] (31, 0) -- (31, 2.685);
  \draw (31, 0) node[anchor=north] {\rotatebox{90}{\code{next}}};
  \draw[ultra thick] (32, 0) -- (32, 3.42);
  \draw (32, 0) node[anchor=north] {\rotatebox{90}{\code{array\_shift}}};
  \draw[ultra thick] (33, 0) -- (33, 3.585);
  \draw (33, 0) node[anchor=north] {\rotatebox{90}{\code{reset}}};
  \draw[ultra thick] (34, 0) -- (34, 3.7175);
  \draw (34, 0) node[anchor=north] {\rotatebox{90}{\code{current}}};
  \draw[ultra thick] (35, 0) -- (35, 3.8325);
  \draw (35, 0) node[anchor=north] {\rotatebox{90}{\code{end}}};
  \draw[ultra thick] (36, 0) -- (36, 5.365);
  \draw (36, 0) node[anchor=north] {\rotatebox{90}{\code{array\_keys}}};
  \draw[ultra thick] (37, 0) -- (37, 7.6975);
  \draw (37, 0) node[anchor=north] {\rotatebox{90}{\code{list}}};
  \draw[ultra thick] (38, 0) -- (38, 8.0875);
  \draw (38, 0) node[anchor=north] {\rotatebox{90}{\code{array\_merge}}};
  \draw[ultra thick] (39, 0) -- (39, 8.3475);
  \draw (39, 0) node[anchor=north] {\rotatebox{90}{\code{is\_array}}};
  \draw[ultra thick] (40, 0) -- (40, 9);
  \draw (40, 0) node[anchor=north] {\rotatebox{90}{\code{each}}};
  \draw[ultra thick] (41, 0) -- (41, 12.025);
  \draw (41, 0) node[anchor=north] {\rotatebox{90}{\code{in\_array}}};
  \draw[ultra thick] (42, 0) -- (42, 19.485);
  \draw (42, 0) node[anchor=north] {\rotatebox{90}{\code{array\_key\_exists}}};
  \draw[ultra thick] (43, 0) -- (43, 29.6475);
  \draw (43, 0) node[anchor=north] {\rotatebox{90}{\code{count}}};

  \draw (0, 0) node[anchor=east] {0};
  \draw (0, 2.5) node[anchor=east] {1000};
  \draw (0, 5) node[anchor=east] {2000};
  \draw (0, 7.5) node[anchor=east] {3000};
  \draw (0, 10) node[anchor=east] {4000};
  \draw (0, 12.5) node[anchor=east] {5000};
  \draw (0, 15) node[anchor=east] {6000};
  \draw (0, 17.5) node[anchor=east] {7000};
  \draw (0, 20) node[anchor=east] {8000};
  \draw (0, 22.5) node[anchor=east] {9000};
  \draw (0, 25) node[anchor=east] {10000};
  \draw (0, 27.5) node[anchor=east] {11000};
  \draw (0, 30) node[anchor=east] {12000};

}
}

\caption{\label{figure:data:collecting_informations} Résultats de l'étude de
plusieurs projets PHP pour connaître les fonctions les plus populaires sur les
tableaux.}

\end{figure}
%
Les trois fonctions les plus utilisées sont \code{count()},
\code{array\_key\_exists()} et \code{in\_array()}. La fonction \code{count()}
compte le nombre d'éléments contenus dans un tableau, la fonction
\code{array\_key\_exists()} vérifie si une clé est présente dans un tableau
(indépendamment de la valeur associée, c'est~à~dire que cette fonction retourne
\code{true} même si la valeur est \code{null}), et enfin, la fontion
\code{in\_array()} vérifie si une valeur est présente dans le tableau. Toutes
ces fonctions travaillent sur un seul tableau à la fois. Cette étude suggère que
nous pouvons considérer ces fonctions booléennes sans effet de bord en tant que
conditions les plus fréquentes sur les tableaux.

\subsection{Conditions sur les tableaux}

Nous étendons la syntaxe d'une déclaration de tableau~:
%
$$\code{a: array($D$, $L$)}$$
%
dans Praspel avec les conditions sur les tableaux suivantes. Une condition sur
une paire est de la forme~:
%
$$\code{a[$K$]: $V$}$$
%
où $K$ et $V$ sont des disjonctions de domaines réalistes. La condition signifie
que les paires constituées de toutes les clés de $K$ et au moins une valeur dans
$V$ sont présentes dans le tableau \code{a}. $K$ accepte seulement les domaines
réalistes qui implémentent les interfaces \code{Constant}, \code{Interval} et
\code{Enumerable}. Cette condition est équivalente à utiliser les fonctions
\code{array\_key\_exists()} et \code{in\_array()} combinées. Si nous voulons
exprimer une contrainte seulement sur $K$, nous pouvons utiliser le symbole
\code{\_}. La condition~:
%
$$\code{a[$K$]: \_}$$
%
signifie que toutes les clés de $K$ doivent être présentes dans le tableau
\code{a}. Cette condition est équivalente à utiliser la fonction
\code{array\_key\_exists} avec toutes les clés de $K$ en conjonction. De même,
la condition~:
%
$$\code{a[\_]: $V$}$$
%
signifie que toutes les valeurs de $V$ doivent être présentes dans le tableau
\code{a}. Cette condition est équivalente à utiliser la fonction
\code{in\_array} avec toutes les valeurs de $V$ en conjonction.

Au lieu d'utiliser le symbole \code{:} nous pouvons utiliser le symbole
\code{!:} pour exprimer la négation. La condition \code{a[$K$]!: $V$} signifie
que toutes les clés de $K$ ont une valeur dans \code{a} et que cette valeur
n'est pas dans $V$. Le fonctionnement est similaire avec le symbole \code{\_}.
Par exemple, la condition \code{a[$K$]!: \_} signifie que toutes les clés de $K$
ont une valeur qui n'est pas dans \code{\_} (rien), donc qu'aucune clé de $K$
n'apparaît dans \code{a}.

Les clés des tableaux sont toujours uniques, mais pas les valeurs. Nous pouvons
exprimer la condition d'unicité sur les valeurs en écrivant~:
%
$$\code{a is unique}$$
%
Dans ce cas, nous ne pouvons pas avoir deux fois la même valeur dans le tableau
\code{a}.

\begin{example}[Conditions sur un tableau]
\label{example:data:array_conditions}
Pour illustrer toutes les sortes de conditions, nous allons utiliser l'exemple
suivant qui utilise \code{a}:

\begin{pre}
length: 0..5 or 10 \\
a     : array([to string('a', 'e', 1)], length) \\
a[0]  : 'b' or 'c' \\
a is unique
\end{pre}

\end{example}

\subsection{Solveur de contraintes}

Sur une conjonction de conditions d'un tableau \code{a}, nous proposons
d'invoquer un solveur de contraintes pour construire un tableau satisfaisant
toutes ces conditions. Cette partie explique comment les conditions sur les
tableaux sont transformées en contraintes pour le solveur, ou plus précisément
en problème de satisfaction de contraintes (\inenglish{Constraint Satisfaction
Problem}, abrégé CSP). Un CSP~\acite{Tsang93} est un triplet composé de
variables, de valeurs pour ces variables (aussi appelés domaines) et de
contraintes. Les variables sont déclarées dans Praspel, les valeurs sont données
par la disjonction de domaines réalistes, et enfin, les contraintes entre ces
variables sont données par les conditions sur les tableaux.

Une de ces conditions est supposée être une déclaration de tableau de la forme
\code{a: array($D$, $L$)}, où $D$ est en forme normale. C'est~à~dire que $D$ est
une liste de $p$ constructions \code{from $F_i$ to $T_i$} avec $1 \leq i \leq
p$, et $L$ est une disjonction de domaines réalistes $L_1, \dots, L_m$ héritant
du domaine réaliste \code{Integer} et étant positifs ou nuls.

Dans l'exemple~\ref{example:data:array_conditions}, $p = 1$, $m = 2$, $L_1 =
[\code{0..5}]$ et $L_2 = \{\code{10}\}$. Dans la description du tableau, aucun
domaine n'a été déclaré. Dans ce cas, le domaine réaliste \code{natural(0, 1)}
est utilisé par défaut~: il représente un entier auto-incrémenté commençant à 0
et avec un pas de 1, soit \code{0}, \code{1}, \code{2}, \code{3} etc. Dans cet
exemple, $F_1 = \code{natural(0, 1)}$ et $T_1 = \code{string('a', 'e', 1)}$.

Nous précisons qu'une disjonction de domaines domaines \code{$D_1$ or $\dots$ or
$D_n$} sera souvent identifié comme l'ensemble $D_1 \union \dots \union D_n$.

\subsubsection{Variables}

Les variables de contraintes sont:
%
\begin{enumerate}

\item la taille du tableau, notée $S$, qui est un entier positif ou nul~;

\item les ensembles $X$ et $Y$ qui sont respectivement le domaine du tableau
(l'ensemble des clés) et le co-domaine (l'ensemble des valeurs)~;

\item le contenu du tableau, noté $H$, qui est une fonction totale de $X$ vers
$Y$ (totale car les clés sont uniques)~;

\item les domaines réalistes $X_1, \dots, X_p$ (respectivement $Y_1, \dots,
Y_p)$ qui sont des sous-ensembles des domaines réalistes $F_1, \dots, F_p$
(respectivement $T_1, \dots, T_p$), compatibles avec toutes les conditions sur
le tableau.

\end{enumerate}

Nous sommes essentiellement intéressés par trouver le contenu de $X$ et les
valeurs de la fonction $H$. Les autres variables sont seulement introduites pour
simplifier l'expression des contraintes.

Quand $x \in X$ est vraie, $H(x) = y$ signifie que la paire $(x, y)$ est dans le
tableau. Nous étendons $H$ aux sous-ensembles de $X$ par la fonction $\hat{H}$
définie par $\hat{H}(E) = \{H(x) \;\mathrm{tel~que}\; x \in E\}$ pour
n'importe quel sous-ensemble $E$ de $X$.

\subsubsection{Contraintes de cardinalité}

Soit $\mathrm{card}(E)$ représentant la cardinalité de l'ensemble fini $E$. Les
contraintes~:
%
$$\mathrm{card}(X) = S \quad\mathrm{et}\quad S \geq 0$$
%
expriment que la taille du tableau est son nombre de clés et que ce nombre est
positif ou nul. Par défaut, il n'y a pas de contrainte d'unicité sur les
co-domaines, donc nous avons seulement la contrainte~:
%
$$\mathrm{card}(X) \geq \mathrm{card}(Y)$$
%
Cependant, en présence de la condition \code{a is unique}, cette contrainte
devient~:
%
$$\mathrm{card}(X) = \mathrm{card}(Y)$$

\subsubsection{Contraintes sur la taille du tableau}

Pendant la propagation des contraintes, le solveur peut raffiner les domaines
réalistes $L_1, \dots, L_m$, représentant les valeurs possibles pour la taille
$S$ du tableau. Cette taille appartient à un de ses domaines réalistes, nous
avons donc la contrainte suivante~:
%
$$S \in L_1 \union \dots \union L_m$$

Dans l'exemple~\ref{example:data:array_conditions}, nous avons $L_1 \subseteq
[\code{0..5}]$ et $L_2 \subseteq \{\code{10}\}$. La taille $S$ est contrainte
par $S \in L_1 \union L_2$.

\subsubsection{Contraintes sur les domaines et co-domaines}

Le domaine $X$ est le co-domaine $Y$ de $H$ sont reliés par la contrainte~:
%
$$Y = \hat{H}(X)$$

Nous attendons du solveur qu'il nous propose une solution pour le domaine $X$
(respectivement le co-domaine $Y$) sous la forme d'une disjonction de domaines
réalistes \code{$X_1$ or $\dots$ or $X_p$} (respectivement \code{$Y_1$ or
$\dots$ or $Y_p$}), compatibles avec toutes les conditions sur le tableau. Nous
devrions avoir les égalités $X = \Union_{1 \;\leq\; i \leq p} X_i$ et $Y =
\Union_{1 \;\leq\; i \leq p} Y_i$, ainsi que les inclusions $X_i \subseteq F_i$
et $Y_i \subseteq T_i$ pour $1 \leq i \leq p$. La paire $(X_i, Y_i)$ devrait
aussi satisfaire la contrainte $\hat{H}(X_i) = Y_i$, c'est~à~dire que $Y_i$ est
le co-domaine de la restriction de $H$ à $X_i$ ($\subseteq X$).

\subsubsection{Contraintes sur les paires}

Pour chaque condition \code{a[$K$]: $V$}, où $K$ et $V$ sont des disjonctions de
domaines réalistes, nous introduisons les contraintes~:
%
$$K \subseteq X \quad\mathrm{et}\quad \hat{H}(K) \subseteq V$$
%
Une condition négative sur une paire \code{a[$K$]!: $V$} est traduire en
contraintes~:
%
$$K \subseteq X \quad\mathrm{et}\quad \hat{H}(K) \intersection V = \emptyset$$

Pour la condition \code{a[0]: 'b' or 'c'} dans
l'exemple~\ref{example:data:array_conditions}, nous avons $K = \{\code{0}\}$ et
$V = \{\code{'b'}, \code{'c'}\}$. Les contraintes sont $\{\code{0}\} \subseteq
X$ et $\hat{H}(\{\code{0}\}) \subseteq \{\code{'b'}, \code{'c'}\}$.

\subsubsection{Contraintes sur les clés et les valeurs}

La condition \code{a[$K$]: \_} est traduite en contrainte $K \subseteq X$ et sa
négation \code{a[$K$]!: \_} en contrainte $K \intersection X = \emptyset$. La
condition \code{a[\_]: $V$} est traduite en contrainte $V \subseteq Y$ et sa
négation \code{a[\_]!: $V$} en contrainte $V \intersection Y = \emptyset$.

\subsubsection{Propagation et consistence}

La propagation des contraintes utilise un algorithme AC3~\acite{Mackworth77},
pour \inenglish{Arc Consistency Algorithm \#3}, implémenté en PHP.

Cet algorithme travaille sur un CSP transformé en graphe dirigé où les nœuds
représentent les variables et les arcs représentent les contraintes entre deux
variables. La consistence d'arc détecte les erreurs le plus tôt possible. Un arc
$(X, Y)$ est considéré comme consistent si et seulement si pour chaque valeur
$x$ de la variable $X$, il existe une valeur $y$ de la variable $Y$ telle que
$(x, y)$ satisfasse la contrainte entre $X$ et $Y$.  Si ce n'est pas le cas, il
faut supprimer les valeurs du domaine de $x$ pour appliquer la consistence sur
l'arc. Cet algorithme permet aussi de propager les contraintes d'une variable à
l'autre en même temps qu'il vérifie la consistence. Seules les contraintes
unaires ou binaires sont gérées par cet algorithme, c'est~à~dire des contraintes
portant sur une ou deux variables maximum. La figure~\ref{figure:data:ac3}
présente l'algorithme AC3, tandis que l'exemple~\ref{example:data:ac3} illustre
cet algorithme.

\begin{figure}

\begin{center}
\begin{algorithmic}

\State $\avar{Variables}: \atype{set\astype{Variable}}$
\State $\avar{Domains}: \lambda (\avar{X}: \atype{Variable} \rightarrow \atype{set\astype{Domain}})$
\State $\avar{Neighbours}: \lambda (\avar{X}: \atype{Variable} \rightarrow \atype{set\astype{Variable}})$

\\\hrulefill

\Function{$\akw{AC3}$}{}

  \Require $\avar{csp}: \atype{CSP}$
  \State $\avar{queue}: \atype{set\astype{(\avar{x}, \avar{y})}} \gets \akw{all~the~arcs}$

  \While{$\avar{queue} \neq \emptyset$}

      \State $(X, Y) \gets \avar{queue}.\acall{pop}()$

      \If{$\acall{AC3\_Arc\_Reduce}(x, y)$}

          \ForAll{$Z \in \avar{Neighbours}(X)$}

              \State $\avar{queue} = \avar{queue} \union \{(Z, X)\}$

          \EndFor

      \EndIf

  \EndWhile

\EndFunction

\\\hrulefill

\Function{$\akw{AC3\_Arc\_Reduce}$}{}

    \Require $\avar{X}: \atype{Variable}$
    \Require $\avar{Y}: \atype{Variable}$
    \Ensure  $\avar{removed}: \atype{boolean} \gets \akw{false}$

    \ForAll{$x \in \avar{Domains}(X)$}

        \If{$\akw{there~is~no} y \in \avar{Domains}(Y) \akw{s.t} y.\acall{reduce}(y)$}

            \State $\akw{delete} x \akw{from} \avar{Domains}(X)$
            \State $\avar{removed} \gets true$

        \EndIf

    \EndFor

    \State $\areturn \avar{removed}$

\EndFunction

\end{algorithmic}
\end{center}

\caption{\label{figure:data:ac3} Algorithme AC3.}

\end{figure}

\begin{example}[Illustration de l'algorithme AC3]
\label{example:data:ac3}

Supposons que nous avons le CSP suivant~:
%
\begin{itemize}

\item variables~: $x$, $y$, $z$~;

\item domaines~: $D(X) = (4, 5, 6, 7)$, $D(Y) = (4, 5, 6, 8, 9)$ et $D(Z) = (3,
5, 6, 7, 9)$~;

\item contraintes~: $X = Y$ et $Y = Z$.

\end{itemize}

Sans algorithme AC3, intuitivement, nous commençons par vérifier la contrainte
$X = Y$, c'est~à~dire que toutes les valeurs des domaines $D(X)$ et $D(Y)$ qui
ne sont pas égales seront supprimées. Alors nous obtenons $D(X) = (4, 5, 6)$,
$D(Y) = (4, 5, 6)$ et $D(Z)$ est inchangé. Ensuite, nous vérifions la contrainte
$Y = Z$. Nous obtenons $D(Y) = (5, 6)$, $D(Z) = (5, 6)$ et $D(X)$ est inchangé.
Mais nous avons un problème~: la contrainte $X = Y$ n'est plus satisfaite. Nous
savons alors qu'il faut re-vérifier la contrainte $X = Y$ pour obtenir $D(X) =
(5, 6)$, mais une machine ne le saurait pas nécessairement.

Maintenant, voyons comment opère l'algorithme AC3. Nous avons les mêmes
variables, domaines et contraintes qu'au début de cet exemple. Le graphe du CSP
est le suivant, il représente les liens par les contraintes entre les
variables~:

\fig{6cm}{!}{AC3.tex}

Nous avons un ensemble $Q$ (représentant la variable $\avar{queue}$ de
l'algorithme présenté dans la figure~\ref{figure:data:ac3}) contenant au départ
tous les arcs du graphe, soit $Q = \{(X, Y), (Y, X), (Y, Z), (Z, Y)\}$. Nous
commençons par traiter la première paire de $Q$ après l'avoir supprimée. Nous
traitons alors la contrainte impliquant $X$ et $Y$, soit $x = y$ mais nous
n'allons travailler que sûr $x$.  Ainsi $D(X) = (4, 5, 6)$, $D(Y)$ et $D(Z)$
restant inchangés. La variable $X$ n'a pas d'arcs vers d'autres variables, donc
nous ne modifions pas $Q$. Ensuite, la paire suivante est $(Y, X)$, donc nous
traitons la contrainte $X = Y$ mais cette fois-ci, nous modifions $Y$. Ainsi
$D(Y) = (4, 5, 6)$, $D(X)$ et $D(Z)$ restant inchangés. La variable $Y$ a un arc
vers $Z$ en plus de vers $X$, donc nous devrions ajouter la paire $(Z, Y)$ mais
elle existe déjà dans $Q$, nous ne faisons rien. La paire suivante dans $Q$ est
$(Y, Z)$, donc nous traitons la contrainte $Y = Z$ en modifiant uniquement $Y$.
Ainsi~: $D(Y) = (5, 6)$, $D(X)$ et $D(Z)$ restant inchangés. La variable $Z$ n'a
pas d'arcs vers une autre variable, donc nous ne modifions pas $Q$. La paire
suivante est $(Z, Y)$, donc nous vérifions la contrainte $Y = Z$ en modifiant
$Z$. Ainsi (en rappelant tous les domaines)~: $D(X) = (4, 5, 6)$, $D(Y) = (5,
6)$ et $D(Z) = (5, 6)$. La variable $Y$ a un arc vers la variable $X$, donc nous
ajoutons la paire $(X, Y)$ dans $Q$. Comme $Q$ était vide, cela devient notre
paire suivante que nous traitons. Nous appliquons alors la contrainte $X = Y$ en
modifiant $X$. Ainsi, nous obtenons~: $D(X) = D(Y) = D(Z) = (5, 6)$. La variable
$X$ n'a pas d'arcs vers d'autres variables que $Y$, donc ne modifions pas $Q$.
Ce dernier est vide, donc l'algorithme se termine.

\end{example}

La complexité de cet algorithme est de $\m{O}(mn^3)$ où $m$ est le nombre de
contraintes (d'arcs) et $n$ le nombre de valeurs de la «~plus grande~»
disjonction de domaines réalistes.

Notre implémentation est une variante de l'algorithme AC3. Les variables n'ont
pas des domaines mais une disjonction de domaines réalistes. Rappelons que nous
manipulons cinq sortes de domaines réalistes classés par les interfaces
\code{Constant}, \code{Interval}, \code{Nonconvex}, \code{Finite} et
\code{Enumerable}. Pour chaque sorte de domaine réaliste, nous avons implémenté
une ou des méthodes de raffinement pour permettre la réduction du domaine
réaliste. Par exemple, \code{Nonconvex} attend une méthode \code{discredit} pour
supprimer une valeur du domaine, \code{Interval} attend les méthodes
\code{reduceRightTo} et \code{reduceLeftTo} respectivement pour réduire la borne
droite et gauche etc.

\begin{example}[Propagation et consistence avec un algorithme AC3]

Si nous reprenons notre exemple~\ref{example:data:array_conditions}, nous avons
les contraintes suivantes~:

\begin{align*}
%
S & \geq 0 \\
%
\mathrm{card}(X) & = S \\
%
\mathrm{card}(X) & = \mathrm{card}(Y) \\
%
S & \in [\code{0..5}] \union \{\code{10}\} \\
%
Y & = \hat{H}(X) \\
%
X & = \ingray{X_1 =\;} \code{natural(0, 1)} \\
%
Y & = \ingray{Y_1 =\;} \code{string('a', 'e', 1)} \\
%
\{\code{0}\} & \subseteq X \\
%
\ingray{\hat{H}(\{\code{0}\})} & \ingray{\;\subseteq \{\code{'b'}, \code{'c'}\}
                                 \quad \text{simplifiée en}} \\
%
H(\code{0}) & = \{\code{'b'}, \code{'c'}\}
%
\end{align*}

Nous appliquons l'algorithme. Nous sélectionnons et retirons la première paire
$(S, X)$ avec la contrainte $\mathrm{card}(X) = S$. Nous révisons les valeurs de
$S$~: la cardinalité de $X = \code{natural(0, 1)}$ est égale à $+\infty$, $S$
reste inchangée. Nous ajoutons la paire $(Y, S)$. Nous sélectionnons et retirons
le deuxième paire $(X, Y)$ lié par la contrainte $\mathrm{card}(X) =
\mathrm{card}(Y)$. Nous n'avons rien à réviser pour l'instant~; en revanche,
nous savons que les deux ensembles auront la même cardinalité. Ensuite, nous
sélectionnons et retirons la paire $(Y, S)$ avec la contrainte $\mathrm{card}(Y)
= S$ (que nous avons déduite par propagation). Nous révisons les valeurs de
$S$~: la variable $Y = \code{string('a', 'e', 1)}$ peut contenir 1 à 5 données
maximum (de \code{a} jusqu'à \code{e}), alors nous enlevons 0 et 10 de $S$. Sa
disjonction est constituée de deux domaines réalistes. Le premier est un
intervalle d'entiers (domaine réaliste \code{Boundinteger}) implémentant entre
autre les interfaces \code{Interval} et \code{Nonconvex}. Nous appelons la
méthode \code{discredit} pour supprimer 0 de l'interval. Le second domaine
réaliste est une constante représentant l'entier 10, nous le supprimons
simplement de la disjonction. Ainsi, nous obtenons $S = [\code{1..5}]$. Nous
ajoutons la paire $(X, Y)$ traitée juste avant. Cette fois-ci, nous pouvons
réviser $Y$ avec la contrainte $\mathrm{card}(X) = \mathrm{card}(Y)$ qui peut
maintenir s'écrire $5 = \mathrm{card}(Y)$. Nous révisons alors \code{natural(0,
1)} pour qu'il ne produise que 5 données au maximum. Ensuite, nous prenons une
autre paire et nous vérifions la consistence de $\{\code{0}\} \subseteq X$, ce
qui est le cas car $\code{0} \in \code{natural(0, 1)}$. Pareil avec
$\{\code{'b'}, \code{'c'}\} \subseteq \code{string('a', 'e', 1)}$. Nous n'avons
plus de paires à vérifier, l'algorithme termine.

\end{example}

La consistence vérifie également qu'il n'y a pas de domaine réaliste vide pour
les quatres variables $S$, $H$, $X$ et $Y$. L'objectif est de détecter les
inconsistences au plus tôt. Toutefois, toutes les inconsistences ne sont pas
détectées, ceci étant un problème NP-difficile.

\subsubsection{\inenglish{Labelling}}

Le \inenglish{labelling} (étiquetage) est le processus qui trouve une valeur
pour chaque variable. Il s'applique si les contraintes sont consistentes. L'idée
est de sélectionner une variable et de lui choisir une valeur, puis de
recommencer avec une autre variable etc. Le choix de la variable à sélectionner
dépend de la dépendance entre les variables (les arcs vus dans la section
précédente).

À chaque choix de valeur, la consistence entre les contraintes sont
vérifiées. Si une inconsistence est détectée, alors le processus va revenir en
arrière, nous parlons de \inenglish{backtracking}.

À chaque fois que l'algorithme sélectionne une nouvelle variable, sa disjonction
de domaines réalistes est clonée (dupliquée)~; ainsi lors du
\inenglish{backtracking}, nou restaurons la disjonction précédente pour
«~restaurer l'état~». En même temps, la valeur qui a conduite à l'inconsistence
est discréditer afin de ne pas être re-sélectionnée.

Dans notre implémentation, le choix d'une valeur utilise la caractéristique de
générabilité des domaines réalistes, avec un générateur numérique
pseudo-aléatoire. Afin de faire converger le solveur rapidement vers une
solution, nous utilisons une heuristique qui consiste à choisir une valeur pour
la variable $S$ en premier.  Cette approche permet de déplier les
quantificateurs $\forall$ et $\exists$ (puisque $F_i$ et $T_i$ sont énumérables,
nous ne manipulons que des ensembles finis). Ensuite, le solveur essaye de
calculer les ensembles $X_i$ et $Y_i$.

Quand toutes les variables sont étiquetées, c'est~à~dire que chacune a une
valeur valide, le solveur retourne la solution.

\begin{example}[\inenglish{Labelling}]

Une solution pour l'exemple~\ref{example:data:array_conditions} est~:
%
\begin{pre}
array( \\
    0 => 'c', \\
    1 => 'd', \\
    2 => 'a', \\
    3 => 'e' \\
)
\end{pre}
%
La taille du tableau est de 4, l'index \code{0} a comme valeur \code{'c'} et
toutes les valeurs sont uniques.

\end{example}
