\section{Génération d'objets}
\label{section:data:objects}

Les sections précédentes expliquaient comment générer des données scalaires,
telles que des booléens, des entiers, réels ou chaînes, et aussi des données
plus complexes, telles que des tableaux. Nous allons à présent décrire comment
générer des objets, puisque le paradigme le plus utilisé en PHP5\footnote{Sortie
en juillet 2004.} est l'objet. \\

Un objet est une aggrégation d'attributs et de méthodes. Les deux sont annotés
par des contrats Praspel, respectivement par la clause \ainvariant et les autres
clauses (voir les entités syntaxiques \grule{attribute-clauses} et
\grule{method-clauses} de la figure~\ref{figure:language:grammar_part0}
page~\pageref{figure:language:grammar_part0} et de la
figure~\ref{figure:language:typical_contract}
page~\pageref{figure:language:typical_contract} pour les positionner). Le
domaine réaliste~:
%
$$\code{class($C$)}$$
%
représente une instance d'un objet de type $C$. L'algorithme de la méthode
\code{sample} de ce domaine réaliste est présenté dans la
figure~\ref{figure:data:object}.  La stratégie pour générer un objet est de
l'instancier en appelant son constructeur, et de définir une valeur à ses
attributs. Les valeurs des arguments du constructeur sont générées grâce à sa
précondition. Les valeurs des attributs sont générées grâce aux invariants. Si
un argument du constructeur ou un attribut est lui-même un objet (par conséquent
spécifié avec le domaine réaliste \code{Class}), ce processus est répété
récursivement. Nous contournons l'encapsulation pour accéder aux attributs ou
méthodes protégés et privés en instrumentant le code source.

\begin{figure}

\begin{center}
\begin{algorithmic}

\Function{$\akw{Object\_sample}$}{}

  \Require $\avar{class}: \atype{Class}$
  \Require $\avar{store}: \atype{set\astype{Object}}$
  \Require $\avar{decision}: \lambda ( \avar{class}: \atype{Class} \times \avar{store}: \atype{set\astype{Object}} \rightarrow \atype{boolean} )$
  \Require $\avar{pick}: \lambda ( \avar{store}: \atype{set\astype{Object}} \rightarrow \atype{Object} )$
  \Ensure  $\avar{object}: \atype{Object}$

  \If{$\akw{false} = \avar{decision}(\avar{class}, \avar{store})$}

      \State $\avar{object} \gets \avar{pick}(\avar{store})$

  \Else

      \State $\avar{constructor}: \atype{Method} \gets \avar{class}.\acall{getConstructor}()$
      \State $\avar{object} \gets \avar{constructor}.\acall{call}(\acall{Generate\_data}(\avar{constructor}))$

      \ForAll{$\avar{attribute} \in \avar{class}.\acall{getAttributes}(\avar{object})$}

          \State $\avar{specification}: \atype{Praspel} \gets \avar{attribute}.\acall{getSpecification}()$
          \State $\avar{invariant}: \atype{Clause} \gets \avar{specification}.\acall{getClause}(\astring{'invariant'})$
          \State $\avar{attribute}.\acall{value} \gets \avar{invariant}.\acall{sample}()$

      \EndFor

  \EndIf

  \State $\areturn \avar{object}$

\EndFunction

\end{algorithmic}
\end{center}

\caption{\label{figure:data:object} Caractéristique de générabilité du domaine
réaliste \code{Class}.}

\end{figure}

Afin de gérer les références circulaires (par exemple quand un premier objet a
un attribut qui pointe sur un deuxième objet, qui a lui-même un attribut qui
pointe vers le premier objet), nous appliquons une stratégie inspirée de
Jartege~\acite{Oriat05} et UDITA~\acite{GligoricGJKKM10}. Cette stratégie
consiste à enregistrer les objets créés dans un groupe. Quand un objet est
demandé, nous avons le choix entre en générer un nouveau ou en choisir un dans
le groupe. Ce choix est dynamiquement réalisé en fonction de l'état du groupe~:
plus il contient d'objets, moins il y aura de nouveaux objets générés. Ce choix
est paramétrable par une fonction de répartition fournie par l'utilisateur.
