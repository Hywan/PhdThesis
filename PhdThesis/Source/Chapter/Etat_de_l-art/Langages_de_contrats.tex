\section{Langages de contrats}
\label{section:sota:bisl}

Les spécifications sont souvent utilisées sous la forme de {\strong
contrats}~\acite{Liskov86, Meyer92}. Nous parlons de programmation par contrats
ou encore de \inenglish{Design by Contract} (abrégé DbC). Le terme {\em contrat}
est une métaphore conceptuelle des conditions et des obligations d'un contrat
d'entreprise. Ce paradigme demande au développeur d'écrire une spécification
formelle qui étend la définition classique des types abstraits~\acite{LiskovZ74}
avec les contraintes formelles suivantes~: des {\strong préconditions}, des
{\strong postconditions} et des {\strong invariants}. Ces contraintes formelles
sont aussi appelées des {\strong clauses}. Il est admis de dire qu'une clause
contient des {\strong assertions}.  Les contrats sont apparus dans le langage
Eiffel~\acite{MeyerNM87}, qui alors se basait sur le paradigme de la
programmation orienté objet. Dans un tel contexte, les préconditions et
postconditions portaient sur les méthodes (aussi appelées procédures ou
routines) et les invariants sur les classes (aussi appelés modules). D'autres
usages des contrats ont été faits hors du paradigme de la programmation orienté
objet, par exemple avec des fonctions d'ordre supérieur~\acite{FindlerF02}. Le
\inenglish{Design by Contract} définit les critères de conformité suivants~:

\begin{itemize}

\item si les invariants et la précondition de la méthode sont satisfaits avant
son appel,

\item alors la méthode s'engage à satisfaire les invariants et sa postcondition
après son exécution.

\end{itemize}

Par extension, nous comprenons qu'il existe deux états pour le SUT
(\inenglish{System Under Test})~: avant l'exécution d'une méthode, il se trouve
dans un {\strong pré-état}, et après l'exécution d'une méthode, il se trouve
dans un {\strong post-état}. Pour les langages supportant les exceptions, il
existe un post-état normal et un post-état exceptionnel, dans le cas où une
exception est levée. Par ailleurs, quand le contrat est évalué, nous disons que
c'est un {\strong succès} si aucune clause n'est violée, sinon c'est un {\strong
échec}. Nous verrons par la suite comment est évalué un contrat.

Les langages décrivant une spécification formelle à travers des contrats pour le
comportement des programmes, et utilisés comme annotations, sont appelés des
{\strong langages de contrats}. Il en existe plusieurs~\acite{HatcliffLLMP12}.
La suite de cette partie en décrit quelques-uns.

\paragraph{Eiffel} Le langage Eiffel~\acite{MeyerNM87} est un langage de
programmation orienté objet. Sa plus grosse contribution au domaine a été
l'utilisation importante du \inenglish{Design by Contract} au sein même du
langage (c'est~à~dire sans être placé en annotation), dont les assertions,
préconditions, postconditions et invariants étaient utilisés pour assurer la
conformité du programme. Les contributions originales importantes comprenaient
l'application au paradigme orienté objet et le support des exceptions. L'aspect
objet apporte quelques spécificités, comme les invariants de classes, la gestion
de l'héritage multiple ou la résolution de conflits avec l'héritage multiple
comme le renommage de certains attributs ou méthodes. De même, la gestion du
\inenglish{dynamic binding} (aussi appelé \inenglish{late binding}, représenté
en général par la variable \code{this} qui représente l'instance de l'objet),
demande à ce que la résolution des appels se fasse à l'exécution et non pas à la
compilation. Les exceptions nécessitent également de définir dans quels
contextes elles peuvent être levées et dans quel état se trouve l'objet après la
levée d'une exception. Cette information est d'autant plus importante que le
langage Eiffel permet de ré-exécuter la routine (méthode) qui a déclenché
l'exception. Les préconditions et postconditions sont exprimées avec les
mots-clés \code{require} et \code{ensure}. Le contenu de ces clauses est exprimé
dans le langage Eiffel lui-même. Le langage introduit également le mot-clé
\code{assert} pour exprimer des assertions ou des invariants de boucles.

Les contrats sont compilés avec le reste du programme mais sont placés dans des
fichiers à part. Lors de la construction finale de l'exécutable du programme, il
est possible d'y inclure ou exclure un ensemble de contrats. Le programme peut
être intensivement testé avec tous les contrats puis être distribué avec
seulement un sous-ensemble de ses contrats pour des parties plus critiques.
Moins il y a de contrats présents dans le programme, plus son exécution sera
rapide, mais il est nécessaire de s'assurer de la fiabilité du programme.

\paragraph{JML} Le \inenglish{Java Modeling Language}~\acite{JML} est un langage
de contrats pour Java~\acite{Java}. À l'inverse du langage Eiffel, JML est un
langage d'annotation pour Java. Il utilise les mots-clés \code{requires} et
\code{ensures} pour introduire les préconditions et les postconditions. Une
postcondition exceptionnelle est introduite par le mot-clé \code{signals}. Les
invariants de classes sont introduits par le mot-clé \code{invariant}, et les
invariants de boucles avec le mot-clé \code{loop\_invariant}. JML propose la
clause \code{also} qui permet de raffiner, par un nouveau contrat, des contrats
hérités d'une classe parente. Les expressions JML sont des expressions Java
complétées par des identifiants réservés, comme \aresult, présent uniquement
dans une postcondition normale, et qui fait référence au résultat de la méthode.
De même, nous trouvons la construction \aold{i}, présente uniquement dans une
postcondition, qui fait référence à la valeur de la donnée $i$ dans le pré-état.
JML propose également des constructions logiques, comme la relation
d'implication (\code{==>}), d'équivalence (\code{<==>}) ou des quantificateurs
comme \code{\bslash{}forall} ($\forall$) et \code{\bslash{}exists} ($\exists$).

La distribution officielle de JML fournit principalement deux outils.  Le
premier, le \inenglish{Runtime Assertion Checker}, sert de
\inenglish{monitoring} au programme Java, et le second, JMLUnit, permet de
transformer un contrat en test (nous le détaillons après). Le RAC de
JML~\acite{Cheon02} est un outil permettant de vérifier les contrats JML lors de
l'exécution du programme. Les contrats JML sont traduits en
Java~\acite{Raghavan2005}. La proximité entre ces deux langages est la clé de la
démarche. En effet, la syntaxe des contrats JML est similaire à celle de Java,
et les mots-clés JML spécifiques (par exemple pour les clauses, les
quantificateurs etc.) sont traduits par des structures Java adéquates.
Néanmoins, cette similitude a des limites~: certaines constructions du langage,
comme les quantifications sur les objets, ne sont pas traduisibles et donc ne
peuvent pas être vérifiées.

Le programme Java d'origine est ainsi enrichi par la vérification des contrats
JML. Un inconvénient est que le binaire résultant est plus important, ce qui
engendre des ralentissements à l'exécution. En revanche, si une contrainte n'est
pas vérifiée, une exception spécifique est déclenchée signalant le type de
clause qui n'a pas été vérifiée (invariant, précondition etc.) ainsi que l'état
visible du système au moment de l'exécution. L'utilisation du RAC confère un
moyen direct et efficace de s'assurer que les contrats JML ne sont pas violés
durant une exécution avec les données de l'exécution.

\paragraph{ACSL} Le \inenglish{ANSI/ISO C Specification Language}~\acite{ACSL}
est un langage de contrats pour C~\acite{C}, inspiré principalement du langage
de spécification de l'outil Caduceus~\acite{FilliatreM07}, dédié à la
vérification déductive de propriétés comportementales de programmes C. Caduceus
est lui-même inspiré de JML. Les langages se ressemblent: les mot-clés
\code{requires} et \code{ensures} introduisent respectivement une précondition
et une postcondition. Les invariants sont introduits avec le mot-clé
\code{invariant}. Plusieurs comportements peuvent être exprimés pour une même
méthode avec le mot-clé \code{behavior}. ACSL propose d'autres clauses pour
exprimer des lemmes, des axiomes ou des fonctions logiques.

Contrairement à JML qui utilise Java pour exprimer ces contraintes, ACSL
n'utilise pas C mais un langage plus simple, qui s'inspire toutefois de C (pour
les opérateurs arithmétiques, logiques, les tableaux etc.) afin de ne pas
perturber l'utilisateur. D'autres contraintes sont présentes car intrinsèques au
langage, comme la logique des pointeurs, et d'autres ne sont pas présentes,
comme les exceptions.

Les outils d'ACSL sont basés sur le \inenglish{Framework for Modular Analyses of
C}, abrégé Frama-C~\acite{FramaC}. Ce dernier propose plusieurs analyses de
programmes C. Les outils d'ACSL l'utilisent pour de la vérification statique.
C'est la grande différence avec JML, qui lui repose sur un RAC, donc une
vérification à l'exécution, ce que ne fait pas ACSL. Des extensions à Frama-C
peuvent être installées pour permettre à des outils externes, comme des
assistants de preuve interactifs ou des prouveurs automatiques de théorèmes,
d'aider à la vérification de certaines spécifications.

\paragraph{Spec\#} Le langage Spec\#~\acite{SpecSharp} permet de faire de la
programmation par contrat pour C\#~\acite{CSharp}. Ce langage est très similaire
à ACSL, avec toutefois moins de clauses. Les préconditions sont introduites avec
le mot-clé \code{requires}, les postconditions avec le mot-clé \code{ensures}.
De même, les invariants sont introduits avec le mot-clé \code{invariant}. En
plus des quantificateurs \code{forall}, \code{exists} et \code{exists unique},
Spec\# propose \code{sum}, \code{product}, \code{min} etc. Les contraintes sont
exprimées avec un langage proche de C\#, toujours pour ne pas perturber
l'utilisateur, mais toutes les constructions de C\# ne sont pas présentes,
seulement un sous-ensemble.  Par ailleurs, Spec\# enrichit le langage C\# en lui
ajoutant des constructions permettant de vérifier l'absence ou la possible
présence de valeurs nulles (avec \code{!} ou \code{?}).

Tout comme ACSL, la vérification des contrats se fait statiquement, et non pas
dynamiquement via un RAC. Et contrairement à JML où les contrats sont
transformés en Java, ils sont ici transformés en Boogie~\acite{BarnettCDJL05},
un langage intermédiaire pour encoder des conditions de vérification pour les
langages impératifs et orientés objets. Boogie s'appuye sur des solveurs
externes, notamment Z3~\acite{DeMoura2008}, pour vérifier les contrats. \\

Se distinguent deux grandes familles de vérification de contrats~: statique, à
l'aide de prouveurs maisons et externes, ou dynamique, à l'aide d'un RAC. Notre
contribution, Praspel, appartient à la seconde famille.
Presque chaque langage de programmation répandu a son propre langage de contrats
sauf PHP, d'où l'importance de Praspel. \\

L'approche par les contrats est en général bien acceptée par les développeurs
car elle est proche du code source du programme. Toutefois, selon la complexité
du langage de spécification, elle sera plus ou moins utilisée. Cette complexité
ressentie peut être compensée par d'autres services que peuvent offrir les
contrats, comme le \inenglish{Contract-based Testing}.
