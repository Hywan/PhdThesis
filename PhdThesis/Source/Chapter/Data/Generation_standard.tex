\section{Génération standard}
\label{section:data:random}

Le processus de génération des données dans Praspel est illustré par la
figure~\ref{figure:data:process}. Il se déroule basiquement en 3~étapes,
détaillées ci-après~:
%
\begin{enumerate}

\item sélection d'une variable~;

\item génération d'une donnée pour cette variable, validation et affectation de
cette donnée à cette variable~;

\item sélection de la variable suivante.

\end{enumerate}

Les variables qui nous intéressent sont écrites dans les préconditions (clause
\arequires) ainsi que dans les invariants (clause \ainvariant). Nous notons par
$V$ la liste de ces variables. Les invariants seront traités dans la
partie~\ref{section:data:objects}. Rappelons qu'une précondition est constituée
de déclarations, de prédicats et de qualifications (voir la
figure~\ref{figure:language:grammar_part2}
page~\pageref{figure:language:grammar_part2}). Le processus est contrôlé par une
variable $\tau$ et une constante $\tau_{\mmax}$, détaillées ci-après.  Dans la
figure~\ref{figure:data:process}, $v$ est une variable de $V$. Les annotations
sur les transitions représentent des actions, les annotations soulignées
représentent des gardes.
%
\begin{figure}

\fig{!}{9cm}{Generation_process.tex}

\caption{\label{figure:data:process} Processus de génération des données
pseudo-aléatoire.}

\end{figure}

Le premier état du processus est l'état \code{pick}. Lors d'une génération de
données, nous commençons par traiter les déclarations. Nous choisissons alors
une variable $v$ dans la liste des variables $V$ (représenté par l'opération
$\mpop$ dans la figure) tant que $V$ est non vide, sinon le processus termine.
Les variables sont choisies dans l'ordre décroissant du nombre de contraintes
portées~: la variable avec le plus de contraintes sera la première.

L'état suivant est \code{sample}. Supposons que la variable $v$ est déclarée
par~:
%
$$\code{$v$: $t_1$(\dots) or \dots or $t_n$(\dots)}$$
%
où $t_1, \dots, t_n$ sont $n$ domaines réalistes. Nous allons choisir
pseudo-aléatoirement le domaine réaliste $t_c$ avec $1 \leq c \leq n$. Sur ce
domaine réaliste $t_c$, nous allons utiliser sa caractéristique de générabilité,
c'est~à~dire appeler sa méthode \code{sample} pour générer une donnée qui
appartient à ce domaine.

Au sein d'un domaine réaliste, quand une donnée est générée, elle est aussitôt
confrontée à sa caractéristique de prédicabilité, c'est~à~dire sa méthode
\code{predicate}, afin de savoir si la donnée est valide ou non. C'est l'état
\code{predicate} de la figure.

Si la précondition a des prédicats (avec la construction \apred{p}), nous avons
deux cas. Si toutes les variables du prédicat ont une valeur, alors le prédicat
sera évalué. Si des variables n'ont pas de valeur, l'évaluation du prédicat sera
repoussée. Si malgré cela, certaines valeurs de variables sont toujours
manquantes (par exemple si elles n'ont pas de domaines réalistes associés), le
prédicat sera évalué à faux avec une erreur spécifique.

Si, pendant ces étapes, la propriété de prédicabilité d'un domaine réaliste ou
un prédicat invalide une donnée, il y a un {\strong rejet}. Suite à un rejet,
une donnée est re-générée. Un nombre maximum $\tau_{\mmax}$ de re-générations
$\tau$ est fixé afin d'éviter des générations et des rejets en boucle trop
importants et donc trop longs. À chaque génération, $\tau$ est incrémenté de 1.
La valeur de $\tau_{\mmax}$ est paramétrable. \\

Mais il n'y a pas que le choix d'un domaine réaliste parmi une disjonction qui
est pseudo-aléatoire. Par défaut, un domaine réaliste génère une donnée de
manière pseudo-aléatoire.  Si nous écrivons \code{boundinteger(7, 42)}, alors un
entier de l'intervalle $[7; 42]$ sera généré pseudo-aléatoirement. Nous avons
précisé dans la partie~\ref{subsection:language:realdom:implementation} qu'une
méthode \code{sample} d'un domaine réaliste reçoit un générateur numérique en
argument. Le seul générateur actuel est pseudo-aléatoire, basé sur l'algorithme
de Mersenne Twister~\acite{MatsumotoN98}. Il permet de générer des entiers et
des réels. Un exemple d'utilisation a été montré dans la
figure~\ref{figure:language:boundinteger}
page~\pageref{figure:language:boundinteger}.

C'est aussi cette méthodologie qu'utilise le domaine réaliste \code{String} pour
générer des chaînes de caractères. Cette approche pseudo-aléatoire pose
rapidement des problèmes. Le nombre de valeurs possibles pour \code{string('a',
'z', 10)}, c'est~à~dire pour une chaîne de 10 caractères entre \code{a} et
\code{z} en minuscules, est de $26^{10}$, soit environ $1.412e^{14}$ valeurs
différentes. La probabilité qu'une de ces valeurs détecte une erreur dans notre
programme est extrêmement faible. C'est un problème inhérent à l'approche
(pseudo-)aléatoire. Cependant, l'ensemble des valeurs générables peut être
considérablement réduit si la donnée est mieux spécifiée~: plus la donnée pourra
être spécifiée avec précision, plus il sera possible de générer des données
pertinentes, c'est~à~dire capables de détecter des erreurs.

Dans les parties suivantes, nous allons nous intéresser à mieux spécifier et
mieux générer des tableaux, des chaînes de caractères et des objets.
