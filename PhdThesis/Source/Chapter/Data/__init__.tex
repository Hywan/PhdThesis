\chapter{Génération de données de test}
\label{chapter:data}

\mminitoc

Le chapitre précédent présentait le langage Praspel. À partir de ce langage,
nous voulons faire du \inenglish{Contract-based Testing}, c'est~à~dire que les
préconditions sont utilisées pour générer des données de test. Ces données sont
spécifiées avec des domaines réalistes. Nous allons nous appuyer sur les
caractéristiques de prédicabiblité et de générabilité des domaines réalistes
pour rendre accessible différentes méthodes du test au sein du langage Praspel.
Nous voulons traiter tous les types de données manipulés au quotidien par les
utilisateurs de PHP.

Ce chapitre présente les différentes techniques utilisées par les domaines
réalistes pour valider et générer des données de test de plusieurs natures~: les
booléens, les entiers et les réels dans la partie~\ref{section:data:random}, les
tableaux dans la partie~\ref{section:data:arrays}, les chaînes de caractères
dans la partie~\ref{section:data:strings} et enfin les objets dans la
partie~\ref{section:data:objects}.

\require{Chapter/Data/Generation_standard.tex}
\require{Chapter/Data/Generation_a_partir_de_solveur_pour_les_tableaux.tex}
\require{Chapter/Data/Generation_a_partir_de_grammaire_pour_les_chaines_de_caracteres.tex}
\require{Chapter/Data/Generation_d-objets.tex}
\require{Chapter/Data/Synthese.tex}
