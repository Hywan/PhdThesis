\section{Perspectives}
\label{section:conclusions:perspectives}

Les perspectives qui se dégagent de ces travaux se classent en trois parties. La
première concerne l'utilisation d'autres méthodes de test dans Praspel. La
deuxième partie concerne des améliorations possibles sur le langage. Enfin, la
dernière partie offre une réflexion sur l'avenir des travaux présentés dans ce
mémoire.

\subsection{Inclure de nouvelles méthodes de test}

Grâce aux domaines réalistes et à la conception modulaire et extensible des
outils de Praspel, il est possible d'assembler plusieurs méthodes de test. Les
expérimentations ont montré certaines limites du langage et des outils. Nous
pensons qu'il est pertinent de s'intéresser aux méthodes suivantes.

\subsubsection{Test aux limites}

Le test aux limites permet de détecter des erreurs usuelles rapidement et
efficacement. Nous avons dit que les domaines réalistes utilisaient un
générateur numérique aléatoire et uniforme pour générer des données. Nous
pensons qu'il est pertinent de réfléchir à une manière de spécifier les limites
des domaines réalistes. Par exemple, avec l'intervalle $[-7; 42]$, les limites
extrêmes sont $-7$, $-6$, 41 et 42, et les limites intermédiaires sont $-1$, 0
et 1. En revanche, définir les limites d'un tableau, d'une chaîne de caractères
ou d'un objet n'est pas aussi trivial et mérite plus d'attention. Une fois ces
limites connues, nous pourrions utiliser un générateur numérique aux limites
dans les domaines réalistes pour produire de nouvelles données aux limites.

\subsubsection{Scénarios et propriétés temporelles}

Dans le contexte du \inenglish{Scenario-based Testing}, nous nous intéressons
aux travaux de \acitei{Castillos13}. Ces travaux s'inspirent des patrons de
propriétés décrits par \acitei{DwyerAC99}. Ces patrons permettent de définir une
propriété temporelle comme un motif qui doit être satisfait à l'intérieur d'une
portée (une portion d'exécution du système). Ils permettent donc d'exprimer
simplement un large panel de propriétés temporelles avec un ensemble restreint
de constructions. Un langage~\acite{CastillosDJKT13} basé sur ces patrons a été
créé. Ce langage comprend des variantes aux motifs et aux portées afin
d'améliorer l'expressivité et corriger certaines incohérences des patrons
initiaux. De nouveaux critères de couverture ont alors été définis et permettent
de vérifier plusieurs propriétés temporelles.

La démarche de~\acitea{Castillos13} est similaire à la nôtre~: proposer un
langage et définir des critères de couverture pour obtenir des objectifs de
test, mais ces objectifs sont différents. Cependant, puisque ce langage est très
simple et permet d'exprimer un grand nombre de propriétés temporelles, nous
pensons que ce langage pourrait être une extension à Praspel. Ainsi, en plus des
attributs et des méthodes, les classes pourraient être annotées par ces
propriétés. Nous les exploiterions pour générer de nouveaux préambules de test,
des objets plus réalistes, des suites de tests fonctionnels ou alors pour
vérifier des propriétés temporelles.

\subsubsection{\inenglish{Grey-box} avec le \inenglish{Search-based Testing}}

Dans les expérimentations, nous avons vu certaines limites de Praspel. La
méthode du \inenglish{Search-based Testing}~\acite{McMinn04} utilise des
métaheuristiques pour générer automatiquement des données de test à partir du
code source. Cette méthode travaille en boîte blanche. Le SUT est en premier
lieu exécuté avec des données aléatoires. Des fonctions de \inenglish{fitness}
(fonctions objectifs) placées sur les points de choix du SUT permettent
d'évaluer la «~distance~» entre les données courantes et les données nécessaires
pour activer ou désactiver ce point de choix. En utilisant la programmation
génétique, ces données initiales sont mutées pour former une nouvelle population
de données. Des algorithmes de sélection vont réduire cette population. Chaque
donnée de cette population sera ensuite réutilisée comme donnée de test pour
exécuter à nouveau le SUT. Les fonctions de \inenglish{fitness} vont produire de
nouveaux résultats qui serviront à guider l'évolution de la population de
données. Par exemple, avec la condition \code{x == y}, la fonction de
\inenglish{fitness} sera $\mathrm{abs}(\code{x} - \code{y})$. Quand cette
fonction tend vers 0, alors la condition est activée.

Les suites de tests générées par Praspel sont basées sur les critères de
couverture des contrats. Les données de test, spécifiées majoritairement avec
les domaines réalistes, sont générées aléatoirement. Si le résultat des
fonctions de \inenglish{fitness} pouvait être remonté dans les domaines
réalistes pour mieux guider les nouvelles générations de données, nous pensons
que nous pourrions détecter plus d'erreurs, ou du moins, couvrir le code source
plus efficacement et plus rapidement. En effet, les domaines réalistes ont
plusieurs interfaces (comme \code{Interval}, \code{Finite}, \code{Nonconvex}
etc.). Ces interfaces imposent l'implémentation de méthodes permettant de
contraindre ou manipuler les données contenues dans le domaine réaliste. Il doit
être possible de transposer les résultats des fonctions de \inenglish{fitness}
dans les domaines réalistes.

Cette approche, d'une part, permettrait d'exploiter le code source au lieu de
rester en boîte noire (nous serions alors en boîte grise), et, d'autre part, ne
compliquerait pas le langage en y ajoutant de nouvelles constructions, mais
offrirait plus de services à l'utilisateur. Ce dernier n'aurait aucun effort à
produire à part mettre à jour ses outils.

\subsubsection{Solveurs arithmétiques sur les entiers et réels}

Quand des entiers sont manipulés, il est courant de vouloir spécifier des
contrain\-tes arithmétiques comme $\code{i} < \code{j} + 1$. Actuellement,
Praspel n'a aucun solveur arithmétique sur les entiers et les réels. Malgré la
pertinence d'une telle fonctionnalité, nous n'avons pas eu le temps de la
développer. Il serait alors nécessaire d'étudier le domaine, de voir comment
nous pouvons y contribuer, et proposer une solution dans Praspel.

\subsection{Améliorer le langage}

Le langage Praspel n'est pas complet. Certaines données pourraient être encore
mieux spécifiées. Nous sommes toujours dans cette démarche d'analyser ce que les
développeurs écrivent dans les contrats et d'améliorer le langage en
conséquence. Nous avons relevé deux points importants, puis un dernier point que
nous aurions aimé traiter mais le temps nous a manqué.

\subsubsection{De nouveaux domaines réalistes}

Durant l'expérimentation, les ingénieurs de test ont trouvé notre approche
pertinente. Toutefois, la bibliothèque standard des domaines réalistes n'était
pas suffisante. Les ingénieurs ont demandé la création de plus de domaines
réalistes représentant des données métiers, comme des numéros de comptes
bancaires, des numéros de téléphones etc. Les ingénieurs ont créé certains de
ces domaines réalistes par eux mêmes mais ils n'ont pas eu le temps de tous les
créer. Ce ne sont pas des perspectives scientifiques mais industrielles.
Toutefois, elles participent à l'adoption de nos outils dans l'industrie.

\subsubsection{Pureté des méthodes}

À l'origine de Praspel, il existait la clause \ais. Elle permet de qualifier la
catégorie de la méthode. Le seul qualificatif était \code{pure}, pour contrôler
la mutabilité de l'environnement. En effet, une méthode peut modifier ou pas son
environnement lors de son exécution. Si elle ne le modifie pas, alors nous
parlons d'une méthode pure, sinon nous parlons d'une méthode impure (par
défaut). D'une manière plus générale, nous parlons de \inenglish{frame
condition}~: quelle partie de l'environement est modifiable ou pas.

Nous pensons qu'il serait pertinent de détecter la pureté d'une méthode. De
même, une méthode pure peut être utilisée dans un contrat car elle ne modifie
par l'état du SUT. Ainsi, détecter la pureté d'une méthode peut permettre
d'écrire des contrats plus élaborés.

\subsection{Vers des outils industriels~?}

Il nous a semblé pertinent de développer nos travaux de thèse dans des projets
\inenglish{open-source} et libres avec une communauté, des utilisateurs et des
contributeurs. Cela nous a été bénéfique mais, comme l'ont fait remarquer les
ingénieurs de test lors de notre expérimentation, il est nécessaire d'écrire
beaucoup de documentation pour permettre une utilisation plus importante de nos
travaux au sein de l'industrie.

\subsubsection{Conférences industrielles}

Toutefois, les communautés de Hoa et d'atoum ont déjà fait plusieurs conférences
industrielles pour la promotion du test, de la qualité logicielle et de Praspel.
Nous pouvons citer~:
%
\begin{itemize}

\item 5 et 6~juin~2012, au ForumPHP à Paris, Ivan Enderlin (auteur de ce mémoire
et créateur de Hoa) et Frédéric Hardy (créateur d'atoum) ont présenté une
conférence en duo sur l'«~Anatomie du test~»~;

\item 28~octobre~2013, soirée AFUP à Lyon, Ivan Enderlin et Julien Bianchi
(contributeur de Hoa et d'atoum) ont présenté une conférence en duo sur Praspel
et atoum~;

\item 23 et 24~juin~2014, au PHPTour à Lyon, Ivan Enderlin et Julien Bianchi ont
présenté une conférence sur Praspel.

\end{itemize}
%
Praspel a également été cité dans d'autres conférences sur la qualité logicielle
et le test.

\subsubsection{Transfert technologique}

Il est difficile de mesurer l'impact de nos contributions et l'usage de nos
algorithmes dans l'industrie. Les quelques chiffres que nous pouvons obtenir
sont le nombre d'installations quotidiennes des bibliothèques que nous avons
créées au sein de Hoa ou auxquelles nous avons contribué.

La bibliothèque \code{Hoa\bslash{}Compiler} contient notre compilateur de
compilateurs $LL(\star)$, le langage de description de grammaire PP et nos
algorithmes de génération de données à partir de grammaires. Elle a été
installée plus de 9000~fois depuis sa création, soit 15~installations en moyenne
par jour. La bibliothèque \code{Hoa\bslash{}Regex} contient la grammaire des
PCRE au format PP et l'algorithme de génération isotropique. Elle est installée
en moyenne 0.5~fois par jour. Nous avons contribué à la bibliothèque
\code{Hoa\bslash{}Math} pour des fonctions combinatoires. Des contributeurs ont
utilisé notre compilateur de compilateurs pour évaluer des formules
arithmétiques à la volée contenues dans des chaînes de caractères. Cette
bibliothèque est installée en moyenne 14~fois par jour. La bibliothèque
\code{Hoa\bslash{}Praspel} est installée en moyenne 1~fois par jour.  D'autres
outils utilisant nos contributions sur les grammaires sont installés entre 3 et
6~fois par jour en moyenne et comptent des milliers d'utilisateurs. 

Comme nous avons créé une extension qui fait le pont entre atoum et Praspel, et
que atoum est installé plus de 100~fois par jour, nous espérons que Praspel va
rapidement se propager et être utilisé. Durant ces travaux de thèse, nous
n'avons pas alloué de temps spécifiquement pour la promotion de Praspel auprès
de l'industrie. Les implémentations des outils sont terminées depuis peu et sont
mis à la disposition de la communauté PHP depuis mars 2014. Au vu des résultats
de l'expérimentation, nous espérons que les communautés de Hoa et d'atoum
sauront promouvoir Praspel. Nous allons les rejoindre et participer à cet effort
au maximum car nous pensons que Praspel et le lien qui s'est créé entre nous et
ces communautés sont bénéfiques pour la recherche.
