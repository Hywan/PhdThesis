\chapter{Introduction}
\label{chapter:introduction}

\minitoc

Les travaux présentés dans cette thèse sont liés à des activités relatives à la
validation et à la vérification de programmes PHP à travers un nouveau langage
de spécification accompagné de ses outils. Ces activités s'articulent autour de
trois axes majeurs~: langage de spécification, génération de données et
génération automatique de tests.

Informellement, nous pouvons voir un test comme étant composé de trois parties~:

\begin{enumerate}

\item un Système Sous Test (\inenglish{System Under Test}, abbrégé {\strong
SUT})~;

\item des {\strong données de test} pour exécuter ce SUT~;

\item un {\strong oracle}, capable de nous calculer le verdict du test~: est-ce
que le produit de l'exécution et l'état du SUT après son exécution sont ceux
attendus ou non~?

\end{enumerate}

Aujourd'hui, dans le monde industriel, la majorité des tests sont écrits
{\strong manuellement}, par des ingénieurs de test. Un SUT leur est fourni, ils
l'exécutent avec des données de test et calculent eux-même le verdict du test.
Dans le cas des tests {\strong automatisés}, l'exécution et le calcul du verdict
est fait par la machine. C'est le rôle qui incombe aux frameworks xUnit, comme
JUnit~\acite{JUnit} pour Java, atoum~\acite{atoum} ou PHPUnit~\acite{PHPUnit}
pour PHP, CUnit~\acite{CUnit} pour C etc. Ce que font les ingénieurs de test
peut se résumer à écrire une spécification informelle exécutable. En effet, ils
transforment le cahier des charges en tests afin de vérifier et valider le
programme. Toutefois, cela reste informel. Et surtout, c'est une tâche
laborieuse qui peut revenir cher \footnote{TODO, environ la moitié du prix d'un
projet}. Et pourquoi ne pas écrire une spécification formelle, et la rendre
exécutable~?  C'est~à~dire d'utiliser un langage formel pour décrire le
fonctionnement du programme, puis dériver, à partir de cette spécification, des
tests qui soient exécutables et qui vont vérifier ou valider plusieurs aspects
de cette spécification. \\

Nous nous intéressons à des programmes écrits en PHP~\acite{PHP} pour plusieurs
raisons. Comparés à des langages plus traditionnels, comme C ou Java, les
langages de scripts accélèrent le processus de développement grâce à la
flexibilité qu'ils offrent avec le typage dynamique, faible et le mélange de
paradigmes. Cependant, cette flexibilité rend plus difficile la compréhension du
comportement de certains programmes, tout comme il est plus difficile de
s'assurer que le programme n'est pas affecté par une modification (nous parlons
de {\strong régressions}). Ce type de langage nous apporte des problématiques
encore peu ou pas abordées dans l'état de l'art actuel. De plus, aucun travaux
ne considère PHP pour la génération de données de test ou la génération de
tests. De même, aucun langage de spécification dédié à PHP n'existait avant les
travaux de cette thèse. Par ailleurs, plus de 85\% du Web fonctionne avec PHP.
Cela implique qu'il y a un marché important avec d'immenses besoins~; que ce
soit des sites de commerces\footnote{Etsy, \url{https://etsy.com/}.}, de
banques, d'assurances, de réseaux sociaux\footnote{Facebook,
\url{https://facebook.com/}.}, de moteurs de recherche\footnote{Yahoo,
\url{https://yahoo.com/}.}, d'encyclopédie\footnote{Wikipedia,
\url{https://wikipedia.org/}.}, gouvernementaux\footnote{France, Belgique,
Suisse, USA, Canada etc.} et d'autres. Nous tenons à préciser que nous nous
intéressons à du test pour PHP et non pas du test pour le Web à travers un
client HTML comme un navigateur. Nous considérons PHP uniquement, tout en
orientant nos travaux vers les problématiques du Web. \\

Cette thèse s'intéresse à la conformité entre un programme et sa spécification,
à travers la création d'un nouveau langage de spécification simple, pragmatique
pour le Web et unifiant plusieurs théories. Chaque théorie introduit se verra
enrichie pour la rendre compatible avec les besoins liés au Web.

\section{Contributions}
\label{section:introduction:contributions}
