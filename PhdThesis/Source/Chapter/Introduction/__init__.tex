\chapter{Introduction}
\label{chapter:introduction}

\mminitoc

\mylettrine[findent=-.5em, nindent=.6em, slope=.6em]{A}{ujourd'hui, nous sommes
dans l'ère} du tout numérique. Le nombre de logiciels produits et utilisés
chaque jour est énorme. Nous utilisons de plus en plus de machines dans notre
quotidien~: téléphones, tablettes, ordinateurs, télévisions, appareils
électroménagers… Une très grande majorité d'entre eux sont connectés à un
réseau, le plus souvent Internet. Ce dernier nous permet d'accéder à des
milliers de millions de sites Web. Plus de 85\% de ces sites Web sont développés
en PHP. La majorité de ces programmes ont besoin d'être vérifiés et validés.

Les travaux présentés dans ce mémoire portent sur la validation de programmes
PHP à travers un nouveau langage de spécification, accompagné de ses outils. Ces
travaux s'articulent selon trois axes majeurs~: langage de spécification,
génération automatique de données de test et génération automatique de tests
unitaires. \\

Ces travaux ont été réalisés au Département d'Informatique des Systèmes
Complexes (abrégé DISC\footnote{Voir \url{http://disc.univ-fcomte.fr/}.}) de
l'institut CNRS Femto-ST\footnote{Voir \url{http://femto-st.fr/}.}, mais
également au Laboratoire lorrain de Recherche en Informatique et ses
Applications (abrégé LORIA\footnote{Voir \url{http://www.loria.fr/}.}) de
l'Institut National de Recherche en Informatique et en Automatique (abrégé
INRIA\footnote{Voir \url{http://inria.fr/}.}). J'ai été accueilli en
septembre~2011 au sein de l'équipe VErification, SpécificatiON, Test et
Ingénierie des mOdèles (abrégé VESONTIO) pour le DISC, et de l'équipe
\inenglish{Combining ApproacheS for the Security of Infinite state Systems}
(abrégé CASSIS) pour le LORIA. Ce mémoire a été financé à travers
Squash\footnote{Voir \url{http://squashtest.org/}.}, un projet
\inenglish{open-source} visant à structurer et industrialiser les activités du
test fonctionnel. Dans le cadre de ce projet, plusieurs partenaires contribuent
pour~: développer un outillage \inenglish{open-source} mature et innovant,
proposer une méthodologie \inenglish{open-source} documentant l'ensemble des
processus de la qualification fonctionnelle et définir des normes et modèles
permettant de mieux mesurer l'activité du test fonctionnel. Les travaux
présentés dans ce mémoire ne s'inscrivent pas dans la même thématique que le
projet Squash. Ces travaux se situent en amont de la chaîne de production. Le
contexte scientifique de ce mémoire et les problématiques qui y sont développés
sont liés aux savoir-faire et thématiques des équipes VESONTIO et CASSIS. Les
deux grandes thématiques sont la modélisation pour vérifier et valider, et la
vérification et la validation symbolique et à contraintes. \\

Informellement, un test est exécuté sur un Système Sous Test (\inenglish{System
Under Test}, abrégé {\strong SUT}) et est composé de deux parties~:
%
\begin{enumerate}

\item des {\strong données de test} pour exécuter ce SUT~;

\item un {\strong oracle}, permettant d'établir le verdict du test~: est-ce que
le produit de l'exécution et l'état du SUT après son exécution sont ceux
attendus ou non~? Les valeurs du verdict sont~: succès, échec ou inconclusif.

\end{enumerate}

Aujourd'hui, dans le monde industriel, la majorité des tests sont écrits
{\strong manuellement}, par des ingénieurs de test. Un SUT leur est fourni, ils
l'exécutent avec des données de test et établissent eux-mêmes le verdict du
test. Dans le cas des tests {\strong automatisés}, l'exécution du SUT et de
l'oracle sont faites par la machine. C'est le rôle qui incombe aux
\inenglish{frameworks} «~xUnit~», comme JUnit~\acite{JUnit} pour Java,
atoum~\acite{atoum} ou PHPUnit~\acite{PHPUnit} pour PHP, CUnit~\acite{CUnit}
pour C etc. Le travail des ingénieurs de test peut se résumer à transformer le
cahier des charges en tests afin de valider le programme. Cela reste toutefois
informel, et surtout, une tâche laborieuse. Pourquoi alors ne pas écrire une
spécification formelle et la rendre exécutable~? C'est~à~dire utiliser un
langage formel pour décrire le fonctionnement du programme, puis dériver, à
partir de cette spécification, des tests qui soient exécutables et qui vont
vérifier ou valider plusieurs aspects de cette spécification.

\section{\inenglish{Behavioral Interface Specification Language}}

Le terme {\strong spécification} signifie généralement une description précise
des comportements et propriétés d'un artéfact. {\strong Vérification} signifie
prouver qu'une implémentation (un programme) satisfait une spécification
particulière dans toutes les exécutions et configurations possibles.
Généralement, une telle preuve est accomplie à l'aide de raisonnements
statiques, c'est~à~dire uniquement par analyse du programme. Lorsqu'un
raisonnement statique n'est pas possible ou suffisant, nous pourrons utiliser
une {\strong validation} dynamique, pour tester l'implémentation, c'est~à~dire
que nous exécuterons le programme dans des configurations différentes (qui sont
des sous-ensembles de toutes les configurations possibles). Une validation
dynamique est un test. Elle peut être accompagnée d'un \inenglish{Runtime
Assertion Checker}, abrégé RAC~\acite{Geller76}, qui permet de valider une
spécification lors d'une exécution du programme.

Un {\strong langage de spécification} formel est une notation précise pour
décrire les comportements et propriétés d'un programme. Les {\strong notations
formelles} aident à rendre les spécifications non-ambiguës, moins dépendantes
des normes culturelles\footnote{Par exemple, une pinte représente $473$ml aux
États-Unis d'Amérique, mais $586$ml en Angleterre.}, et ainsi moins sujettes à
une mauvaise interprétation. Les langages de spécification sont conçus pour
répondre à des problèmes généraux ou spécialisés. Le plus souvent ils sont
spécialisés, en fonction de la catégorie de problèmes ou langages qu'ils
visent~: ensembliste, logique, typage statique et fort, ou dynamique et faible,
orienté objet ou fonctionnel etc.

Par ailleurs, puisqu'une notation est formelle, elle peut être exploitée par une
machine pour y appliquer un traitement automatique. Les spécifications formelles
peuvent alors aider à {\strong automatiser le test}, autant pour générer les
données de test~\acite{BernotGM91, KorelA98, SankarH94, Jalote92} que pour
calculer le verdict du test~\acite{CheonL02}. C'est dans ces deux directions que
nous allons nous tourner.

Les spécifications peuvent documenter les {\strong interfaces} d'un programme.
Par interfaces, il faut comprendre les parties du programme permettant de
manipuler les données. Une catégorie de ces spécifications est appelée {\strong
langage d'annotations}. De tels langages peuvent jouer un rôle central dans la
programmation. Par exemple, ils facilitent la maintenance du code, qui n'est
généralement pas suffisament expressif pour faire comprendre au développeur ses
tenants et ses aboutissants. Par ailleurs, lors d'une phase de
\inenglish{debugging}, les annotations permettent de localiser plus facilement
l'erreur. Par exemple, si une propriété est violée, il sera plus facile à
l'utilisateur de situer l'erreur dans un comportement spécifique. Dans le cas où
nous travaillons en {\strong boîte blanche} (\inenglish{white-box}),
c'est~à~dire où nous avons accès au code source du programme, les annotations
peuvent faire partie de la grammaire du langage, ou alors, elles sont écrites
dans des commentaires. Dans le cas où nous travaillons en {\strong boîte noire}
(\inenglish{black-box}), c'est~à~dire où nous n'avons pas ou partiellement accès
au code source, les annotations peuvent parfois être externes au programme, par
exemple sur un langage de définition d'interface (\inenglish{Interface
Definition Language}, abrégé IDL) ou sur un modèle, ce qui nous rapprocherait du
\inenglish{Model-based Testing}, abrégé MbT~\acite{Utting07}. La différence
entre boîte noire et boîte blanche peut parfois être plus
subtile~\acite{Gaudel11}~: nous considérons que nous sommes en boîte noire quand
nous n'exploitons pas la structure du code, basiquement le contenu des fonctions
et méthodes, sinon, nous sommes en boîte blanche.

Les langages d'annotations sont des \inenglish{Behavioral Interface
Specification Language}, abrégé BISL. C'est dans cette catégorie de langages que
se situent nos contributions.

\section{Contexte et problématique}

Ce mémoire s'intéresse à la conformité entre un programme et sa spécification, à
travers la création d'un nouveau langage de spécification simple, pragmatique
pour le Web et les développeurs, et assemblant plusieurs méthodes du test.
Chaque méthode introduite se verra enrichie pour la rendre compatible avec les
besoins liés au Web. Les parties suivantes expliquent notre motivation, le
contexte et les enjeux.

\subsection{\inenglish{PHP: Hypertext Preprocessor}}

PHP~\acite{PHP} se définit comme étant un «~langage de script universel
populaire qui est particulièrement adapté pour le développement Web~». PHP se
définit également avec les adjectifs suivants~: «~rapide, flexible et
pragmatique~». C'est aussi un langage multiparadigmes. Il est par exemple
procédural avec des fonctions nommées, orienté objet avec des classes (héritage
vertical), des interfaces (typage) et des traits (héritage horizontal), ou
encore fonctionnel avec des fonctions anonymes ($\lambda$) et des fermetures
($\overline{\lambda}$). Son système de typage est dynamique et faible,
c'est~à~dire que les types sont non-déclarés et qu'il est possible de transtyper
les données. PHP est un langage de script interprété. Il existe plusieurs
interpréteurs, appelés machines virtuelles. Nous pouvons voir PHP comme un
langage «~glue~» (comme la majorité des langages de script), c'est~à~dire qu'il
communique avec à peu près tout et est au centre de beaucoup de logiciels. La
syntaxe du langage est inspirée de C, C++, Java ou encore Perl.  Depuis
plusieurs années, le langage connaît des améliorations importantes afin de faire
oublier les erreurs de conception passées qui lui sont trop souvent associées.
Nous nous intéressons à des programmes écrits en PHP pour plusieurs raisons.

Comparés à des langages plus traditionnels, comme C ou Java, les langages de
scripts accélèrent le processus de développement grâce à la flexibilité qu'ils
offrent avec un typage dynamique, faible et un mélange de paradigmes. Cependant,
cette flexibilité rend plus difficile la compréhension du comportement de
certains programmes, tout comme il est plus difficile de s'assurer que le
programme n'est pas affecté par une modification (il est alors question de
régression).

Par ailleurs, plus de 85\% du Web fonctionne avec PHP. Cela implique qu'il y a
un marché important avec d'immenses besoins~; que ce soient des sites de
commerces (comme Etsy\footnote{Voir \url{https://etsy.com/}.}), de banques,
d'assurances, de réseaux sociaux (comme Facebook\footnote{Voir
\url{https://facebook.com/}.}), de moteurs de recherche (comme
Yahoo!\footnote{Voir \url{https://yahoo.com/}.}), d'encyclopédies (comme
Wikipedia\footnote{Voir \url{https://wikipedia.org/}.}), gouvernementaux
(France, Belgique, Suisse, USA, Canada etc.) et autres. Tous ces domaines ont
besoin de qualité logicielle.

\subsection{Test unitaire}

Plusieurs travaux cherchent à améliorer la sécurité des programmes écrits en PHP
à l'aide d'analyses statiques~\acite{YuAB10, BalzarottiCFJKKV08, WassermannS08,
WassermannS07, Xie06} du code source, des flots des données pour trouver des
chaînes de caractères malicieuses, des injections SQL etc. Toutefois, un grand
nombre des solutions proposées souffrent de limitations
majeures~\acite{HauzarK12} comme une détection faible des erreurs, un taux de
faux-positifs important ou même un support faible des constructions du langage~;
l'aspect dynamique et multiparadigmes du langage n'aidant pas. D'autres travaux
testent PHP à un niveau plus élevé, par exemple en testant le site Web écrit en
HTML directement~\acite{ArtziKDTDPE10, KiezunGJE09, McAllisterKK08, Benedikt02}.
Ces travaux se situent du côté client et observent les résultats produits par
PHP côté serveur. Ces travaux sont intéressants pour, d'une part, tester la
sécurité, et d'autre part, tester des programmes existants.

Toutefois les méthodes de développement ont énormément évolué ces dernières
années avec la qualité logicielle~\acite{LepineSF09}. Les horribles programmes
tels ceux que nous avons lus ou écrits en PHP tendent à disparaître. De nouveaux
\inenglish{frameworks} ou de nouvelles bibliothèques PHP ont fait leur
apparition et mettent en avant l'usage intensif de bonnes pratiques pour le
développement de programmes maintenables et testables. Des
\inenglish{frameworks} de test de plusieurs natures ont également fait leur
apparition et un programme non testé ne sera même plus installé. 

En outre, aucun travail de recherche n'a fait de propositions pour spécifier des
programmes écrits en PHP. Pourtant cela permettrait d'éviter en amont une
multitude de problèmes. Enfin, aucun travail ne considère PHP pour la génération
automatique de données de test ou encore la génération automatique de tests
unitaires.

Nous profitons de cette tendance d'amélioration de la qualité logicielle et de
l'absence de travaux dans le domaine des langages de spécification pour
introduire un langage de spécification dans PHP. Nos travaux s'inscrivent au
niveau des tests unitaires et au niveau du programme PHP lui-même, c'est~à~dire
que nous ne tenons pas compte du contexte d'utilisation du programme
(client-serveur, client seul, démon etc.).

\subsection{Langage de spécification}

Comme PHP est un langage «~glue~», il manipule des données de toutes sortes.
C'est une première contrainte. De plus, les utilisateurs de PHP sont aussi bien
des débutants que des experts. La palette des utilisateurs est donc très large.
C'est un avantage et un inconvénient du langage~: il est simple à utiliser et
permet d'obtenir des résultats rapidement mais il permet aussi de faire des
choses très poussées. Toutefois, tous ces utilisateurs suivent la tendance
d'amélioration de la qualité logicielle. C'est une deuxième contrainte. Enfin,
le langage PHP est très pragmatique, c'est ce qui le rend si facile d'accès. Il
est possible d'écrire un programme PHP en très peu de lignes de code sans
connaître et installer des dizaines de bibliothèques ou d'autres outils. Les
utilisateurs du langage sont habitués à ce pragmatisme. C'est une troisième
contrainte.

Si nous proposons un langage de spécification pour PHP, il doit impérativement
tenir compte de ces trois contraintes. Il doit être simple pour être utilisé et
compris par la majorité des utilisateurs. Il doit être pragmatique pour que le
travail de spécification et de génération automatique de tests unitaires puisse
être fait rapidement et comme il faut. L'objectif n'est pas de traiter tous les
problèmes mais les problèmes les plus courants et les plus préoccupants. Et
enfin, puisque culturellement PHP est un langage «~glue~» et que nous voulons
traiter plusieurs problèmes, le langage de spécification que nous proposons doit
assembler plusieurs méthodes du test.

Nous pensons que la programmation par contrat est pertinente pour répondre à
toutes ces contraintes et pour générer automatiquement des tests sunitaires.

\section{Contributions}
\label{section:introduction:contributions}

Dans ce mémoire, nous proposons un nouveau langage de contrat pour PHP. À partir
de ce langage, nous voulons être capables de générer automatique des tests
unitaires. Ainsi, nous avons trois axes de réflexion~: le langage lui-même (sa
syntaxe et sa sémantique), les algorithmes de génération de données, et enfin
les algorithmes de génération de tests unitaires. Nous voulons que les
générations soient automatiques, c'est~à~dire que l'humain intervienne le moins
possible.

\subsection{Praspel et les domaines réalistes}

Le langage de contrat que nous proposons s'appelle {\strong Praspel}~:
\inenglish{PHP Realistic Annotation and SPEcification Language}. Cette première
contribution, présentée dans le chapitre~\ref{chapter:language}, s'intéresse au
langage en lui-même, à savoir sa syntaxe et sa sémantique. Nous avons également
étudié comment évaluer un contrat pour qu'il vérifie des données à l'exécution.
Ce langage est inspiré de JML~\acite{JML} (s'applique à Java) mais se
différencie sur la façon de spécifier les données~: Java a un typage déclaré et
fort, PHP a un typage dynamique et faible. Pour spécifier les données, Praspel
s'appuie en grande partie sur les {\strong domaines réalistes}, des structures
permettant de valider et générer des données et pouvant être emboîtées afin de
représenter des données plus complexes. Cette contribution, à savoir Praspel et
les domaines réalistes, a été publiée dans~\acitei{EnderlinDGO11}.

\subsection{Générateurs de données}

Un contrat peut être utilisé pour générer automatiquement des tests unitaires.
Un test est constitué de deux parties~: des données de test et un oracle. Cette
contribution, présentée dans le chapitre~\ref{chapter:data}, s'intéresse à
générer des données de test. Ces données sont spécifiées au mieux par le
contrat. Nous devons proposer des solutions capables de générer toutes sortes de
données~: des booléens, des entiers, des réels, des chaînes de caractères, des
tableaux, des objets etc. Les premières générations étaient aléatoires uniformes
pour des booléens, des entiers et des réels. Toutefois, cette approche a
rapidement montré ses limites avec des chaînes de caractères, des tableaux ou
des objets. Nous avons proposé des algorithmes de génération de chaînes de
caractères s'appuyant sur des grammaires. Ensuite, nous nous sommes concentrés
sur les tableaux avec un solveur et des contraintes dédiées. Ces contributions
ont donné lieu à deux publications, respectivement \acitei{EnderlinDGB12} et
\acitei{EnderlinGB13}. Enfin, nous avons amélioré la génération des objets en
combinant toutes ces approches.

\subsection{Critères de couverture des contrats}

Cette dernière contribution, présentée dans le chapitre~\ref{chapter:tests},
«~boucle la boucle~». D'une part, nous avons un langage de contrat qui peut être
évalué pour valider et vérifier les données manipulées par le programme. D'autre
part, nous avons des algorithmes capables de générer automatiquement des données
de test à partir d'un morceau d'un contrat. La dernière étape s'intéresse au
fait qu'un contrat peut exprimer plusieurs comportements et nous devons nous
assurer que tous ces comportements sont testés. Nous avons alors défini
plusieurs critères de couverture sur les contrats afin d'avoir des objectifs de
test à atteindre.  Par conséquent, nous avons développé des algorithmes générant
autant de tests unitaires que nécessaire pour satisfaire certains critères. Les
données nécessaires pour exécuter ces tests seront générées automatiquement avec
la contribution précédente. Cette contribution, accompagnée des contributions
précédentes, ont été soumises au journal \inenglish{Software and Systems
Modeling} (abrégé SoSyM) pour une édition spéciale sur les méthodes formelles
intégrées.

\section{Plan du mémoire}

Ce mémoire s'articule en 8~chapitres plus annexes et bibliographie.  Le
chapitre~1, ce chapitre, contient l'introduction, le contexte, la problématique
et les contributions. Le chapitre~2 contient l'état de l'art nécessaire pour
comprendre cette thèse et situer nos contributions. Les chapitres 3 à 5
présentent nos contributions~: les domaines réalistes ainsi que Praspel, au
niveau syntaxique et sémantiquen dans le chapitre~3, les différents algorithmes
et stratégies de génération de données de test dans le chapitre~4 et les
critères de couverture pour obtenir des objectifs de test à satisfaire afin de
produire des tests unitaires pertinents dans le chapitre~5. Le chapitre~6 décrit
les outils développés durant ce mémoire ou les outils incluant nos travaux. Le
chapitre~7 valide nos contributions à travers des expérimentations. Et enfin, le
chapitre~8 présente les conclusions de nos travaux et décrits plusieurs
perspectives nous semblant pertinentes.
