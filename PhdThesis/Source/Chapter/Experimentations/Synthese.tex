\section{Synthèse}
\label{section:experimentation:other}

Dans ce chapitre, nous avons présenté 3~expérimentations qualitatives autour de
Praspel~: sur un projet d'enseignement, sur 7~programmes du «~monde réel~» grâce
à un panel d'ingénieurs bénévoles et sur une extension de PHP. Elles s'ajoutent
aux expérimentations quantitatives réalisées dans nos précédents
articles~\acite{EnderlinDGO11, EnderlinDGB12, EnderlinGB13}. Ces
expérimentations nous ont montré l'efficacité des contrats pour la génération
automatique de tests unitaires, ainsi que l'efficacité de nos générateurs de
données de tests.  Plusieurs erreurs ont été trouvées dans des programmes qui
étaient déjà testés, et parfois déjà en production. Nous avons vu que Praspel et
ses outils génèrent des suites de tests simples mais pertinentes couvrant les
spécifications. Le temps économisé grâce à la génération automatique de tests
unitaires est réinvesti dans l'écriture de tests manuels plus avancés.
L'utilisation de générateur de données de tests permet d'augmenter la couverture
des données et la confiance dans les suites de tests. Ces générateurs permettent
également de réduire le nombre de tests dans des suites. Ces expérimentations
nous ont également montré les limites de Praspel, notamment avec des codes dits
«~techniques~»,  ce qui est une opportunité pour de futurs travaux.
