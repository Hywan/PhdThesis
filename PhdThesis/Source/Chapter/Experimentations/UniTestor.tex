\section{Étude de cas~: UniTestor}
\label{section:experimentation:unitestor}

Les Journées nationales du Développement Logiciel, abrégé JDév'\footnote{Voir
\url{http://devlog.cnrs.fr/jdev2013}.}, est une Action Nationale de Formation
inter-établissements, soutenue par la Mission pour l'interdisciplinité du CNRS,
de l'INRIA et de l'INRA, organisée en 2013 à l'École Polytechnique. Une équipe
de notre laboratoire a été invitée afin d'y donner des cours et des ateliers.
C'est dans ce cadre que nous avons développé UniTestor, un robot écrit en PHP
nous permettant d'aborder la majorité des aspects du test unitaire~: écrire un
préambule de test, définir des assertions, boucher des objets etc. 16 élèves ont
alors écrit manuellement une suite de tests pour ce robot. Nous la comparons
avec une suite de tests générée automatiquement avec Praspel.

\subsection{Présentation générale}

UniTestor est une simulation de robot pour l'exploration spatiale. Son diagramme
de classes UML simplifié est présenté dans la
figure~\ref{figure:experimentation:unitestor}. Le robot est représenté par la
classe \code{Robot}. Il est constitué de plusieurs périphériques, que nous
allons détailler, lui envoyant des informations. L'ensemble du robot comporte au
total 5~classes et 21~méthodes, ce qui représente 340~lignes de code (sans les
contrats Praspel).
%
\begin{figure}

\fig{\textwidth}{!}{UniTestor.pdf}

\caption{\label{figure:experimentation:unitestor} Diagramme de classes UML du
robot UniTestor.}

\end{figure}

\subsubsection{Équipements et capteurs}

Le robot est capable de se déplacer en utilisant des coordonnées relatives ou
absolues avec respectivement les méthodes \code{move} et \code{moveTo}. Pour se
déplacer, il utilise une balise de géolocalisation représentée par la classe
\code{Coordinates}. À tout moment, il peut connaître la nature du terrain avec
le capteur de sol représenté par la classe \code{LandSensor}. La nature et la
configuration du terrain influent sur l'énergie utile à un déplacement. Le robot
a également sa propre horloge, représentée par la classe \code{Clock}, lui
permettant de connaître des informations sur le temps. Enfin, la classe
\code{Vector} permet de manipuler des paires de coordonnées.

\subsubsection{Les opérations de UniTestor}

La seule opération importante est le déplacement. Lorsque le robot se déplace,
il consomme de l'énergie, représentée par un pourcentage. En fonction du
terrain, plus ou moins d'énergie sera nécessaire. L'énergie nécessaire est
calculée par la fonction \code{getNeededEnergy} du capteur de sol. Mais le robot
a des batteries qui se rechargent en fonction du temps. Le temps qui s'écoule
est donné par l'horloge et la recharge est linéaire (paramétrable par une
constante).

\subsection{\inlatin{Modus operandi} et résultats}

Nous avons commencé par annoter le code du robot par des contrats Praspel.
Ensuite, nous avons généré automatiquement une suite de tests, appelée la suite
AGT (\inenglish{Automatically Generated Tests}) satisfaisant le critère de
couverture de contrat \inenglish{All}-$G$ (défini dans la
partie~\ref{subsection:test:combination}). Nous avons comparé cette suite de
tests avec la suite de tests manuels des étudiants, appelée la suite MT
(\inenglish{Manual Tests}), complétée pour atteindre un taux de couverture de
code de 100\% quand ce n'était pas le cas.

Nous avons observé que le nombre moyen de tests générés par les étudiants est de
53, alors que pour 21 méthodes, soit 21 contrats, Praspel a généré
automatiquement 29 tests, soit une réduction de 45\%. Pour caractériser la
pertinence des tests, nous avons utilisé comme métrique la couverture de code.
Le critère de couverture de code le plus fin que nous propose atoum est le
critère tous-les-arcs. Ainsi, nous avons observé que les deux suites de tests
avaient un score de 100\%. De plus, 1h30 a été nécessaire aux étudiants pour
rédiger en moyenne 53~tests, alors qu'il nous a fallu 15~minutes pour écrire
21~contrats.  La phase de génération de tests a pris moins d'une seconde~: cela
comprend la génération automatique des données de tests, à savoir des entiers,
des réels, des chaînes de caractères spécifiées par des expressions régulières
et des objets, en particulier plusieurs instances complètes du robot (avec ses
périphériques).

Ainsi, nous obtenons le même score de couverture du code avec pratiquement
moitié moins de tests et en 6~fois moins de temps. Ces premiers résultats
montrent l'efficacité de l'approche du test automatisé et le gain qu'apportent
les contrats~: écrire des contrats est moins pénible et nécessite moins de temps
que d'écrire une suite de tests complète.

Cependant, cette expérimentation est biaisée~: les étudiants étaient en
apprentissage de l'outil et un temps d'adaptation leur était nécessaire. De
plus, nous sommes des experts de Praspel, donc nous rédigeons des contrats très
rapidement. Afin d'évaluer notre approche de manière plus objective, nous avons
demandé à un panel de bénévoles, composé d'ingénieurs de test et d'entreprises,
d'appliquer une expérimentation similaire sur leur propre code.
