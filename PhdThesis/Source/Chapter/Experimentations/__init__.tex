\chapter{Expérimentations}
\label{chapter:experimentations}

\mminitoc

\mylettrine[findent=.5em, nindent=-2.3em, slope=2.3em]{C}{inq expérimentations
sont présentées} dans ce chapitre. Les deux premières expérimentations sont
{\strong quantitatives} et ont été réalisées en interne. Elles ont été publiées
dans plusieurs articles~\acite{EnderlinDGO11, EnderlinDGB12, EnderlinGB13}. Les
trois suivantes sont {\strong qualitatives} et ont été réalisées en externe,
c'est~à~dire par d'autres personnes que nous.

Nous proposons d'introduire un langage de spécification pour PHP, simple,
pragmatique et assemblant plusieurs méthodes du test. Le but est de présenter
toutes ces méthodes et de simplifier leur utilisation dans le quotidien des
ingénieurs de test et développeurs. Nous allons commencer, dans la première
partie de ce chapitre, par valider les techniques introduites avant de nous
tourner vers les ingénieurs, dans la seconde partie de ce chapitre, pour qu'ils
jugent notre travail. Nous pensons qu'il est pertinent d'obtenir une expertise
humaine à travers les retours et remarques d'ingénieurs de test expérimentés.

La première expérimentation, présentée dans la
partie~\ref{section:experimentation:grammar}, valide notre approche basée sur
les grammaires pour les chaînes de caractères. La deuxième expérimentation,
présentée dans la partie~\ref{section:experimentation:solver}, valide notre
approche basée sur le solveur pour les tableaux. La troisième expérimentation,
présentée dans la partie~\ref{section:experimentation:unitestor}, a pour
contexte un projet d'enseignement utilisant un robot, appelé UniTestor,
développé par nos soins dans le cadre d'un événement de formation à l'École
Polytechnique. La quatrième expérimentation, présentée dans la
partie~\ref{section:experimentation:real}, a été réalisée auprès d'un panel
d'ingénieurs de test bénévoles afin d'obtenir des retours concrets sur des
programmes du «~monde réel~». Enfin, la dernière expérimentation, présentée dans
la partie~\ref{section:experimentation:php}, présentera un usage inattendu de
Praspel pour du test de performance et de non-régression d'une extension
officielle de PHP.

\require{Chapter/Experimentations/String.tex}
\require{Chapter/Experimentations/Array.tex}
\require{Chapter/Experimentations/UniTestor.tex}
\require{Chapter/Experimentations/Cas_concrets.tex}
\require{Chapter/Experimentations/Performance_et_regression.tex}
\require{Chapter/Experimentations/Synthese.tex}
