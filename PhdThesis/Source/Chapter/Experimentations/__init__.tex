\chapter{Expérimentations}
\label{chapter:experimentations}

\minitoc

Ce chapitre présente majoritairement deux expérimentations. Même si nous pouvons
noter la présence de certains chiffres, l'objectif n'est pas de faire des
expérimentations {\strong quantitatives} mais {\strong qualitatives}.
%Nous ne voulons pas montrer l'efficacité de notre approche par des chiffres sur
%des «~faux~» problèmes\footnote{Définir}.
Il est question dans ce mémoire d'introduire un langage de spécification pour
PHP, qui soit nouveau, simple, pragmatique et rassemblant plusieurs méthodes du
test. Le but est de présenter toutes ces méthodes et de simplifier leurs
utilisations dans le quotidien des ingénieurs de tests et développeurs. C'est
pour cela que nous avons par ailleurs choisi une approche avec les contrats car
ils offrent beaucoup de services et demandent peu d'efforts de la part de
l'utilisateur.

Dans les chapitres précédents, nous avons présenté Praspel, un nouveau langage
de spécification pour PHP. Nous avons par la suite présenté différents
algorithmes de générations de données, inspirés de différentes méthodes du test.
Puis, nous avons défini des critères de couvertures sur ces contrats pour
mesurer la pertinence d'une suite de tests. Enfin, nous avons proposé une
implémentation de toutes ces contributions ainsi qu'une extension dans un outil
industriel répandu pour qu'elles soient utilisables dans le quotidien des
ingénieurs de tests. C'est vers eux que nous allons nous tourner pour qu'ils
jugent notre travail. Nous pensons qu'obtenir leurs retours et remarques mis
en perspective par leur expérience est plus pertinent.

La première expérimentation a pour contexte un projet d'enseignement utilisant
un robot appelé UniTestor développé par nos soins dans le cadre des JDév' à
l'École Polytechnique. La seconde expérimentation a été réalisée auprès d'un
panel bénévol d'ingénieurs de tests et d'entreprises afin d'obtenir des retours
concrets sur des programmes du «~monde réel~».

Ce chapitre se décompose de la manière suivante. Tout d'abord, dans la
partie~\ref{section:experimentation:unitestor}, nous présentons le cas d'étude
du robot UniTestor où nous comparons deux suites de tests~: une manuelle et une
générée automatiquement avec Praspel et son extension pour atoum. Nous verrons
que Praspel et ses outils rivalisent avec les tests manuels à couverture de code
équivalente mais avec moins de tests. Nous verrons également que le temps de
rédaction des contrats est nettement inférieur à l'écriture de tests manuels.
Puis, pour valider nos résultats sur des cas concrets, nous réappliquons une
méthodologie similaire dans la partie~\ref{section:experimentation:real} avec un
panel bénévol d'ingénieurs de tests et d'entreprises. En plus des retours et
remarques, nous verrons que Praspel et ses outils ont été permis de détecter des
erreurs dans du code déjà testé et en production, et ce rapidement et
efficacement. Enfin, …

\section{Étude de cas~: UniTestor}
\label{section:experimentation:unitestor}

Les Journées nationales du Développement Logiciel, abrégé JDév'\footnote{Voir
\url{http://devlog.cnrs.fr/jdev2013}.}, est une Action Nationale de Formation
inter-établissements, soutenue par la Mission pour l'interdisciplinité du
CNRS\footnote{Voir \url{http://mrct.cnrs.fr/}.}, l'INRIA\footnote{Voir
\url{http://www.inria.fr/}.} et l'INRA\footnote{Voir
\url{http://www.inra.fr/}.}, organisée en 2013 à l'École Polytechnique. Une
équipe de notre laboratoire a été invitée afin d'y donner des cours et des
ateliers. C'est dans ce cadre que nous avons développé UniTestor, un robot écrit
en PHP nous permettant d'aborder la majorité des aspects du tests unitaires~:
écrire un préambule de test, définir des assertions, boucher des objets etc. 16
élèves ont alors écrits manuellement une suite de tests pour ce robot, que nous
avons complété. Nous la comparons avec une suite de tests générée
automatiquement avec Praspel.

\subsection{Présentation générale}

UniTestor est une simulation de robot pour l'exploration spatiale. Son diagramme
de classes UML simplifié est présenté dans la
figure~\ref{figure:experimentation:unitestor}. Le robot est représenté par la
classe \code{Robot}. Il est constitué de plusieurs périphériques, que nous
allons détaillé, lui envoyant des informations. L'ensemble du robot comporte au
total 5~classes dont 21~méthodes, ce qui représente 340~lignes de code (sans les
contrats Praspel).
%
\begin{figure}

\fig{\textwidth}{!}{UniTestor.pdf}

\caption{\label{figure:experimentation:unitestor} Diagramme de classes UML du
robot UniTestor.}

\end{figure}

\subsubsection{Équipements et capteurs}

Le robot est capable de se déplacer en utilisant des coordonnées relatives ou
absolues avec respectivement les méthodes \code{move} et \code{moveTo}. Pour se
déplacer, il utilise une balise de géolocalisation représentée par la classe
\code{Coordinates}. À tout moment, il peut connaître la nature du terrain avec
le capteur de sol représenté par la classe \code{LandSensor}. La nature et la
configuration du terrain influent sur l'énergie utile à un déplacement. Le robot
a également sa propre horloge, représentée par la classe \code{Clock}, lui
permettant de connaître des informations sur le temps. Enfin, la classe
\code{Vector} permet de manipuler des paires de coordonnées.

\subsubsection{Les opérations de UniTestor}

La seule opération importante est le déplacement. Lorsqu'il se déplace, il
consomme de l'énergie, représentée par un pourcentage. En fonction du terrain,
plus ou moins d'énergie sera nécessaire. L'énergie nécessaire est calculée par
la fonction \code{getNeededEnergy} du capteur de sol. Mais le robot a des
batteries qui se recharge en fonction du temps. Le temps qui s'écoule est donné
par l'horloge et la recharge est linéaire (paramétrable par une constante).

\subsection{Protocole expérimental et résultats}

Nous avons commencé par annoter le code du robot par des contrats Praspel.
Ensuite, nous avons générer automatiquement une suite de tests AGT
(\inenglish{Automatically Generated Tests}) satisfaisant le critère de
couverture de contrat \inenglish{All}-$G$ (défini dans la
partie~\ref{subsection:test:combination}). Nous avons comparé cette suite de
test avec celle des étudiants, appelé MT (\inenglish{Manual Tests}), complétée
pour atteindre un taux de couverture de code de 100\%.

Nous avons observé que le nombre moyen de tests générés par les étudiants est de
53~MT, alors que pour 21 méthodes, soit 21 contrats, Praspel a généré 29~AGT,
soit une réduction de 45\%. Pour caractériser la pertinence des tests, nous
avons utiliser comme métrique la couverture de code. Le critère de couverture de
code le plus fin que nous propose atoum est le critère tous-les-arcs. Ainsi,
nous avons observé que les deux suites de tests avaient un score de 100\%. De
plus, 1h30 a été nécessaire aux étudiants pour rédiger en moyenne 53~tests,
alors qu'il nous a fallu 15~minutes pour écrire 21~contrats.  La phase de
génération de tests a pris moins d'une seconde~: cela comprend la génération
automatique des données de tests, à savoir des entiers, des réels, des chaînes
de caractères spécifiées par des expressions régulières et des objets, en
particulier plusieurs instances complètes du robot (avec ses périphériques).

Ainsi, nous obtenons le même score de couverture du code avec pratiquement
moitié moins de tests et en 6~fois moins de temps. Ces premiers résultats
montrent l'efficacité de l'approche du test automatisé et le gain qu'apporte les
contrats~: écrire des contrats est moins difficile et nécessite moins de temps
que d'écrire une suite de tests complète.

Cependant, cette expérimentation est biaisée~: les étudiants étaient en
apprentissage de l'outil et un temps d'adaptation leur était nécessaire. De
plus, nous sommes des experts de Praspel, donc nous rédigeons des contrats très
rapidement. Afin d'évaluer notre approche de manière plus objective, nous avons
demandé à un panel de bénévoles, composé d'ingénieurs de test et d'entreprises,
d'appliquer une expérimentation similaire sur leur propre code.

\section{Étude de cas concrète~: ingénieurs bénévoles}
\label{section:experimentation:real}

Introduction, bla bla.

\subsection{Présentation générale}

Nous avons fait un appel auprès de différents réseaux sociaux et différentes
commnautés pour rassembler des bénévoles pour cette expérimentation.
7~ingénieurs de tests de nationalités différentes, dont 1~représentant une
entreprise, ont répondu à l'appel. Ces ingénieurs font du test depuis en moyenne
5~ans. Ils ont appliqué l'expérimentation sur leurs propres programmes ou sur
des programmes sur lesquels ils participent. Ces programmes constituent nos cas
concrets. Voici la description de ces programmes~:

\begin{itemize}

\item outil de \inenglish{debugging}~: analyse les performances d'exécutions de
certaines parties de programmes et génère des rapports dans deux formats
différents. Les parties abordées durant cette expérimentation étaient la mesure
des données et la génération des rapports.

\item outil de test pour le langage JSON~: une autre extension à atoum était en
développement pour proposer des nouvelles assertions sur le langage JSON. Les
parties abordées durant cette expérimentation étaient les assertions et les
tests de cette extension, c'est~à~dire s'assurer que l'outil de test testait
correctement le langage JSON.

\item outil d'échanges de messages~: les messages sont écrits en XML et
contiennent des données de toutes sortes, notamment des dates. C'est cette
partie qui a été majoritairement abordée durant cette expérimentation car
centrale à cet outil.

\item générateur de données de tests~: pour tester certaines parties techniques
d'un outil, un générateur de données de tests spécifique a été créé. Afin de
s'assurer que ce générateur ne comporte pas d'erreurs et étant lui-même assez
volumineux, il a sa propre suite de tests. Les parties qui ont été abordées
durant cette expérimentation étaient le calcul de ces données, les tests
unitaires et les tests fonctionnels.

\item outils de comptabilité~: 2~outils développés sur-mesure pour le calcul de
comptabilités. Les parties abordées étaient multiples.

\item outil d'évaluation de formules Mathématiques~: des utilisateurs écrivent
eux-mêmes des formules algébriques décrivant l'évolution des tarifs en temps
réels pour certains produits. Une bibliothèque de fonctions, de constantes et de
variables est mise à la disposition de ces utilisateurs. La partie abordée était
l'évaluation de ces formules analysées avec un compilateur et une grammaire
spécifique.

\end{itemize}

Tous ces programmes font intervenir tous les types de données, à savoir des
booléens, des entiers, des réels, des chaînes de caractères, des tableaux et des
objets. Dans le cas des chaînes de caractères, la plupart pouvaient être
spécifiées par une expression régulière, et quelqu'unes par des grammaires.

Sur ces programmes variés, nous avons demandé à notre panel d'appliquer les
protocoles expérimentaux suivants.

\subsection{Protocoles expérimentaux}

Cette expérimentation s'est déroulée en deux temps. Tout d'abord, nous avons
demandé au panel d'appliquer le protocole expérimental suivant~:
%
\begin{enumerate}

\item sélectionner plusieurs méthodes de toutes sortes déjà testées
manuellement~;

\item annoter ces méthodes avec des contrats Praspel~;

\item générer automatiquement une suite de tests satisfaisant le critère de
couverture de contrat \inenglish{All}-$G$, et l'exécuter~;

\item comparer la suite MT (\inenglish{Manual Tests}) avec la suite AGT
(\inenglish{Automatically Generated Tests}).

\end{enumerate}

Puis, nous leur avons demandé d'expérimenter les algorithmes de générations de
données seuls, sans Praspel. Nous rappelons que l'extension introduisant Praspel
dans atoum permet la définition et l'utilisation des domaines réalistes
facilement (voir la partie~\ref{subsection:tools:extension}). Il n'y avait pas
de protocole expérimental pré-défini mais nous leur avons demandé d'observer
l'impact de certains générateurs sur la couverture de code ainsi que l'impact
sur les suites de tests (par exemple une réduction du nombres de tests).

L'expérimentation a été ouverte pendant une première semaine avec un premier
panel puis pendant deux autres semaines avec un second panel plus important. Les
parties suivantes regroupent toutes les données et informations des deux panels.

%On présente les ingénieurs et ce qu'on va faire~: groupe de méthodes, 2 TS etc.
%Que va-t-on comparer~? Les métriques sont~: nombres de tests, code coverage,
%temps.

\subsection{Comparaison de la couverture de code des suites de tests}

%Ici, on compare les suites de test en fonction de la couverture de tests.
%Regarder les données manipulées des deux côtés, comment ça a été testé (aux
%limites, aléatoirement etc.). Donner un sens aux chiffres (100\% ne veut rien
%dire). La couverture de code tous-les-arcs n'est probablement pas la meilleure,
%ça devrait se voir à un moment ou un autre.

Un moyen de comparer les suites de tests est d'utiliser la couverture de code
avec le critère tous-les-arcs, comme pour l'expérimentation précédente.

Nous avons observé que les AGT couvrent autant que les MT, à $\pm 5\%$ près.

Quand les AGT couvraient moins que les MT, c'est que des bouchons étaient
nécessaires (des \inenglish{mocks}), c'est~à~dire décrire un comportement bien
spécifique et tester le SUT dans ces conditions. Les bouchons ne concernent que
les objets ou les fonctions, et Praspel n'est pas capable de générer de tels
bouchons.

Dans certains cas, malgré une couverture de code de 100\% pour les MT et les
AGT, des erreurs ont été trouvées. En effet, la couverture tous-les-arcs n'est
pas très précise et des erreurs peuvent apparaître après des conditions qui
n'auraient pas dû être activées. Comme le critère de couverture
\inenglish{All}-$G$ sur les contrats combinent tous les domaines réalistes de
toutes les variables, certaines combinaisons peuvent activer ces conditions
d'une manière qui n'avait pas été testée manuellement.

\subsection{Temps de rédaction des suites de tests}

%Ils sont tous habitués à écrire des tests. Personne ne connaît Praspel. Est-ce
%qu'on compte le temps d'apprentissage~? Ils n'ont peut-être pas le temps
%d'écriture de la suite MT en tête~: on travaillera pas estimation.
%
%Faire 2 fois l'XP en fait. Une fois avec apprentissage, une fois sans.

Tous les membres du panel sont habitués à écrire des tests. Ils en écrivent des
dizaines par jour. Aucun d'entre eux n'avaient écrire un contrat en Praspel
avant cette expérimentation. Nous leur avons demandé de comparer le temps de
rédaction de la suite MT avec la suite AGT.

Pour ceux qui n'étaient pas habitués à la programmation par contrat, plusieurs
heures (entre 4 et 10h) ont été nécessaires pour bien comprendre le principe et
savoir comment l'appliquer. Cette période comprend aussi l'analyse des AGT~:
savoir les lire, les comprendre et les positionner par rapport aux MT. Durant
cette période, écrire des MT était bien plus rapide que d'écrire des contrats et
produire des AGT. Les autres, même s'ils connaissaient la programmation par
contrat, ne l'avaient jamais pratiquée. Les problèmes rencontrés étaient les
même~: savoir lire, comprendre et analyser les AGT. Le panel pondère toutefois
cet effort~: comparer aux autres outils qu'ils manipulent quotidiennement,
l'installation de Praspel et son extension pour atoum est très rapide, le
langage est simple à utiliser, et même si les notions sous-jacentes (les
algorithmes et les méthodes du test utilisées) ne sont pas connues ou comprises,
les domaines réalistes sont également une notion simple à comprendre et à
utiliser.

Une fois le cap de l'apprentissage franchi, le panel a observé que pour peu de
code à tester (moins de 5 ou 6~méthodes), il était plus rapide d'écrire des MT.
En revanche, lorsqu'il faut tester plus de code (ce qui représente 95\% des
cas), l'utilisation des contrats s'avèrent très efficace. Il faut entre 2 et
4~fois moins de temps pour écrire les contrats et produire les AGT que d'écrire
les MT à la main (de 2h à 30mn par exemple). Pour mettre ces résultats en
perspective avec la partie précédente, le temps qui a été gagné est réinvesti
pour écrire des MT plus «~poussés~», comprendre plus complets.

Le panel nous a fait remarquer que malheureusemnt, les tests ne sont pas encore
une priorité dans les budgets alloués au développement logiciel. Par conséquent,
ils ont un temps limité pour tester leurs applications. Pendant le temps
imparti, ils n'étaient capable d'écrire que des tests dits «~basiques~» ou alors
ils ne testaient que les parties les plus critiques du code. Avec les contrats,
ils spécifient toutes les méthodes car ils sont écrits en même temps que le
code. La proximité entre le code et les contrats \via les annotations est un
avantage. Par conséquent, plus de code est testé sans nécessiter autant de
temps que si ça devait être fait manuellement. Le temps qui a été économisé est
ainsi réinvesti pour des tests plus complets, qui n'auraient pu être faits en
temps normal. Autrement dits, les tests «~usuels~» sont gérés par Praspel, ce
qui permet aux ingénieurs de tests de se concentrer sur les parties plus
critiques.

Nous aurions aimé faire l'expérimentation 2~fois de suite pour comparer la
courbe d'apprentissage, mais nous n'en avons pas eu le temps. Nous rappelons que
le panel était constitué de bénévoles.

\subsection{Générations de données de tests}
%
%Les suites AGT ne sont pas toutes complètes. Il faut compléter avec des MT.
%Est-ce que les algos de génération de données sont utiles~? Sont-ils aussi
%efficaces~? Ont-ils réécrit des tests en utilisant la génération de données~?
%Quels résultats a-t-il apporté~?

Nous avons précisé dans les parties précédentes que les suites d'AGT ne sont pas
toujours complètes. En effet, Praspel n'est pas capable de tout spécifier et
certaines parties du code n'est pas toujours testable facilement. Heureusement,
la partie précédente nous a montré que le temps gagné grâce aux contrats est
réinvesti dans l'écriture de MT. Dans cette étape, les ingénieurs de tests ont
utilisés nos générateurs de données. Rappelons qu'atoum ne possède aucun
générateur de données. Le seul outil qu'il propose est de se brancher sur un
dictionnaire de données écrit manuellement.

Premier constat. Les domaines réalistes permettent de spécifier simplement et
facilement une grande majorité des données. Pouvoir utiliser les domaines
réalistes dans du code PHP directement permet de mixer et d'orienter la
génération facilement. Par exemple, écrire des boucles où chaque pas produit une
donnée légèrement différente de la précédente, ou encore utiliser des données
générées pour modifier d'autres données provenant de bases de données ou
d'autres sources. Comme les données de tests ne sont plus déclarées
manuellement, cela simplifie grandement l'écriture des tests. Le panel a précisé
que ce n'était même pas comparable~: il n'y avait rien avant et c'était un
travail pénible, qui a maintenant disparu. Par ailleurs, cela simplifie aussi la
lecture des tests. En effet, le contenu du test se restreint à la logique du
test plutôt qu'à la manipulation des données de tests. Le panel insiste sur le
fait que les tests sont par conséquent plus facilement maintenables.

Deuxième constat. Sans générateur automatique de données, le panel nous avoue
que les données de tests utilisées étaient très peu nombreuses~: entre 1 à 5
données par tests, 1.3 en moyenne. Avec nos générateurs de données, un grande
nombre de données sont générables «~instantanément~» pour reprendre leurs
termes. En effet, le temps de génération est tellement rapide qu'il en devient
négligeable.

Troisième constat. Générer beaucoup de données de tests n'a aucun sens si elles
sont toutes inutiles. Le panel a à la majorité apprécié les algorithmes de
générations de chaînes de caractères. Pouvoir spécifier des chaînes de
caractères avec des expressions régulières ou des grammaires, et ensuite générer
toutes les données de manière exhaustive ou alors par couverture de la
grammaire, a été remarquablement apprécié. Plusieurs points sont à noter. Tout
d'abord, avoir une exhaustivité ou une couverture sur les données manipulées est
un avantage important~: cela augmente la confiance dans les tests. La notion de
couverture de données est aussi importante que la couverture du code pour plus
de la moitié de notre panel. Ensuite, spécifier des chaînes de caractères avec
des expressions régulières est facile. En effet, il arrive très fréquemment que
ces expressions régulières soient déjà définies dans l'implémentation. Il suffit
alors de les réutiliser. Quand nous avons demandé comment le panel générait des
chaînes de caractères avant, nous avons été surpris de voir qu'un seul ingénieur
en générait en concaténant des ensembles de sous-chaînes de caractères. Les
autres écrivaient quelques chaînes de caractères manuellement. Dans le cas des
grammaires, un ingénieur a fait remarquer qu'en plus d'apprendre le langage
Praspel, il fallait apprendre le langage PP pour décrire des grammaires. Les
autres membres du panel n'ont pas relevé cet effort arguant qu'aux vues des
services rendus, cet effort était très rapidement amorti. Enfin, dans le cas des
tableaux, le panel a apprécié le fait de pouvoir générer à moindre coût des
centaines de tableaux respectant certaines contraintes. Là encore, très peu de
tableaux étaient utilisés en tant que données de tests manuellement avant.

Cette partie de l'expérimentation a révélé plus d'erreurs que dans la
précédente. L'utilisation des grammaires avec le générateur exhaustive a permi
de détecter des erreurs chez tous les membres du panel ayant utilisé cette
technique. À un moment, la correction d'une erreur n'a pas invalidé des tests
qui auraient dû l'être. Par conséquent, cela a montré des erreurs dans les tests
eux-mêmes. Ceci n'aurait probablement jamais été découvert sans l'utilisation
exhaustive de données de tests.

Bla bla bla.

\subsubsection{Tests paramétrés}

Certains auront probablement fait des tests paramétrés (ça se fait déjà un peu
dans atoum, ça s'est fait dans UniTestor, je les encouragerai à le faire). Quels
résultats~? C'est une approche «~hybride~», est-ce qu'elle comble certains
manques de Praspel~? Ou même des MT~?

\subsection{Retours des ingénieurs/expertise humaine}

Plutôt sous la forme d'une interview~? Est-ce que l'approche des contrats est
pertinente~? Est-ce que vous avez l'impression de gagner du temps~? De
l'argent~? De la qualité logiciel~? Qu'est-ce qu'il faudrait pour gagner en
maturité~? Les bugs qui ont été trouvés (s'il y en a), peut-on estimer leurs
coût~: en temps, en personne, en argent (par rapport à la criticité du bug)~?

Autre question, moins empirique~: est-ce que l'ingénieur aurait écrit ces tests
là, ou pas, des tests trop simples, des tests trop complexes, pertinents, ça
remplit nos objectifs qualités, de couverture etc.

Expressivité de Praspel~: est-ce que ce le langage est assez complet, est-ce
qu'ils peuvent tout exprimer avec ou pas~?

Est-ce qu'ils sont limités par les générateurs de données~? Est-ce qu'ils les
ont un peu hacker pour obtenir la donnée qu'ils voulaient~? Est-ce qu'ils
voulaient faire quelque chose qui n'était pas possible~? Est-ce qu'ils se
s'auraient poser la question sans Praspel~?

\subsubsection{Bénéfices des contrats}

Qu'est-ce que cela apporte à votre méthodologie~? Vous travaillez plus
lentement, plus rapidement~? Facile à prendre en main et à comprendre~?

\section{Autre cas d'études ponctuels}

Parler de ext/jsond (remplaçant de ext/json dans PHP)~? Test de non-régression
et de performance, basé sur notre article A-MOST'12.

\section{Autre}

Est-ce qu'ils ont trouvé des bugs dans Praspel~?

% En vrac~:
%     ext/jsond
%     Hoa\Compiler
%     Hoa\Math
%     Hoa\Ruler
%     Rezzza et son million d'utilisateur

\section{Synthèse}
\label{section:experimentation:other}
