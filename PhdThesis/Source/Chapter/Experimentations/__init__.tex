\chapter{Expérimentations}
\label{chapter:experimentations}

\minitoc

% En vrac~:
%     ext/jsond
%     Hoa\Compiler
%     Hoa\Math
%     Hoa\Ruler
%     Rezzza et son million d'utilisateur

Ce chapitre présente l'utilisation de Praspel et de ses algorithmes dans
principalement deux contextes. D'une part sur un projet d'enseignement utilisant
un robot appelé UniTestor développé par nos soins dans le cadre des JDév' à
l'École Polytechnique, et d'autre part auprès d'un panel bénévol d'ingénieurs de
tests et d'entreprises pour avoir des retours concrets sur des programmes du
«~monde réel~».

Ce chapitre se décompose de la manière suivante. Tout d'abord, dans la
partie~\ref{section:experimentation:atoum}, nous présentons comment nous avons
intégré Praspel à un outil industriel très populaire afin de faciliter son
utilisation et sa diffusion. Ensuite, dans la
partie~\ref{section:experimentation:unitestor}, nous présentons le cas d'étude
du robot UniTestor où nous comparons deux suites de tests~: une manuelle et une
générée automatiquement avec Praspel.  Puis, pour valider nos résultats, nous
réappliquons une méthodologie similaire dans la
partie~\ref{section:experimentation:real} avec un panel bénévol d'ingénieurs de
tests et d'entreprises. Nous en profitons pour obtenir des retours de personnes
expérimentées sur notamment la pertinence de notre approche ou encore son
intégration dans des projets existants. Enfin, dans la
partie~\ref{section:experimentation:other}, nous analysons des cas d'études
spécifiques qui ont été réalisés ponctuellement.

\section{Intégration dans atoum}
\label{section:experimentation:atoum}

\section{Étude de cas~: UniTestor}
\label{section:experimentation:unitestor}

Les Journées nationales du Développement Logiciel, abrégé JDév'\footnote{Voir
\url{http://devlog.cnrs.fr/jdev2013}.}, est une Action Nationale de Formation
inter-établissements, soutenue par la Mission pour l'interdisciplinité du
CNRS\footnote{Voir \url{http://mrct.cnrs.fr/}.}, l'INRIA\footnote{Voir
\url{http://www.inria.fr/}.} et l'INRA\footnote{Voir
\url{http://www.inra.fr/}.}, organisée en 2013 à l'École Polytechnique. Une
équipe de notre laboratoire a été invité afin d'y donner des cours et des
ateliers. C'est dans ce cadre que nous avons développé UniTestor, un robot écrit
en PHP nous permettant d'aborder la majorité des aspects du tests unitaires~:
écrire un préambule de test, définir des assertions, boucher des objets etc. 16
élèves ont alors écrits manuellement une suite de tests pour ce robot, que nous
avons complété. Nous la comparons avec une suite de tests générée
automatiquement avec Praspel.

\subsection{Présentation générale}

UniTestor est une émulation de robot pour l'exploration spatiale. Son diagramme
de classes UML est présenté dans la
figure~\ref{figure:experimentation:unitestor}. Le robot est représenté par la
classe \code{Robot}. Il est constitué de plusieurs périphériques lui envoyant
des informations. L'ensemble du robot comporte 5 classes dont 21 méthodes, ce
qui représente 340 lignes de code (sans les contrats Praspel).
%
\begin{figure}

\tikzstyle{class}=[rectangle, draw=black, thick, text centered, text width=4cm]
\tikzstyle{umlarrow}=[->, >=latex, shorten >=1pt, thick]
\tikzstyle{umlline}=[-, shorten >=3pt, thick]

\drawfig{\textwidth}{!}{node distance=6.5cm}{
    \node (Robot) [class, rectangle split, rectangle split parts=2]
    {
        \code{Robot}
        \nodepart{second}
        \code{move} \\
        \code{moveTo} \\
        \code{getTimeToReach} \\
        \code{getClock} \\
        \code{getCoordinates} \\
        \code{getEnergy} \\
        \code{getLandSensor}
    };

    \node (Coordinates) [class, rectangle split, rectangle split parts=2, below left of=Robot]
    {
        \code{Coordinates}
        \nodepart{second}
        \code{getX} \\
        \code{getY}
    };

    \node (Clock) [class, rectangle split, rectangle split parts=2, below of=Robot]
    {
        \code{Clock}
        \nodepart{second}
        \code{reset} \\
        \code{getDifference} \\
        \code{getTimestamp} \\
        \code{getCurrentTime}
    };

    \node (LandSensor) [class, rectangle split, rectangle split parts=2, below right of=Robot]
    {
        \code{LandSensor}
        \nodepart{second}
        \code{getNeededEnergy}
    };

    \node (Vector) [class, rectangle split, rectangle split parts=2, above left of=Coordinates, text width=5cm]
    {
        \code{Vector}
        \nodepart{second}
        \code{getFirstCoordinates} \\
        \code{getSecondCoordinates} \\
        \code{getLength}
    };

    \draw[umlarrow] (Coordinates.north) -- ++(0,0.5)  -| (Robot.south);
    \draw[umlline]  (LandSensor.north)  -- ++(0,0.75) -| (Robot.south);
    \draw[umlline]  (Clock.north)                     -| (Robot.south);
    \draw[umlarrow] (Coordinates.west)                -| (Vector.south);

}

\caption{\label{figure:experimentation:unitestor} Diagramme de classes UML du
robot UniTestor.}

\end{figure}

\subsubsection{Équipements et capteurs}

Le robot est capable de se déplacer en utilisant des coordonnées relatives ou
absolues avec respectivement les méthodes \code{move} et \code{moveTo}. Pour se
déplacer, il utilise une balise de géolocalisation représentée par la classe
\code{Coordinates}. À tout moment, il peut connaître la nature du terrain avec
le capteur de sol représenté par la classe \code{LandSensor}. La nature et la
configuration du terrain influent sur l'énergie utile à un déplacement. Le robot
a également sa propre horloge, représentée par la classe \code{Clock}, lui
permettant de connaître des informations sur le temps. Enfin, la classe
\code{Vector} permet de manipuler des paires de coordonnées.

\subsubsection{Les opérations de UniTestor}

La seule opération importante est le déplacement. Lorsqu'il se déplace, il
consomme de l'énergie, représentée par un pourcentage. En fonction du terrain,
plus ou moins d'énergie sera nécessaire. L'énergie nécessaire est calculée par
la fonction \code{getNeededEnergy} du capteur de sol. Mais le robot a des
batteries qui se recharge en fonction du temps. Le temps qui s'écoule est donné
par l'horloge et la recharge est linéaire (paramétrable par une constante).

\subsection{Génération de tests pour l'application}

Nous avons commencé par annoter le code du robot par des contrats Praspel.
Ensuite, nous avons générer automatiquement une suite de tests AGT
(\inenglish{Automatically Generated Tests}) satisfaisant le critère de
couverture de contrat \inenglish{All}-$G$ (défini dans la
partie~\ref{subsection:test:combination}).  Nous avons comparé cette suite de
test avec celle des étudiants, appelé MT (\inenglish{Manual Tests}).

%\subsubsection{Calcul des séquences de tests}
%\subsubsection{Exécution des tests}

Le nombre moyen de tests générés par les étudiants est de 53~MT, alors que pour
21 méthodes, soit 21 contrats, Praspel a généré 29~AGT, soit une réduction de
45\%. Pour caractériser la pertinence des tests, nous avons utiliser comme
métrique la couverture de code. Le critère de couverture de code le plus fin que
nous propose atoum est le critère tous-les-arcs. Ainsi, nous avons observé que
les deux suites de tests avaient un score de 100\%. De plus, 1h30 a été
nécessaire aux étudiants pour rédiger en moyenne 53~tests, alors qu'il nous a
fallu 15~minutes pour écrire 21~contrats.  La phase de génération de tests a
pris moins d'une seconde~: cela comprend la génération automatique des données
de tests, à savoir des entiers, des réels, des chaînes de caractères spécifiées
par des expressions régulières et des objets, en particulier plusieurs instances
complètes du robot (avec ses périphériques).

Ainsi, nous obtenons le même score de couverture du code avec pratiquement
moitié moins de tests et en 6~fois moins de temps. Ces premiers résultats
montrent l'efficacité de l'approche du test automatisé et le gain qu'apporte les
contrats~: écrire des contrats est moins difficile et nécessite moins de temps
que d'écrire une suite de tests complète.

Cependant, cette expérimentation est biaisée~: les étudiants étaient en
apprentissage de l'outil, et même s'il est {\em simple} et {\em intuitif}, un
temps d'adaptation est nécessaire. De plus, nous sommes des experts de Praspel,
donc nous rédigeons des contrats très rapidement. Afin d'évaluer notre approche
de manière plus objective, nous avons demandé à un panel de bénévoles, composé
d'ingénieurs de test et d'entreprises, d'appliquer une expérimentation similaire
sur leur propre code.

\section{Étude de cas concrète~: ingénieurs bénévoles}
\label{section:experimentation:real}

Nous avons sélectionné un panel de ???\footnote{TODO}~ingénieurs, dont
???\footnote{TODO}~entreprises. Tous ont une expérience dans dans la
qualité logiciel allant de ???\footnote{TODO} à ???\footnote{TODO}~années. Les
entreprises travaillent dans le domaine de ???\footnote{TODO}. Nous leur avons
demandé d'appliquer une méthodologie similaire à celle présentée dans la partie
précédente mais sur leur propre code.

\subsection{Présentation générale}

Nous avons demandé au panel d'appliquer le protocole expérimental suivant~:

\begin{enumerate}

\item sélectionner plusieurs méthodes déjà testées manuellement\footnote{Selon
quels critères~?}~;

\item annoter ces méthodes avec des contrats Praspel~;

\item générer automatiquement une suite de tests satisfaisant le critère de
couverture de contrat \inenglish{All}-$G$, et l'exécuter~;

\item comparer la suite MT avec la suite AGT.

\end{enumerate}

%On présente les ingénieurs et ce qu'on va faire~: groupe de méthodes, 2 TS etc.
%Que va-t-on comparer~? Les métriques sont~: nombres de tests, code coverage,
%temps.

\subsection{Comparaison de la couverture de code des suites de tests}

Ici, on compare les suites de test en fonction de la couverture de tests.
Regarder les données manipulées des deux côtés, comment ça a été testé (aux
limites, aléatoirement etc.). Donner un sens aux chiffres (100\% ne veut rien
dire). La couverture de code tous-les-arcs n'est probablement pas la meilleure,
ça devrait se voir à un moment ou un autre.

\subsection{Temps de rédaction des suites de tests}

Ils sont tous habitués à écrire des tests. Personne ne connaît Praspel. Est-ce
qu'on compte le temps d'apprentissage~? Ils n'ont peut-être pas le temps
d'écriture de la suite MT en tête~: on travaillera pas estimation.

Faire 2 fois l'XP en fait. Une fois avec apprentissage, une fois sans.

\subsection{Générations de données de tests}

Les suites AGT ne sont pas toutes complètes. Il faut compléter avec des MT.
Est-ce que les algos de génération de données sont utiles~? Sont-ils aussi
efficaces~? Ont-ils réécrit des tests en utilisant la génération de données~?
Quels résultats a-t-il apporté~?

\subsubsection{Tests paramétrés}

Certains auront probablement fait des tests paramétrés (ça se fait déjà un peu
dans atoum, ça s'est fait dans UniTestor, je les encouragerai à le faire). Quels
résultats~? C'est une approche «~hybride~», est-ce qu'elle comble certains
manques de Praspel~? Ou même des MT~?

\subsection{Retours des ingénieurs/expertise humaine}

Plutôt sous la forme d'une interview~? Est-ce que l'approche des contrats est
pertinente~? Est-ce que vous avez l'impression de gagner du temps~? De
l'argent~? De la qualité logiciel~? Qu'est-ce qu'il faudrait pour gagner en
maturité~? Les bugs qui ont été trouvés (s'il y en a), peut-on estimer leurs
coût~: en temps, en personne, en argent (par rapport à la criticité du bug)~?

Autre question, moins empirique~: est-ce que l'ingénieur aurait écrit ces tests
là, ou pas, des tests trop simples, des tests trop complexes, pertinents, ça
remplit nos objectifs qualités, de couverture etc.

Expressivité de Praspel~: est-ce que ce le langage est assez complet, est-ce
qu'ils peuvent tout exprimer avec ou pas~?

Est-ce qu'ils sont limités par les générateurs de données~? Est-ce qu'ils les
ont un peu hacker pour obtenir la donnée qu'ils voulaient~? Est-ce qu'ils
voulaient faire quelque chose qui n'était pas possible~? Est-ce qu'ils se
s'auraient poser la question sans Praspel~?

\subsubsection{Bénéfices des contrats}

Qu'est-ce que cela apporte à votre méthodologie~? Vous travaillez plus
lentement, plus rapidement~? Facile à prendre en main et à comprendre~?

\section{Autre cas d'études ponctuels}

Parler de ext/jsond (remplaçant de ext/json dans PHP)~? Test de non-régression
et de performance, basé sur notre article A-MOST'12.

\section{Autre}

Est-ce qu'ils ont trouvé des bugs dans Praspel~?

\section{Synthèse}
\label{section:experimentation:other}
