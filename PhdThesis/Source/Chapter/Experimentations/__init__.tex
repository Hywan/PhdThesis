\chapter{Expérimentations}
\label{chapter:experimentations}

\minitoc

Ce chapitre présente trois expérimentations. Même si nous pouvons noter la
présence de certains chiffres, l'objectif n'est pas de faire des
expérimentations {\strong quantitatives} mais {\strong qualitatives}. Des
expérimentations quantitatives ont été réalisées dans nos
articles~\acite{EnderlinDGO11, EnderlinDGB12, EnderlinGB13}. Il est question
dans ce mémoire d'introduire un langage de spécification pour PHP, qui soit
simple, pragmatique et rassemblant plusieurs méthodes du test. Le but était de
présenter toutes ces méthodes et de simplifier leurs utilisations dans le
quotidien des ingénieurs de test et développeurs. C'est vers eux que nous allons
nous tourner pour qu'ils jugent notre travail. Nous pensons qu'il est pertinent
d'obtenir une expertise humaine à travers les retours et remarques d'ingénieurs
de test expérimentés.

La première expérimentation a pour contexte un projet d'enseignement utilisant
un robot, appelé UniTestor, développé par nos soins dans le cadre des JDév' à
l'École Polytechnique. La seconde expérimentation a été réalisée auprès d'un
panel d'ingénieur de test bénévoles et d'entreprises afin d'obtenir des retours
concrets sur des programmes du «~monde réel~».

Ce chapitre se décompose de la manière suivante. Tout d'abord, dans la
partie~\ref{section:experimentation:unitestor}, nous présentons le cas d'étude
du robot UniTestor où nous comparons deux suites de tests~: une manuelle et une
générée automatiquement avec Praspel et son extension pour atoum. Puis, pour
valider nos résultats sur des cas concrets, nous réappliquons une méthodologie
similaire dans la partie~\ref{section:experimentation:real} avec un panel
d'ingénieurs de test bénévoles. Enfin, un usage inattendu de Praspel sera abordé
dans la partie~\ref{section:experimentation:php} pour du test de performance et
de non-régression d'une extension officielle de PHP.

\require{Chapter/Experimentations/UniTestor.tex}
\require{Chapter/Experimentations/Cas_concrets.tex}
\require{Chapter/Experimentations/Performance_et_regression.tex}
\require{Chapter/Experimentations/Synthese.tex}
