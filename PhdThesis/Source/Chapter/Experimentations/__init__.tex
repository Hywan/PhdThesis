\chapter{Expérimentations}
\label{chapter:experimentations}

\minitoc

Ce chapitre présente majoritairement deux expérimentations. Même si nous pouvons
noter la présence de certains chiffres, l'objectif n'est pas de faire des
expérimentations {\strong quantitatives} mais {\strong qualitatives}.
%Nous ne voulons pas montrer l'efficacité de notre approche par des chiffres sur
%des «~faux~» problèmes\footnote{Définir}.
Il est question dans ce mémoire d'introduire un langage de spécification pour
PHP, qui soit nouveau, simple, pragmatique et rassemblant plusieurs méthodes du
test. Le but est de présenter toutes ces méthodes et de simplifier leurs
utilisations dans le quotidien des ingénieurs de test et développeurs. C'est
pour cela que nous avons par ailleurs choisi une approche avec les contrats car
ils offrent beaucoup de services et demandent peu d'efforts de la part de
l'utilisateur.

Dans les chapitres précédents, nous avons présenté Praspel, un nouveau langage
de spécification pour PHP. Nous avons par la suite présenté différents
algorithmes de générations de données, inspirés de différentes méthodes du test.
Puis, nous avons défini des critères de couvertures sur ces contrats pour
mesurer la pertinence d'une suite de tests. Enfin, nous avons proposé une
implémentation de toutes ces contributions ainsi qu'une extension dans un outil
industriel répandu pour qu'elles soient utilisables dans le quotidien des
ingénieurs de test. C'est vers eux que nous allons nous tourner pour qu'ils
jugent notre travail. Nous pensons qu'obtenir une expertise humaine à travers
leurs retours et remarques mis en perspective par leur expérience est pertinent.

La première expérimentation a pour contexte un projet d'enseignement utilisant
un robot appelé UniTestor développé par nos soins dans le cadre des JDév' à
l'École Polytechnique. La seconde expérimentation a été réalisée auprès d'un
panel bénévol d'ingénieurs de test et d'entreprises afin d'obtenir des retours
concrets sur des programmes du «~monde réel~».

Ce chapitre se décompose de la manière suivante. Tout d'abord, dans la
partie~\ref{section:experimentation:unitestor}, nous présentons le cas d'étude
du robot UniTestor où nous comparons deux suites de tests~: une manuelle et une
générée automatiquement avec Praspel et son extension pour atoum. Nous verrons
que Praspel et ses outils rivalisent avec les tests manuels à couverture de code
équivalente mais avec moins de tests. Nous verrons également que le temps de
rédaction des contrats est nettement inférieur à l'écriture de tests manuels.
Puis, pour valider nos résultats sur des cas concrets, nous réappliquons une
méthodologie similaire dans la partie~\ref{section:experimentation:real} avec un
panel bénévol d'ingénieurs de test et d'entreprises. En plus des retours et
remarques, nous verrons que Praspel et ses outils ont permis de détecter des
erreurs dans du code déjà testé et en production, et ce rapidement et
efficacement. Enfin, …

\require{Chapter/Experimentations/UniTestor.tex}
\require{Chapter/Experimentations/Cas_concrets.tex}
\require{Chapter/Experimentations/Performance_et_regression.tex}
\require{Chapter/Experimentations/Synthese.tex}
