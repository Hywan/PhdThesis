\section{Solveur sur les tableaux}
\label{section:experimentation:solver}

Cette partie présente l'expérimentation que nous avons menée dans
l'article~\acitei{EnderlinGB13}. Dans cette partie, nous utilisons le mot
{\strong système} pour parler d'un ensemble de propriétés sur les tableaux.
Cette expérimentation évalue l'efficacité du solveur présenté dans la
partie~\ref{section:data:arrays} page~\pageref{section:data:arrays},
c'est~à~dire sa capacité à supprimer ou réduire le rejet lors de la génération
de données à partir d'un système. Nous mesurons le nombre de
\inenglish{backtracks} dans le solveur, le temps de génération de données à
partir d'un système satisfaisable et le nombre de systèmes non satisfaisables
détectés. L'expérimentation est composée de trois étapes~:
%
\begin{enumerate}

\item génération de systèmes (à partir d'une grammaire et des algorithmes
présentés dans la partie~\ref{subsection:data:algorithms}
page~\pageref{subsection:data:algorithms})~;

\item exécution du solveur sur chaque système généré~;

\item mesures.

\end{enumerate}
%
Nous générons automatiquement des systèmes dont les propriétés portent sur des
tableaux contenant des chaînes de caractères et des entiers, et dont la taille
des tableaux varie entre 5 et 20.

Praspel est décrit par une grammaire. Afin de générer des systèmes, nous
réutilisons les travaux présentés dans la partie~\ref{section:data:strings}
page~\pageref{section:data:strings} concernant la génération de chaînes de
caractères à partir de grammaire. Trois algorithmes ont été présentés~:
aléatoire uniforme, exhaustif borné et basé sur la couverture. Puisque la
grammaire des propriétés sur les tableaux est petite, nous ne pouvons pas
utiliser le dernier algorithme car il ne sera pas capable de générer une
collection de systèmes suffisamment importante. L'algorithme exhaustif borné est
plus coûteux que l'algorithme aléatoire uniforme mais il est précis et plus
adapté à des petites grammaires. Nous retenons cet algorithme. Ce dernier va
générer des séquences de lexèmes dont la taille sera bornée à $n$. Ces séquences
seront ensuite concrétisées pour, dans notre cas, former un système. Nous
rappelons que les lexèmes sont concrétisés avec un algorithme aléatoire
isotropique.

Avec $n = 3$, nous pouvons générer des propriétés de la forme \code{a is unique}
(avec \code{a} un tableau). Avec $n = 6$, nous pouvons générer des propriétés de
la forme \code{a[0]: 0}. Avec $n = 8$, nous pouvons générer des propriétés des
la forme \code{a[0]: 0 or 1}. Avec $n = 11$, nous pouvons générer des propriétés
comme celle de l'exemple~\ref{example:data:array_proprietes}
page~\pageref{example:data:array_proprietes} soit \code{a[0]: 'b' or 'c' and a
is unique}.

La deuxième étape invoque le solveur sur chaque système généré afin de générer
un tableau satisfaisant ce système. Chaque tableau est validé par les prédicats
des domaines réalistes et par les propriétés des tableaux pour vérifier la
justesse (\inenglish{soundness}) du solveur, c'est~à~dire qu'il génère des
données qui sont correctes.

Durant cette étape, des marqueurs permettent de compter uniquement le temps de
génération des tableaux sans compter le temps de compilation (génération des
systèmes, analyses des systèmes, création du modèle objet de Praspel etc.). Nous
mesurons également le nombre de \inenglish{backtracks} dans le solveur et le
nombre de systèmes rejetés. Quand un système est rejeté, ses
\inenglish{backtracks} ne sont pas comptabilisés.

Puisque la concrétisation des séquences de lexèmes utilise un algorithme
aléatoire isotropique, les valeurs peuvent être très différentes d'un système
généré à l'autre. Ces différences peuvent conduire à des amplitudes dans les
temps de génération et les nombres de \inenglish{backtracks}. Pour éviter des
pics dans les résultats, nous calculons la moyenne de 100~exécutions.

\begin{figure}

\centering
\begin{tabular}{|c|c|c|c|c|c|}

\hline

$n$                          &
  systèmes                   &
  \inenglish{backtracks}     &
  \inenglish{backtracks} par &
  systèmes                   &
  temps de                   \\

                             &
  générés                    &
                             &
  système                    &
  rejetés                    &
  génération (ms)            \\

\hline

10                           &
  \phantom{0}14              &
  \phantom{00}0              &
  0\phantom{.00}             &
  0                          &
  \phantom{00}6.484 \\

15                           &
  \phantom{0}86              &
  \phantom{0}34              &
  0.40                       &
  0                          &
  \phantom{0}42.167          \\

18                           &
 210                         &
 \phantom{0}91               &
 0.43                        &
 0                           &
 141.694                     \\

19                           &
 275                         &
 103                         &
 0.37                        &
 0                           &
 229.001                     \\

20                           &
 492                         &
 114                         &
 0.23                        &
 0                           &
 372.241                     \\
\hline
\end{tabular}

\caption{\label{figure:experimentation:arrays} Résultat de l'expérimentation sur
les tableaux}

\end{figure}

La figure~\ref{figure:experimentation:arrays} montre les résultats de notre
expérimentation. Dans la première colonne, $n$ est la taille maximum de la
séquence de lexèmes représentant un système. La deuxième colonne indique le
nombre de systèmes distincts générés avec cette taille. La troisième, cinquième
et sixième colonne indiquent respectivement les nombres moyens de
\inenglish{backtracks}, de systèmes rejetés pour 100~exécutions et la moyenne
des temps de génération des tableaux. La quatrième colonne indique le taux de
\inenglish{backtracks} par système.

Pour $n \leq 20$, nous observons qu'aucun système n'est rejeté, ce qui est une
importante amélioration par comparaison avec la génération standard qui est
aléatoire (voir la partie~\ref{section:data:random}
page~\pageref{section:data:random}). La génération standard introduisait un taux
de rejet supérieur à 90\% face à de tels systèmes. Toutes les données générées
satisfont leur spécification. Le solveur génère avec succès et rapidement des
données avec un faible nombre de \inenglish{backtracks}. $n = 20$ représente
approximativement 3~propriétés avec des disjonctions~: l'algorithme exhaustif
borné génère 492~systèmes distincts et les exécutions produisent seulement
114~\inenglish{backtracks}, soit approximativement 1~\inenglish{backtrack} pour
4~systèmes. C'est un bon résultat. Néanmoins, nous n'avons pas été capables de
caractériser le nombre de \inenglish{backtracks} par rapport au nombre de
propriétés. Cela aurait pu permettre d'observer la réaction du solveur face à
certaines formes de contraintes, et potentiellement de lui ajouter des
optimisations. Enfin, cette expérimentation nous a permis de détecter une erreur
dans le solveur. Cette erreur conduisait systématiquement au rejet de systèmes
ayant un certain motif. Par conséquent, nous remarquons que cette démarche de
validation est efficace pour détecter des erreurs.
