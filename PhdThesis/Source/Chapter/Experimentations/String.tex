\section{Test à partir de grammaires}
\label{section:experimentation:grammar}

Cette partie contient deux expérimentations publiées dans
l'article~\acitei{EnderlinDGB12}. La première a été conçue pour valider que le
compilateur de compilateurs $LL(\star)$ et les algorithmes de génération de
chaînes de caractères fonctionnent correctement. Cette expérimentation est une
auto-validation. La seconde expérimentation montre l'utilisation de Praspel pour
valider des applications Web.

\subsection{Auto-validation du compilateur et des algorithmes}

Notre première expérimentation vise à valider notre approche
\inenglish{grammar-based testing}, soit à s'assurer que, d'une part, le
compilateur de compilateurs $LL(\star)$ fonctionne correctement, c'est~à~dire
qu'il accepte des données correctes et rejette des données incorrectes, et que,
d'autre part, les générateurs de chaînes de caractères fonctionnent correctement
également, c'est~à~dire qu'ils ne génèrent pas de données incorrectes par
rapport à une grammaire.

Nous travaillons alors en deux étapes. Premièrement, nous générons des données
automatiquement à partir de grammaires qui utilisent toutes les constructions du
langage de description de grammaires PP. Deuxièmement, toutes ces données seront
analysées par notre propre compilateur et d'autres compilateurs. L'objectif est
triple~:
%
\begin{enumerate}

\item vérifier que les données correctes sont acceptées par tous les
compilateurs~;

\item les données incorrectes sont refusées par tous les compilateurs~;

\item tous les compilateurs ont le même avis sur la validité des données.

\end{enumerate}

Nous considérons une grammaire de JSON (\inenglish{JavaScript Object
Notation}~\acite{JSON}) et une grammaire simplifiée des PCRE (\inenglish{Perl
Compatible Regular Expressions}~\acite{Hazel05}). Ce choix est motivé par le
fait que nous travaillons dans le domaine du web et que ces grammaires sont
implémentées au sein de langages du Web, comme PHP ou Javascript, avec des API
simples nous permettant de valider les données que nous allons générer.
%
\begin{figure}

\small

\centering
\begin{tikzbox}{boxxpstrings}{}
\begin{tabular}{lc|c|c|c|c|c|c}

\cline{1-8}

\multicolumn{1}{|l|}{Grammaire / $n$} &
  1                                   &
  2                                   &
  3                                   &
  4                                   &
  5                                   &
  6                                   &
  7 \tikzref{a}                       \\

\cline{1-8}

\multicolumn{1}{|l|}{JSON}            &
  4                                   &
  0                                   &
  6                                   &
  \phantom{0}4                        &
  \phantom{0}30                       &
  \phantom{0}20                       &
  \phantom{0}180                      \\

\cline{1-8}

\multicolumn{1}{|l|}{PCRE}            &
  1                                   &
  4                                   &
  9                                   &
  36                                  &
  117                                 &
  420                                 &
  1525 \tikzref{b}                    \\

\cline{1-8}                           \\
\cline{2-8}

                                      &
  \tikzref{c} 8                       &
  9                                   &
  10                                  &
  11                                  &
  12                                  &
  13                                  &
  \multicolumn{1}{c|}{14}             \\

\cline{2-8}

                                      &
  \phantom{0}128                      &
  \phantom{0}1156                     &
  \phantom{00}848                     &
  \phantom{00}8060                    &
  \phantom{000}6256                   &
  \phantom{00}59596                   &
  \multicolumn{1}{c|}{\scshape to}    \\

\cline{2-8}

                                      &
  \tikzref{d} 5608                    &
  21021                               &
  79528                               &
  304201                              &
  1173288                             &
  4559049                             &
  \multicolumn{1}{c|}{\scshape to}    \\

\cline{2-8}

\end{tabular}
\end{tikzbox}
%
\begin{tikzannotation}
  \node [xshift=-2mm, yshift=.5pt] (A) at (a.north -| boxxpstrings.east) {};
  \node [yshift=-.5pt] (B) at (b.south -| A) {};
  \draw[myzigzag, draw=foreground] (A.center) -- (B.center);

  \node [xshift=-2mm, yshift=.5pt] (C) at (c.north -| d) {};
  \node [yshift=-.5pt] (D) at (d.south -| C) {};
  \draw[myzigzag, draw=foreground] (C.center) -- (D.center);
\end{tikzannotation}

\caption{\label{figure:experimentation:strings} Nombre de données de taille $n$
pour chaque grammaire.}

\end{figure}

La figure~\ref{figure:experimentation:strings} montre un aperçu des tailles des
grammaires en terme de nombres de données d'une certaine taille $n$ qui peuvent
être produites. Dans la figure, {\scshape to} signifie \inenglish{Time Out} et
indique que le temps de génération des données a dépassé 10~minutes. Ces tailles
ont été calculées à partir du nombre de données générées par l'algorithme
exhaustif borné.

Nous avons tout d'abord expérimenté avec la grammaire de JSON pour générer des
descriptions d'objet JSON. Nous avons utilisé l'algorithme exhaustif borné et
l'algorithme basé sur la couverture.

\paragraph{Algorithme exhaustif borné} À cause du nombre important de règles
imbriquées, l'algorithme exhaustif borné n'est pas capable de produire des
données de taille raisonnable couvrant toutes les règles (et ainsi générer des
descriptions complexes d'objet) malgré le fait que nous générions toutes les
descriptions d'objets de taille $n \leq 9$ en un temps raisonnable (quelques
minutes).

\paragraph{Algorithme basé sur la couverture} L'algorithme basé sur la
couverture produit moins de données mais il génère des descriptions complexes
d'objets contenant jusqu'à 32~lexèmes. Il est intéressant de noter que toutes
les règles sont couvertes avec très peu de données, ici 3~données en moyenne
sont nécessaires pour couvrir toute la grammaire de JSON. Nous avons également
remarqué que cet algorithme se comporte comme l'algorithme du postier Chinois
dans le domaine du test de FSM (\inenglish{Finite State Machines}) et tend à
produire une première donnée longue qui couvre le maximum de règles, puis
produit des données plus petites pour couvrir les quelques règles n'ayant pas
été couvertes précédemment. Aussi, contrairement aux algorithmes précédents qui
sont relativement lents (l'algorithme aléatoire uniforme nécessite une étape de
dénombrement exponentielle et l'algorithme exhaustif borné est hautement
combinatoire), cet algorithme est capable de produire des données complexes en
quelques millisecondes. Ainsi, nous l'avons utilisé de manière répétée pour
produire d'importants ensembles de données dont la variété était assurée par les
choix aléatoires dans la résolution des points des choix, comme expliqué dans la
partie~\ref{subsection:data:coverage_based_generation}.

\paragraph{Validation et erreurs} Durant cette évaluation, nous avons trouvé une
erreur dans deux de nos algorithmes, qui était détectée par le compilateur.
L'origine de l'erreur était une mauvaise gestion des caractères échappés, et par
conséquent, des données incorrectes étaient produites à partir d'une grammaire
valide.

Pour évaluer notre compilateur de compilateurs, nous avons réinjecté ces données
dans le compilateur. Nous n'avons relevé aucune erreur durant cette étape~:
toutes les données générées par nos algorithmes ont été correctement analysées
par notre propre compilateur. Additionnellement, nous avons validé ces données
avec l'analyseur JSON de Gecko (créé par Mozilla et utilisé par exemple dans
Firefox pour Javascript) et l'analyseur JSON de PHP. Toutes les données générées
ont été correctement analysées par ces deux autres outils.

Pour tester davantage, nous avons modifié la grammaire en y introduisant des
erreurs, afin de générer des données incorrectes. Nous avons considéré des
opérations simples de mutation de grammaires~\acite{Offutt06}, telles que~:
remplacement des opérateurs de répétitions (\code{+} devient \code{*},
modification des bornes minimales et maximales), suppression d'un lexème ou d'un
appel à une sous-règle dans une règle, ajout de point de choix vers des règles
existantes etc. Nous avons vérifié si les nouvelles données générées étaient
soit acceptées soit refusées par notre compilateur et nous avons comparé les
verdicts avec ceux des deux autres outils. Durant cette validation, nous avons
trouvé une erreur dans notre compilateur qui considérait incorrectes des données
correctes. La cause de cette erreur était une mauvaise gestion du
\inenglish{backtracking}. \\

Après correction de cette erreur, nous avons recommencé l'expérimentation en
entier avec la grammaire des PCRE au lieu de JSON. Aucune erreur n'a été
détectée, validant alors nos algorithmes et notre compilateur.

\subsection{Validation d'applications Web}

Nous avons aussi conçu une expérimentation qui utilise Praspel pour la
validation d'applications Web. Nous avons sélectionné des projets d'étudiants du
cours de «~Langage Web~», dans lequel les étudiants apprennent le langage PHP et
l'utilisent pour générer des documents écrits en HTML. En premier exercice, les
étudiants devaient écrire des fonctions qui vérifient des données provenant de
formulaires, incluant des adresses emails. Nous avons utilisé le domaine
réaliste des adresses emails pour générer des données de test pour ces
fonctions. Aucune erreur n'a été détectée~: toutes les fonctions validaient
correctement les adresses emails générées. Toutefois, nous avons suspecté que
ces fonctions étaient en fait trop «~faibles~» et accepteraient des adresses
emails incorrectes (par exemple, contenant deux symboles «~\code{@}~»). Pour
détecter ces erreurs, nous avons décidé d'utiliser des mutations similaires à
l'expérimentation précédente, mais en utilisant une grammaire pour les adresses
emails, ce afin de générer des adresses invalides. Nous avons comparé les
résultats de toutes ces fonctions avec notre propre fonction qui utilise le
domaine réaliste initial validant des adresses emails correctes. Plusieurs
erreurs ont été détectées. Ce premier exercice montre qu'il est possible
d'utiliser Praspel et les domaines réalistes pour générer des données valides et
invalides.

En second exercice, les étudiants devaient générer des blocs de code HTML qui
seraient par la suite assemblés. Le SUT est une fonction, existant sous
7~versions différentes et faite par 7~étudiants. Cette fonction est utilisée
pour sélectionner une date, représentée par trois balises \code{select},
respectivement pour sélectionner le jour, le mois et l'année. Les paramètres de
cette fonction sont trois entiers représentant les valeurs par défaut de ces
trois balises. La spécification informelle de cette fonction est la suivante~:
%
\begin{itemize}

\item le code produit doit contenir trois balises \code{select} en une ligne~;

\item le code HTML produit doit être «~protégé~» (tous les accents doivent être
remplacés par des entités XML)~;

\item les jours vont de 1 à 31, les mois de 1 à 12 et les années de 2011 à
1911~;

\item uniquement une option (un élément d'une balise \code{select}) peut être
sélectionnée par défaut.

\end{itemize}
%
Nous avons conçu plusieurs prédicats basés basés sur plusieurs degrés de
granularité.
%
\begin{enumerate}

\item Le premier niveau vérifie que le code HTML est bien structuré. Pour y
arriver, nous avons utilisé une grammaire simplifiée du langage XML, vérifiant
que toutes les balises ouvertes sont correctement fermées et que les attributs
sont syntaxiquement corrects. Tous les étudiants ont passé ce niveau.

\item Le deuxième niveau vérifie que le code HTML généré contient trois balises
\code{select} syntaxiquement correctes. Pour y arriver, nous avons utilisé une
grammaire d'un sous-ensemble du langage HTML représentant ces trois balises
\code{select}. 4~étudiants n'ont pas passé ce niveau.

\item Finalement, le troisième niveau ajoute un visiteur dédié qui vérifie le
contenu du code généré~: les valeurs des options pour chaque balise
\code{select}, l'existence d'exactement une option sélectionnée par défaut par
balise etc. Seulement 2~des étudiants restants ont passé ce niveau.

\end{enumerate}

En plus de nous montrer que seulement 2~étudiants sur 7~sont capables de suivre
des spécifications simples, cette expérimentation nous a montré que Praspel est
très commode pour tester des applications. L'approche par les grammaires est
utile pour spécifier le format attendu de code HTML. En plus, la flexibilité
offerte par les visiteurs permet de valider des cas de tests plus complexes, en
vérifiant par exemple des contraintes structurelles qui ne pourraient pas être
exprimées au sein d'une grammaire.
