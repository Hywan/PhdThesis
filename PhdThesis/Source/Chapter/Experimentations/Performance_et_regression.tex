\section{Performance et régression de PHP}
\label{section:experimentation:php}

Les générateurs de données de Praspel ont également été utilisés dans un cadre
de test de performance et de non-régression. PHP a une extension appelée
\code{ext/json} qui permet de manipuler le langage JSON au sein même de PHP
(principalement de l'encodage et du décodage). Cette extension devait être
remplacée, pour des raisons de compatibilité de licences\footnote{Voir
\url{https://bugs.php.net/63520}.}\footnote{Voir
\url{https://wiki.php.net/rfc/free-json-parser}.}. Une nouvelle extension, que
nous appelerons \code{ext/jsond}, a dû être développée.

Pour comparer les performances entre ces deux extensions, il était nécessaire
d'avoir un grand nombre de données JSON. Nous avons alors proposé d'utiliser la
grammaire de JSON et le générateur de chaînes de caractères à partir de
grammaire avec l'algorithme exhaustif pour générer des centaines de milliers de
données de la plus petite jusqu'à une certaine taille.

Une fois toutes ces données générées, les auteurs de \code{ext/jsonp} ont pu
comparer les performances en terme de vitesse et de mémoire des extensions. De
même, ils ont pu comparer que toutes les données étaient manipulées de la même
façon par les deux extensions. L'objectif premier n'était pas de détecter des
erreurs dans les extensions mais de savoir si \code{ext/jsonp} n'introduisait
pas de régression par rapport à \code{ext/json}.

Il nous a fallu approximativement 2~minutes pour écrire le test, qui est un test
paramétré. Les auteurs de l'extension ont généré environ 1~million de données en
20~minutes. Ce nombre a été estimé suffisant pour tester les performances et la
non-régression.

Cet usage inattendu montre l'intérêt des générateurs automatiques de données de
tests dans d'autres types de tests.
