\section{Définitions}
\label{section:test:definitions}

Cette partie présente plusieurs définitions sur les chemins, les tests, les cas
de tests etc. Ces définitions sont importantes pour exprimer et comprendre
correctment les critères de couverture. \\

Tout au long de cette partie, nous considérons une méthode PHP \code{f} et nous
désignons par $G(\code{f})$, ou $G$, son contrat $(V, D, i, N, E, U)$, comme
défini dans la partie~\ref{section:test:contract}.

\begin{definition}[Chemins dans un graphe de contrat]

Un {\strong chemin} est une séquence de transitions consécutives séparées par
des points. Deux transitions $(v, \alpha, w)$ et $(x, \beta, y)$ sont
consécutives si et seulement si $w = x$. Nous disons qu'un sommet $v$ {\strong
est dans} le chemin $p$ si et seulement si le chemin $p$ contient une transition
$(v, \alpha, w)$ ou $(w, \alpha, v)$.

\end{definition}

Nous associons deux sortes de chemins à un contrat $G = (V, D, i, N, E, U)$.

\begin{definition}[Routes et sentiers dans un graphe de contrat]

Une {\strong route} de $G$ est un chemin dont les transitions sont dans $D$.
Soit $r$ une route de $G$. Remplaçons toutes les transitions $(v, e, w)$ dans
$r$ par le chemin $(v, \varepsilon, i_H) \cdot q \cdot (n_H, \varepsilon, w)$ où
$q$ est un chemin de $i_H$ à $n_H$ du graphe $H = (V_H, D_H, i_H, n_H)$ de
l'expression $e$. Le résultat est appelé un {\strong sentier} de $r$. Un sentier
d'un graphe de contrat $G$ est n'importe quel sentier obtenu ainsi, à partir
d'une route de $G$.

\end{definition}

\begin{definition}[Test unitaire, cas de test et suite de tests]

Un {\strong test unitaire} pour une méthode PHP \code{f} est un ensemble de
données (appelées entrées), qui fixent les valeurs pour les
arguments de \code{f}. Un {\strong cas de test} est une paire $(s, t)$ composée
d'un état $s$ (le contexte dans lequel est exécuté le test) et un test unitaire
$t$. Pour construire l'état $s$, nous avons deux approches~:
%
\begin{enumerate}

\item créer un objet et invoquer ses méthodes afin de changer son état~; ou

\item instrumenter le code pour détourner l'encapsulation et définir l'état de
l'objet directement.

\end{enumerate}
%
Une {\strong suite de tests} est un ensemble de cas de test.

\end{definition}

Quand nous exécutons un cas de test, nous sommes capables d'y associer des
chemins du graphe de contrat $G$, appelés chemins {\strong activés par} le cas
de test, démarrant depuis l'état initial de $G(\code{f})$ et terminant dans un
de ses états finaux (soit normaux soit exceptionnels).

Nous représentons par $s'$ l'état après l'exécution du test, par $o$ la valeur
retournée par la méthode \code{f} et par $e$ l'exception levée le cas échéant.
Remarquons que si la méthode lève une exception, alors $o$ est indéfinie. À
l'inverse, si la méthode termine normalement (c'est~à~dire sans lever
d'exception), $e$ est considérée comme indéfinie.

Afin de définir quels sont les chemins (autant les routes que les sentiers) d'un
contrat qui sont activés par un cas de test $(s, t)$, nous explorons le graphe
$G(\code{f})$ et nous collectons les chemins en évaluant les étiquettes des
transitions tout en respectant le test $t$~:
%
\begin{itemize}

\item si la transition est étiquetée par une expression $R$ (respectivement
$I$) provenant d'une précondition (respectivement d'un invariant évalué dans le
pré-état), la transition est activée si et seulement si $R$ (respectivement $I$)
est vraie dans le pré-état $s$~;

\item si la transition est étiquetée par une expression $E$ provenant d'une
postcondition (normale ou exceptionnelle), la transition est activée si et
seulement si $E$ est vraie, quand \aresult est remplacée par la valeur retournée
$o$, les sous-expressions \aold{\empty} sont remplacées par leur valeur dans le
pré-état $s$, et les autres sous-expressions sont évaluées dans le post-état
$s'$~;

\item si la transition est étiquetée par une expression $I$ provenant d'un
invariant évalué dans le post-état, la transition est activée si et seulement si
$I$ est vraie dans le post-état $s'$~;

\item si la transition est étiquetée par une déclaration de domaine réaliste
\code{$i$: $d$}, la transition est activée si et seulement si la valeur courante
de $i \in t$ appartient au domaine réaliste $d$ (grâce à la caractéristique de
prédicabilité). Quand cette déclaration provient d'une précondition
(c'est~à~dire dans un graphe d'expression d'une précondition), la valeur de $i$
est choisie dans le pré-état $s$ du test si $i$ est un attribut de classe, ou
dans les arguments de la méthode si $i$ est un argument de la méthode. Quand la
déclaration est exprimée dans une postcondition (normale ou exceptionnelle), la
valeur de $i$ est choisie dans le post-état $s'$~;

\item si la transition est étiquetée par un prédicat, la transition est activée
si et seulement si le prédicat est satisfait avec l'état courant du système ($s$
dans le cas d'une précondition, $s'$ pour un prédicat \inenglish{before-after},
c'est~à~dire contenu dans une postcondition, dont les expressions \aold{\empty}
sont évaluées dans le pré-état $s$)~;

\item si la transition est étiquetée par un \grule{exception-identifier} $T_C$,
alors la transition est activée si et seulement si l'exception levée $e$ n'est
pas indéfinie et est une instance de $T_C$.

\end{itemize}

Il est possible qu'un test active plusieurs chemins en même temps. Par exemple,
dans la figure~\ref{figure:test:throwable_graph}, si $T_{C_1}$ étend $T_{C_2}$
et qu'une exception de classe $T_{C_1}$ est levée, alors les chemins $v_1 \cdot
v_2 \cdot v_4 \cdot v_5$ et $v_1 \cdot v_2 \cdot v_7 \cdot v_5$ seront activés.
\\

Nous représentons également par $R_\code{f}(S)$ l'ensemble des routes de
$G(\code{f})$ activées par une suite de tests $S$. Similairement, nous
représentons par $T_\code{f}(S)$ l'ensemble des sentiers de $G(\code{f})$
activés par une suite de tests $S$ et traversant des graphes d'expressions. \\

Les exemples suivants sont basés sur le contrat de la
figure~\ref{figure:test:irc} pour la méthode \code{compute} avec deux arguments:
\code{server} qui doit être une classe de type
\code{\bslash{}Irc\bslash{}Server} et \code{buffer} qui doit contenir un message
IRC.
%
\begin{figure}

\begin{bigpre}
/** \\
 * \arequires server: class('\bslash{}Irc\bslash{}Server'); \\
 * \abehavior message \{ \\
 *     \arequires buffer: /^privmessage .+/ or /^message .+/; \\
 *     \aensures  \aresult: 1..; \\
 * \} \\
 * \abehavior ping \{ \\
 *     \arequires buffer: /^ping\$/ and \\
 *               \bslash{}pred('\$server->bufferState   >= 0 or \\
 *                      network\_buffer\_state() >  0'); \\
 *     \aensures  \aresult: 1..; \\
 * \} \\
 * \adefault \{ \\
 *     \athrowable \bslash{}Irc\bslash{}Exception\bslash{}MalformedMessage e \\
 *                    with e->code: 400..491; \\
 * \} \\
 */ \\
public static function compute ( \$server, \$buffer ) \{ … \}
\end{bigpre}

\caption{\label{figure:test:irc} Contrat et méthode manipulant un flux IRC.}

\end{figure}
%
\begin{figure}

\fig{\textwidth}{!}{IRC.tex}

\caption{\label{figure:test:irc_graph} Graphe de contrat associé à la figure
\ref{figure:test:irc}.}

\end{figure}
%
Le graphe de contrat associé est présenté dans la
figure~\ref{figure:test:irc_graph}. Tous les états ont été renommés en $v_i$
avec $1 \leq i \leq 22$, pour éviter toute ambiguïté lorsque nous parlerons des
chemins. Les graphes d'expressions sont représentés, ce qui permet de parcourir
tous les sentiers. Les routes sont représentées par des arcs en
pointillés. Certaines étiquettes sont tronquées pour réduire la taille du
graphe. Dans cet exemple, nous avons deux comportements, un pour un message
privé ou public (représenté par une expression régulière à travers le domaine
réaliste \code{regex} sous sa forme simplifiée) et un pour un «~ping~». Dans le
dernier comportement, la précondition a un prédicat pour spécifier que le réseau
est dans un état spécifique. Enfin, le comportement par défaut spécifie qu'une
exception de type \code{\bslash{}Irc\bslash{}Exception\bslash{}MalformedMessage}
doit être levée si le tampon n'est ni un message ni un ping. De plus, le code
d'une exception doit être un entier entre 400 et 491.

\begin{example}[Test et chemin]

La syntaxe \code{object($c$)} représente une instance valide d'une classe $c$.
Le test $t_1 = \code{compute(object(\bslash{}Irc\bslash{}Server), 'message
foobar')}$ va activer le chemin $(v_1, P, v_4) \cdot (v_4, P_\code{message},
v_7) \cdot (v_7, Q_\code{message}, v_{10})$. En effet, la donnée
\code{object(\bslash{}Irc\bslash{}Server)} représente la valeur de
\code{\$server}. Cet objet est bien une instance de la classe
\code{\bslash{}Irc\bslash{}Server}, donc la transition entre $v_1$ et $v_4$ est
activée. Ensuite, la donnée \code{'message foobar'} représente la valeur de
\code{\$buffer}. Cette donnée active la transition entre $v_4$ et $v_7$
étiquetée par \code{buffer: /\^\empty privmessage .+/ or /\^\empty message .+/}
car la donnée est une chaîne de caractères qui commence par \code{message}.

\end{example}
