\section{Critères de couverture}
\label{section:test:criteria}

Nous définissons maintenant les critères de couverture de contrat, par des
propriétés de l'ensemble des chemins du graphe associé à un contrat Praspel
annotant une méthode PHP.

\subsection{Critère de Couverture de Clause}

Le Critère de Couverture Clause vise à couvrir la structure d'un contrat. Ainsi,
le Critère de Couverture de Clause est satisfait par une suite de tests $S$
d'une méthode \code{f} de graphe de contrat $G(\code{f}) = (V, D, i, N, E, U)$
si tous les états de $V \bslash{}U$ sont dans une route de $R_\code{f}(S)$.

\begin{example}[Suite de tests satisfaisant le Critère de Couverture de Clause]

La suite de tests $\{t_1, t_2, t_3\}$ avec~:
%
\begin{itemize}

\item
$t_1 =$ \code{compute(object(\bslash{}Irc\bslash{}Server), 'message foobar')}~;

\item
$t_2 =$ \code{compute(object(\bslash{}Irc\bslash{}Server), 'ping')}~;

\item
$t_3 =$ \code{compute(object(\bslash{}Irc\bslash{}Server), 'xyz')}.

\end{itemize}
%
satisfait le Critère de Couverture de Clause, c'est~à~dire que ses cas de tests
couvrent toutes les clauses, avec les chemins respectifs suivants (sur les
routes)~:
%
\begin{align*}
%
R_\code{compute}(\{t_1\}) = \{\;
          & (v_1, P, v_4) \\
  \cdot\; & (v_4, P_\code{message}, v_7) \\
  \cdot\; & (v_7, Q_\code{message}, v_{10}) \;\} \\
%
R_\code{compute}(\{t_2\}) = \{\;
          & (v_1, P, v_4) \\
  \cdot\; & (v_4, P_\code{ping}, v_{15}) \\
  \cdot\; & (v_{15}, Q_\code{ping}, v_{18}) \;\} \\
%
R_\code{compute}(\{t_3\}) = \{\;
          & (v_1, P, v_4) \\
  \cdot\; & (v_4, P_\m{D}, v_{19}) \\
  \cdot\; & (v_{19}, T_\m{D}, v_{23}) \;\}
%
\end{align*}

\end{example}

\subsection{Critère de Couverture de Domaine}

Le Critère de Couverture de Domaine raffine le Critère de Couverture de Clause
en considérant la déclaration de domaines réalistes à l'intérieur des graphes
d'expression. Plus formellement, le Critère de Couverture de Domaine est
satisfait par une suite de tests $S$ pour une méthode \code{f} si toutes les
transitions de la forme \code{$i$: $d$} dans le graphe de contrat $G(\code{f})$
(qui sont dans les graphes d'expressions étiquetant les transitions de
$G(\code{f})$ et contenant le nœud $\mathit{dom}$) apparaissent dans au moins un
sentier de $T_\code{f}(S)$.

\begin{example}[Suite de tests satisfaisant le Critère de Couverture de Domaine]

La suite de tests $\{t_1, t_2, t_4\}$ avec~:
%
\begin{itemize}

\item
$t_4 =$ \code{compute(object(\bslash{}Irc\bslash{}Server), 'privmessage foobar')}.

\end{itemize}
%
satisfait le Critère de Couverture de Domaine: la variable \code{server}, les
trois déclarations de domaines réalistes de la variable \code{buffer} et la
variable \aresult sont couvertes. Cette suite de tests produit les chemins
respectifs suivants (sur les sentiers)~:
%
\begin{align*}
%
T_\code{compute}(\{t_1\}) = \{\;
          & (v_1, \varepsilon, v_2) \\
  \cdot\; & (v_2, \code{server: class('\bslash{}Irc\bslash{}Server')}, v_3) \\
  \cdot\; & (v_3, \varepsilon, v_4) \\
  \cdot\; & (v_4, \varepsilon, v_5) \\
  \cdot\; & (v_5, \code{buffer: /\^\empty message .+/}, v_6) \\
  \cdot\; & (v_6, \varepsilon, v_7) \\
  \cdot\; & (v_7, \varepsilon, v_8) \\
  \cdot\; & (v_8, \code{\aresult: 1..}, v_9) \\
  \cdot\; & (v_9, \varepsilon, v_{10}) \;\} \\
%
T_\code{compute}(\{t_2\}) = \{\;
          & (v_1, \varepsilon, v_2) \\
  \cdot\; & (v_2, \code{server: class('\bslash{}Irc\bslash{}Server')}, v_3) \\
  \cdot\; & (v_3, \varepsilon, v_4) \\
  \cdot\; & (v_4, \varepsilon, v_{11}) \\
  \cdot\; & (v_{11}, \code{buffer: /\^\empty ping\$/}, v_{12}) \\
  \cdot\; & (v_{12}, \varepsilon, v_{13}) \\
  \cdot\; & (v_{13}, \apred{…}, v_{14}) \\
  \cdot\; & (v_{14}, \varepsilon, v_{15}) \\
  \cdot\; & (v_{15}, \varepsilon, v_{16}) \\
  \cdot\; & (v_{16}, \code{\aresult: 1..}, v_{17}) \\
  \cdot\; & (v_{17}, \varepsilon, v_{18}) \;\} \\
%
T_\code{compute}(\{t_4\}) = \{\;
          & (v_1, \varepsilon, v_2) \\
  \cdot\; & (v_2, \code{server: class('\bslash{}Irc\bslash{}Server')}, v_3) \\
  \cdot\; & (v_3, \varepsilon, v_4) \\
  \cdot\; & (v_4, \varepsilon, v_5) \\
  \cdot\; & (v_5, \code{buffer: /\^\empty privmessage .+/}, v_6) \\
  \cdot\; & (v_6, \varepsilon, v_7) \\
  \cdot\; & (v_7, \varepsilon, v_8) \\
  \cdot\; & (v_8, \code{\aresult: 1..}, v_9) \\
  \cdot\; & (v_9, \varepsilon, v_{10}) \;\} \\
%
\end{align*}

\end{example}

\subsection{Critère de Couverture de Prédicat}

Similairement au Critère de Couverture de Domaine, le Critère de Couverture de
Prédicat s'applique aux prédicats dans les graphes d'expression, et vise à tous
les couvrir. Plus formellement, le Critère de Couverture de Prédicat est
satisfait par une suite de tests $S$ pour une méthode \code{f} si toutes les
transitions qui ne sont pas de la forme \code{$i$: $d$} dans le graphe de
contrat $G(\code{f})$ (qui sont dans les graphes d'expressions étiquetant les
transitions de $G(\code{f})$ et contenant le nœud $\mathit{pred}$) apparaissent
dans au moins un sentier de $T_\code{f}(S)$.

\begin{example}[Suite de tests satisfaisant le Critère de Couverture de
Prédicat]

La suite de tests $\{t_5, t_6\}$ avec~:
%
\begin{itemize}

\item
$t_5 =$ \code{compute(object(\bslash{}Irc\bslash{}Server), 'ping')} \\
avec \apred{\code{'\$server->bufferState >= 0'}}~;

\item
$t_6 =$ \code{compute(object(\bslash{}Irc\bslash{}Server), 'ping')} \\
avec \apred{\code{'network\_buffer\_state() > 0'}}.

\end{itemize}
%
satisfait le Critère de Couverture de Prédicat. En effet, cette suite de tests
produit les chemins suivants (sur les sentiers)~:
%
\begin{align*}
%
T_\code{compute}(\{t_5\}) = \{\;
          & (v_1, \varepsilon, v_2) \\
  \cdot\; & (v_2, \code{server: class('\bslash{}Irc\bslash{}Server')}, v_3) \\
  \cdot\; & (v_3, \varepsilon, v_4) \\
  \cdot\; & (v_4, \varepsilon, v_{11}) \\
  \cdot\; & (v_{11}, \code{buffer: /\^\empty ping\$/}, v_{12}) \\
  \cdot\; & (v_{12}, \varepsilon, v_{13}) \\
  \cdot\; & (v_{13}, \apred{\code{'\$server->bufferState >= 0'}}, v_{14}) \\
  \cdot\; & (v_{14}, \varepsilon, v_{15}) \\
  \cdot\; & (v_{15}, \varepsilon, v_{16}) \\
  \cdot\; & (v_{16}, \code{\aresult: 1..}, v_{17}) \\
  \cdot\; & (v_{17}, \varepsilon, v_{18}) \;\} \\
%
T_\code{compute}(\{t_6\}) = \{\;
          & (v_1, \varepsilon, v_2) \\
  \cdot\; & (v_2, \code{server: class('\bslash{}Irc\bslash{}Server')}, v_3) \\
  \cdot\; & (v_3, \varepsilon, v_4) \\
  \cdot\; & (v_4, \varepsilon, v_{11}) \\
  \cdot\; & (v_{11}, \code{buffer: /\^\empty ping\$/}, v_{12}) \\
  \cdot\; & (v_{12}, \varepsilon, v_{13}) \\
  \cdot\; & (v_{13}, \apred{\code{'network\_buffer\_state() > 0'}}, v_{14}) \\
  \cdot\; & (v_{14}, \varepsilon, v_{15}) \\
  \cdot\; & (v_{15}, \varepsilon, v_{16}) \\
  \cdot\; & (v_{16}, \code{\aresult: 1..}, v_{17}) \\
  \cdot\; & (v_{17}, \varepsilon, v_{18}) \;\} \\
%
\end{align*}

\end{example}

\subsection{Combinaisons}
\label{subsection:test:combination}

Avoir de tels critères de couverture n'est pas suffisant pour produire ou
qualifier des cas de tests {\em réalistes}. En effet, couvrir toutes les clauses
d'un contrat n'assure pas que tous les domaines réalistes ont été couverts. À
l'inverse, tous les domaines n'assure pas que toutes les clauses du contrat
aient été exploitées. Ainsi, afin d'avoir des cas de tests plus réalistes, ces
critères de couverture de contrat peuvent être combinés ensemble. Nous proposons
les combinaisons suivantes~:
%
\begin{itemize}

\item le Critère de Couverture Clause~+~Domaine vise à couvrir la structure et
les domaines réalistes d'un contrat~;

\item le Critère de Couverture Clause~+~Prédicat vise à couvrir la structure et
les prédicats d'un contrat~;

\item le Critère de Couverture \inenglish{All}-$G$ vise à satisfaire les
critères Clause, Domaine et Prédicat en même temps.

\end{itemize}
%
Ce {\strong treillis} de critères de couverture de contrats est résumé dans la
figure~\ref{figure:test:lattice}.

\begin{figure}

\fig{11cm}{!}{Criteria.tex}

\caption[Hiérarchie des critères.]{\label{figure:test:lattice} Hiérarchie des
critères. $A \rightarrow B$ signifie que si le critère $A$ est satisfait, alors
$B$ est également satisfait.}

\end{figure}

\begin{example}[Suite de tests satisfaisant les combinaisons des Critères de
Couverture de Contrat]

La suite de tests $\{t_1, t_2, t_3, t_4\}$ satisfait le Critère de Couverture
Clause~+~Domaine. La suite de tests $\{t_1, t_3, t_5, t_6\}$ satisfait le
Critère de Couverture Clause~+~Prédicat. La suite de tests $\{t_1, t_3, t_4,
t_5, t_6\}$ satisfait le Critère de Couverture~\inenglish{All}-$G$.

\end{example}
