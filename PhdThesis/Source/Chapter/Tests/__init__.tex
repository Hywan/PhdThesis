\chapter{Critères de couverture de contrat}
\label{chapter:tests}

\mminitoc

Les chapitres précédents présentaient le langage de contrat Praspel et la
génération de données pour plusieurs types. L'étape suivante est la génération
automatique de tests unitaires. Pour accomplir cette étape, il est nécessaire
d'avoir des objectifs de tests. Comme nous sommes en boîte noire, nous allons
une fois de plus exploiter le contrat pour obtenir nos objectifs de tests.

Dans ce chapitre, nous définissons des critères de couverture sur les contrats.
Ces critères vont nous offrir des objectifs de tests. Dans la
partie~\ref{section:test:contract}, nous transformons les contrats en graphe. Ce
formalisme nous permet de plus facilement définir et exprimer les critères sous
la forme d'un chemin dans un graphe. Il sera également plus facile de vérifier
qu'un critère est satisfait. Cette démarche est similaire à la définition des
critères de couverture de code~\acite{MillerM63}. Sur la base de quelques
définitions introduites dans la partie~\ref{section:test:definitions}, nous
établissons les critères de couverture sur un contrat transformé en graphe dans
la partie~\ref{section:test:criteria}.

\require{Chapter/Tests/Couverture_de_contrat.tex}
\require{Chapter/Tests/Definitions.tex}
\require{Chapter/Tests/Criteres_de_couverture.tex}
\require{Chapter/Tests/Synthese.tex}
