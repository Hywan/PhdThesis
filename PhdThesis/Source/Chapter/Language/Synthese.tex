\section{Synthèse}
\label{section:language:summary}

Dans ce chapitre, nous avons tout d'abord présenté les domaines réalistes, des
structures dédiées au test avec deux caractéristiques~: la générabilité et la
prédicabilité, permettant respectivement de générer et de valider une donnée par
rapport au domaine réaliste. Les domaines réalistes sont paramétrables, ce qui
permet de représenter des structures imbriquées. Les domaines réalistes sont
classifiables selon deux axes~: l'héritage et les interfaces.

Nous avons ensuite présenté au niveau syntaxique et sémantique un nouveau
langage de spécification appelé Praspel. Ce dernier repose sur les domaines
réalistes pour exprimer des contrats sur du code. Même s'il se veut être un
langage agnostique, il est tourné vers PHP en essayant de répondre à tous ses
aspects dynamiques.

Praspel permet d'assembler plusieurs méthodes du domaine du test au sein d'un
même langage, de manière pragmatique comme nous le verrons avec les chapitres
suivants. La contribution de chapitre a été publiée dans
l'article~\acitei{EnderlinDGO11}. Nous connaissons maintenant le langage. Nous
savons qu'il permet de générer et valider des données (de test). Le chapitre
suivant s'intéresse à la génération (et à la validation) de plusieurs types de
données.
