\chapter{Description des outils}
\label{chapter:tools}

\minitoc

Ce chapitre est consacré à la description des outils~: comment est implémenté
Praspel, comment il est maintenu, bla bla.

\section{Praspel}
\label{section:tools:praspel}

Cette partie est consacrée à présenter l'implémentation et l'outillage de
Praspel. Tout d'abord, la partie~\ref{subsection:tools:model} présente
l'architecture et le modèle objet de Praspel. Ensuite, la
partie~\ref{subsection:tools:interpretation} présente comment l'interpréter,
suivie de la partie~\ref{subsection:tools:compilation} qui présente comment le
compiler. Puis, la partie~\ref{subsection:tools:evaluation} présente comment
l'évaluer. Et enfin, comme Praspel est un langage de spécification destiné à PHP
construit de façon modulaire et extensible, pour faciliter son développement et
sa maintenance, nous avons choisi de nous appuyer sur le projet Hoa, présenté
dans la partie~\ref{subsection:tools:hoa}.

\subsection{Modèle objet}
\label{subsection:tools:model}

\begin{figure}

\drawfig{10cm}{!}{node distance=3cm}{
    \tikzstyle{rectangle}=[draw=black, thick, text centered, text width=2cm]
    \tikzstyle{arrow}=[->, >=latex, shorten >=1pt, thick]

    \node (model) [rectangle, minimum height=3cm] {
        Object model

        \raisebox{0cm}[1.5cm]{
            \begin{tikzpicture}[node distance=4cm]
            \node [state, scale=.15] (foo1) {};
            \node [state, scale=.15, below left of=foo1] (foo2) {};
            \node [state, scale=.15, below right of=foo1] (foo3) {};
            \node [state, scale=.15, below left of=foo3] (foo4) {};
            \node [state, scale=.15, below right of=foo3] (foo5) {};
            \draw [thick] (foo1) -- (foo2) (foo1) -- (foo3) (foo3) -- (foo4) (foo3) -- (foo5);
            \end{tikzpicture}
        }
    };
    \node (language) [left of=model] {
        Language
    };
    \node (php) [right of=model] {
        PHP
    };
    \node (verdict) [rectangle, below of=model, minimum height=1cm, node distance=4cm] {
        Verdict
    };
    \node (callable) [left of=verdict] {
        Callable
    };

    \draw[arrow] (language.north)
                     -- node[auto] {\small interpreter}
                     ++(0,  1.68)
                     -| ([xshift=-5pt] model.north);
    \draw[arrow] ([xshift=+5pt] model.north)
                     --
                     ++(0,  0.5)
                     -| node[auto, yshift=-.9cm] {\small compiler} (php.north);
    \draw[arrow] (php.south)
                     -- node[auto] {\small execution}
                     ++(0, -1.72)
                     -| ([xshift=+5pt] model.south);
    \draw[arrow] ([xshift=-5pt] model.south)
                     --
                     ++(0, -0.5)
                     -| node[auto, yshift=.8cm] {\small disassembler} (language.south);
    \draw[arrow] (model.south)
                     --
                     node[auto, yshift=-.13cm] {evaluation} (verdict.north);
    \draw[arrow] (callable.east)
                     --
                     (verdict.west);
}

\caption{\label{figure:tools:praspel} Fonctionnement schématique de Praspel.}

\end{figure}

Praspel est un langage qui peut être aussi bien interprété que compilé, pour
être ensuite évalué. La figure~\ref{figure:tools:praspel} présente les
différents processus autour de Praspel. La pièce centrale de Praspel est le
modèle objet~: une variable est un objet, une clause est un objet, une
spécification (un contrat) est un objet etc. Tous ces objets sont imbriqués et
forme une structure qui s'apparente à un arbre. La racine de ce modèle est
représenté par un objet appelé
\code{Hoa\bslash{}Praspel\bslash{}Model\bslash{}Specification}.

Dans les parties suivantes, nous détaillons les différents processus autour de
ce modèle objet.

\subsection{Interprétation}
\label{subsection:tools:interpretation}

Le processus qui transforme le langage Praspel en modèle objet est appelé une
interprétation. Grâce à la grammaire de Praspel exprimée avec le langage PP et
un compilateur, un AST sera produit. Cet AST sera par la suite visité afin de
construire le modèle objet. Pour y arriver, nous avons deux façons de faire. La
première consiste à utiliser la méthode \code{Hoa\bslash{}Praspel::interprete}
qui est raccourci~; ainsi~:
%
\begin{pre}
\$model = Hoa\bslash{}Praspel::interprete('@requires i: 7..42;');
\end{pre}
%
La seconde façon de faire détail tous les outils, à savoir l'utilisation du
compilateur pour produire un AST, puis l'appel d'un visiteur pour interpréter
cet AST et produire le modèle objet~; ainsi~:
%
\begin{pre}
\$compiler    = Hoa\bslash{}Compiler\bslash{}Llk::load( \\
    new Hoa\bslash{}File\bslash{}Read('hoa://Library/Praspel/Grammar.pp') \\
); \\
\$ast         = \$compiler->parse('@requires i: 7..42;'); \\
\$interpreter = new Hoa\bslash{}Praspel\bslash{}Visitor\bslash{}Interpreter(); \\
\$model       = \$interpreter->visit(\$ast);
\end{pre}

Durant la production de l'AST, les erreurs syntaxiques seront détectées. Et
durant la production du modèle objet, les erreurs sémantiques seront détectées.

\subsection{Compilation}
\label{subsection:tools:compilation}

Le processus qui transforme le modèle en code PHP est appelé une compilation.
Cette transformation est appliquée par un visiteur sur le modèle objet. Quand
le code PHP produit est exécuté, il reconstruit le modèle objet depuis lequel il
a été généré. Cela permet de «~sauvegarder~» le modèle en évitant la phase
d'interprétation qui peut être coûteuse dans certaines situations. Pour y
arriver, nous devons faire~:
%
\begin{pre}
\$compiler = new Hoa\bslash{}Praspel\bslash{}Visitor\bslash{}Compiler(); \\
echo \$compiler->visit(\$model);
\end{pre}
%
Cet exemple produira le code PHP suivant~:
%
\begin{pre}
\$praspel = new \bslash{}Hoa\bslash{}Praspel\bslash{}Model\bslash{}Specification(); \\
 \\
\$requires = \$praspel->getClause('requires'); \\
\$requires['i']->in = realdom()->boundinteger(7, 42);
\end{pre}
%
qui permet de construire le modèle objet contenu dans \code{\$model},
c'est~à~dire que \code{\$model} est strictement équivalent à \code{\$praspel}.
Cet exemple nous permet également d'avoir un aperçu de l'API du modèle objet~:
elle se veut le plus simple et compréhensible possible.

\subsection{Désassemblage}
\label{subsection:tools:disassembler}

Le processus qui transforme le modèle objet vers le langage Praspel est appelé
un désassemblage~: c'est l'opération inverse de l'interprétation. Il permet par
exemple d'indiquer des erreurs avec précisions sous la forme originale de
Praspel~; ainsi~:
%
\begin{pre}
\$disassembler = new Hoa\bslash{}Praspel\bslash{}Visitor\bslash{}Praspel(); \\
echo \$disassembler->visit(\$model);
\end{pre}
Cette exemple produira le résultat suivant~:
%
\begin{pre}
@requires i: boundinteger(7, 42);
\end{pre}
%
Notons que le sucre syntaxique \code{7..42} a disparu.

Ce dernier processus, complété des précédents, nous permet de passer d'une
représentation vers n'importe quelle autre représentation de Praspel~: soit
textuelle, soit objet, soit PHP.

\subsection{Évaluation avec un \inenglish{Runtime Assertion Checker}}
\label{subsection:tools:evaluation}

Ce sont les \inenglish{assertion checkers} qui sont responsables d'évaluer
Praspel. À partir d'une spécification, racine du modèle objet, et d'un
\inenglish{callable}, c'est~à~dire une méthode ou une fonction qui représente
notre système sous test, un \inenglish{assertion checker} est capable d'évaluer
la spécification sur le système sous test. Les données nécessaires à
l'évaluation lui sont soit fournies, soit générer par lui-même à partir de la
spécification.

Actuellement, nous nous appuyons sur un \inenglish{runtime assertion checker}
(RAC) pour calculer le verdict du test. Ce verdict est basé sur la vérification
d'assertions à l'exécution. Quand la vérification d'une assertion échoue, une
erreur spécifique est produite. Les erreurs du RAC (aussi appelées les
\inenglish{Praspel failures} ou erreurs Praspel) peuvent être de cinq sortes~:
%
\begin{enumerate}

\item \inenglish{precondition failure}, quand une précondition n'est pas
satisfaite lors de l'invocation de la méthode~;

\item \inenglish{postcondition failure}, quand une postcondition n'est pas
satisfaite après l'exécution de la méthode~;

\item \inenglish{throwable failure}, quand l'exécution de la méthode lève une
exception inattendue~;

\item \inenglish{invariant failure}, quand un invariant de classe est cassé~; et

\item \inenglish{internal precondition failure}, qui correspond à la propagation
d'une \inenglish{precondition failure} à un niveau supérieur.

\end{enumerate}
%
Le test réussit si aucune erreur Praspel n'est détectée. Autrement, il échoue,
et l'erreur avec des informations supplémentaires est consignée.

À présent, illustrons comment utiliser le RAC de Praspel où le système sous test
est une fonction \code{f}~:
%
\begin{pre}
function f ( \$i ) \{ \} \\
 \\
\$rac = new Hoa\bslash{}Praspel\bslash{}AssertionChecker\bslash{}Runtime( \\
    \$model, \\
    xcallable('f') \\
); \\
\$rac->setData(['i' => 13]); \\
\$verdict = \$rac->evaluate();
\end{pre}
%
Dans ce cas, la variable \code{\$verdict} contiendra \code{true} car la
précondition (\code{\arequires i: 7..42}) est bien respectée. Si nous voulons
que les données soient générées automatiquement, nous devrons appeler la méthode
\code{\$ac->au\-to\-ma\-ti\-cal\-ly\-Ge\-ne\-ra\-te\-Data(true)} en plus de
définir le générateur numérique par défaut des domaines réalistes afin de
pouvoir appeler les méthodes \code{sample} sur ces derniers~:
%
\begin{pre}
Hoa\bslash{}Realdom::setDefaultSampler(new Hoa\bslash{}Sampler\bslash{}Random()); \\
\$rac = new Hoa\bslash{}Praspel\bslash{}AssertionChecker\bslash{}Runtime( \\
    \$model, \\
    xcallable('f') \\
); \\
\$rac->automaticallyGenerateData(true); \\
\$verdict = \$rac->evaluate();
\end{pre}

\begin{example}[Vérifications des assertions]

Dans la figure~\ref{figure:language:short_contract}
page~\pageref{figure:language:short_contract}, le système sous test peut être la
méthode \code{store}, le contrat est le contrat associé à cette méthode et les
données sont soit fournies, soit générées automatiquement.  Considérons par
exemple la méthode \code{store} et son contrat dans la
figure~\ref{figure:language:short_contract}.
%
Avec le jeu (\code{'foo'}, \code{null}), nous aurons une \inenglish{precondition
failure}~: la contrainte \code{file: class('File')} n'est pas respectée car la
méthode \code{predicate(\$q)} du domaine réaliste \code{Class} avec $\code{\$q}
= \code{'foo'}$ retourne \code{false}, et aucun autre domaine réaliste n'est
spécifié pour cette variable.
%
Avec le jeu de paramètres (\code{new File(…)}, \code{null}), avec un fichier
vide, nous activerons le comportement par défaut et nous n'observerons aucune
erreur sur les préconditions. Nous sommes maintenant dans le post-état. Si la
méthode \code{store} retourne un booléen et que la méthode \code{isAttached} de
l'objet \code{File} retourne \code{true}, aucune erreur non plus sur la
postcondition.  Si cette méthode retourne autre chose que \code{true} ou que la
méthode \code{store} retourne autre chose qu'un booléen, nous aurons une
\inenglish{postcondition failure}. Si une exception est levée, nous aurons une
\inenglish{throwable failure}.
%
Avec le même jeu de paramètres mais un fichier de grande taille ne pouvant être
enregistré, nous activerons le comportement \code{full}. Nous devrons avoir une
exception levée de type \code{AllocationException}, avec une méthode
\code{getFilesystem} qui retournera l'instance de notre système de fichier. Le
fichier ne devra pas non plus être attaché, \ie sa méthode \code{isAttached}
devra retourner \code{false}. Si une autre exception est levée ou que la
postcondition exceptionnelle n'est pas respectée, nous aurons une
\inenglish{throwable failure}. Si aucune exception n'est levée, nous aurons une
\inenglish{postcondition failure}.
%
Si avant ou après l'exécution de la méthode \code{store}, un invariant est
cassé, nous aurons une \inenglish{invariant failure}. Par exemple, si l'attribut
\code{\_map} contient autre chose que des objets \code{File} ou que sa taille
dépasse 65535.
%
Enfin, si la méthode \code{store} fait appel à la méthode \code{getUsage} en ne
respectant pas sa précondition (ici en lui donnant un argument par exemple),
alors une \inenglish{precondition failure} sera émise depuis la méthode
\code{getUsage}, qui sera ensuite traduite en \inenglish{internal precondition
failure} depuis la méthode \code{store}.

\end{example}

\subsection{Hoa, un ensemble de bibliothèques PHP}
\label{subsection:tools:hoa}

Hoa~\acite{Hoa} se définit comme {\em un ensemble de bibliothèques PHP}
modulaires{\em , } extensibles {\em et } structurées. De plus, {\em Hoa souhaite
être un pont entre le monde de la recherche et de l'industrie}. Ce projet
comporte des bibliothèques solides pour une multitude de domaines. Entre autre,
il offre une consistence entre les versions de PHP et est très respectueux des
standards. Hoa est également impliqué dans plusieurs consortiums ou groupes de
décisions concernant plusieurs standards de l'Informatique ou du Web. Par
conséquent, il est plus facile d'écrire, de maintenir et de s'assurer de la
qualité des programmes constituants Praspel.

De plus, le fait que Hoa souhaite être un pont entre le monde de la recherche et
de l'industrie nous offre l'opportunité de faire valider nos expérimentations
par sa communauté ou ses utilisateurs. Mais aussi, Hoa est développé sous la
licence libre \inenglish{New BSD License}\footnote{Voir
\url{http://hoa-project.net/En/About.html\#License}.}, soit une licence
\inenglish{open-source}. Cela implique que les outils que nous avons développé
durant cette thèse, comme Praspel, le compilateur etc., sont gratuits et libres.
Cette approche offre deux avantages majeurs. Tout d'abord, des contributeurs de
tout horizon peuvent nous aider à corriger ou améliorer nos outils. Ça a été le
cas à plusieurs reprises où des contributeurs ont corrigé des erreurs dans
Praspel ou ont amélioré considérablement le compilateur, notamment en y ajoutant
le support Unicode, en améliorant les performances de l'analyseur lexical (pour
l'analyse de très grandes données) etc. Ensuite, puisque les outils sont
gratuits, des entreprises peuvent l'utiliser et nous offrir des retours
pertinents, comme nous le verrons dans le
chapitre~\ref{chapter:experimentations} avec les expérimentations.

Nos outils prennent la forme de bibliothèques dans Hoa, à savoir
\code{Hoa\bslash{}Realdom} pour les domaines réalistes, \code{Hoa\bslash{}Praspel}
pour le langage Praspel, \code{Hoa\bslash{}Compiler} pour le compilateur et
\code{Hoa\bslash{}Regex} pour les expressions régulières.

\subsubsection{\code{Hoa\bslash{}Realdom}}

La bibliothèque \code{Hoa\bslash{}Realdom} représente la bibliothèque standard
des domaines réalistes. Ils sont implémentés comme présenté dans la
partie~\ref{section:language:realdoms} page~\pageref{section:language:realdoms},
à savoir que chaque domaine réaliste est représenté par une classe.

Actuellement, une liste de 28~domaines réalistes est proposée~:
%
\begin{itemize}

\item \code{Array}, représentant des tableaux~;

\item \code{Bag}, un sac pouvant contenir plusieurs domaines réalistes de
natures différentes~;

\item \code{Boolean}, représentant les booléens~;

\item \code{Boundfloat}, représentant un intervalle de réels~;

\item \code{Boundinteger}, représentant un intervalle d'entiers~;

\item \code{Class}, représentant des instances de classes~;

\item \code{Color}, représentant des couleurs au format \code{\#{\em rrggbb}}~;

\item \code{Constarray}, représentant les tableaux de Praspel, c'est~à~dire une
description de tableau~;

\item \code{Constboolean}, représentant les booléens de Praspel~;

\item \code{Constfloat}, représentant les réels de Praspel~;

\item \code{Constinteger}, représentant les entiers de Praspel~;

\item \code{Constnull}, représentant la valeur nulle de Praspel~;

\item \code{Conststring}, représentant les chaînes de caractères de Praspel~;

\item \code{Date}, représentant une date formatée~;

\item \code{Empty}, représentant une donnée vide~;

\item \code{Even}, représentant les nombres pairs~;

\item \code{Float}, représentant les réels~;

\item \code{Grammar}, représentant les chaînes de caractères spécifiées par une
grammaire~;

\item \code{Integer}, représentant les entiers~;

\item \code{Natural}, représentant les naturels (entiers auto-incrémentés)~;

\item \code{Number}, représentant les nombres (entiers ou réels)~;

\item \code{Object}, représentant un objet donné~;

\item \code{Odd}, représentant les nombres impairs~;

\item \code{Regex}, représentant les chaînes de caractères spécifiées par une
expression régulière~;

\item \code{Smallfloat}, représentant des petits réels~;

\item \code{Smallinteger}, représentant des petits entiers~;

\item \code{String}, représentant des chaînes de caractères~;

\item \code{Timestamp}, représentant des valeurs dans le temps~;

\item \code{Undefined}, représentant des valeurs non-définies.

\end{itemize}

Pour obtenir toutes les informations sur un domaine réaliste, nous pouvons nous
aider de l'outil en ligne de commande \code{hoa realdom:reflection}.

\begin{example}[Informations sur le domaine réaliste \code{Boundinteger}]

Par exemple, pour obtenir des informations sur le domaine réaliste
\code{Boundinteger}, en ligne de commande, nous ferons~:
%
\begin{bigpre}
\$ hoa realdom:reflection boundinteger \\
Realdom boundinteger \{ \\
 \\
    Implementation Hoa\bslash{}Realdom\bslash{}Boundinteger; \\
 \\
    Parent Hoa\bslash{}Realdom\bslash{}Integer; \\
 \\
    Interfaces \{ \\
 \\
        ArrayAccess; \\
        Countable; \\
        IteratorAggregate; \\
        Traversable; \\
        Hoa\bslash{}Realdom\bslash{}IRealdom\bslash{}Enumerable; \\
        Hoa\bslash{}Realdom\bslash{}IRealdom\bslash{}Finite; \\
        Hoa\bslash{}Realdom\bslash{}IRealdom\bslash{}Interval; \\
        Hoa\bslash{}Realdom\bslash{}IRealdom\bslash{}Nonconvex; \\
        Hoa\bslash{}Realdom\bslash{}Number; \\
        Hoa\bslash{}Visitor\bslash{}Element; \\
    \} \\
 \\
    Parameters \{ \\
 \\
        [#0 optional] Constinteger lower = -9223372036854775808; \\
        [#1 optional] Constinteger upper = 9223372036854775807; \\
    \} \\
\}
\end{bigpre}
%
Comme informations, nous avons le nom de la classe qui représente
l'implémentation du domaine réaliste, nous avons toutes les interfaces utilisées
par cette implémentation (ici plusieurs de PHP et des domaines réalistes, comme
présenté dans la partie~\ref{subsection:language:realdom:classification}
page~\pageref{subsection:language:realdom:classification}), ainsi que les
paramètres avec leurs positions, filtres et valeurs par défaut.

\end{example}

Cette bibliothèque peut être utilisée seule, sans Praspel.

\subsubsection{\code{Hoa\bslash{}Praspel}}

La bibliothèque \code{Hoa\bslash{}Praspel} est responsable de tout le support de
Praspel, de l'interprétation à l'évaluation en passant par la compilateur, la
couverture des contrats etc.

La bibliothèque est découpée de la façon suivante~:
%
\begin{itemize}

\item \code{AssertionChecker}, permet d'évaluer le langage~;

\item \code{Bin}, contient des scripts, dont un \inenglish{shell} permettant
d'analyser et évaluer du code Praspel à la volée~;

\item \code{Exception}, contient toutes les catégories d'exceptions de Praspel~;

\item \code{Iterator}, contient entre autre les générateurs de tests unitaires à
partir de plusieurs critères de couverture sur un contrat~;

\item \code{Model}, le cœur du langage Praspel est représenté par un modèle
objet~;

\item \code{Preambler}, permet de créer un préambule de test pour n'importe quel
système sous test~;

\item \code{Visitor}, contient l'interpréteur et plusieurs compilateurs.

\end{itemize}

À la racine de la bibliothèque, nous trouvons entre autre la grammaire, une
classe nommée \code{Hoa\bslash{}Praspel\bslash{}Praspel} qui rassemble les
opérations usuelles sur le langage et une classe représantant une trace
d'évaluation d'un contrat.

\subsubsection{\code{Hoa\bslash{}Compiler}}

\subsubsection{\code{Hoa\bslash{}Regex}}

\section{atoum}

\section{Synthèse}
