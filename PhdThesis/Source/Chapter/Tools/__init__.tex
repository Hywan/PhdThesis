\chapter{Implémentations et outillage}
\label{chapter:tools}

\mminitoc

\mylettrine[findent=.5em, nindent=.5em, slope=-.5em]{D}{ans les chapitres précédents} nous avons présenté
le langage Praspel et plusieurs algorithmes de générations de données ainsi que
des critères de couverture. Ce chapitre est consacré à leur implémentation. Il
est organisé en trois parties. Tout d'abord, dans la
partie~\ref{section:tools:hoa}, nous présentons Hoa, un ensemble de
bibliothèques PHP. Nos contributions sont implémentées sous la forme de
bibliothèques au sein de Hoa, qui est un projet \inenglish{open-source}.
Ensuite, dans la partie~\ref{section:tools:praspel}, nous présentons notre
implémentation de Praspel, son fonctionnement et son découpage. Enfin, dans la
partie~\ref{section:tools:atoum}, nous présentons l'intégration de ces outils
dans atoum, un \inenglish{framework} de tests unitaires, ainsi qu'une extension
qui fait le pont entre Praspel et atoum.

\require{Chapter/Tools/Hoa.tex}
\require{Chapter/Tools/Praspel.tex}
\require{Chapter/Tools/atoum.tex}
\require{Chapter/Tools/Synthese.tex}
