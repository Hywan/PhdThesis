\section{Synthèse}
\label{section:tools:summary}

Dans ce chapitre, nous avons présenté l'implémentation de Praspel sous forme de
plusieurs bibliothèques dans le projet Hoa (\code{Hoa\bslash{}Praspel},
\code{Hoa\bslash{}Real\-dom}, \code{Hoa\bslash{}Com\-pi\-ler} et
\code{Hoa\bslash{}Regex}). Nous avons vu que la pièce centrale de Praspel est
son modèle objet. Le langage peut être analysé, exporté, importé et désassemblé.
Un \inenglish{Runtime Assertion Checker} est proposé pour évaluer un SUT à
l'exécution spécifié par un contrat représenté par son modèle objet.
L'évaluation peut se faire avec des données de test fournies manuellement ou
générées automatiquement.

Praspel a été intégré dans atoum, un \inenglish{framework} de tests unitaires,
afin de profiter du moteur d'exécution, de la production des rapports, de
l'intégration à plusieurs IDE et outils industriels, des API avancées à
l'intérieur d'atoum etc. atoum n'avait par ailleurs pas de générateurs de
données de test~: c'est pourquoi à travers l'extension
\code{atoum\bslash{}praspel-extension}, nous proposons de nouvelles API pour
générer des données de test grâce aux travaux présentés dans ce mémoire. En
plus, cette extension permet de transformer une suite de test générée par
Praspel en une suite de test exécutable avec atoum. Enfin, cette extension est
incluse dans la distribution standard d'atoum «~sous stéroïde~».

Grâce à cette extension, aux utilisateurs et aux licences
\inenglish{open-source} de Hoa et d'atoum, nous avons pu mener des
expérimentations présentées dans le chapitre suivant.
