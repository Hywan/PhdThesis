\section{Synthèse}
\label{section:sota:summary}

Dans ce chapitre, nous avons décrit des langages de contrat de plusieurs
langages de programmation. Nous avons remarqué que les langages majeurs ont leur
propre langage de contrat. Ce n'est pas le cas de PHP. Nous avons vu par la
suite comment ces langages de contrat étaient utilisés pour générer des suites
de test. Cette technique s'appelle le \inenglish{Contract-based Testing}~: les
invariants et les préconditions sont utilisés pour générer des données de test
et les postconditions fournissent un oracle permettant de calculer le verdict du
test. Un RAC (\inenglish{Runtime Assertion Checker}) permet de valider ce
verdict à l'exécution. Enfin, nous avons vu d'autres techniques de génération de
données notamment pour des chaînes de caractères et des tableaux.

Certains des outils ne respectent pas vraiment le \inenglish{Contract-based
Testing} car ils demandent à l'utilisateur de définir des données manuellement.
Toutefois, ils offrent un gain de temps non négligeable par la production d'une
suite de test exécutable. D'autres outils génèrent eux-mêmes les données mais à
l'aide de solveurs de contraintes car les assertions à l'intérieur des contrats
ou dans le code du programme sont exprimées avec des prédicats ou des
expressions similaires. Malgré le niveau d'expressivité important que cela
offre, il est difficile pour les solveurs de générer des données satisfaisant
toutes ces contraintes ou de passer à l'échelle, c'est~à~dire d'appliquer ses
techniques sur de très grands programmes. Ainsi, les données de test générées
vont permettre de détecter les erreurs à l'exécution mais rien n'assure que le
programme fait bien ce que l'utilisateur en attend.

Nous remarquons qu'il existe plusieurs techniques et plusieurs approches.
Malheureusement, malgré leur intérêt, leur «~éparpillement~» les rend
inutilisables pour un ingénieur de test car la chaîne de développement
deviendrait trop complexe. Une solution est alors de les assembler de manière
cohérente.

Nous proposons Praspel, un langage de spécification pour PHP à partir de
contrats. Ce langage trouve son inspiration dans JML ou ACSL. Les outils de
Praspel sont capables de générer automatiquement des suites de test exécutables
à partir de contrats. Les données de test seront générées automatiquement. Un
juste milieu a été trouvé pour exprimer les assertions à l'intérieur des
contrats, supportant la validation et la génération pour toutes les
constructions. C'est un langage simple (pour l'utilisateur), pragmatique et
suffisament souple pour assembler plusieurs méthodes du domaine du test, comme
nous le verrons dans les chapitres suivants.
