\section{\inenglish{Contract-based Testing}}
\label{section:sota:cbt}

Nous avons vu comment les contrats sont utilisés pour {\strong filtrer} les
données manipulées par un programme ou pour vérifier que le programme ne
produira pas d'erreur. Cependant, l'un des intérêts et usages principaux des
contrats est pour le {\strong test}. Le {\strong test à partir de contrats}
(\inenglish{Contract-based Testing}, abrégé CbT) a été introduit pour exploiter
les langages de contrats afin de tester des programmes~\acite{Aichernig03}~:

\begin{itemize}

\item les invariants et les préconditions sont utilisés pour {\strong générer
des données} de test~;

\item les postconditions {\strong fournissent un oracle} permettant de calculer
le verdict du test à l'exécution.

\end{itemize}
%
Rappelons qu'un test est composé de trois parties~: un SUT, des données de test
et un oracle. En général, le contrat annote le programme, donc nous connaissons
le SUT. Les deux dernières parties nous sont données en exploitant le contrat.

À partir d'une spécification formelle, sous la forme d'un contrat, nous sommes
capables de générer des suites de tests automatiquement. Comme nous sommes au
niveau du code source du programme, et plus précisément, sur les méthodes des
classes (dans un contexte orienté objet), les tests seront {\strong unitaires}.

Différents outils exploitant des contrats pour générer des suites de tests sont
présentés ci-dessous.

\paragraph{JMLUnit} L'outil JMLUnit~\acite{CheonL02, ZimmermanN10} permet de
transformer un contrat JML en une suite de tests exécutable avec JUnit. L'outil
assure un moyen simple et efficace d'automatiser la génération d'oracle de
tests. L'objectif est de décharger l'utilisateur du calcul du verdict de test, à
savoir si le test unitaire est un succès ou un échec. L'outil génère
automatiquement deux classes pour chaque interface ou classe Java annotée par du
JML~: un squelette où déclarer les données de tests et les oracles permettant de
calculer le verdict des tests.

Les données de tests ne sont pas générées automatiquement~: un dictionnaire de
données de tests pré-définies selon certaines stratégies est utilisé à la place.
Pour des types de données non-primitifs au langage (donc des données de tests
qui ne sont pas des booléens, des nombres, des chaînes de caractères etc.), le
dictionnaire ne contient qu'une valeur nulle. C'est alors à l'utilisateur de
calculer lui-même les données de tests.

Comme l'outil se concentre sur le test unitaire, chaque test ne concerne qu'une
seule méthode. Les postconditions et invariants JML sont transformés en
prédicats Java et constituent les oracles de tests. Si l'exécution d'une méthode
avec un certain ensemble de données de tests ne lève aucune erreur, alors le
test est considéré comme un succès, sinon il est considéré comme un échec. Il
est toutefois possible d'obtenir un troisième résultat qui est une violation
d'une précondition d'une autre méthode appelée par la méthode testée.  Ce cas
est considéré comme inconclusif par l'outil car aucune précondition ou
postcondition n'a été violée.

Enfin, le rapport de test est le même que celui produit habituellement par
JUnit.

\paragraph{Jartege} L'outil Jartege~\acite{Oriat05} est un générateur de tests
aléatoires pour des programmes écrits en Java. Jartege produit des séquences de
test d'une certaine taille et un nombre de séquences, fixés par
l'utilisateur.

Jartege sélectionne aléatoirement des méthodes à exécuter. Chaque méthode est
annotée par un contrat JML. Les données de tests, soient les valeurs des
paramètres de ces méthodes, sont également générées aléatoirement à partir de la
précondition. Cette dernière sert aussi de filtre pour sélectionner les méthodes
à exécuter. L'exécution des méthodes et les postconditions et invariants sont
utilisés pour connaître la conformité du programme avec sa spécification. La
vérification des contraintes est ainsi conjointe à la création des séquences de
test.

Pour les types primitifs de Java, Jartege demande à l'utilisateur de leur
définir des bornes pour la génération aléatoire, ceci afin de générer des
données de tests plus pertinentes. L'outil permet à l'utilisateur de fournir une
fonction qui détermine la probabilité de créer un nouvel objet ou d'en
réutiliser un déjà existant. Selon les structures de données manipulées, cette
fonction permettra de détecter plus ou moins d'erreurs.

\paragraph{JML-Testing-Tools} L'outil JML-Testing-Tool~\acite{BouquetDLU05,
BouquetDL06} est développé comme une extension aux outils BZ Testing
Tools~\acite{LegeardPU02}, un ensemble d'outils d'animation et de génération
automatique de programmes B~\acite{Abrial05}, Z~\acite{Spivey89} ou de
spécifications Statechart~\acite{Harel87}. L'outil JML-Testing-Tool comprend un
animateur symbolique et un générateur de tests.

Le premier permet d'animer symboliquement des méthodes Java en utilisant des
spécifications écrites en JML. Le code des méthodes n'est jamais utilisé car
l'outil manipule des modèles. JML-Testing-Tool permet d'appeler des méthodes
avec des paramètres symboliques. Les valeurs de ces paramètres sont des
contraintes définies sur le domaine lié au type des paramètres. Ces contraintes
sont supportées par un solveur spécifique. Par conséquent, JML-Testing-Tool est
capable de représenter un grand nombre d'états d'un programme. Une valuation
pour obtenir l'ensemble de tous les états concrets est possible. L'outil est
également capable de vérifier les propriétés JML à la volée et d'afficher un
contre-exemple si l'une d'entre elles est violée.

Le générateur de tests mélange les spécifications JML et le modèle. L'outil va
tout d'abord extraire un ensemble de comportements (normaux et exceptionnels) à
partir de la spécification JML. Un comportement est représenté par un prédicat
en forme disjonctive. Cet ensemble de comportements va être réduit en fonction
du critère de couverture du modèle choisi par l'utilisateur. L'outil propose
quatre critères de couverture qui représentent des règles de réécritures de ces
prédicats. Les données de tests sont générées automatiquement en utilisant les
valeurs aux limites des paramètres des méthodes. Cette approche fonctionne pour
les types primitifs de Java avec un domaine ordonné. L'outil considère également
qu'une valeur aux limites d'un objet est une valeur aux limites de l'un de ses
attributs. Quand toutes les valeurs sont spécifiées, elles sont instanciées
grâce à l'animateur symbolique et les tests peuvent être exécutés.

\paragraph{TestEra} L'outil TestEra~\acite{MarinovK01} est un langage
d'annotation de code Java. Ses préconditions, postconditions et invariants
expriment des contraintes avec le langage Alloy~\acite{Jackson02}. Alloy est un
langage de spécification déclaratif pour exprimer des contraintes structurelles
complexes. Son objectif est d'instancier des micro-modèles, à l'aide de son
propre analyseur Alcoa~\acite{JacksonSS00}. TestEra passe des invariants
accompagnés de préconditions à l'analyseur d'Alloy, qui va lui générer plusieurs
instances de modèles satisfaisant cette spécification. Chaque instance générée
par Alcoa représente un test. TestEra va concrétiser ces instances une par une,
afin de produire des données valides en Java. Ces dernières sont utilisées en
tant que données de tests, c'est~à~dire qu'elles vont servir à exécuter le SUT.
Ensuite, TestEra regarde les sorties produites par le SUT ainsi que son
post-état, qui sont abstraits pour se ramener au formalisme d'Alloy afin d'y
être confrontés à la postcondition et aux invariants. Si cette dernière
vérification échoue, un contre-exemple est produit. Sinon, le processus itère à
l'instance suivante.

\paragraph{Korat} L'outil Korat~\acite{BoyapatiKM02} est la suite de TestEra.
L'objectif de Korat est de couvrir des structures de données
complexes non-isomorphiques de taille finie. Pour cela, il demande à
l'utilisateur d'écrire un prédicat en Java (appelé \code{repOK}), qui va valider
ou invalider les structures générées automatiquement. Il y a plusieurs avantages
à cette démarche~: l'utilisateur est familier avec le langage puisqu'il
l'utilise pour développer, il y a plusieurs outils et environnements de
développement qui peuvent aider à la rédaction ou l'analyse de ce prédicat, et
enfin, le prédicat peut potentiellement déjà exister dans le programme. Korat va
ensuite générer des structures à partir des entrées de ce prédicat. Pour cela,
Korat va analyser le prédicat et indexer quels sont les accès aux méthodes et
attributs sur la structure qui influencent la décision du prédicat. Korat va
ensuite jouer sur ces méthodes et attributs afin de générer plusieurs structures
candidates. Korat demande également à l'utilisateur d'écrire une méthode de
\inenglish{finitization} qui précise la taille et la nature des données portées
par la structure. L'idée est de préciser à Korat dans quelle mesure il peut
modifier la structure pour générer de nouvelles structures candidates. Cette
technique assure qu'il y a un ensemble fini de structures candidates qui vont
être sélectionnées.

Comme TestEra avant lui, Korat est utilisé depuis des annotations d'un programme
Java. Plus précisément, les prédicats de Korat sont appelés depuis JML et la
postcondition fournit l'oracle du test. Les tests sont alors exécutés sur un
programme Java enrichi.

Korat se démarque de TestEra en proposant ses propres algorithmes de génération
de données candidates à partir de prédicats Java. Cette approche demande un
effort supplémentaire à l'utilisateur, car en plus d'écrire des contrats en JML,
il doit écrire une méthode pour un prédicat et une autre pour une
\inenglish{finitization}. Notons que cet effort est partiellement compensé par
le fait que, d'une part, écrire des spécifications est une bonne pratique, et
que d'autre part, l'utilisateur connaît déjà le langage dans lequel écrire ces
méthodes et que les prédicats lui seront très propablement utiles dans son
programme.

\paragraph{UDITA} Le langage UDITA~\acite{GligoricGJKKM10} répond à l'objectif
suivant~: s'assurer que des structures de données complexes sont bien supportées
par un programme Java, c'est~à~dire qu'un programme sait manipuler sans erreur
toutes les formes d'une certaine structure de données complexes.  Pour atteindre
cet objectif, UDITA se présente comme un langage étendu du langage Java
permettant de décrire des structures de données complexes. Ces descriptions vont
être utilisées pour générer des ensembles finis de ces structures. UDITA est la
suite de Korat. Nous retrouvons ainsi la démarche de demander à l'utilisateur
d'écrire lui-même des méthodes de génération de données (description des
structures) mais les contrats ont totallement disparu. UDITA ajoute à Java deux
extensions~: des primitives de choix non-déterministes et une primitive pour
restreindre ces choix. Ces constructions sont familières aux utilisateurs de
vérificateur de modèle comme Java PathFinder~\acite{VisserHBPL03}, sur lequel il
s'appuye. UDITA est d'ailleurs intégré au \inenglish{framework} Java PathFinder
en tant qu'extension. Grâce aux points de choix non-déterministes, l'utilisateur
ne décrit qu'une seule fois sa structure, et UDITA est capable d'en générer
plusieurs. L'outil propose aussi à l'utilisateur des générateurs basiques, comme
un générateur de booléens, d'entiers, de nouveaux objets ou d'objets déjà créés.
Ces deux derniers permettent à l'utilisateur de décrire des structures de
données récursives ou liées. En plus d'une description, UDITA demande à
l'utilisateur d'écrire un filtre, notion identique aux prédicats de Korat, si ce
n'est qu'UDITA ne va pas s'en servir pour la génération de données, mais
uniquement pour filtrer les données générées~: chaque donnée qui franchit le
filtre sera mise de côté pour servir de données de test.

Contrairement à Korat, UDITA ne demande pas de méthode de
\inenglish{finitization}, mais s'appuie sur ces points de choix
non-déterministes pour explorer l'espace de la structure tout en restant
borné~\acite{Marinov03, SullivanYCKJ04}. L'efficacité de la méthode s'appuie
sur les choix retardés~\acite{NollS07}, c'est~à~dire sur des évaluations
non-déterministes tardives~\acite{FischerKS09}.

\paragraph{JSConTest} Le langage JSConTest~\acite{HeideggerT12} permet d'annoter
des programmes Javascript avec des contrats. Javascript est un langage à typage
non déclaré. C'est pourquoi JSConTest propose que le contrat ne contienne que
la signature de la fonction annotée, soit ses entrées et sa sortie,
respectivement une précondition et une postcondition. JSConTest ne permet pas
d'exprimer d'invariants ni de comportements. Le fait de préciser les types
permet à JSConTest de réduire l'effort d'inférence en se limitant à un minimum
de fonctions à analyser. Les résultats de l'inférence servent à guider la
génération aléatoire de données de test. Le langage permettant d'exprimer la
signature est enrichi de quelques constructions également pour guider la
génération aléatoire. Un outil permet d'exécuter les tests et d'en calculer le
verdict grâce à un RAC.

JSConTest, contrairement à JML ou UDITA, demande très peu d'efforts à
l'utilisateur. Les auteurs pensent qu'écrire la signature ne peut pas être
négatif pour le développeur, surtout dans un langage où les types ne sont pas
déclarés. Cependant, le niveau d'expressivité sur les préconditions et
postconditions reste très faible. En revanche, comme démontré dans le travail
sur les types graduels~\acite{SiekT07}, pour un langage à typage dynamique et
faible comme Javascript ou PHP, même un contrat très simple qui spécifie le
typage des données peut être très utile et offrir de bons résultats, notamment
dans la détection de régressions. \\

%Le test à partir de contrat répond au constat suivant. La validation
%dynamique souffre d'une limitation majeure~: elle fournit des résultats limités.
%Les spécifications sont souvent vérifiées avec les données des exécutions
%courantes et rarement de manière exhaustive. Ainsi, la validation dynamique
%résulte souvent à une détection incomplète et tardive (potentiellement en
%production) de défauts. Et pourtant, «~un test montre la présence, jamais
%l'absence d'une erreur~» (Edsger W.~Dijkstra dans~\acitep{Buxton70}). Et un bon
%test est un test qui échoue.  Pour savoir ce qu'est un «~bon test~», il existe
%plusieurs métriques pour le qualifier~: les critères de couverture de
%code~\acite{MillerM63, Myers04, GargantiniGM12}. Si plusieurs tests remplissent
%les critères les plus élevés de la couverture de code, alors une importante
%partie des erreurs serait détectée. Toutefois, il faut être conscient que ces
%critères ne sont pas non plus exhaustives. En effet, ce n'est pas parce qu'un
%test à activer tous les chemins que toutes les erreurs ont été évitées. De plus,
%des erreurs peuvent se cacher dans l'environnement d'exécution du programme.
%Notons que si nous travaillons en boîte noire, ces critères ne sont pas
%appliquables, et nous pouvons uniquement nous référer aux postconditions des
%contrats.
