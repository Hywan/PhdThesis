\chapter{Perspectives}
\label{chapter:perspectives}

\minitoc

Les perspectives qui se dégagent de ces travaux se classent en trois parties. La
première concerne l'utilisation d'autres méthodes du test dans Praspel. La
deuxième partie concerne des améliorations possibles sur le langage. Et enfin,
la dernière partie offre une réflexion sur l'avenir des travaux présentés dans
ce mémoire.

\section{Inclure de nouvelles méthodes du test}

Grâce aux domaines réalistes et à la conception modulaire et extensible des
outils de Praspel, il est possible d'assembler plusieurs méthodes du test. Les
expérimentations ont montré certaines limites du langage et des outils. Nous
pensons qu'il est pertinent de s'intérer aux méthodes suivantes.

\subsection{Test aux limites}

Le test aux limites permet de détecter des erreurs usuelles rapidement et
efficacement. Nous avons dit que les domaines réalistes utilisaient un
générateur numérique aléatoire et uniforme pour générer des données. Nous
pensons qu'il est pertinent de réfléchir à une manière de spécifier les limites
des domaines réalistes. Par exemple, avec l'intervalle $[-7; 42]$, les limites
extrêmes sont $-7$, $-6$, 41 et 42, et les limites intermédiaires sont $-1$, 0
et 1. En revanche, définir les limites d'un tableau, d'une chaîne de caractères
ou d'un objet n'est pas aussi trivial et mérite plus d'attention. Une fois ses
limites connues, nous pourrions utiliser un générateur numérique aux limites
dans les domaines réalistes pour produire de nouvelles données aux limites.

\subsection{Scénario et propriétés temporelles}

Dans le contexte du \inenglish{Scenario-based Testing}, nous nous intéressons
aux travaux de \acitei{Castillos13}. Ces travaux s'inspirent des patrons de
propriétés décrits par \acitei{DwyerAC99}. Ces patrons permettent de définir une
propriété temporelle comme un motif qui doit être satisfait à l'intérieur d'une
portée (une portion d'exécution du système).  Ils permettent donc d'exprimer
simplement un large panel de propriétés temporelles avec un ensemble restreint
de constructions. Un nouveau langage~\acite{CastillosDJKT13} basé sur ces
patrons a été créé. Ce langage comprend des variantes aux motifs et aux portées
ajoutées pour améliorer l'expressivité et corriger certaines incohérences. Des
nouveaux critères de couverture ont alors été définis et permettent de vérifier
plusieurs propriétés temporelles.

La démarche de~\acitea{Castillos13} est similaire à la notre mais les objectifs
de tests sont différents. Cependant, puisque ce langage est très simple et
permet d'exprimer un grand nombre de propriétés temporelles, nous pensons que ce
langage pourrait être une extension à Praspel. Ainsi, en plus des attributs et
des méthodes, les classes pourraient être annotées par ces propriétés. Nous les
exploiterions pour générer de nouveaux préambules de test, des suites de tests
fonctionnels ou alors pour vérifier des propriétés temporelles.

\subsection{\inenglish{Grey-box} avec le \inenglish{Search-based Testing}}

Dans les expérimentations, nous avons vu certaines limites de Praspel. La
méthode du \inenglish{Search-based Testing}~\acite{McMinn04} utilise des
métaheuristiques pour générer automatiquement des données de tests. Le SUT est
en premier lieu exécuté avec des données aléatoires. Des fonctions de
\inenglish{fitness} (fonctions objectifs) placées sur les points de choix du SUT
permettent d'évaluer la «~distance~» entre les données courantes et les données
nécessaires pour activer ou désactiver ce point de choix. En utilisant la
programmation génétique, ces données initiales sont mutées pour former une
nouvelle population de données. Des algorithmes de sélections vont réduire cette
population. Chaque donnée de cette population sera ensuite réutilisée comme
données de tests pour exécuter à nouveau le SUT. Les fonctions de
\inenglish{fitness} vont produire de nouveaux résultats qui serviront à guider
l'évolution de la population de données. Par exemple, avec la condition \code{x
== y}, la fonction de \inenglish{fitness} sera $\mathrm{abs}(\code{x} -
\code{y})$. Quand cette fonction tend vers 0, alors la condition est activée.

Les suites de tests générées par Praspel sont basées sur les critères de
couverture des contrats. Les données de test, spécifiées majoritairement avec
les domaines réalistes, sont générées aléatoirement. Si le résultat des
fonctions de \inenglish{fitness} pouvait être remonté dans les domaines
réalistes pour mieux guider les nouvelles générations de données, nous pensons
que nous pourrions détecter plus d'erreurs, ou du moins, couvrir le code source
plus efficacement et plus rapidement. En effet, les domaines réalistes ont
plusieurs interfaces (comme \code{Interval}, \code{Finite}, \code{Nonconvex}
etc.). Ces interfaces imposent l'implémentation de méthodes permettant de
contraindre ou manipuler les données contenues dans le domaine réaliste. Il doit
être possible de transposer les résultats des fonctions de \inenglish{fitness}
dans les domaines réalistes.

Cette approche, d'une part, permettrait d'exploiter le code source au lieu de
rester en boîte noire (nous serions alors en boîte grise), et, d'autre part, ne
compliquerait pas le langage en y ajoutant de nouvelles constructions, mais
offrirait plus de services à l'utilisateur. Ce dernier n'aurait aucun effort à
produire à part mettre à jour ses outils.

\section{Améliorer le langage}

Le langage Praspel n'est pas complet. Certaines choses peuvent être mieux
spécifiées. Nous sommes toujours dans cette démarche d'analyser ce que les
développeurs écrivent dans les contrats et d'améliorer le langage en
conséquence. Nous avons relevé deux points importants.

\subsection{Solveurs sur les entiers et réels}

Quand des entiers sont manipulés, il est courant de vouloir spécifier des
constraintes comme $\code{i} < \code{j} + 1$. Actuellement, Praspel n'a aucun
solveur sur les entiers et les réels. Malgré la pertinence d'une telle
fonctionnalité, nous n'avons pas eu le temps de la dévélopper. Il serait alors
nécessaire d'étudier le domaine, de voir comment nous pouvons y contribuer, et
proposer une solution dans Praspel.

\subsection{De nouveaux domaines réalistes}

Durant l'expérimentation, les ingénieurs de tests ont trouvé notre approche
pertinente. Toutefois, la bibliothèque standard des domaines réalistes n'était
pas suffisante. Les ingénieurs ont demandé la création de plus de domaines
réalistes représentant des données métiers, comme des numéros de comptes
bancaires, des numéros de téléphones etc. Ce ne sont pas des perspectives
scientifiques mais industrielles. Toutefois, elles participent à l'adoption de
nos outils dans l'industrie.

\section{Vers des outils industriels~?}
