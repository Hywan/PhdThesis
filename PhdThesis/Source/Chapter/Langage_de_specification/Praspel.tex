\section{Praspel}
\label{section:language:praspel}

Praspel est un acronyme anglais signifiant {\em PHP Realistic Annotation and
SPEcification Language}. C'est un langage et un framework pour du test à partir
de contrats en PHP (voir la
Section~\ref{subsection:language:contract-based_testing}). Ce langage s'appuye
sur les domaines réalistes.

Praspel est un {\em langage d'annotation} car il s'écrit dans les commentaires
du code source du programme. PHP comprend trois catégories de commentaires~: en
ligne (avec \code{//} ou \code{\#}), multi-lignes (entre \code{/*} et \code{*/})
et en bloc (entre \code{/**} et \code{*/}). C'est dans cette dernière catégorie
que nous écrirons Praspel, catégorie qui est culturellement dédiée à l'écriture
des annotations, documentations API etc.

Praspel est un {\em Behavioral Interface Specification Language} (BISL) basé sur
les contrats. Un {\em contrat} est un modèle du comportement du code, décrit à
travers des contraintes formelles, alors appelées clauses, comme les
préconditions, les postconditions et les invariants. Ces contraintes sont
généralement localisées dans le code source autour des données. Dans le cas de
Praspel, les invariants sont localisés sur les attributs des classes et les
préconditions avec les postconditions sont localisées sur les paramètres des
méthodes. La sémantique d'un contrat est la suivante~:

\begin{itemize}

\item l'appelant de la méthode s'engage à satisfaire la précondition~;

\item seulement dans ce cas, la méthode appelée s'engage à satisfaire sa
postcondition~;

\item les invariants, quant à eux, doivent être satisfait avant et après
l'exécution de la méthode.

\end{itemize}

Le reste de la section décrit les parties de la grammaire de Praspel, en forme
normale.

\subsection{Clauses}
\label{subsection:language:clauses}

\def\gsep{$\quad::=\quad$}
\def\mvert{$\;\;|\;\;$}
\newcommand{\token}[1]{\underline{\code{php-#1}}}
\newcommand{\grule}[1]{\textit{#1}}

\begin{figure}
\centering
\begin{tabular}{rcl}
\grule{specification} & \gsep &
    \grule{attribute-clauses} \mvert \grule{method-clauses} \\

\grule{attribute-clauses} & \gsep &
    \grule{invariant-clause}$^?$ \\

\grule{method-clauses} & \gsep &
    $($ \grule{description-clause} \code{;} $)^?$ \\ & &
    \grule{rbet-clauses} \\

\grule{rbet-clauses} & \gsep &
    $($ \grule{requires-clause} \code{;} $)^?$ \\ & &
    $($ \\ & &
    $\quad\,\,($ \grule{behavior-clause}$^+$ \grule{default-clause}$^?$ $)^?$ \\ & &
    \mvert $($ \grule{ensures-clause} \code{;} $)^?$
    $($ \grule{throwable-clause} \code{;} $)^?$ \\ & &
    $)$ \\
\end{tabular}

\caption{\label{figure:language:grammar_part0} Grammaire de Praspel en forme
normale~: les règles de plus haut niveau.}

\end{figure}

\begin{figure}
\centering
\begin{tabular}{rcl}
\grule{invariant-clause} & \gsep &
    \ainvariant \grule{expression} \\

\grule{requires-clause} & \gsep &
    \arequires \grule{expression} \\

\grule{behavior-clause} & \gsep &
    \abehavior \token{identifier} \code{\{} \\ & &
    \quad $($ \grule{description-clause} \code{;} $)^?$ \\ & &
    \quad \grule{rbet-clauses} \\ & &
    \code{\}} \\

\grule{default-clause} & \gsep &
    \adefault \code{\{} \\ & &
    \quad $($ \grule{description-clause} \code{;} $)^?$ \\ & &
    \quad $($ \grule{ensures-clause} \code{;} $)^?$ \\ & &
    \quad $($ \grule{throwable-clause} \code{;} $)^?$ \\ & &
    \code{\}} \\

\grule{ensures-clause} & \gsep &
    \aensures \grule{expression} \\

\grule{throwable-clause} & \gsep &
    \athrowable \grule{exceptional-expression} \\

\grule{description-clause} & \gsep &
    \adescription \token{string} \\
\end{tabular}

\caption{\label{figure:language:grammar_part1} Grammaire de Praspel en forme
normale~: les règles des clauses.}

\end{figure}

Un contrat est composé de clauses, dont la syntaxe est décrite dans les
Figures~\ref{figure:language:grammar_part0},
\ref{figure:language:grammar_part1}, \ref{figure:language:grammar_part2} et
\ref{figure:language:grammar_part3}. Dans ces figures, le style \code{token}
représente un lexème de la grammaire de Praspel, \grule{rule} représente une
entité syntaxique (règle) de la grammaire, et \token{token} représente un lexème
de la grammaire de PHP (pour être le plus proche possible du langage manipulé
par le développeur). La notation $e^r_s$ signifie que le motif $e$ est répété
$r$ fois, et séparé par $s$. $r$ peut être $?$, $+$ ou $*$, respectivement pour
0 ou 1 fois, 1 ou plusieurs fois et 0 ou plusieurs fois. Si $s$ n'est pas vide,
alors $s$ doit être un lexème.

\begin{figure}
\begin{pre}
class \(C\) \{ \\
\\
    /** \\
     * @invariant \(I\); \\
     */ \\
    protected \$a; \\
\\
    /** \\
     * @requires \(R\); \\
     * @behavior \(\alpha\) \{ \\
     *     @requires  \(R\sb{\alpha}\); \\
     *     @behavior \(\beta\) \{ \\
     *         @requires  \(R\sb{\alpha.\beta}\); \\
     *         @ensures   \(E\sb{\alpha.\beta}\); \\
     *         @throwable \(T\sb{\alpha.\beta}\); \\
     *     \} \\
     * \} \\
     * @behavior \(\gamma\) \{ \\
     *     @requires  \(R\sb{\gamma}\); \\
     *     \(\dots\) \\
     * \} \\
     * @default \{ \\
     *     @ensures   \(E\sb{\m{D}}\); \\
     *     @throwable \(T\sb{\m{D}}\); \\
     * \} \\
     */ \\
    public function f ( ) \{ \} \\
\}
\end{pre}

\caption{\label{figure:language:typical_contract} Un contrat Praspel avec toutes
les clauses.}

\end{figure}

