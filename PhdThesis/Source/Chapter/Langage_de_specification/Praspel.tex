\def\gsep{$\quad::=\quad$}
\def\mvert{$\;\;|\;\;$}
\def\mvertp{\phantom{$\;\;|\;\;$}}
\newcommand{\gtoken}[1]{\underline{\code{php-#1}}}
\newcommand{\grule}[1]{\textit{#1}}

\section{Praspel}
\label{section:language:praspel}

Praspel est un acronyme anglais signifiant \inenglish{PHP Realistic Annotation
and SPEcification Language}. C'est un langage et un \inenglish{framework} de
test à partir de contrats en PHP. Ce langage repose sur les domaines réalistes.

Praspel est un {\strong langage d'annotation} car il s'écrit dans les
commentaires du code source du programme. PHP comprend trois catégories de
commentaires~: en ligne (avec \code{//} ou \code{\#}), multi-lignes (entre
\code{/*} et \code{*/}) et en bloc (entre \code{/**} et \code{*/}). Cette
dernière catégorie est culturellement dédiée à l'écriture des annotations,
documentations API etc., et est par conséquent celle où nous écrirons Praspel.

Praspel est un \inenglish{Behavioral Interface Specification Language} ({\strong
BISL}) basé sur les contrats. Un {\strong contrat} est un modèle du comportement
du code, décrit à travers des contraintes formelles, alors appelées {\strong
clauses}, comme les préconditions, les postconditions et les invariants. Ces
contraintes sont généralement localisées dans le code source autour des données.
Dans le cas de Praspel, les invariants sont localisés sur les attributs des
classes et les préconditions avec les postconditions sont localisées sur les
paramètres des méthodes. La sémantique d'un contrat est la suivante~:

\begin{itemize}

\item l'appelant de la méthode s'engage à satisfaire la précondition~;

\item seulement dans ce cas, la méthode appelée s'engage à satisfaire sa
postcondition~;

\item les invariants, quant à eux, doivent être satisfait avant et après
l'exécution de la méthode.

\end{itemize}

Nous décrivons maintenant les parties de la grammaire de Praspel en forme
normale. La grammaire complète se trouve dans
l'annexe~\ref{appendices:grammar_of_praspel}
page~\pageref{appendices:grammar_of_praspel}.

\subsection{Clauses}
\label{subsection:language:clauses}

\begin{figure}
\centering
\begin{tabular}{rcl}
\grule{specification} & \gsep &
    \grule{attribute-clauses} \mvert \grule{method-clauses} \\

\grule{attribute-clauses} & \gsep &
    \grule{invariant-clause}$^?$ \\

\grule{method-clauses} & \gsep &
    $($ \grule{is-clause} \code{;} $)^?$ \\ & &
    $($ \grule{description-clause} \code{;} $)^?$ \\ & &
    \grule{rbdet-clauses} \\

\grule{rbdet-clauses} & \gsep &
    $($ \grule{requires-clause} \code{;} $)^?$ \\ & &
    $($ \\ & &
    $\quad\,\,($ \grule{behavior-clause}$^+$ \grule{default-clause}$^?$ $)^?$ \\ & &
    \mvert $($ \grule{ensures-clause} \code{;} $)^?$
    $($ \grule{throwable-clause} \code{;} $)^?$ \\ & &
    $)$ \\
\end{tabular}

\caption{\label{figure:language:grammar_part0} Grammaire de Praspel~: les
entités syntaxiques de plus haut niveau.}

\end{figure}

Un contrat est composé de clauses, dont la syntaxe est décrite dans les
figures~\ref{figure:language:grammar_part0},
\ref{figure:language:grammar_part1}. Dans ces figures, le style \code{token}
représente un lexème de la grammaire de Praspel, \grule{rule} représente une
entité syntaxique (règle) de la grammaire, et \gtoken{token} représente un
lexème de la grammaire de PHP (pour être le plus proche possible du langage
manipulé par le développeur). La notation $e^r_s$ signifie que le motif $e$ est
répété $r$ fois, et séparé par $s$. $r$ peut être $?$, $+$ ou $*$,
respectivement pour 0 ou 1 fois, 1 ou plusieurs fois et 0 ou plusieurs fois. Si
$s$ n'est pas vide, alors $s$ doit être un lexème.

\begin{figure}
\begin{bigpre}
class C \{ \\
\\
    /** \\
     * @invariant \(I\); \\
     */ \\
    protected \$a; \\
\\
    /** \\
     * @is \(S\); \\
     * @requires \(P\); \\
     * @behavior \(\alpha\) \{ \\
     *     @description 'Apply the \(\alpha\) process.'; \\
     *     @requires \(P\sb{\alpha}\); \\
     *     @behavior \(\beta\) \{ \\
     *         @requires  \(P\sb{\alpha.\beta}\); \\
     *         @ensures   \(Q\sb{\alpha.\beta}\); \\
     *         @throwable \(T\sb{\alpha.\beta}\); \\
     *     \} \\
     * \} \\
     * @behavior \(\gamma\) \{ \\
     *     @requires \(P\sb{\gamma}\); \\
     *     \(\dots\) \\
     * \} \\
     * @default \{ \\
     *     @description 'Apply the fallback process.'; \\
     *     @ensures   \(Q\sb{\m{D}}\); \\
     *     @throwable \(T\sb{\m{D}}\); \\
     * \} \\
     */ \\
    public function f ( ) \{ \} \\
\}
\end{bigpre}

\caption{\label{figure:language:typical_contract} Un contrat Praspel typique
avec toutes les clauses.}

\end{figure}

La figure~\ref{figure:language:grammar_part0} montre les familles de clauses,
dont le contenu est décrit par la figure~\ref{figure:language:grammar_part1}.
Ainsi, nous apprenons que les entités syntaxiques \grule{attribute-clauses} et
\grule{method-clauses} définissent respectivement les clauses annotant les
attributs d'une classe et ses méthodes.  L'entité syntaxique
\grule{method-clauses} montre l'ordre des clauses en forme normale. La grammaire
actuelle de Praspel est bien plus souple~: elle accepte des clauses dans
n'importe quel ordre mais pour faciliter le raisonnement, nous ne présenterons
et ne travaillerons que sur une forme normale. La
figure~\ref{figure:language:typical_contract} illustre la localisation typique
des clauses sur une classe \code{C} avec un attribut \code{a} et une méthode
\code{f}. La clause \ainvariant est localisée juste avant l'attribut \code{a}.
Les autres clauses sont des clauses de méthode.  Elles sont localisées juste
avant l'en-tête des méthodes. La plupart de ces clauses introduisent des
expressions Praspel, représentée par les entités syntaxiques \grule{expression}
ou \grule{exceptional-expression}, détaillées dans la
partie~\ref{subsection:language:expressions}.

\begin{figure}
\centering
\begin{tabular}{rcl}
\grule{invariant-clause} & \gsep &
    \ainvariant \grule{expression} \\

\grule{requires-clause} & \gsep &
    \arequires \grule{expression} \\

\grule{behavior-clause} & \gsep &
    \abehavior \gtoken{identifier} \code{\{} \\ & &
    \quad $($ \grule{description-clause} \code{;} $)^?$ \\ & &
    \quad \grule{rbdet-clauses} \\ & &
    \code{\}} \\

\grule{default-clause} & \gsep &
    \adefault \code{\{} \\ & &
    \quad $($ \grule{description-clause} \code{;} $)^?$ \\ & &
    \quad $($ \grule{ensures-clause} \code{;} $)^?$ \\ & &
    \quad $($ \grule{throwable-clause} \code{;} $)^?$ \\ & &
    \code{\}} \\

\grule{ensures-clause} & \gsep &
    \aensures \grule{expression} \\

\grule{throwable-clause} & \gsep &
    \athrowable \grule{exceptional-expression} \\

\grule{is-clause} & \gsep &
    \ais \code{pure} \\

\grule{description-clause} & \gsep &
    \adescription \gtoken{string} \\
\end{tabular}

\caption{\label{figure:language:grammar_part1} Grammaire de Praspel~: les
entités syntaxiques des clauses.}

\end{figure}

La précondition d'une méthode est exprimée à l'aide de la clause \arequires, la
postcondition à l'aide de la clause \aensures. La précondition exprime des
contraintes sur le {\strong pré-état} du système, alors que la postcondition
exprime des contraintes sur le {\strong post-état} du système. Afin de modifier
l'état du système, il doit être exécuté en appelant une méthode. La clause
\aensures représente une {\strong postcondition normale}, mais le système peut
aussi lever une exception et être alors placé dans un {\strong post-état
exceptionnel}. La clause \athrowable exprime des contraintes sur cet état
particulier~: elle exprime les conditions sous lesquelles des exceptions peuvent
être levées par la méthode et aussi le post-état du système associé (par exemple
une exception a été levée mais le système est suffisamment propre pour tenter
une nouvelle exécution). La syntaxe de $T$ ($T_{\alpha.\beta}$ et $T_\m{D}$ dans
la figure~\ref{figure:language:typical_contract}) est définie par l'entité
syntaxique \grule{exceptional-expression} dans la
figure~\ref{figure:language:grammar_part2}. $T$ est de la forme \code{$T_C$ with
$T_E$} où $T_C$ est une liste de nom de classes asssociée à un identifiant et
$T_E$ est une expression, appelée {\strong postcondition exceptionnelle}, dont
la syntaxe est détaillée dans la partie~\ref{subsection:language:expressions}.
Tous les identifiants définis dans $T_C$ peuvent apparaître dans $T_E$. La
sémantique est la suivante~: si l'exception levée est une instance d'une classe
représentant une exception listée dans $T_C$, alors elle est assignée à
l'identifiant associée et la postcondition exceptionnelle $T_E$ doit être
satisfaite.

Une méthode peut avoir différents {\strong comportements} relatifs à ses
arguments et l'état du système. Praspel propose la clause \abehavior pour
représenter un comportement, identifié par un nom, et peut potentiellement
contenir une description informelle, grâce à la clause \adescription~; ces deux
informations peuvent s'avérer très utiles pour offrir un retour à l'utilisateur.
Les clauses \arequires, \aensures, \athrowable et \abehavior elle-même peuvent
apparaître à l'intérieur d'une clause \abehavior. Cette structure décrit des
comportements {\strong imbriqués}, comme illustré dans la
figure~\ref{figure:language:typical_contract} dans laquelle le comportement
$\beta$ est imbriqué dans le comportement $\alpha$. À côté de ça, nous pouvons
également décrire des comportements {\strong alternatifs} en juxtaposant des
clauses \abehavior. Ceci est toujours illustré dans la
figure~\ref{figure:language:typical_contract} avec le comportement $\gamma$,
frère du comportement $\alpha$. Les comportements sont, dans la pratique,
{\strong mutuellement exclusifs}. Cependant, la syntaxe autorise l'écriture de
comportements non-déterministes.

La clause \adefault est strictement équivalente à une clause \abehavior avec une
clause \arequires implicite décrivant la conjonction de toutes les négations des
clauses \arequires des comportements frères précédents.  Par exemple, dans la
figure~\ref{figure:language:typical_contract}, la clause \arequires de \adefault
pourrait s'écrire $\neg P_\alpha \land \neg P_\gamma$.

Afin d'«~activer~» un comportement donné, sa clause \arequires doit être
satisfaite. Après l'exécution, le système est dans le post-état. Nous faisons
face à deux situations. Si c'est un post-état normal, alors la clause \aensures
du dernier comportement activé doit être satisfaite. Si c'est un post-état
exceptionel, alors c'est la clause \athrowable qui doit être satisfaite, \ie les
deux $T_C$ et $T_E$ doivent être satisfaits, comme décrit précédemment.

Enfin, la clause \ais permet de qualifier la catégorie de la méthode. Le seul
qualificatif actuel est \code{pure}, pour contrôler la mutabilité de
l'environnement. En effet, une méthode peut modifier ou pas son environnement
lors de son exécution. Si elle ne le modifie pas, alors nous parlons d'une
méthode {\strong pure}, sinon nous parlons d'une méthode {\strong impure} (par
défaut). Seules les méthodes pures peuvent être utilisées dans les contrats. \\

\begin{figure}
\begin{bigpre}
class Filesystem \{ \\
\\
    const SIZE = 1024; \\
\\
    /** \\
     * @invariant _usage: boundinteger(0, static::SIZE); \\
     */ \\
    protected \$_usage = 0; \\
\\
    /** \\
     * @invariant _map: array([to class('File')], 0..0xffff); \\
     */ \\
    protected \$_map = array(); \\
\\
    /** \\
     * @is pure; \\
     * @ensures \bslash{}result: this->_usage; \\
     */ \\
    public function getUsage ( ) \{ /* … */ \} \\
\\
    /** \\
     * @requires file: class('File') and \\
     *           index: 0..0xffff or void; \\
     * @behavior full \{ \\
     *     @description 'The filesystem is full.'; \\
     *     @requires  \bslash{}pred('\$this->getUsage() + \$file->getSize() \\
     *                          > static::SIZE'); \\
     *     @throwable AllocationException e with \\
     *                    file->isAttached(): false and \\
     *                    e->getFilesystem(): this; \\
     * \} \\
     * @default \{ \\
     *     @ensures file->isAttached(): true and \\
     *              \bslash{}result: boolean(); \\
     * \} \\
     */ \\
    public function store ( File \$file, \$index = null ) \{ /* … */ \} \\
\}
\end{bigpre}

\caption{\label{figure:language:short_contract} Exemple d'une classe
\code{Filesystem} annotée par des contrats.}

\end{figure}

\begin{example}[Contrat d'un système de fichier] La
figure~\ref{figure:language:short_contract} montre un exemple concret d'une
classe \code{Filesystem} annotée, avec deux méthodes \code{getUsage} et
\code{store}. Cet exemple sera utilisé pour illustrer chaque notion introduite
dans la suite de cette partie.

La méthode \code{getUsage} est déclarée comme \code{pure}, ce qui veut dire
qu'elle ne modifie pas son environnement. Elle n'a pas de clause \arequires,
donc elle refusera tous les arguments. Elle a une postcondition que nous
décrirons dans les parties suivantes. La méthode \code{store} a deux
comportements~: un comportement \code{full} quand le système de fichier est
plein, et un comportement par défaut. Le premier comportement décrit une
précondition et une postcondition uniquement exceptionnelle. En l'absence de
postcondition normale, toute valeur retournée sera invalide. Ici, seule
l'exception \code{AllocationException} peut être levée. Elle sera associée à
l'identifiant \code{e}, puis le reste de la clause devra être valide. Dans le
second comportement (par défaut), aucune exception ne peut être levée car la
clause \athrowable est absente. Nous pouvons voir également que les attributs
\code{\_usage} et \code{\_map} sont annotés d'invariants avec la clause
\ainvariant.

\end{example}

\subsection{Expressions}
\label{subsection:language:expressions}

Les figures~\ref{figure:language:grammar_part2} et
\ref{figure:language:grammar_part3} décrivent les expressions Praspel. Nous
avons principalement trois sortes d'expressions~: déclarations, prédicats et
contraintes, respectivement présentées dans les parties suivantes.

\begin{figure}
\begin{center}
\begin{tabular}{rcl}
\grule{expression} & \gsep &
  (\grule{declaration}$^+_\code{and}$ $\code{and})^?$ \\ & &
  (\grule{constraint}$^+_\code{and}$ $\code{and})^?$  \\ & &
   \grule{predicate}$^?_\code{and}$ \\

\grule{exceptional-expression} & \gsep &
    $($ $($ \grule{exception-identifier} $)^+_\code{or}$ \\ & &
    \code{with} \grule{expression} $)^+_{\code{or}}$ \\

\grule{exception-identifier} & \gsep &
    \gtoken{classname} \gtoken{identifier} \\
\end{tabular}
\end{center}

\caption{\label{figure:language:grammar_part2} Grammaire de Praspel~: les
entités syntaxiques d'expressions.}

\end{figure}

\begin{figure}
\begin{center}
\begin{tabular}{rcl}
\grule{declaration} & \gsep &
    \code{let}$^?$ \grule{extended-identifier} \code{:} \grule{disjunction} \\

\grule{constraint} & \gsep &
    \grule{qualification} \mvert \grule{contains} \\

\grule{qualification} & \gsep &
    \grule{identifier} \code{is} \gtoken{identifier}$^+_\code{,}$ \\

\grule{contains} & \gsep &
    \grule{extended-identifier} \code{contains} \grule{constant}$^+_\code{or}$ \\

\grule{predicate} & \gsep &
    \code{\bslash{}pred(} \gtoken{string} \code{)} \\

\grule{disjunction} & \gsep &
    $($
    \grule{constant} \mvert \grule{realdom} \mvert \grule{extended-identifier}
    $)^+_\code{or}$ \\

\grule{realdom} & \gsep &
    \gtoken{identifier} \code{(} \grule{argument}$^?_\code{,}$ \code{)} \\

\grule{argument} & \gsep &
    \code{default} \mvert \grule{realdom} \mvert \grule{constant} \mvert
    \grule{array} \\ & &
    \mvert \grule{extended-identifier} \\

\grule{constant} & \gsep &
    \grule{scalar} \mvert \grule{array} \\

\grule{scalar} & \gsep &
    \code{null} \mvert \gtoken{boolean} \mvert \grule{number} \\ & &
    \mvert \gtoken{string} \mvert \grule{range} \\

\grule{number} & \gsep &
    \gtoken{binary} \mvert \gtoken{octal} \mvert \gtoken{hexa} \\ & &
    \mvert \gtoken{decimal} \\

\grule{range} & \gsep &
    \grule{number} \code{..} \grule{number} \\

\grule{array} & \gsep &
    \code{[} \grule{pair}$^?_\code{,}$ \code{]} \\

\grule{pair} & \gsep &
    \code{from}$^?$ \grule{disjunction} \code{to} \grule{disjunction} \\ & &
    \mvert \code{to}$^?$ \grule{disjunction} \\

\grule{extended-identifier} & \gsep &
    \grule{array-access} \\

\grule{array-access} & \gsep &
    \grule{identifier} $($ \code{[} \grule{scalar} \code{]} $)^?$ \\

\grule{identifier} & \gsep &
    \gtoken{identifier} \\ & &
    \mvert \code{this} $($ \code{->} \gtoken{identifier} $)^*$ \\ & &
    \mvert $($ \code{self} \mvert \code{static} \mvert \code{parent} $)$ \\ & &
    \mvertp $($ \code{::} \gtoken{identifier} $)^+$ \\ & &
    \mvert \code{\bslash{}old(} \grule{extended-identifier} \code{)} \\ & &
    \mvert \code{\bslash{}result} \\
\end{tabular}
\end{center}

\caption{\label{figure:language:grammar_part3} Grammaire de Praspel~: entités
syntaxiques de construction des expressions.}

\end{figure}

\subsubsection{Déclarations}

Une {\em déclaration} assigne un ou plusieurs domaines réalistes à une variable
à travers l'opérateur \code{:}. Une variable est soit un attribut de classe,
spécifié dans une clause \ainvariant, soit un paramètre de la méthode, spécifié
dans les autres clauses. La valeur d'une variable peut appartenir à plusieurs
domaines réalistes si la variable est définie par une {\strong disjonction} de
domaines réalistes, représentée par le mot-clé \code{or}. Dans l'exemple de la
figure~\ref{figure:language:short_contract}, la variable \code{index} du contrat
de la méthode \code{store} peut avoir deux valeurs~: soit un entier entre 0 et
65535, soit \code{null}. Une disjonction de domaines réalistes peut contenir
n'importe quelle sorte de domaine réaliste, similairement à l'aspect dynamique
de PHP.

Toutes les variables utilisées dans le contrat appartiennent au système,
c'est~à~dire qu'elles existent en tant qu'argument d'une méthode ou attribut de
classe. Toutefois, il est possible de déclarer une variable qui appartient au
modèle si elle est précédée par le mot-clé \code{let}. Nous pouvons voir cette
variable comme étant {\strong locale} au modèle. Cette catégorie de variables
est très utile pour manipuler des représentations intermédiaires de données,
comme une longueur qui serait utilisée par plusieurs tableaux par exemple.

La variable spéciale \aresult représente la valeur {\strong retournée} par la
méthode. Elle ne peut être déclarée que dans la clause \aensures. Les clauses
\aensures et \athrowable peuvent également faire {\strong référence} à la valeur
d'une variable dans le pré-état grâce à la construction \aold{i}, où $i$ est le
nom d'une variable.

\subsubsection{Prédicats}

Toutes les constructions présentes dans Praspel supportent deux aspects~: la
validation et la génération, à travers les caractéristiques de prédicabilité et
de générabilité des domaines réalistes. Chaque concept introduit dans le langage
doit supporter ces deux aspects. Mais il arrive qu'il soit impossible d'exprimer
certaines contraintes avec les constructions actuelles. C'est pourquoi nous
avons la construction «~boîte-noire~» \apred{p}, où $p$ est du code PHP. Cette
construction permet à l'utilisateur d'exprimer des contraintes arbitraires en
utilisant PHP lui-même au lieu de Praspel. Le code $p$ doit être un prédicat en
forme normal {\strong conjonctive} (CNF). La construction \apred{p} ne supporte
que la validation et non pas la génération. Le prédicat
\apred{\code{'\$this->getUsage() + \$file->getSize()
> static::SIZE'}} dans la figure~\ref{figure:language:short_contract} signifie
que s'il n'y a plus de place pour ajouter un nouveau fichier, alors cette partie
de la précondition est satisfaite.

Parce que la construction \apred{p} est une boîte-noire, elle introduit du rejet
lors de la génération de données car les contraintes $p$ ne sont pas
considérées. En effet, quand nous générons des données, nous les validons auprès
des prédicats. Si un prédicat est trop fort, il invalide la donnée, et nous en
générons de nouvelles. Une telle situation peut conduire à une boucle infinie.
Heureusement, il existe un nombre maximum d'essais (qui est paramétrable). Quand
ce nombre est atteint, le générateur abandonne et émet une erreur qui sera
journalisée.

Une tâche récurrente est d'observer et analyser ce que les utilisateurs écrivent
le plus dans cette construction, et l'extraire dans Praspel afin de supporter
les deux aspects~: validation et génération. L'objectif est de faire évoluer le
langage en supportant plus de constructions usuelles nativement au lieu de les
avoir dans une boîte-noire, et ainsi, entres autres, réduire le rejet lors de la
génération de données. Ce processus est illustré dans la
partie~\ref{section:data:arrays} page~\pageref{section:data:arrays}.

\subsubsection{Contraintes}

Praspel a principalement deux sortes de contraintes~: soit en utilisant la
syntaxe des déclarations, soit en utilisant le mot-clé \code{is} (différent de
la clause \ais). Des exemples de contraintes sont présentées dans la
partie~\ref{section:data:arrays} page~\pageref{section:data:arrays}.

\subsubsection{Identifiants}

Dans la figure~\ref{figure:language:short_contract} ainsi que dans la règle
\grule{identifier} de la grammaire de la
figure~\ref{figure:language:grammar_part3}, pour manipuler un identifiant, nous
voyons les mots-clés spéciaux \code{this}, \code{self}, \code{static} et
\code{parent}, avec deux opérateurs différents \code{->} et \code{::}. Tout
d'abord, pour un accès sur un objet (accès dynamique), nous utilisons
l'opérateur \code{->}, sinon, pour un accès sur une classe (accès statique),
nous utilisons l'opérateur \code{::}. Nous faisons référence à l'objet courant
avec le mot-clé \code{this}, et nous faisons référence à la classe courante avec
les mots-clés \code{self} et \code{static}. Enfin, nous faisons référence à la
classe parente avec le mot-clé \code{parent}.

La différence entre \code{self} et \code{static} est que \code{self} fera
toujours référence à la classe qui déclare, alors que \code{static} fera
référence à la classe qui utilise. Le mot-clé \code{static} est un équivalent à
un \code{this} statique. Ce mécanisme est appelé le \inenglish{Late Static
Binding}\footnote{Voir \url{http://php.net/lsb}.}.

\begin{example}[Différence entre \code{self} et \code{static}]

Soit une classe \code{Big\-File\-system} qui hérite de \code{Filesystem} et
redéfinit uniquement la constante \code{SIZE}~:
%
\begin{pre}
class BigFilesystem extends Filesystem \{ \
\\
\\
    const SIZE = 1048576; // \(2\sp{20}\) \\
\}
\end{pre}
%
Et soit la méthode \code{getUsage} de \code{Filesystem} qui utilise la constante
avec \code{self::SIZE}. Alors, lorsque nous appelons la méthode \code{getUsage}
depuis \code{File\-system}, ce sera la constante \code{SIZE} de
\code{Filesystem} qui sera utilisée. La même chose se produira lorsque nous
appelons la méthode \code{getUsage} depuis \code{Big\-File\-system} malgré le
fait que \code{Big\-File\-system} ait sa propre constante \code{SIZE}, car
\code{self} fait référence à la classe qui déclare. Dans ce cas, ce n'est pas le
comportement attendu. Si maintenant \code{getUsage} utilise la constante avec
\code{static::SIZE}, alors lorsque nous appelons la méthode \code{getUsage}
depuis \code{Filesystem}, ce sera la constante \code{SIZE} de \code{Filesystem}
qui sera utilisée. Et lorsque nous appelons la méthode \code{getUsage} depuis
\code{BigFilesystem}, ce sera la constante \code{SIZE} de \code{BigFilesytem}
qui sera utilisé.

\end{example}

\subsection{Description de tableaux}
\label{subsection:language:array}

La description de tableaux est une partie très importante de Praspel.

Dans PHP, un tableau est toujours un {\strong tableau associatif} (ou une
\inenglish{map}, un {\strong dictionnaire}), \ie une collection de paires
clé-valeur, où chaque clé apparaît au maximum une fois. Les clés peuvent être de
type nul, booléen, entier, réel ou chaîne de caractères. PHP accepte ces types
de clés mais les booléens sont transtypés en entier et les réels sont réduits à
leur partie entière. Les valeurs quant à elles peuvent être de n'importe quel
type. Un tableau peut être {\strong homogène} ou {\strong hétérogène}. Dans un
tableau homogène, toutes les clés ont le même type, ainsi que toutes les
valeurs. Dans un tableau hétérogène, les clés peuvent avoir des types distincts,
tout comme les valeurs.  Les clés peuvent être {\strong auto-incrémentées}, en
ajoutant 1 à la dernière clé entière, à partir de 0. La {\strong longueur} (ou
la {\strong taille}) d'un tableau est son nombre de paires. Un tableau n'a pas
de longueur prédéfinie, mais sa longueur (stockée en interne par le moteur PHP)
peut être connu grâce à la fonction PHP \code{count()}. Un tableau n'a également
pas de profondeur prédéfinie, \ie il peut contenir un nombre arbitraire de
sous-tableaux.

Dans Praspel, \code{array($D$, $L$)} dénote le domaine réaliste des tableaux
dont les domaines et codomaines sont décrits par $D$ et dont la longueur
appartient à la disjonction $L$ de domaines réalistes d'entiers non-négatifs.
$D$ est une liste séparée par une virgule, entre \code{[} et \code{]}, de
{\strong descriptions} de paires de la forme \code{from $K$ to $V$}, où $K$ et
$V$ sont des disjonctions de domaines réalistes, respectivement pour les clés et
les valeurs. Quand le mot-clé \code{from} est manquant, nous introduisons le
domaine réaliste représentant un entier auto-incrémenté démarrant de 0 avec un
pas de 1.

\begin{example}[Tableaux homogènes et hétérogènes]

La syntaxe des descriptions de tableaux est illustrée avec les déclarations de
tableaux suivantes~:
%
\begin{pre}
a1: array([to boolean()], 7..42) \\
a2: array([from 0..5 or 10 to integer()], 7) \\
a3: array([from 0..10 to boolean(), \\
           from 20..30 to float()], 7) \\
a4: array([from 0..10 or 20..30 to boolean() or float()], 7)
\end{pre}
%
L'identifiant \code{a1} est déclaré comme un tableau homogène de booléens avec
une longueur comprise entre 7 et 42. L'identifiant \code{a2} est déclaré comme
un tableau homogène de longueur 7, dont les clés sont des entiers entre 0 et 5
ou simplement 10, et dont les valeurs sont des entiers. Les identifiants
\code{a3} et \code{a4} sont déclarés comme des tableaux hétérogènes. Les deux
tableaux peuvent contenir les paires $(\code{5}, \code{true})$ et $(\code{25},
\code{4.2})$, mais \code{a4} peut contenir la paire $(\code{5}, \code{4.2})$,
alors que \code{a3} ne peut pas la contenir.

\end{example}

Dans la figure~\ref{figure:language:short_contract}, l'attribut \code{\_map} est
spécifié comme un tableau d'objets \code{File} d'une longueur de 0 et 65535,
avec~:
%
\begin{pre}
\ainvariant _map: array([to class('File')], 0..0xffff);
\end{pre}

Nous introduisons une {\strong forme normale} pour supprimer les disjonctions
dans les descriptions des paires, en appliquant itérativement la règle de
réécriture suivante~:
%
\begin{center}
\begin{tabular}{c}
\code{from $F_1$ or $F_2$ to $T_1$ or $T_2$} \\
\hline
\code{from $F_1$ to $T_1$} \\
\code{from $F_1$ to $T_2$} \\
\code{from $F_2$ to $T_1$} \\
\code{from $F_2$ to $T_2$}
\end{tabular}
\end{center}
%
Une description de tableau est en forme normale quand elle ne peut plus être
réduite par cette règle.

\begin{example}[Description de tableau en forme normale]

La déclaration suivante de \code{a4} est en forme normale~:

\begin{pre}
a4: array([from  0..10 to boolean(), \\
           from  0..10 to float(), \\
           from 20..30 to boolean(), \\
           from 20..30 to float()], 7)
\end{pre}

\end{example}
