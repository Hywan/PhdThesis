\section{Domaines réalistes}
\label{section:language:realdoms}

Un des moyens les plus répandus pour caractériser les données manipulées par un
programme est d'utiliser un système de {\em typage}. Il en existe grossièrement
deux grandes familles~: {\em statique} et {\em dynamique}, que nous pouvons
comprendre comme types déclarés syntaxiquement ou déduits lors de la compilation
ou de l'exécution. Dans tous les cas, le type d'une donnée est connu au moment
de l'{\em exécution} du programme. Le typage permet de {\em vérifier} la
cohérence entre les données manipulées. Ainsi, la plupart des systèmes de typage
ne permettent pas l'addition d'une chaîne de caractères avec un entier, sauf si
ce système est dit {\em faible}. Prenons l'exemple de PHP qui a un système de
typage dynamique (déduit à l'exécution) et faible~: l'addition de la chaîne de
caractères \code{'1.2'} avec l'entier \code{4} produit le réel \code{5.2}. Dans
ce cas, le langage opère un transtypage, c'est~à~dire qu'à partir de règles
internes il transforme les données vers un autre type.

Notons que dans un langage comme PHP où les types ne sont pas déclarés
syntaxiquement, il est culturellement admis d'écrire une documentation API qui
décrit la nature des données manipulées par le programme, comme ses types.

Nous profitons de la présence quasi permente d'un système de typage dans tous
les langages pour introduire la notion de {\em domaine réaliste}. Naïvement,
nous pouvons dire qu'ils raffinent les types usuels comme les entiers, les
chaînes de caractères, les tableaux etc., ainsi que des types plus complexes. Le
mot {\em réaliste} signifie qu'ils sont conçus pour spécifier des domaines de
données pertinentes pour un contexte spécifique. Par exemple, une adresse email
peut être un domaine réaliste~: plusieurs logiciels identifient leurs
utilisateurs avec leur adresse email. Ceci est plus qu'une chaîne de
caractères~; en effet, ça doit respecter certaines règles, notamment
syntaxiques.

Les domaines réalistes ont été conçus afin de répondre aux problématiques de
test. C'est pourquoi ils ont les deux propriétés suivantes~:

\begin{itemize}

\item la {\em prédicabilité} permet de vérifier qu'une donnée appartient à
l'ensemble des valeurs décrites par le domaine réaliste~;

\item la {\em générabilité} permet de générer une valeur décrite par le domaine
réaliste.

\end{itemize}

Ces deux propriétés permettent respectivement de {\em valider} et de {\em
vérifier} un programme.

\subsection{Implémentation}
\label{subsection:language:realdom:implementation}

Nous proposons une implémentation des domaines réalistes en PHP. Dans cette
implémentation, un domaine réaliste est représenté par une {\em classe}, et ses
propriétés de prédicabilité et de générabilité par les {\em méthodes}
respectives suivantes~:

\begin{itemize}

\item \code{predicate(\$q)}, où \code{\$q} est la valeur à valider~;

\item \code{sample(\$sampler)}, où \code{\$sampler} est un {\em générateur
numérique}~: il génère uniquement des entiers et des réels, et permet de guider
la génération d'une donnée~; par exemple nous allons utiliser ce générateur
numérique pour choisir un caractère parmi une plage de caractères donnés.  Pour
l'instant, le seul générateur numérique disponible est pseudo-aléatoire.

\end{itemize}

\subsection{Hiérarchie}
\label{subsection:language:realdom:hierarchy}

Parce que les domaines réalistes sont implémentés à travers des classes et que
PHP est orienté objet, nous pouvons faire de l'{\em héritage}. Par conséquent,
un domaine réaliste enfant peut hériter et {\em rafiner} les propriétés de son
parent. Par exemple, le domaine réaliste \code{Color} hérite de \code{String},
et \code{Boundinteger} hérite de \code{Integer}. Tous les domaines réalistes
doivent être enfant de la classe \code{Realdom}.

La Figure~\ref{figure:language:realdom_boundinteger} montre une implémentation
légère du domaine réaliste \code{Boundinteger}. Il hérite du domaine réaliste
\code{Integer} en \circled{1}. La propriété de prédicabilité rafine celle du
parent en \circled{2}, et ajoute de nouvelles contraintes en \circled{3} et en
\circled{4} qui vérifient que \code{\$q} appartient à l'intervalle décrit par le
domaine réaliste.

\begin{figure}[t]
\begin{pre}
class Boundinteger extends Integer /* \(\circled{1} \)*/ \{ \\
\\
    protected \$_arguments = array( \\
        'Constinteger lower' => PHP_INT_MIN, /* \(\circled{5}\) */ \\
        'Constinteger upper' => PHP_INT_MAX  /* \(\circled{6}\) */ \\
    ); \\
\\
    public function predicate ( \$q ) \{ \\
\\
        return    parent::predicate(\$q)                     /* \(\circled{2}\) */ \\
               && \$q >= \$this['lower']->getConstantValue()  /* \(\circled{3}\) */ \\
               && \$q <= \$this['upper']->getConstantValue(); /* \(\circled{4}\) */ \\
    \} \\
\\
    public function sample ( \$sampler ) \{ \\
\\
        return \$sampler->getInteger( \\
            \$this['lower']->sample(\$sampler), \\
            \$this['upper']->sample(\$sampler) \\
        ); \\
    \} \\
\}
\end{pre}

\caption{\label{figure:language:realdom_boundinteger} Implémentation légère du
domainer réaliste \code{Boundinteger}. Nous pouvons voir l'héritage, les
propriétés de prédicabilité et de générabilité à travers les méthodes
\code{predicate} et \code{sample}, le rafinement et la déclaration des
paramètres.}

\end{figure}

\subsection{Paramèters}
\label{subsection:language:realdom:parameters}

{\em A l'instar} des fonctions, les domaines réalistes sont {\em paramétrables}.
Les données reçues par le domaine réaliste sont appelées ses {\em arguments}.
Les paramètres sont très utiles pour représenter des structures de données
complexes, telles que des tableaux imbriqués, des graphes, des automates, des
arbres etc.

\begin{example}[Paramétrer un domaine réaliste]

Par exemple, le domaine réaliste \code{string(0x61, 0x7a, boundinteger(4, 12))}
représente une chaîne de caractères dont la taille est comprise entre 4 et 12 et
dont les code-points des caractères sont entre \code{0x61} et \code{0x7a}. Nous
pouvons écrire \code{4..12}, qui est un sucre syntaxique de
\code{boundinteger(4, 12)}.

\end{example}

Quand nous décrivons les paramètres d'un domaine réaliste, nous pouvons indiquer
leur type. Le type est exprimé à travers le ou les noms d'un des domaines
réalistes. En effet, un domaine réaliste accepte plusieurs sortes d'arguments et
est capable de faire la traduction d'un type vers un autre lui-même. Ainsi, nous
restons proche de la philosophie de PHP. Par exemple, pour le domaine réaliste
\code{String}, ses deux premiers paramètres peuvent être des entiers ou des
caractères. Alors, nous pouvons écrire \code{string('a', 'z', 4..12)} sans
produire d'erreur, mais \code{string(true, 'z', 4..12)} produira une erreur.

La Figure~\ref{figure:language:realdom_boundinteger} montre la déclaration de
deux paramètres~: \code{lower} en \circled{5} et \code{upper} en \circled{6},
avec respectivement en valeur par défaut \code{PHP\_INT\_MIN} et
\code{PHP\_INT\_MAX}. Si aucune valeur par défaut n'est précisée, alors le
paramètre est {\em obligatoire}, sinon il est {\em optionnel}.

\subsection{Classification}
\label{subsection:language:realdom:classification}

L'héritage est une classification. Une autre classification {\em transversale}
est possible grâce aux {\em interfaces}. Elles aident à caractériser le
«~paysage~» des domaines réalistes. Les interfaces les plus intéressantes sont~:

\begin{itemize}

\item \code{Constant}, pour représenter un domaine réaliste immuable avec
seulement une valeur, comme \code{42} (du domaine réaliste \code{Constinteger}),
\code{true} (du domaine réaliste \code{Constboolean}) etc.~;

\item \code{Interval}, pour représenter un intervalle à travers une borne
inférieure et supérieure, qui peuvent être réduites dynamiquement (à
l'exécution) comme \code{Boundinteger}~;

\item \code{Finite}, pour représenter un domaine réaliste avec un ensemble de
valeurs dénombrable~;

\item \code{Nonconvex}, pour représenter un domaine réaliste avec des trous, \ie
où des valeurs ont été exclues, et donc, ne peuvent plus être générées~;

\item \code{Enumerable}, pour représenter un domaine réaliste dont les valeurs
peuvent être énumérées, comme \code{Integer}.

\end{itemize}

Ainsi, par exemple, un domaine réaliste qui implémente les interfaces
\code{Interval} et \code{Nonconvex} représente un intervalle à trous.
