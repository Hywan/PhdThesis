\chapter{Implémentations et outillage}
\label{chapter:tools}

\minitoc

Ce chapitre est consacré à la description des outils. Ils sont organisés en deux
parties. Tout d'abord dans la partie~\ref{section:tools:praspel}, nous
présentons notre implémentation de Praspel, son fonctionnement, son découpage et
l'outil avec lequel nous l'avons implémenté. Ensuite, dans la
partie~\ref{section:tools:atoum}, nous présentons l'intégration de ces outils
dans atoum, un \inenglish{framework} de tests unitaires, ainsi qu'une extension
qui fait le pont entre Praspel et atoum.

\begin{figure}

\centering

\begin{tikzbox}{boxtoolsdispatching}{xshift=-2.2cm}
\begin{pre}
class C \{ \\
 \\
    /** \\
     * \arequires  …; \tikzref{boxtoolsdispatchingA} \\
     * \aensures   …; \\
     * \athrowable …; \tikzref{boxtoolsdispatchingB} \\
     */ \\
    public function f ( ) \{ \tikzref{boxtoolsdispatchingC} \\
 \\
        // … \\
    \} \tikzref{boxtoolsdispatchingD} \\
\}
\end{pre}
\end{tikzbox}
%
\begin{tikzannotation}
    \draw [mybrace]
        (boxtoolsdispatchingA.north -| boxtoolsdispatching.east)
        -- node (P) [mybracetext] {Praspel}
        (boxtoolsdispatchingB.south -| boxtoolsdispatching.east);
    \draw [mybrace]
        (boxtoolsdispatchingC.north -| boxtoolsdispatching.east)
        -- node (S) [mybracetext] {SUT}
        (boxtoolsdispatchingD.south -| boxtoolsdispatching.east);

    \node (Pc) [right of=P, node distance=1.8cm] {\circled{P}};
    \draw [mywavyarrow] (P) -- (Pc);

    \node (Sc) at (S -| Pc) {\circled{S}};
    \draw [mywavyarrow] (S) -- (Sc);
\end{tikzannotation}

\caption{\label{figure:tools:dispatching} Répartitions des composants d'un
programme.}

\end{figure}

La figure~\ref{figure:tools:dispatching} nous rappelle que, dans une classe, les
commentaires qui annotent les méthodes contiennent des contrats écrits en
Praspel. Les contrats seront donc extraits de ces commentaires pour être envoyés
dans les outils de Praspel. Les méthodes vont constituer le SUT et seront
utilisées lors de l'exécution des tests. Dans les prochaines figures, ces
données sont identifiées par une lettre cerclée afin d'indiquer leur flux dans
les différents outils. Ainsi, la notation Praspel \tikz { \draw
[mywavyarrow] (0, 0) -- (1, 0); }\circled{P} signifie que le composant Praspel
sera représenté par \circled{P} dans les figures suivantes.

\require{Chapter/Implementations_et_outillage/Praspel.tex}
\require{Chapter/Implementations_et_outillage/atoum.tex}
\require{Chapter/Implementations_et_outillage/Synthese.tex}
