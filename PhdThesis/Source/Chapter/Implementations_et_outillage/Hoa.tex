\section{Implémentation dans Hoa}
\label{section:tools:hoa}

Hoa~\acite{Hoa} est un ensemble de bibliothèques PHP. De plus, Hoa est un pont
entre le monde de la recherche et de l'industrie. À l'origine créé par Ivan
Enderlin, l'auteur de ce mémoire, ce projet est maintenant soutenu par une
communauté et une association. L'avantage d'utiliser ce projet est double.

Le premier avantage est que Hoa comporte des bibliothèques pour une multitude de
domaines en plus d'être impliqué dans plusieurs consortiums ou groupes de
décisions concernant plusieurs standards de l'Informatique ou du Web. Par
conséquent, il est plus facile d'écrire, de maintenir et de s'assurer de la
qualité des programmes constituant Praspel.

Le second avantage est que, grâce à ce projet, nous pouvons avoir une meilleure
connaissance des besoins, et, grâce à cette thèse, nous essayons d'y répondre.
L'objectif est de valoriser nos travaux de recherche à travers Hoa (valider
notre démarche) mais aussi de valoriser Hoa à travers cette thèse. En effet, Hoa
est un pont entre le monde de la recherche et de l'industrie. Ainsi le projet
nous offre l'opportunité de faire valider nos expérimentations par sa communauté
ou ses utilisateurs. Mais aussi, Hoa est développé sous la licence libre
\inenglish{New BSD License}\footnote{Voir
\url{http://hoa-project.net/About.html\#License}.}, soit une licence
\inenglish{open-source}. Cela implique que les outils que nous avons développé
durant cette thèse, comme Praspel, le compilateur etc., sont gratuits et libres.
Cette approche offre deux avantages majeurs. Tout d'abord, des contributeurs de
tout horizon peuvent nous aider à corriger ou améliorer nos outils. Ça a été le
cas à plusieurs reprises où des contributeurs ont corrigé des erreurs dans
Praspel ou ont amélioré considérablement le compilateur, notamment en y ajoutant
le support Unicode, en améliorant les performances de l'analyseur lexical (pour
l'analyse de très grandes données) etc. Ensuite, puisque le projet est gratuit,
l'industrie peut l'utiliser et nous offrir des retours pertinents, comme nous le
verrons dans le chapitre~\ref{chapter:experimentations} avec les
expérimentations.

Nos contributions prennent la forme de bibliothèques, à savoir, dans la
partie~\ref{subsection:tools:hoa-realdom}, \code{Hoa\bslash{}Real\-dom} pour les
domaines réalistes, dans la partie~\ref{subsection:tools:hoa-praspel},
\code{Hoa\bslash{}Praspel} pour le langage Praspel, dans la
partie~\ref{subsection:tools:hoa-compiler-regex}, \code{Hoa\bslash{}Compiler}
pour le compilateur et \code{Hoa\bslash{}Re\-gex} pour les expressions
régulières. D'autres contributions plus minimes ne sont pas détaillées ici,
comme l'ajout de fonctionnalités dans la bibliothèque \code{Hoa\bslash{}Math}.

\subsection{\code{Hoa\bslash{}Realdom}}
\label{subsection:tools:hoa-realdom}

La bibliothèque \code{Hoa\bslash{}Realdom}\footnote{Voir
\url{http://central.hoa-project.net/Resource/Library/Realdom}.} représente la
bibliothèque standard des domaines réalistes. Ils sont implémentés comme
présenté dans la partie~\ref{section:language:realdoms}
page~\pageref{section:language:realdoms}, à savoir que chaque domaine réaliste
est représenté par une classe.  Actuellement, une liste de 28~domaines réalistes
est proposée~:
%
\begin{itemize}

\item \code{Array}, représentant des tableaux~;

\item \code{Bag}, un sac pouvant contenir plusieurs domaines réalistes de
natures différentes~;

\item \code{Boolean}, représentant les booléens~;

\item \code{Boundfloat}, représentant un intervalle de réels~;

\item \code{Boundinteger}, représentant un intervalle d'entiers~;

\item \code{Class}, représentant des instances de classes~;

\item \code{Color}, représentant des couleurs au format \code{\#{\em rrggbb}}~;

\item \code{Constarray}, représentant les tableaux de Praspel, c'est~à~dire une
description de tableau~;

\item \code{Constboolean}, représentant les booléens de Praspel~;

\item \code{Constfloat}, représentant les réels de Praspel~;

\item \code{Constinteger}, représentant les entiers de Praspel~;

\item \code{Constnull}, représentant la valeur nulle de Praspel~;

\item \code{Conststring}, représentant les chaînes de caractères de Praspel~;

\item \code{Date}, représentant une date formatée~;

\item \code{Empty}, représentant une donnée vide~;

\item \code{Even}, représentant les nombres pairs~;

\item \code{Float}, représentant les réels~;

\item \code{Grammar}, représentant les chaînes de caractères spécifiées par une
grammaire~;

\item \code{Integer}, représentant les entiers~;

\item \code{Natural}, représentant les naturels (entiers auto-incrémentés)~;

\item \code{Number}, représentant les nombres (entiers ou réels)~;

\item \code{Object}, représentant un objet donné~;

\item \code{Odd}, représentant les nombres impairs~;

\item \code{Regex}, représentant les chaînes de caractères spécifiées par une
expression régulière~;

\item \code{Smallfloat}, représentant des petits réels~;

\item \code{Smallinteger}, représentant des petits entiers~;

\item \code{String}, représentant des chaînes de caractères~;

\item \code{Timestamp}, représentant des valeurs dans le temps~;

\item \code{Undefined}, représentant des valeurs non-définies.

\end{itemize}

Pour obtenir toutes les informations sur un domaine réaliste, nous pouvons nous
aider de l'outil en ligne de commande \code{hoa realdom:reflection}.

\begin{example}[Informations sur le domaine réaliste \code{Boundinteger}]

Par exemple, pour obtenir des informations sur le domaine réaliste
\code{Boundinteger}, en ligne de commande, nous ferons~:
%
\begin{bigpre}
\$ hoa realdom:reflection boundinteger \\
Realdom boundinteger \{ \\
 \\
    Implementation Hoa\bslash{}Realdom\bslash{}Boundinteger; \\
 \\
    Parent Hoa\bslash{}Realdom\bslash{}Integer; \\
 \\
    Interfaces \{ \\
 \\
        ArrayAccess; \\
        Countable; \\
        IteratorAggregate; \\
        Traversable; \\
        Hoa\bslash{}Realdom\bslash{}IRealdom\bslash{}Enumerable; \\
        Hoa\bslash{}Realdom\bslash{}IRealdom\bslash{}Finite; \\
        Hoa\bslash{}Realdom\bslash{}IRealdom\bslash{}Interval; \\
        Hoa\bslash{}Realdom\bslash{}IRealdom\bslash{}Nonconvex; \\
        Hoa\bslash{}Realdom\bslash{}Number; \\
        Hoa\bslash{}Visitor\bslash{}Element; \\
    \} \\
 \\
    Parameters \{ \\
 \\
        [#0 optional] Constinteger lower = -9223372036854775808; \\
        [#1 optional] Constinteger upper = 9223372036854775807; \\
    \} \\
\}
\end{bigpre}
%
Nous trouvons comme informations le nom de la classe qui représente
l'implémentation du domaine réaliste, toutes les interfaces utilisées par cette
implémentation (ici plusieurs de PHP et des domaines réalistes, comme présenté
dans la partie~\ref{subsection:language:realdom:classification}
page~\pageref{subsection:language:realdom:classification}), ainsi que les
paramètres avec leurs positions, filtres et valeurs par défaut.

\end{example}

Cette bibliothèque peut être utilisée seule, sans Praspel.

\subsection{\code{Hoa\bslash{}Praspel}}
\label{subsection:tools:hoa-praspel}

La bibliothèque \code{Hoa\bslash{}Praspel}\footnote{Voir
\url{http://central.hoa-project.net/Resource/Library/Praspel}.} est responsable
de tout le support de Praspel. Le fonctionnement de Praspel est détaillé dans la
partie~\ref{section:tools:praspel}. La bibliothèque est découpée de la façon
suivante~:
%
\begin{itemize}

\item \code{AssertionChecker}, permet l'évaluation du modèle objet~;

\item \code{Bin}, contient des scripts, dont un \inenglish{shell} permettant
d'analyser et évaluer du code Praspel à la volée~;

\item \code{Exception}, contient toutes les catégories d'exceptions de Praspel~;

\item \code{Iterator}, contient entre autre les générateurs de tests unitaires à
partir de plusieurs critères de couverture sur un contrat~;

\item \code{Model}, contient le modèle objet, à savoir la pièce central du
langage Praspel~;

\item \code{Preambler}, permet de créer un préambule de test pour n'importe quel
système sous test~;

\item \code{Visitor}, permet de passer d'une forme à une autre du langage.

\end{itemize}

À la racine de la bibliothèque, nous trouvons entre autre la grammaire au format
PP (que nous retrouvons dans l'annexe~\ref{appendices:grammar_of_praspel}), une
classe nommée \code{Hoa\bslash{}Praspel} qui rassemble les opérations usuelles
sur le langage et une classe représantant une trace d'évaluation d'un contrat.

\subsection{\code{Hoa\bslash{}Compiler} et \code{Hoa\bslash{}Regex}}
\label{subsection:tools:hoa-compiler-regex}

Le compilateur LL($\star$) que nous avons développé dans la
partie~\ref{section:data:strings} page~\pageref{section:data:strings} est
représenté par la bibliothèque
\code{Hoa\bslash{}Compiler\bslash{}Llk}\footnote{Voir
\url{http://central.hoa-project.net/Resource/Library/Compiler}.} (un autre
compilateur LL(1) existait préalablement). Cette bibliothèque comprend aussi le
support du langage PP ainsi que des algorithmes de générations à partir de
grammaires, soient~: aléatoire uniforme, exhaustif borné et par couverture. Ces
algorithmes sont présents dans la sous-bibliothèque
\code{Hoa\bslash{}Com\-pi\-ler\bslash{}Llk\bslash{}Sam\-pler}.  Seuls les deux
derniers fonctionnent comme des itérateurs.

\begin{example}[Génération exhaustive bornée avec Hoa]

Pour générer toutes les données au format JSON dont la taille maximum des
séquences de lexèmes ne dépasse pas 10, nous écrirons~:
%
\begin{pre}
\$grammar = new Hoa\bslash{}File\bslash{}Read('hoa://Library/Json/Grammar.pp'); \\
\$parser  = Hoa\bslash{}Compiler\bslash{}Llk::load(\$grammar); \\
\$token   = new Hoa\bslash{}Regex\bslash{}Visitor\bslash{}Isotropic( \\
    new Hoa\bslash{}Math\bslash{}Sampler\bslash{}Random() \\
); \\
\$length  = 10; \\
\$sampler = new Hoa\bslash{}Compiler\bslash{}Llk\bslash{}Sampler\bslash{}BoundedExhaustive( \\
    \$parser, /* \(\circled{1}\) */ \\
    \$token,  /* \(\circled{2}\) */ \\
    \$length  /* \(\circled{3}\) */ \\
); \\
 \\
foreach(\$sampler as \$data) \\
    // compute \$data
\end{pre}
%
La variable \code{\$grammar} contient un flux en lecture de la grammaire JSON.
La variable \code{\$parser} contient l'analyseur lexical et syntaxique
représentés par la grammaire. La variable \code{\$token} contient le générateur
de valeurs pour les lexèmes. Le seul proposé actuellement est isotropique, comme
présenté dans la partie~\ref{subsection:data:isotropic_generation}
page~\pageref{subsection:data:isotropic_generation}. Enfin, la variable
\code{\$length} représente la taille maximum de la séquence. Les trois dernières
variables sont distribuées sur le générateur exhaustif borné~: les analyseurs en
\circled{1}, le générateur de valurs pour les lexèmes en \circled{2} et la
taille maximum de la séquence en \circled{3}.

\end{example}

La grammaire des expressions régulières se trouve également dans la bibliothèque
\code{Hoa\bslash{}Regex}\footnote{Voir
\url{http://central.hoa-project.net/Resource/Library/Regex}.} dans
\code{hoa://Library/Regex/Grammar.pp}.
