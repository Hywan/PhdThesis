\section{Intégration dans atoum}
\label{section:tools:atoum}

atoum~\acite{atoum}, sans majuscule, est aujourd'hui l'un des
\inenglish{frameworks} de tests unitaires les plus utilisés dans l'industrie.
Gratuit et \inenglish{open-source}, il compte une communauté active et plusieurs
utilisateurs et partenaires industriels importants. atoum est également inclue
dans plusieurs IDE et outils majeurs, comme Netbeans\footnote{Voir
\url{https://netbeans.org/}.}, Eclipse\footnote{Voir
\url{http://eclipse.org/}.}, Vim\footnote{Voir \url{http://vim.org/}.},
Travis-CI\footnote{Voir \url{http://travis-ci.com/}.}, Jenkins\footnote{Voir
\url{http://jenkins-ci.org/}.}, Coveralls\footnote{Voir
\url{https://coveralls.io/}.} etc.

Les suites de tests générées par Praspel doivent être exécutables de plusieurs
façons (en ligne, isolée ou autre), des rapports doivent être produits, les
outils doivent être intégrés à des IDE etc. Tout ceci représente un important
travail. atoum comporte déjà toutes ces fonctionnalités. Nous avons donc choisi
de créer une extension pour introduire Praspel dans atoum. De plus, atoum ne
possède aucun générateur de données, alors que Praspel en propose plusieurs.
Cela a donc été encore plus naturel de rapprocher les deux projets. Cette
extension, appelée \code{atoum/praspel-extension}, est incluse dans la
distribution standard d'atoum «~sous-stéroïde~», c'est~à~dire avec des outils
qui ajoutent des fonctionnalités à atoum.

Dans un premier temps, nous allons présenter atoum, puis dans un second temps,
nous montrerons les fonctionnalités offertes par l'extension.

\subsection{Présentation d'atoum}

\begin{figure}

\fig{10.5cm}{!}{atoum.tex}

\caption{\label{figure:tools:atoum} Fonctionnement schématique d'atoum.}

\end{figure}

La figure~\ref{figure:tools:atoum} illustre le fonctionnement schématique
d'atoum et de tous les \inenglish{framework} de tests unitaires~: ils permettent
d'exécuter un ensemble d'assertions sur un SUT et est capable de produire
différents rapports. Pour utiliser atoum, il faudra alors basiquement procéder
en deux étapes~: écrire des assertions et les exécuter. Dans le vocabulaire
d'atoum, un test est une méthode qui contient plusieurs assertions sur le SUT et
une suite de tests est une classe. Ainsi, nous allons tester que le résultat
retourné par une certaine méthode \code{say}, en \circled{1}, d'une classe
\code{Project\bslash{}HelloWorld} retourne bien la chaîne de caractères
\code{Hello World!}, en \circled{2}~:
%
\begin{pre}
class HelloWorld extends \bslash{}atoum\bslash{}test \{ \\
 \\
    public function testSay ( ) \{ \\
 \\
        \$helloWorld = new \bslash{}Project\bslash{}HelloWorld(); \\
        \$this->string(\$helloWorld->say())      /* \(\circled{1}\) */ \\
                  ->isEqualTo('Hello World!'); /* \(\circled{2}\) */ \\
    \} \\
\}
\end{pre}
%
Puis nous exécutons le test en ligne de commande ({\em modulo} une intégration
dans un IDE)~:
%
\begin{pre}
\$ atoum --file Path/To/Our/Test.php \\
… \\
> Total test duration: 0.00 second. \\
> Total test memory usage: 0.25 Mb. \\
> Code coverage value: 100.00\% \\
> Running duration: 0.08 second. \\
> Success (1 test, 1/1 method, 0 void method, \\
           0 skipped method, 2 assertions)!
\end{pre}
%
Le rapport simple nous informe que le test s'est exécuté en moins de
0.005~seconde, que la mémoire utilisée a été de 0.25~Mb, que la couverture de
code (avec le critère tous-les-arcs) a été de 100\%, que l'ensemble s'est
exécuté en 0.08~seconde et que l'exécution de la seule suite de tests, ne
contenant qu'un seul test représentant deux assertions, a été un succès. \\

Au delà de l'intégration avec plusieurs outils industiels ou la production de
rapports détaillés, nous pouvons citer trois points forts de cet outil.

\subsubsection{Moteurs d'exécution}

Tout d'abord, il permet nativement (comprendre sans \inenglish{plugin}
supplémentaire) l'exécution des tests en ligne, en isolation ou en parallèle.
Une exécution en ligne va exécuter tous les tests les uns après les autres de
manière séquentielle dans le même processus. Une exécution en isolation va
exécuter tous les tests les uns après les autres mais dans un processus
nouvellement crée à chaque fois afin d'isoler le test des exécutions
précédentes. Et enfin, une exécution en parallèle va exécuter tous les tests en
isolation mais de manière concurrente. Cela permet d'accélérer les exécutions ou
d'assurer une fiabilité dans les résultats des tests.

\subsubsection{Collection d'asserteurs}

Le second point fort est sa collection importante d'asserteurs. Bien plus que de
simples \code{assertTrue} ou \code{assertFalse} que nous retrouvons dans la
majorité des autres \inenglish{frameworks} de tests unitaires, atoum propose des
asserteurs personnalisés pour chaque type de données.

\begin{example}[Asserteurs spécifiques avec atoum]

Par exemple, une simple assertion \code{1 - 0.97 === 0.33} sera fausse dans tous
les langages malgré le fait que ce résultat nous paraisse naturel. C'est
pourquoi atoum propose l'asserteur \code{isNearlyEqualTo} sur le groupe
d'asserteurs \code{float}~:
%
\begin{pre}
\$this->float(1 - 0.97)->isNearlyEqualTo(0.33);
\end{pre}
%
De même, nous pouvons trouver l'asserteur \code{containsValues} sur le groupe
\code{array} qui va vérifier que certaines valeurs sont bien présentes dans un
tableau~:
%
\begin{pre}
\$this->array([1, 1, 2, 3, 5, 8])->containsValues([1, 3]);
\end{pre}
%
Il existe encore plusieurs autres groupes d'asserteurs, notamment \code{output}
qui permet de tester les sorties du programme. Notons que ce groupe hérite du
groupe \code{string}~: l'héritage entre groupes est une autre fonctionnalité
intéressante.

\end{example}

\subsubsection{\inenglish{Mock}}

Enfin, le dernier point fort d'atoum est son système de \inenglish{mock}
(bouchon). Pour obtenir le \inenglish{mock} de n'importe quel objet, il suffit
de l'instancier sur l'espace de nom \code{Mock}.

\begin{example}[Bouchonner avec atoum]

Si nous voulons obtenir le \inenglish{mock} de la classe
\code{Project\bslash{}HelloWorld}, nous instancierons simplement la classe
\code{Mock\bslash{}Project\bslash{}HelloWorld}, où nous pourrons ensuite
contrôler le résultat de certaines méthodes~:
%
\begin{pre}
class HelloWorld extends \bslash{}atoum\bslash{}test \{ \\
 \\
    public function testSay ( ) \{ \\
 \\
        \$helloWorld = new \bslash{}Mock\bslash{}Project\bslash{}HelloWorld(); \\
        \$this->calling(\$helloWorld)->say = 'Hello!'; \\
        \$this->string(\$helloWorld->say()) \\
                  ->isEqualTo('Hello!'); \\
    \} \\
\}
\end{pre}
%
Nous avons créé un \inenglish{mock} puis nous avons spécifié que la méthode
\code{say} retournera toujours \code{Hello!}.

\end{example}

Nous pouvons également forcer une méthode à retourner une exception à partir de
son troisième appel par exemple. Nous pouvons également bouchonner des fonctions
et objets standards de PHP. En somme, les fonctionnalités sont avancées et sont
accessibles facilement.

\subsection{Extension~: Praspel dans atoum}
\label{subsection:tools:extension}

Cette extension\footnote{Voir
\url{http://central.hoa-project.net/Resource/Contributions/Atoum/PraspelExtension}.}
introduit Praspel dans atoum. Elle propose deux fonctionnalités~:
%
\begin{itemize}

\item la génération automatique de données de test, qui n'est pas présente dans
atoum~;

\item la compilation de suites de tests générées avec Praspel en suite de tests
écrites avec l'API d'atoum~; les tests seront donc exécutables avec atoum.

\end{itemize}

\subsubsection{Asserteurs de générations de données de tests}

Dans un premier temps, nous donnons accès aux domaines réalistes dans atoum à
travers l'asserteur \code{realdom} qui permet de créer des disjonctions de
domaines réalistes. La syntaxe utilisée dans PHP est la plus proche possible de
la syntaxe utilisée dans Praspel.

\begin{example}[Déclaration de domaines réalistes dans atoum]

Pour déclarer l'intervalle d'entiers $[7; 13] \union [42; 153]$, nous écrirons~:
%
\begin{pre}
\$data = \$this->realdom->boundinteger( 7,  13) \\
                  ->or->boundinteger(42, 153);
\end{pre}
%
L'équivalent en Praspel serait \code{boundinteger(7, 13) or boundinteger(42,
153)} ou plus simplement \code{7..13 or 42..153}.

\end{example}

Ensuite, nous pouvons utiliser au choix l'asserteur \code{sample} ou
\code{sampleMany} pour générer une ou plusieurs données.

\begin{example}[Génération d'entiers avec atoum]

Pour générer respectivement 1 entier ou 7 entiers représentés par l'intervalle
de l'exemple précédent, nous devons écrire~:
%
\begin{pre}
\$integer  = \$this->sample(\$data); \\
\$integers = \$this->sampleMany(\$data, 7);
\end{pre}

\end{example}

\begin{example}[Génération de données plus fines avec atoum]

Nous proposons encore deux exemples pour générer plusieurs dates ou chaînes de
caractères spécifiées par une expression régulière. Tout d'abord, nous allons
générer des dates au format \code{d/m H:i} (par exemple \code{28/03 12:34})
entre hier et lundi prochain, qui sont des repères temporels relatifs~:

%
\begin{pre}
\$data = \$this->realdom->date( \\
    'd/m H:i', \\
    \$this->realdom->boundinteger( \\
        \$this->realdom->timestamp('yesterday'), \\
        \$this->realdom->timestamp('next Monday') \\
    ) \\
); \\
 \\
foreach(\$this->sampleMany(\$data, 7) as \$date) \\
    // compute \$date
\end{pre}
%
En Praspel, la déclaration du domaine réaliste s'écrirait de la façon suivante~:
%
\begin{pre}
date( \\
    'd/m H:i', \\
    boundinteger( \\
        timestamp('yesterday'), \\
        timestamp('next Monday') \\
    ) \\
)
\end{pre}
%
Ensuite, nous allons générer des chaînes de caractères qui contiennent
le mots-clé \code{foo} suivi des mots-clés \code{bar} ou \code{baz}~:
%
\begin{pre}
\$data = \$this->realdom->regex('/.+foo.+ba(r|z).+/'); \\
 \\
foreach(\$this->sampleMany(\$data, 7) as \$string) \\
    // compute \$string
\end{pre}

\end{example}

\subsubsection{Générations de suites de test unitaires}

\begin{figure}

\centering

\begin{tikzbox}{boxtoolsext}{}
    \fig{\textwidth}{!}{Extension.tex}
\end{tikzbox}
%
\begin{tikzannotation}
    \node (sc) [left of=boxextsut, xshift=-4.25cm, yshift=-1.9cm] {};
    \node (Sc) [left of=sc, node distance=1.5cm] {\circled{S}};
    \draw [mywavyarrow] (Sc) -- (sc);

    \node (mc) [above of=boxextom, xshift=.4cm, yshift=-1.75cm] {};
    \node (Mc) at (Sc -| mc) {\circled{M}};
    \draw [mywavyarrow] (Mc) -- (mc);
\end{tikzannotation}

\caption{\label{figure:tools:extension} Fonctionnement de la génération de tests
unitaires avec l'extension.}

\end{figure}

Dans la partie~\ref{section:test:criteria} page~\pageref{section:test:criteria},
nous avons défini des critères de couvertures sur les contrats. Les critères
sont satisfaits par un ensemble de chemins dans les contrats. Un chemin est
représenté par un test. Praspel est capable de générer des tests correspondant à
ces chemins à partir d'un modèle objet d'un contrat. C'est le point de départ du
processus illustré dans la figure~\ref{figure:tools:extension}. Nous y
retrouvons \circled{M} et \circled{S}. À partir du modèle objet, une suite de
tests est générée. Ces tests que nous pourrions qualifiés d'abstraits (car non
exécutables) sont compilés vers des tests écrits avec l'API d'atoum (donc
exécutables). Cette étape est regroupée en un seul nœud dans la figure. atoum va
ensuite sélectionner chaque test dans cette suite et les exécuter en utilisant
le SUT associé et des données. Le nœud atoum dans la figure applique le même
processus que dans la figure~\ref{figure:tools:rac}.

La compilation de la suite de tests abstraits vers concrets s'effectue en ligne
de commande~: plusieurs classes seront analysées pour en extraire des contrats
Praspel, qui seront à leurs tours analysés pour pouvoir appliquer les
algorithmes de générations de tests unitaires sur leur modèle objet. Puis,
l'extension va compiler ces tests vers des tests écrits au format atoum. Ces
tests seront enregistrés dans les fichiers appropriés grâce à l'exportation,
décrite dans la partie~\ref{subsection:tools:compilation}. Nous pourrons par la
suite exécuter ces fichiers avec atoum.

Si nous reprenons notre classe \code{Project\bslash{}HelloWorld} avec sa méthode
\code{say}, nous aurons le contrat suivant~:
%
\begin{pre}
class HelloWorld \{ \\
 \\
    /** \\
     * \aensures \aresult: 'Hello World!'; \\
     */ \\
    public function say ( ) \{ \\
 \\
        return 'Hello World!'; \\
    \} \\
\}
\end{pre}
%
Nous allons ensuite générer automatiquement les tests associés en utilisant la
ligne de commande suivante~:
%
\begin{pre}
\$ praspel generate --class 'Project\bslash{}HelloWorld' --test-root .
\end{pre}
%
L'option \code{-\empty{}-class} précise la ou les classes à analyser. L'option
\code{-\empty{}-test-root} précise où se trouve la racine des tests, en
l'occurence à l'endroit où nous exécutons la ligne de commande. Un fichier par
classe sera généré. Le résultat du fichier que nous venons de générer est le
suivant~:

\begin{bigpre}
class HelloWorld extends \bslash{}Atoum\bslash{}PraspelExtension\bslash{}Test \{ \\
 \\
    // … \\
 \\
    public function test say n\(\sp{\circ}\)1 ( ) \{ \\
 \\
        \$this \\
            ->if(\$this->praspel->ensures['\bslash{}result']->in = \\
                realdom()->const('Hello World!')) \\
            ->then \\
                ->praspel->verdict('\bslash{}Project\bslash{}HelloWorld');  \\
 \\
        return; \\
    \} \\
\}
\end{bigpre}

\noindent Nous pouvons enfin exécuter ce test avec atoum grâce à la ligne de
commande suivante~:
%
\begin{pre}
\$ atoum --file Path/To/Our/Generated/Test.php \\
… \\
> Total test duration: 0.01 second. \\
> Total test memory usage: 0.50 Mb. \\
> Code coverage value: 100.00\% \\
> Running duration: 0.06 second. \\
> Success (1 test, 1/1 method, 0 void method, \\
           0 skipped method, 1 assertion)!
\end{pre}
%
Sans surprise, la suite de tests a été exécutée avec succès.
