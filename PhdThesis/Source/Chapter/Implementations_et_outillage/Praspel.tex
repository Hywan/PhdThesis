\section{Praspel}
\label{section:tools:praspel}

Cette partie présente l'implémentation et l'outillage de Praspel. Tout d'abord,
la partie~\ref{subsection:tools:interpretation} présente comment le langage est
analysé. Ensuite, la partie~\ref{subsection:tools:model} présente l'architecture
de Praspel et son modèle objet, la pièce centrale. Les
parties~\ref{subsection:tools:compilation}
et~\ref{subsection:tools:disassembler} présentent respectivement comment
exporter ou importer le modèle et comment le désassembler. Enfin, la
partie~\ref{subsection:tools:evaluation} présente comment évaluer le modèle.

\begin{figure}

\centering

\begin{tikzbox}{boxtoolsdispatching}{xshift=-2.2cm}
\begin{pre}
class C \{ \\
 \\
    /** \\
     * \arequires  …; \tikzref{boxtoolsdispatchingA} \\
     * \aensures   …; \\
     * \athrowable …; \tikzref{boxtoolsdispatchingB} \\
     */ \\
    public function f ( ) \{ \tikzref{boxtoolsdispatchingC} \\
 \\
        // … \\
    \} \tikzref{boxtoolsdispatchingD} \\
\}
\end{pre}
\end{tikzbox}
%
\begin{tikzannotation}
    \draw [mybrace]
        (boxtoolsdispatchingA.north -| boxtoolsdispatching.east)
        -- node (P) [mybracetext] {Praspel}
        (boxtoolsdispatchingB.south -| boxtoolsdispatching.east);
    \draw [mybrace]
        (boxtoolsdispatchingC.north -| boxtoolsdispatching.east)
        -- node (S) [mybracetext] {SUT}
        (boxtoolsdispatchingD.south -| boxtoolsdispatching.east);

    \node (Pc) [right of=P, node distance=1.8cm] {\circled{P}};
    \draw [mywavyarrow] (P) -- (Pc);

    \node (Sc) at (S -| Pc) {\circled{S}};
    \draw [mywavyarrow] (S) -- (Sc);
\end{tikzannotation}

\caption{\label{figure:tools:dispatching} Répartitions des composants d'un
programme.}

\end{figure}

La figure~\ref{figure:tools:dispatching} nous rappelle que, dans une classe, les
commentaires qui annotent les méthodes contiennent des contrats écrits en
Praspel. Les contrats seront donc extraits de ces commentaires pour être envoyés
dans les outils de Praspel. Les méthodes vont constituer le SUT et seront
utilisées lors de l'exécution des tests. Dans les prochaines figures, ces
données sont identifiées par une lettre cerclée afin d'indiquer leur flux dans
les différents outils. Ainsi, la notation %
\tikz[baseline=(p.base)] {
    \node (p) [anchor=east] at (0, 0) {Praspel};
    \node (pc) at (1, 0) {\circled{P}};
    \draw [mywavyarrow] (p) -- (pc);
} %
signifie que le composant Praspel sera représenté par \circled{P} dans les
figures suivantes. Le même raisonnement s'applique pour le SUT représenté par
\circled{S}.

\begin{figure}

\drawfig{10.4cm}{!}{node distance=3cm}{

    \node (model) [rectangle, minimum height=3cm] {
        modèle objet

        \raisebox{0cm}[1.5cm]{
            \begin{tikzpicture}[node distance=4cm]
            \node [state, scale=.15] (foo1) {};
            \node [state, scale=.15, below left of=foo1] (foo2) {};
            \node [state, scale=.15, below right of=foo1] (foo3) {};
            \node [state, scale=.15, below left of=foo3] (foo4) {};
            \node [state, scale=.15, below right of=foo3] (foo5) {};
            \draw [thick] (foo1) -- (foo2) (foo1) -- (foo3) (foo3) -- (foo4) (foo3) -- (foo5);
            \end{tikzpicture}
        }
    };
    \node (language) [left of=model] {
        \tikzref{boxtoolsomA} langage
    };
    \node (php) [right of=model] {
        PHP
    };

    \draw [arrow] (language.north)
                      -- node[auto] {\small analyses}
                      ++(0, 1.68)
                      -| ([xshift=-5pt] model.north);
    \draw [arrow] ([xshift=+5pt] model.north)
                      --
                      ++(0, 0.5)
                      -| node[auto, yshift=-.9cm] {\small exportation} (php.north);
    \draw [arrow] (php.south)
                      -- node[auto] {\small importation}
                      ++(0, -1.72)
                      -| ([xshift=+5pt] model.south);
    \draw [arrow] ([xshift=-5pt] model.south)
                      --
                      ++(0, -0.5)
                      -| node[auto, yshift=.8cm] {\small désassemblage} (language.south);

    % does not work in annotations, don't know why…
    \node (Mc) [above of=model, node distance=3.5cm] {\circled{M}};
    \draw [mywavyarrow] (model) -- (Mc);
}
%
\begin{tikzannotation}
    \node (pc) [yshift=-.1cm] at (boxtoolsomA -| boxtoolsomA) {};
    \node (Pc) [left of=pc, node distance=1.8cm] {\circled{P}};
    \draw [mywavyarrow] (Pc) -- (pc);
\end{tikzannotation}

\caption{\label{figure:tools:praspel} Fonctionnement schématique de Praspel.}

\end{figure}

Praspel existe sous différentes formes. Nous les expliquons en nous appyant sur
la figure~\ref{figure:tools:praspel} qui montre l'agencement de ces différentes
formes et le passage de l'une vers l'autre~:
%
\begin{itemize}

\item le langage est la forme textuelle de Praspel, provenant des commentaires
des méthodes. Il est défini par la grammaire dans la
partie~\ref{section:language:praspel} page~\pageref{section:language:praspel}.
En utilisant des analyseurs, le langage est transformé en modèle objet~;

\item le modèle objet \circled{M} est la pièce centrale de Praspel~: c'est un
ensemble d'objets en mémoire réprésentant chaque partie du langage (clause,
variable, opérateur etc.). Tous ces objets sont imbriqués et forme une structure
qui s'apparente à un arbre~;

\item le modèle objet peut être exporté sous forme de code PHP qui, quand il est
exécuté, reconstruit en mémoire le modèle objet depuis lequel il a été généré~;

\item le désassemblage permet de transformer le modèle objet sous sa forme
textuelle.

\end{itemize}
%
Les parties suivantes détaillent ces différentes étapes.

\subsection{Analyses}
\label{subsection:tools:interpretation}

À l'aide de la grammaire de Praspel, définie dans la
partie~\ref{section:language:praspel} page~\pageref{section:language:praspel},
exprimée avec le langage PP (voir l'annexe~\ref{appendices:grammar_of_praspel})
et un compilateur, défini dans les sections~\ref{subsection:data:pp} et
\ref{subsection:data:compiler-compiler} pages~\pageref{subsection:data:pp}
et~\pageref{subsection:data:compiler-compiler}, l'AST du contrat analysé sera
produit. Cet AST sera ensuite visité afin de construire le modèle objet. Pour y
arriver, nous avons deux façons de faire. La première détaille toutes les
étapes, à savoir l'utilisation du compilateur pour produire un AST, puis l'appel
d'un visiteur pour construire le modèle objet~; ainsi~:

%
\begin{pre}
\$parser      = Hoa\bslash{}Compiler\bslash{}Llk::load( \\
    new Hoa\bslash{}File\bslash{}Read('hoa://Library/Praspel/Grammar.pp') \\
);                                        \tikzref{codetoolspraspel} \\
\$ast         = \$parser->parse('@requires i: 7..42;'); \\
\$interpreter = new Hoa\bslash{}Praspel\bslash{}Visitor\bslash{}Interpreter(); \\
\tikzref{codetoolsom}\$model       = \$interpreter->visit(\$ast);
\end{pre}
%
La variable \code{\$parser} contient l'analyseur lexical et syntaxique chargé
depuis la grammaire. La variable \code{\$ast} contient l'AST produit par
l'analyse lexicale et l'analyse syntaxique d'une chaîne de caractères
représentant du code Praspel. Ces analyses sont exécutées en appelant la méthode
\code{parse} sur le compilateur. Ensuite, nous instancions un visiteur dans la
variable \code{\$interpreter} qui va visiter, grâce à la méthode \code{visit},
l'AST pour le transformer en modèle objet.
%
\begin{tikzannotation}
    \node (Pc) [right of=codetoolspraspel, node distance=3cm, yshift=-.5cm] {\circled{P}};
    \draw [mywavyarrow] (Pc) to[out=135, in=45, distance=.5cm] (codetoolspraspel.south);

    \node (Mc) [above of=codetoolsom, node distance=1.4cm, xshift=-.45cm] {\circled{M}};
    \draw [mywavyarrow] (codetoolsom.west) to[out=140, in=-90] (Mc);
\end{tikzannotation}

La seconde façon de faire consiste à utiliser un raccourci qui est la méthode
(statique) \code{interprete}~; ainsi~:
%
\begin{pre}
\$model = Hoa\bslash{}Praspel::interprete('@requires i: 7..42;');
\end{pre}
%
produira le même résultat.

Durant la production de l'AST, les erreurs {\strong lexicales} et {\strong
syntaxiques} seront détectées. Et durant la production du modèle objet, les
erreurs {\strong sémantiques} seront détectées.

\subsection{Modèle objet}
\label{subsection:tools:model}

Le modèle objet est la pièce centrale de Praspel. Chaque construction du langage
est représentée par une classe. Le diagramme de classes du modèle objet est
présenté dans la figure~\ref{figure:tools:om}. Toutes les clauses sont
représentées par la classe abstraite \code{Clause}. La classe \code{Behavior}
représente la clause \abehavior. Elle peut contenir une clause \arequires
représentée par la classe \code{Requires}, une clause \aensures représentée par
la classe \code{Ensures}, une clause \athrowable représentée par la classe
\code{Throwable}, une clause \adescription représentée par la classe
\code{Description et, une clause \adefault représentée par la classe
\code{DefaultBehavior}. Une clause \abehavior peut contenir d'autres clauses
\abehavior \via la classe \code{Collection} qui représente une collection de
clauses. Une spécification, représentée par la classe \code{Specification}, est
un comportement qui peut contenir des clauses supplémentaires~: \ainvariant
représentée par la classe \code{Invariant} et \ais représentée par la classe
\code{Is}.

Certaines clauses contiennent des expressions (voir la règle syntaxique
\grule{expression} dans la grammaire de Praspel présentée dans la
figure~\ref{figure:language:grammar_part2}
page~\pageref{figure:language:grammar_part2}). Ces expressions sont représentées
par la classe \code{Declaration}. Une expression est composée de variables,
représentée par la classe \code{Variable}. Une variable est, du point de vue des
domaines réalistes, un \inenglish{holder}, c'est~à~dire une structure qui porte
des domaines réalistes.

Il existe trois sortes de variable~: normale, implicite et empruntée. Une
variable normale est une variable classique du contrat, représentée par la
classe \code{Variable}. Une variable implicite représentée par la classe
\code{Implicit} est une variable non-déclarée, comme \code{this} dans le cas des
méthodes. En effet, en PHP, la variable contenant l'objet est toujours
implicite, c'est~à~dire non déclarée. Ce comportement se retrouve dans Praspel
où il n'est pas nécessaire de spécifier \code{this}. Enfin, une variable
empruntée est une variable extérieure à la méthode, comme \code{this->foo} qui
représente un attribut de classe, ou une variable extérieure à l'état courant du
système, comme \aold{\code{f}} qui fait référence la variable \code{foo} dans le
pré-état lorsque que le système est dans son post-état.

\begin{sidewaysfigure}

\fig{!}{8.5cm}{Object_model.pdf}

\caption{\label{figure:tools:om} Modèle objet de Praspel.}

\end{sidewaysfigure}

Les détails du modèle objet ne sont pas présents dans la
figure~\ref{figure:tools:om} pour des raisons de simplicité. Le lien entre la
bibliothèque \code{Hoa\bslash{}Praspel} et \code{Hoa\bslash{}Realdom} se fait
par les variables qui sont des \inenglish{holders}. Autant les domaines
réalistes peuvent être utilisés sans Praspel, autant l'inverse n'est pas
envisageable.

Tous les objets de ce modèle sont imbriqués pour former une structure qui
s'apparente à un arbre. La racine de cet arbre est représentée par une instance
de la classe \code{Specification}. Cet arbre est visitable, c'est~à~dire que
nous pouvons lui appliquer des visiteurs, comme nous allons le voir dans les
parties suivantes.

\subsection{Exportation et importation}
\label{subsection:tools:compilation}

Le modèle objet peut être exporté de la mémoire vers un fichier sous la forme
de code PHP. Quand ce code PHP est exécuté, c'est une importation car il
construit en mémoire le modèle objet depuis lequel il a été généré.
Cela permet de «~sauvegarder~» le modèle en évitant la phase
des analyses qui peuvent être coûteuses dans certaines situations. Ainsi, à
partir d'un modèle contenu dans la variable \code{\$model} et d'un visiteur,
nous allons produire du code PHP que nous allons afficher~:
%
\begin{pre}
\$compiler = new Hoa\bslash{}Praspel\bslash{}Visitor\bslash{}Compiler(); \\
echo \$compiler->visit(\$mo\tikzref{codetoolsom}del);
\end{pre}
%
Cet exemple produira le code PHP suivant~:
%
\begin{pre}
\$praspel = new \bslash{}Hoa\bslash{}Praspel\bslash{}Model\bslash{}Specification(); \\
 \\
\$requires = \$praspel->getClause('requires'); \\
\$requires['i']->in = realdom()->boundinteger(7, 42);
\end{pre}
%
qui permet de construire le modèle objet contenu dans \code{\$model},
c'est~à~dire que la valeur de la variable \code{\$model} est strictement égale à
la valeur de la variable \code{\$praspel}. Cet exemple nous permet également
d'avoir un aperçu de l'API du modèle objet~: elle se veut le plus simple et
compréhensible possible. Une spécification est instanciée. Sur cette
spécification, la clause \arequires est obtenue pour y déclarer une variable
\code{i} dont les valeurs sont décrites par (ou présentes dans, \code{->in}) la
disjonction de domaines réalistes, créée à l'aide de la fonction \code{realdom}.
Cette fonction se comporte comme une macro.
%
\begin{tikzannotation}
    \node (Mc) [right of=codetoolsom, node distance=1.6cm] {\circled{M}};
    \draw [mywavyarrow] (Mc) to[out=-150, in=-70, distance=.2cm] (codetoolsom);
\end{tikzannotation}

\subsection{Désassemblage}
\label{subsection:tools:disassembler}

Le désassemblage permet de transformer le modèle objet en mémoire sous sa forme
textuelle. C'est l'opération inverse des analyses de la
partie~\ref{subsection:tools:interpretation}. Il permet par exemple d'indiquer
des erreurs avec précisions sous la forme originale de Praspel. C'est tout à
fait comparable à un \inenglish{pretty-printer}. Ainsi~:
%
\begin{pre}
\$disassembler = new Hoa\bslash{}Praspel\bslash{}Visitor\bslash{}Praspel(); \\
echo \$disassembler->visit(\$mo\tikzref{codetoolsom}del);
\end{pre}
%
Encore une fois, nous instancions un visiteur dans la variable
\code{\$disas\-sem\-bler} pour ensuite visiter le modèle objet. Cette exemple
produira le résultat suivant~:
%
\begin{pre}
@requires i: boundinteger(7, 42);
\end{pre}
%
Notons que le sucre syntaxique \code{7..42} a disparu.
%
\begin{tikzannotation}
    \node (Mc) [right of=codetoolsom, node distance=1.6cm] {\circled{M}};
    \draw [mywavyarrow] (Mc) to[out=-150, in=-70, distance=.2cm] (codetoolsom);
\end{tikzannotation}

Ce dernier processus, complété des précédents, nous permet de passer d'une
représentation vers n'importe quelle autre représentation de Praspel~: soit
textuelle, soit objet, soit PHP.

\subsection{Évaluation avec un \inenglish{Runtime Assertion Checker}}
\label{subsection:tools:evaluation}

\begin{figure}

\centering

\begin{tikzbox}{boxtoolsrac}{xshift=1.1cm}
    \fig{11cm}{!}{RAC.tex}
\end{tikzbox}
%
\begin{tikzannotation}
    \node (mc) [xshift=.2cm, yshift=-.97cm] at (boxracom -| boxtoolsrac.west) {};
    \node (Mc) [left of=mc, node distance=1.3cm] {\circled{M}};
    \draw [mywavyarrow] (Mc) -- (mc);

    \node (sc) [xshift=.2cm, yshift=-.5cm] at (boxracsut -| boxtoolsrac.west) {};
    \node (Sc) [left of=sc, node distance=1.3cm] {\circled{S}};
    \draw [mywavyarrow] (Sc) -- (sc);
\end{tikzannotation}

\caption{\label{figure:tools:rac} Fonctionnement schématique du
\inenglish{Runtime Assertion Checker}.}

\end{figure}

Ce sont les \inenglish{assertion checkers} qui sont responsables d'évaluer
Praspel. Le fonctionnement d'un \inenglish{assertion checker} est illustré dans
la figure~\ref{figure:tools:rac}. Il travaille avec 3~composants~: un modèle
objet qui représente un contrat (provenant par exemple des commentaires des
méthodes), un SUT (par exemple la méthode commentée) et des données. Ces
composants forment un test~! Un \inenglish{assertion checker} est donc capable
de calculer le verdict d'un test.

Actuellement, nous nous appuyons sur un \inenglish{Runtime Assertion Checker},
abrégé RAC, pour calculer le verdict du test. Ce verdict est basé sur la
vérification d'assertions à l'exécution. Quand la vérification d'une assertion
échoue, une erreur spécifique est produite. Les erreurs du RAC (aussi appelées
les \inenglish{Praspel failures} ou erreurs Praspel) peuvent être de 5~sortes~:
%
\begin{enumerate}

\item \inenglish{precondition failure}, quand une précondition n'est pas
satisfaite lors de l'invocation de la méthode~;

\item \inenglish{postcondition failure}, quand une postcondition n'est pas
satisfaite après l'exécution de la méthode~;

\item \inenglish{throwable failure}, quand l'exécution de la méthode lève une
exception inattendue~;

\item \inenglish{invariant failure}, quand un invariant de classe est cassé~; et

\item \inenglish{internal precondition failure}, qui correspond à la propagation
d'une \inenglish{precondition failure} à un niveau supérieur.

\end{enumerate}
%
Le test réussit si aucune erreur Praspel n'est détectée. Autrement, il échoue,
et l'erreur avec des informations supplémentaires est consignée.

À présent, illustrons comment utiliser le RAC de Praspel où le système sous test
est une fonction \code{f} avec un seul paramètre \code{\$i}~:
%
\begin{pre}
\$rac = new Hoa\bslash{}Praspel\bslash{}AssertionChecker\bslash{}Runtime( \\
   \tikzref{codetoolsom} \$model, \\
    xcallable('f')\tikzref{codetoolssut} \\
); \\
\$rac->setData(['i' => 13]); \\
\$verdict = \$rac->evaluate();
\end{pre}
%
\begin{tikzannotation}
    \node (Mc) [left of=codetoolsom, node distance=1.7cm] {\circled{M}};
    \draw [mywavyarrow] (Mc) -- (codetoolsom.center);

    \node (Sc) [right of=codetoolssut, node distance=1.9cm] {\circled{S}};
    \draw [mywavyarrow] (Sc) -- (codetoolssut.east);
\end{tikzannotation}
%
Dans ce cas, la variable \code{\$verdict} contiendra \code{true} car la
précondition (\code{\arequires i: 7..42}) est bien respectée. Si nous voulons
que les données soient générées automatiquement, nous devrons appeler la méthode
\code{\$ac->au\-to\-ma\-ti\-cal\-ly\-Ge\-ne\-ra\-te\-Data(true)} en plus de
définir le générateur numérique par défaut des domaines réalistes afin de
pouvoir appeler les méthodes \code{sample} sur ces derniers~:
%
\begin{pre}
Hoa\bslash{}Realdom::setDefaultSampler(new Hoa\bslash{}Math\bslash{}Sampler\bslash{}Random()); \\
\$rac->automaticallyGenerateData(true); \\
\end{pre}

\begin{example}[Vérifications des assertions]

Dans la figure~\ref{figure:language:short_contract}
page~\pageref{figure:language:short_contract}\footnote{On la replace ici~?}, le
système sous test peut être la méthode \code{store}, le contrat est le contrat
associé à cette méthode et les données sont soit fournies, soit générées
automatiquement. Considérons par exemple la méthode \code{store} et son contrat
dans la figure~\ref{figure:language:short_contract}.
%
Avec le jeu (\code{'foo'}, \code{null}), nous aurons une \inenglish{precondition
failure}~: la contrainte \code{file: class('File')} n'est pas respectée car la
méthode \code{predicate(\$q)} du domaine réaliste \code{Class} avec $\code{\$q}
= \code{'foo'}$ retourne \code{false}, et aucun autre domaine réaliste n'est
spécifié pour cette variable.
%
Avec le jeu de paramètres (\code{new File(…)}, \code{null}), avec un fichier
vide, nous activerons le comportement par défaut et nous n'observerons aucune
erreur sur les préconditions. Nous sommes maintenant dans le post-état. Si la
méthode \code{store} retourne un booléen et que la méthode \code{isAttached} de
l'objet \code{File} retourne \code{true}, aucune erreur non plus sur la
postcondition.  Si cette méthode retourne autre chose que \code{true} ou que la
méthode \code{store} retourne autre chose qu'un booléen, nous aurons une
\inenglish{postcondition failure}. Si une exception est levée, nous aurons une
\inenglish{throwable failure}.
%
Avec le même jeu de paramètres mais un fichier de grande taille ne pouvant être
enregistré, nous activerons le comportement \code{full}. Nous devrons avoir une
exception levée de type \code{AllocationException}, avec une méthode
\code{getFilesystem} qui retournera l'instance de notre système de fichier. Le
fichier ne devra pas non plus être attaché, \ie sa méthode \code{isAttached}
devra retourner \code{false}. Si une autre exception est levée ou que la
postcondition exceptionnelle n'est pas respectée, nous aurons une
\inenglish{throwable failure}. Si aucune exception n'est levée, nous aurons une
\inenglish{postcondition failure}.
%
Si avant ou après l'exécution de la méthode \code{store}, un invariant est
cassé, nous aurons une \inenglish{invariant failure}. Par exemple, si l'attribut
\code{\_map} contient autre chose que des objets \code{File} ou que sa taille
dépasse 65535.
%
Enfin, si la méthode \code{store} fait appel à la méthode \code{getUsage} en ne
respectant pas sa précondition (ici en lui donnant un argument par exemple),
alors une \inenglish{precondition failure} sera émise depuis la méthode
\code{getUsage}, qui sera ensuite traduite en \inenglish{internal precondition
failure} depuis la méthode \code{store}.

\end{example}
